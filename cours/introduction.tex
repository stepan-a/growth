\synctex=1

\documentclass[10pt,notheorems]{beamer}

\usepackage{etex}
\usepackage{fourier-orns}
\usepackage{ccicons}
\usepackage{amssymb}
\usepackage{amstext}
\usepackage{amsbsy}
\usepackage{amsopn}
\usepackage{amscd}
\usepackage{amsxtra}
\usepackage{amsthm}
\usepackage{float}
\usepackage{color, colortbl}
\usepackage{mathrsfs}
\usepackage{bm}
\usepackage{lastpage}
\usepackage[nice]{nicefrac}
\usepackage{setspace}
\usepackage{ragged2e}
\usepackage{listings}
\usepackage{algorithms/algorithm}
\usepackage{algorithms/algorithmic}
\usepackage[frenchb]{babel}
\usepackage{tikz,pgfplots,pgfplotstable}
\pgfplotsset{compat=newest}
\usetikzlibrary{patterns, arrows, decorations.pathreplacing, decorations.markings, calc}
\pgfplotsset{plot coordinates/math parser=false}
\newlength\figureheight
\newlength\figurewidth
%\usepackage[utf8x]{inputenc}
\usepackage{cancel}
\usepackage{tikz-qtree}
\usepackage{dcolumn}
\usepackage{adjustbox}
\usepackage{environ}
\usepackage[cal=boondox]{mathalfa}
\usepackage{manfnt}
\usepackage{hyperref}
\hypersetup{
  colorlinks=true,
  linkcolor=blue,
  filecolor=black,
  urlcolor=black,
}
\usepackage{venndiagram}

% Git hash
\usepackage{xstring}
\usepackage{catchfile}
\immediate\write18{git rev-parse HEAD > git.hash}
\CatchFileDef{\HEAD}{git.hash}{\endlinechar=-1}
\newcommand{\gitrevision}{\StrLeft{\HEAD}{7}}

\newcommand{\trace}{\mathrm{tr}}
\newcommand{\vect}{\mathrm{vec}}
\newcommand{\tracarg}[1]{\mathrm{tr}\left\{#1\right\}}
\newcommand{\vectarg}[1]{\mathrm{vec}\left(#1\right)}
\newcommand{\vecth}[1]{\mathrm{vech}\left(#1\right)}
\newcommand{\iid}[2]{\mathrm{iid}\left(#1,#2\right)}
\newcommand{\normal}[2]{\mathcal N\left(#1,#2\right)}
\newcommand{\dynare}{\href{http://www.dynare.org}{\color{blue}Dynare}}
\newcommand{\sample}{\mathcal Y_T}
\newcommand{\samplet}[1]{\mathcal Y_{#1}}
\newcommand{\slidetitle}[1]{\fancyhead[L]{\textsc{#1}}}

\newcommand{\R}{{\mathbb R}}
\newcommand{\C}{{\mathbb C}}
\newcommand{\N}{{\mathbb N}}
\newcommand{\Z}{{\mathbb Z}}
\newcommand{\binomial}[2]{\begin{pmatrix} #1 \\ #2 \end{pmatrix}}
\newcommand{\bigO}[1]{\mathcal O \left(#1\right)}
\newcommand{\red}{\color{red}}
\newcommand{\blue}{\color{blue}}

\renewcommand{\qedsymbol}{C.Q.F.D.}

\newcolumntype{d}{D{.}{.}{-1}}
\definecolor{gray}{gray}{0.9}
\newcolumntype{g}{>{\columncolor{gray}}c}

\setbeamertemplate{theorems}[numbered]

\theoremstyle{plain}
\newtheorem{theorem}{Théorème}

\theoremstyle{definition} % insert bellow all blocks you want in normal text
\newtheorem{definition}{Définition}
\newtheorem{properties}{Propriétés}
\newtheorem{lemma}{Lemme}
\newtheorem{property}[properties]{Propriété}
\newtheorem{example}{Exemple}
\newtheorem*{idea}{Éléments de preuve} % no numbered block

\setbeamertemplate{footline}{
  {\hfill\vspace*{1pt}\href{http://creativecommons.org/licenses/by-sa/3.0/legalcode}{\ccbysa}\hspace{.1cm}
    \raisebox{-.1cm}{\href{https://github.com/stepan-a/growth}{\includegraphics[scale=.015]{../img/git.png}}}\enspace
    \href{https://github.com/stepan-a/growth/blob/\HEAD/cours/introduction.tex}{\gitrevision}\enspace--\enspace\today\enspace
  }}

\setbeamertemplate{navigation symbols}{}
\setbeamertemplate{blocks}[rounded][shadow=true]
\setbeamertemplate{caption}[numbered]

\newenvironment{notes}
{\bgroup \justifying\bgroup\tiny\begin{spacing}{1.0}}
  {\end{spacing}\egroup\egroup}

\newenvironment{exercise}[1]
{\bgroup \small\begin{block}{Ex. #1}}
  {\end{block}\egroup}

\newenvironment{defn}[1]
{\bgroup \small\begin{block}{Définition. #1}}
  {\end{block}\egroup}

\newenvironment{exemple}[1]
{\bgroup \small\begin{block}{Exemple. #1}}
  {\end{block}\egroup}


\begin{document}

\title{Croissance\\\small{Introduction}}
\author[S. Adjemian]{St\'ephane Adjemian}
\institute{\texttt{stephane.adjemian@univ-lemans.fr}}
\date{Septembre 2020}

\begin{frame}
  \titlepage{}
\end{frame}

\section{Introduction}

\begin{frame}
  \frametitle{Deux regards sur la croissance}
  \begin{itemize}

  \item On peut s'intéresser à la croissance économique dans
    deux directions:

    \bigskip

    \begin{itemize}

    \item[$\Rightarrow$] L'évolution dans \textbf{le temps} de la croissance.\newline

      Pourquoi le taux de croissance d'une économie change t-il significativement en différentes périodes de son histoire~?\newline

    \item[$\Rightarrow$] L'hétérogénéité dans \textbf{la coupe des nations} de la croissance.\newline

      Pourquoi le taux de croissance de la Corée du sud est-il plus important de celui des économies occidentales sur la période récente~?

    \end{itemize}

    \bigskip

  \item Ces directions ne sont pas forcément interchangeables. Nous
    montrerons néanmoins dans ce cours qu'un modèle développé pour expliquer la
    croissance dans la première direction peut expliquer une bonne
    part de l'hétérogénéité internationale (observée) des taux de
    croissance.

  \end{itemize}

\end{frame}


\begin{frame}
  \frametitle{Séries temporelles}
  \framesubtitle{Les États Unis d'Amérique du Nord, I}
  \begin{center}
    \input{../tikz/us-real-gdp-per-capita.tex}
  \end{center}
\end{frame}


\begin{frame}
  \frametitle{Séries temporelles}
  \framesubtitle{Les États Unis d'Amérique du Nord, I'}
  \begin{center}
    \begin{tikzpicture}

  \begin{axis}[/pgf/number format/1000 sep={},scale=1, title= \small{PIB par tête}, enlargelimits=false, color=blue!30!black]
    \addplot[style={black,mark=none}] table[x index=0, y expr=exp(\thisrowno{1}), col sep=space]{../data/usa_logged_rgdp_per_capita1.dat};
  \end{axis}

\end{tikzpicture}
  \end{center}
\end{frame}


\begin{frame}
  \frametitle{Séries temporelles}
  \framesubtitle{Les États Unis d'Amérique du Nord, II}

  \begin{description}

  \item[\textbf{Rappel 1}] Le taux de croissance annuel moyen ($\bar g$) est calculé à partir du facteur de croissance ($G$)~:
    \[
      \bar g = (G^{1/T}-1)\times 100
    \]
    où $T$ est le nombre d'années sur lequel le facteur de croissance est calculé.\newline

  \item[\textbf{Rappel 2}] Le taux de croissance entre t et t-1 peut être approximé par le logarithme népérien du facteur de croissance~:
    \[
      g_t \approx 100\times\log \left(\frac{y_t}{y_{t-1}}\right)
    \]
    Cette approximation est « bonne » quand le taux de croissance est proche de zéro.
  \end{description}

\end{frame}

\begin{notes}
  \begin{itemize}

  \item \textbf{Rappel 2.} Nous savons que le taux de croissance est égal au facteur de croissance moins 1. En effet
    \[
      \begin{split}
        G_t-1 &= \frac{y_t}{y_{t-1}}-1\\
        &= \frac{y_t-y_{t-1}}{y_{t-1}} \equiv g_t
      \end{split}
    \]
    De façon équivalente nous avons donc~:
    \[
      G_t = 1+g_t
    \]
    En appliquant le logarithme népérien, il vient~:
    \[
      \log (G_t) = \log(1+g_t)
    \]
    Par ailleurs, voir le cours de mathématiques ou la page wikipédia sur les \href{https://fr.wikipedia.org/wiki/S%C3%A9rie_de_Taylor}{séries de Taylor}, nous savons que~:
    \[
      log(1+x) = \sum_{i=1}^{\infty}\frac{(-1)^{n+1}}{n!}x^n
    \]
    tant que $x\in]-1,1[$. Nous pouvons donc approximer $\log(1+g)$ par $g$, car si $g$ est petit (par exemple si le taux de croissance est de 2\% nous avons $g=0,02$) alors les termes suivants de la série de Taylor (on a $g^2=4\times 10^{-4}$, $g^3 = 8\times 10^{-6}$, \ldots) sont très petits (négligeables). En utilisant cette approximation, nous avons finalement:
    \[
      g_t \approx \log(G_t)
    \]
    Nous ré-utiliserons, plus loin dans le cours, l'approximation de Taylor pour simplifier les modèles sur lequels nous travaillons.
  \end{itemize}
\end{notes}



\begin{frame}
  \frametitle{Séries temporelles}
  \framesubtitle{Les États Unis d'Amérique du Nord, III}

  \begin{itemize}
  \item Sur la période 1800-2016 le taux de croissance annuel moyen du PIB par tête est $\bar g \approx 1.53\%$.\newline

  \item Sur la même période, le facteur de croissance, le ratio du PIB de 2016 au PIB de 1800, est $G \approx 26,8$. En un peu plus de deux siècles la production par tête d'un américain a été multpliée par environ 27.\newline

  \item Approximativement, le taux de croissance moyen est aussi donné graphiquement par la pente du PIB par tête en logarithme.\newline

  \item On remarque que le taux de croissance est temporairement très supérieur au taux de croissance moyen après les creux (voir par exemple la crise de 1929).
  \end{itemize}

\end{frame}


\begin{frame}
  \frametitle{Séries temporelles}
  \framesubtitle{Les États Unis d'Amérique du Nord, IV}

  \begin{center}
    \input{../tikz/us-real-gdp-per-capita-with-regression.tex}
  \end{center}

\end{frame}



\begin{frame}
  \frametitle{Séries temporelles}
  \framesubtitle{France, I}

  \begin{center}
    \input{../tikz/fra-1-real-gdp-per-capita.tex}
  \end{center}

\end{frame}

\begin{frame}
  \frametitle{Séries temporelles}
  \framesubtitle{France, I'}

  \begin{center}
    \begin{tikzpicture}

  \begin{axis}[/pgf/number format/1000 sep={},scale=1, title= \small{Logarithme du PIB par tête (US\$ 2011)}, enlargelimits=false, color=blue!30!black]
    \addplot[style={black,mark=none}] table[x index=0, y index=1, col sep=space]{../data/fra_logged_rgdp_per_capita2.dat};
  \end{axis}

\end{tikzpicture}
  \end{center}

\end{frame}


\begin{frame}
  \frametitle{Séries temporelles}
  \framesubtitle{France, II}

  \begin{itemize}

  \item La croissance est un phénomène récent en France, postérieur à la révolution.\newline

  \item On identifie facilement les deux guerres mondiales. On observe que le taux de croissance est temporairement plus élevé après ces épisodes.\newline

  \item On identifie les \textit{trentes glorieuses}, et la baisse notable du taux de croissance à partir des années 70-80.\newline

  \item Sur la période 1280-2016, le taux de croissance annuel moyen est $\bar g \approx 0,49$, mais sur la période 1820-2016 on a un taux de croissance annuel moyen nettement plus important $\bar g \approx 1,69$.\newline

  \item En un peu plus de 7 siècle la France a multiplié son niveau de PIB par tête par un peu plus de 36.

  \end{itemize}

\end{frame}

\begin{frame}
  \frametitle{Séries temporelles}
  \framesubtitle{Grande Bretagne}

  \begin{center}
    \begin{tikzpicture}

  \begin{axis}[/pgf/number format/1000 sep={},scale=1, title= \small{Logarithme du PIB par tête}, enlargelimits=false, color=blue!30!black]
    \addplot[style={black,mark=none}] table[x index=0, y index=1, col sep=space]{../data/gbr_logged_rgdp_per_capita.dat};
  \end{axis}

\end{tikzpicture}
  \end{center}

\end{frame}


\begin{frame}
  \frametitle{Contrefactuel}
  \framesubtitle{Le prix d'un point de croissance}

  \begin{itemize}

  \item Un point de croissance annuel moyen sur d'aussi longues périodes vaut très cher un termes de niveau de PIB.\newline

  \item Entre 1870 (après la guerre civile) et 2000 le PIB par tête aux États Unis (\$ 1996) a été multiplié par 14, avec un taux de croissance annuel moyen de 2,26\%. Sur cette période le PIB par tête est passé de  3736\$ à 53015\$.\newline

  \item En imaginant un monde où les États Unis auraient connus un taux de croissance annuel moyen inférieur de 1 point sur ces 130 années, nous observerions en 2000 un PIB par tête de seulement 19026\$, soit un facteur de croissance de 5,1. À peine plus d'un tiers du niveau effectivement observé en 2000~!\ldots\newline

  \item \ldots Entre l'Île Maurice et les Seychelles en 2000~!

  \end{itemize}

\end{frame}


\begin{frame}
  \frametitle{Comparaisons internationales, I}

  \bigskip

  \begin{itemize}

  \item Comment comparer des niveaux de PIB~?\newline

  \item En 1960:

    \begin{itemize}
    \item La Suisse est le pays le plus riche en termes de PIB par tête (20086\$ base 2011).
    \item Le Botswana est le pays le plus pauvre en termes de PIB par tête (405\$ base 2011).
    \item[$\Rightarrow$] Le PIB par tête est 49,6$\times$ moindre au Botswana~!\newline
    \end{itemize}

    $\Rightarrow$ Le PIB par tête est 49,6$\times$ moindre au Botswana~!\newline

  \item En 2000:

    \begin{itemize}
    \item Le Luxembourg est le pays le plus riche en termes de PIB par tête (57062\$ base 2011).
    \item La République Démocratique du Congo (ex Zaïre) est le pays le plus pauvre en termes de PIB par tête (496\$ base 2011).
    \item[$\Rightarrow$] Le PIB par tête est 115$\times$ moindre en République Démocratique du Congo~!\newline
    \end{itemize}

  \item Les différences sont très importantes et augmentent.\newline

  \end{itemize}

\end{frame}


\begin{frame}
  \frametitle{Comparaisons internationales, II}

  \bigskip

  \begin{itemize}

  \item Le taux de croissance annuel moyen (dans le temps) moyen (dans la coupe)~:
    \[
      \bar{\bar g} = \frac{1}{N}\sum_{i=1}^{N} \bar g_{i}
    \]

  \item Entre 1960 et 2000, $\bar{\bar g} \approx 2\%$ (pour un échantillon de 111 pays).\newline

  \item L'hétérogénéité augmente:
    \begin{itemize}
    \item 1960: $\sigma (log(y)) \approx 0.93$
    \item 2000: $\sigma (log(y)) \approx 1.29$
    \end{itemize}

    \bigskip

  \item Le pire: $\bar g_{i} = -4,2\%$ pour la République Démocratique du Congo.\newline

  \item Le meilleur: $\bar g_{i} = 8,9\%$ pour Malte.\newline

  \end{itemize}

\end{frame}


\begin{frame}
  \frametitle{Comparaisons internationales, III}

  \bigskip

  \begin{itemize}

  \item \textbf{Désastres} (taux de croissance négatif): BEN (-0.1\%), BGD (-0.1\%), CAF (-1.1\%), COD (-4.2\%), DZA (-0.35\%), GHA (-0.68\%), GIN (-0.63\%), MDG (-0.53\%), NER (-1.6\%), NGA (-3.5\%), NIC (-0.5\%), RWA (-0.56\%), SEN (-0.24\%), SYR (-0.97\%), TCD (-0.83\%), TZA (-0.094\%), ZMB (-0.91\%).\newline

  \item \textbf{Miracles} (taux de croissance supérieur à 5\%): BWA (8.1\%), EGY (5\%), GNQ (5.5\%), HKG (5.1\%), KOR (7.7\%), MLT (8.9\%), SGP (6.7\%), TWN (6.7\%).\newline

  \item Des pays initialement pauvres qui s'enrichissent, les pays initialement mieux dotés qui s'appauvrissent, des pauvres qui restent pauvres et des riches toujours plus riches.

  \end{itemize}

\end{frame}


\begin{frame}
  \frametitle{Comparaisons des taux de croissance}
  \framesubtitle{Petit échantillon (55 nations de 1950 à 2017)}

  \begin{Center}
    \begin{tikzpicture}
\tikzstyle{every node}=[font=\tiny]
\begin{axis}[
enlargelimits=false, color=blue!30!black,
scale=1,
xmin=6,
xmax=10,
xlabel style={font=\color{white!15!black}},
xlabel={Logarithme du PIB par tête en 1950 (US\$ 2011)},
ymin=-1.5,
ymax=4.5,
ylabel style={font=\color{white!15!black}},
ylabel={Taux de croissance},
axis background/.style={fill=white}
]
\node[right, align=left]
at (axis cs:7.863,2.793) {ARG};
\node[right, align=left]
at (axis cs:9.416,2.036) {AUS};
\node[right, align=left]
at (axis cs:8.565,3.297) {AUT};
\node[right, align=left]
at (axis cs:8.873,2.54) {BEL};
\node[right, align=left]
at (axis cs:7.697,1.466) {BOL};
\node[right, align=left]
at (axis cs:7.32,3.359) {BRA};
\node[right, align=left]
at (axis cs:9.322,2.027) {CAN};
\node[right, align=left]
at (axis cs:9.559,2.281) {CHE};
\node[right, align=left]
at (axis cs:7.585,-1.41) {COD};
\node[right, align=left]
at (axis cs:8.045,2.13) {COL};
\node[right, align=left]
at (axis cs:8.08,2.323) {CRI};
\node[right, align=left]
at (axis cs:7.987,3.486) {CYP};
\node[right, align=left]
at (axis cs:8.443,3.533) {DEU};
\node[right, align=left]
at (axis cs:9.138,2.41) {DNK};
\node[right, align=left]
at (axis cs:8.021,1.773) {ECU};
\node[right, align=left]
at (axis cs:6.44,4.398) {EGY};
\node[right, align=left]
at (axis cs:8.141,3.557) {ESP};
\node[right, align=left]
at (axis cs:6.026,2.029) {ETH};
\node[right, align=left]
at (axis cs:8.672,2.909) {FIN};
\node[right, align=left]
at (axis cs:8.832,2.648) {FRA};
\node[right, align=left]
at (axis cs:9.144,2.159) {GBR};
\node[right, align=left]
at (axis cs:7.65,1.868) {GTM};
\node[right, align=left]
at (axis cs:7.654,1.053) {HND};
\node[right, align=left]
at (axis cs:6.735,3.08) {IND};
\node[right, align=left]
at (axis cs:8.487,4.19) {IRL};
\node[right, align=left]
at (axis cs:8.997,2.466) {ISL};
\node[right, align=left]
at (axis cs:8.545,2.956) {ISR};
\node[right, align=left]
at (axis cs:8.313,3.388) {ITA};
\node[right, align=left]
at (axis cs:7.836,4.221) {JPN};
\node[right, align=left]
at (axis cs:7.319,1.067) {KEN};
\node[right, align=left]
at (axis cs:7.841,2.429) {LKA};
\node[right, align=left]
at (axis cs:9.452,2.673) {LUX};
\node[right, align=left]
at (axis cs:7.133,2.856) {MAR};
\node[right, align=left]
at (axis cs:8.409,2.042) {MEX};
\node[right, align=left]
at (axis cs:8.438,2.413) {MUS};
\node[right, align=left]
at (axis cs:7.876,0.729) {NGA};
\node[right, align=left]
at (axis cs:8.093,0.697) {NIC};
\node[right, align=left]
at (axis cs:8.966,2.697) {NLD};
\node[right, align=left]
at (axis cs:9.049,2.973) {NOR};
\node[right, align=left]
at (axis cs:9.278,1.849) {NZL};
\node[right, align=left]
at (axis cs:7.145,2.159) {PAK};
\node[right, align=left]
at (axis cs:7.593,3.651) {PAN};
\node[right, align=left]
at (axis cs:7.631,2.581) {PER};
\node[right, align=left]
at (axis cs:7.209,2.626) {PHL};
\node[right, align=left]
at (axis cs:7.882,3.364) {PRT};
\node[right, align=left]
at (axis cs:7.386,2.138) {SLV};
\node[right, align=left]
at (axis cs:9.12,2.352) {SWE};
\node[right, align=left]
at (axis cs:6.989,4.073) {THA};
\node[right, align=left]
at (axis cs:8.572,3.06) {TTO};
\node[right, align=left]
at (axis cs:8.04,3.297) {TUR};
\node[right, align=left]
at (axis cs:6.719,1.168) {UGA};
\node[right, align=left]
at (axis cs:8.683,1.818) {URY};
\node[right, align=left]
at (axis cs:9.587,1.997) {USA};
\node[right, align=left]
at (axis cs:8.601,0.52) {VEN};
\node[right, align=left]
at (axis cs:8.557,1.223) {ZAF};
\end{axis}
\end{tikzpicture}
%%% Local Variables:
%%% mode: latex
%%% TeX-master: t
%%% End:

  \end{Center}

\end{frame}


\begin{frame}
  \frametitle{Comparaisons des taux de croissance}
  \framesubtitle{Grand échantillon (111 nations de 1960 à 2000)}

  \begin{Center}
    \begin{tikzpicture}
\tikzstyle{every node}=[font=\tiny]
\begin{axis}[
enlargelimits=false, color=blue!30!black,
scale=1,
xmin=6,
xmax=10.3,
xlabel style={font=\color{white!15!black}},
xlabel={Logarithme du PIB par tête en 1960},
ymin=-4.5,
ymax=9.1,
ylabel style={font=\color{white!15!black}},
ylabel={Taux de croissance},
axis background/.style={fill=white}
]
\node[right, align=left]
at (axis cs:7.956,3.809) {ARG};
\node[right, align=left]
at (axis cs:9.563,2.293) {AUS};
\node[right, align=left]
at (axis cs:9.103,3.355) {AUT};
\node[right, align=left]
at (axis cs:6.517,0.004) {BDI};
\node[right, align=left]
at (axis cs:9.116,2.936) {BEL};
\node[right, align=left]
at (axis cs:7.388,-0.101) {BEN};
\node[right, align=left]
at (axis cs:6.585,1.211) {BFA};
\node[right, align=left]
at (axis cs:7.265,-0.104) {BGD};
\node[right, align=left]
at (axis cs:7.479,1.301) {BOL};
\node[right, align=left]
at (axis cs:7.656,3.561) {BRA};
\node[right, align=left]
at (axis cs:9.025,2.401) {BRB};
\node[right, align=left]
at (axis cs:6.003,8.129) {BWA};
\node[right, align=left]
at (axis cs:7.243,-1.052) {CAF};
\node[right, align=left]
at (axis cs:9.5,2.573) {CAN};
\node[right, align=left]
at (axis cs:9.908,1.801) {CHE};
\node[right, align=left]
at (axis cs:8.463,2.03) {CHL};
\node[right, align=left]
at (axis cs:6.922,3.559) {CHN};
\node[right, align=left]
at (axis cs:7.377,0.616) {CIV};
\node[right, align=left]
at (axis cs:7.261,1.377) {CMR};
\node[right, align=left]
at (axis cs:7.904,-4.155) {COD};
\node[right, align=left]
at (axis cs:7.44,0.903) {COG};
\node[right, align=left]
at (axis cs:8.152,1.741) {COL};
\node[right, align=left]
at (axis cs:7.523,0.929) {COM};
\node[right, align=left]
at (axis cs:7.02,2.879) {CPV};
\node[right, align=left]
at (axis cs:8.43,1.8) {CRI};
\node[right, align=left]
at (axis cs:8.261,4.782) {CYP};
\node[right, align=left]
at (axis cs:9.204,3.072) {DEU};
\node[right, align=left]
at (axis cs:9.395,2.572) {DNK};
\node[right, align=left]
at (axis cs:7.98,2.374) {DOM};
\node[right, align=left]
at (axis cs:9.152,-0.347) {DZA};
\node[right, align=left]
at (axis cs:8.285,0.582) {ECU};
\node[right, align=left]
at (axis cs:6.531,5.011) {EGY};
\node[right, align=left]
at (axis cs:8.644,3.801) {ESP};
\node[right, align=left]
at (axis cs:6.259,0.081) {ETH};
\node[right, align=left]
at (axis cs:9.089,3.258) {FIN};
\node[right, align=left]
at (axis cs:7.928,1.915) {FJI};
\node[right, align=left]
at (axis cs:9.212,2.877) {FRA};
\node[right, align=left]
at (axis cs:8.154,2.726) {GAB};
\node[right, align=left]
at (axis cs:9.385,2.464) {GBR};
\node[right, align=left]
at (axis cs:8.35,-0.685) {GHA};
\node[right, align=left]
at (axis cs:8.001,-0.63) {GIN};
\node[right, align=left]
at (axis cs:7.828,0.242) {GMB};
\node[right, align=left]
at (axis cs:6.638,1.775) {GNB};
\node[right, align=left]
at (axis cs:6.861,5.548) {GNQ};
\node[right, align=left]
at (axis cs:8.507,3.889) {GRC};
\node[right, align=left]
at (axis cs:7.691,2.078) {GTM};
\node[right, align=left]
at (axis cs:8.434,5.125) {HKG};
\node[right, align=left]
at (axis cs:7.608,1.183) {HND};
\node[right, align=left]
at (axis cs:7.192,0.982) {HTI};
\node[right, align=left]
at (axis cs:7.156,2.506) {IDN};
\node[right, align=left]
at (axis cs:6.948,1.651) {IND};
\node[right, align=left]
at (axis cs:8.695,4.566) {IRL};
\node[right, align=left]
at (axis cs:8.24,1.762) {IRN};
\node[right, align=left]
at (axis cs:9.293,2.912) {ISL};
\node[right, align=left]
at (axis cs:9.013,3.407) {ISR};
\node[right, align=left]
at (axis cs:8.831,3.96) {ITA};
\node[right, align=left]
at (axis cs:8.505,0.14) {JAM};
\node[right, align=left]
at (axis cs:7.978,0.479) {JOR};
\node[right, align=left]
at (axis cs:8.508,4.977) {JPN};
\node[right, align=left]
at (axis cs:7.377,0.553) {KEN};
\node[right, align=left]
at (axis cs:7.015,7.723) {KOR};
\node[right, align=left]
at (axis cs:7.899,1.277) {LKA};
\node[right, align=left]
at (axis cs:6.858,2.475) {LSO};
\node[right, align=left]
at (axis cs:9.665,3.269) {LUX};
\node[right, align=left]
at (axis cs:7.137,3.239) {MAR};
\node[right, align=left]
at (axis cs:7.271,-0.532) {MDG};
\node[right, align=left]
at (axis cs:8.686,1.974) {MEX};
\node[right, align=left]
at (axis cs:6.557,1.25) {MLI};
\node[right, align=left]
at (axis cs:6.578,8.905) {MLT};
\node[right, align=left]
at (axis cs:6.202,0.744) {MOZ};
\node[right, align=left]
at (axis cs:7.014,1.544) {MRT};
\node[right, align=left]
at (axis cs:8.204,3.552) {MUS};
\node[right, align=left]
at (axis cs:6.745,0.041) {MWI};
\node[right, align=left]
at (axis cs:7.773,4.115) {MYS};
\node[right, align=left]
at (axis cs:8.255,0.71) {NAM};
\node[right, align=left]
at (axis cs:7.221,-1.639) {NER};
\node[right, align=left]
at (axis cs:8.303,-3.486) {NGA};
\node[right, align=left]
at (axis cs:8.382,-0.503) {NIC};
\node[right, align=left]
at (axis cs:9.347,2.992) {NLD};
\node[right, align=left]
at (axis cs:9.289,3.877) {NOR};
\node[right, align=left]
at (axis cs:6.553,1.709) {NPL};
\node[right, align=left]
at (axis cs:9.46,1.826) {NZL};
\node[right, align=left]
at (axis cs:7.093,2.066) {PAK};
\node[right, align=left]
at (axis cs:7.881,3.166) {PAN};
\node[right, align=left]
at (axis cs:7.887,1.526) {PER};
\node[right, align=left]
at (axis cs:7.525,2.042) {PHL};
\node[right, align=left]
at (axis cs:8.314,4.321) {PRT};
\node[right, align=left]
at (axis cs:7.485,1.861) {PRY};
\node[right, align=left]
at (axis cs:7.218,4.435) {ROU};
\node[right, align=left]
at (axis cs:6.869,-0.558) {RWA};
\node[right, align=left]
at (axis cs:7.87,-0.239) {SEN};
\node[right, align=left]
at (axis cs:7.88,6.748) {SGP};
\node[right, align=left]
at (axis cs:7.495,2.213) {SLV};
\node[right, align=left]
at (axis cs:9.373,2.755) {SWE};
\node[right, align=left]
at (axis cs:8.626,3.328) {SYC};
\node[right, align=left]
at (axis cs:7.779,-0.973) {SYR};
\node[right, align=left]
at (axis cs:7.181,-0.829) {TCD};
\node[right, align=left]
at (axis cs:7.017,0.446) {TGO};
\node[right, align=left]
at (axis cs:6.979,4.778) {THA};
\node[right, align=left]
at (axis cs:9.153,0.721) {TTO};
\node[right, align=left]
at (axis cs:7.422,4.086) {TUN};
\node[right, align=left]
at (axis cs:8.454,2.356) {TUR};
\node[right, align=left]
at (axis cs:7.774,6.702) {TWN};
\node[right, align=left]
at (axis cs:7.068,-0.094) {TZA};
\node[right, align=left]
at (axis cs:6.671,0.953) {UGA};
\node[right, align=left]
at (axis cs:8.815,1.159) {URY};
\node[right, align=left]
at (axis cs:9.768,2.443) {USA};
\node[right, align=left]
at (axis cs:8.817,0.324) {VEN};
\node[right, align=left]
at (axis cs:8.685,0.975) {ZAF};
\node[right, align=left]
at (axis cs:7.418,-0.913) {ZMB};
\node[right, align=left]
at (axis cs:7.596,0.67) {ZWE};
\end{axis}
\end{tikzpicture}
  \end{Center}

\end{frame}


\begin{frame}
  \frametitle{Comparaisons internationales, III}
  \framesubtitle{Dynamique de la distribution}

  \begin{center}
    \begin{tikzpicture}

  \begin{axis}[/pgf/number format/1000 sep={},scale=1, enlargelimits=false, color=blue!30!black, xlabel style={font=\color{white!15!black}}, xlabel={Logarithme du PIB par tête}]
    \addplot[style={black,mark=none,smooth,thick}] table[x index=0, y index=1, col sep=space]{../data/rgdpc-density-1960.dat};
    \addlegendentry{$1960$};
    \addplot[style={red,mark=none,smooth,thick}] table[x index=0, y index=1, col sep=space]{../data/rgdpc-density-2000.dat};
    \addlegendentry{$2000$};
  \end{axis}

\end{tikzpicture}
  \end{center}

\end{frame}


\begin{frame}
  \frametitle{Comparaisons internationales, IV}
  \framesubtitle{Lumière sur une inégale répartition, I}

  \begin{center}
    \includegraphics[scale=0.12]{../img/maps/lights-world.jpg}
  \end{center}

\end{frame}


\begin{frame}
  \frametitle{Comparaisons internationales, IV'}
  \framesubtitle{Lumière sur une inégale répartition, II}

  \begin{center}
    \includegraphics[scale=0.3]{../img/maps/lights-eu.jpg}
  \end{center}

\end{frame}


\begin{frame}
  \frametitle{Comparaisons internationales, IV''}
  \framesubtitle{Concentration de la création de richesses}

  \begin{center}
    \includegraphics[scale=0.5]{../img/maps/gdp-density.png}
  \end{center}

\end{frame}


\begin{frame}
  \frametitle{Comparaisons internationales, IV'''}
  \framesubtitle{Inégalité de PIB par tête (2005)}

  \begin{center}
    \includegraphics[scale=0.2]{../img/maps/gdp-pc-2005.png}
  \end{center}

\end{frame}


\begin{frame}
  \frametitle{Objectif du cours}

  \begin{itemize}

  \item Si on veut comprendre pourquoi les niveaux de vie différent
    autant entre les pays, nous devons déterminer les raisons d'écarts
    aussi prononcés entre leurs taux de croissance.\newline

  \item Même des petites différences entre ces taux résultent dans de
    grandes différences sur les niveaux.\newline

  \item Conséquences sur les niveaux de consommation et de bien être.\newline

  \item Il est essentiel de comprendre ce qui peut expliquer
    l'hétérogénéité observée des taux de croissance dans le temps et
    dans la coupe.\newline

  \item Dans ce cours, on ne cherchera pas à exliquer la croissance
    sur le très long terme (voir les premières figures), à expliquer
    pourquoi la croissance apparaît subitement après la première
    révolution industrielle\ldots On s'intéresse au court/moyen terme.

  \end{itemize}

\end{frame}


\begin{frame}
  \frametitle{Les faits stylisés de Kaldor}

  Même si nous ne cherchons qu'à expliquer l'hétérogénéité des
  dynamiques de croissance sur le court/moyen termes, nous devons
  utiliser des modèles dont les prédictions quant au long terme ne
  contredisent pas ce que nous pouvons observer.\newline

  \begin{enumerate}

  \item La production par tête croît à un taux relativement constant.

  \item Le capital physique par tête croît avec le temps.

  \item Le taux de rendement du capital physique est approximativement constant.

  \item Le rapport du capital physique à la production est approximativement constant.

  \item Les parts respectives du travail et du capital physique dans la valeur ajoutée sont approximativement constantes.

  \item Le taux de croissance de la production par tête est trés variable d'un pays à l'autre.
  \end{enumerate}

\end{frame}


\begin{frame}
  \frametitle{Objectif du cours (reprise)}

  \begin{itemize}

  \item On s'intéresse à l'hérogénéité des dynamiques de croissance,
    depuis le siècle dernier (après seconde guerre mondiale).\newline

  \item Nous mettrons en évidence le rôle de l'investissement dans la
    dynamique de croissance. On montrera aussi le rôle de l'éducation
    -- un investissement -- dans l'hétérogénéité des dynamiques de
    croissance.\newline

  \item On utilise un modèle de croissance développé dnas les années
    cinquante~: le modèle de Solow, ainsi que plusieurs extensions
    apportées à ce modèle depuis. Par exemple pour évaluer le rôle que
    peut avoir l'éducation dans la croissance.\newline

  \item On montrera que, même si ce modèle fût initialement développé
    pour expliquer l'évolution de la croissance dans temps, ce modèle,
    une fois amendé, peut expliquer une large part de l'hétérogénéité
    observée des taux de croisannce.

  \end{itemize}

\end{frame}


\begin{frame}
  \frametitle{Formalisme en temps continu}

  \begin{itemize}

  \item Dans ce cours on s'intéresse à la dynamique d'aggrégats économiques (le PIB, le stock de capital physique, \ldots).\newline

  \item Pour étudier la dynamique de variables (ou systèmes de variables) on dispose des outils suivants~:

    \begin{itemize}

    \item[-] Les équations récurrentes si le temps est discret ($t\in\mathbb N$), par exemple~:
      \[
        x_t = \alpha + \beta x_{t-1}
      \]

    \item[-] Les équations différentielles si le temps est continu ($t\in\mathbb R$), par exemple~:
      \[
        \dot x(t) = \alpha + \beta x(t)
      \]
    \end{itemize}

    \medskip

  \item Dans ce cours, comme dans la littérature, nous adopterons un
    formalisme en temps continu.\newline

  \item Ce choix peut sembler contre-intuitif, relativement à
    l'observation des aggrégats, mais a l'avantage de la simplicité.

  \end{itemize}

\end{frame}


\begin{frame}
  \frametitle{Formalisme en temps continu}
  \framesubtitle{Le taux de croissance}

  \begin{itemize}

  \item Habituellement nous définissons le taux de croissance d'une variable (en temps discret) $x_t$ comme~:
    \[
      g_t = \frac{x_t-x_{t-1}}{x_{t-1}} \equiv G_t - 1
    \]
    c'est-à-dire comme la variation de $x$, que l'on note souvent
    $\Delta x_t = x_{t}-x_{t-1}$, rapporté au niveau de $x$ en $t-1$.\newline

  \item Sous l'hypothèse d'un temps continu, on adopte une définition analogue~:
    \[
      g(t) = \frac{\dot x(t)}{x(t)}
    \]
    où $\dot x(t)$ est la dérivée de $x$ par rapport au temps $\dot x(t) = \frac{\mathrm dx(t)}{\mathrm dt}$.\newline

  \item Le numérateur $\dot x(t)$ est la version continue de $\Delta x_t$, c'est-à-dire la variation de $x$ induite par une variation infinitésimale de $t$.

  \end{itemize}

\end{frame}


\begin{frame}
  \frametitle{Formalisme en temps continu}
  \framesubtitle{Facteur et taux de croissance, I}

  \begin{itemize}

  \item La définition du facteur de croissance ne change pas~:
    \[
      G(0,T) = \frac{x(T)}{x(0)}
    \]
    le niveau en $T$ rapporté au niveau initial (ici 0). Comment relier ce facteur de croissance avec le taux de croissance~?\newline

  \item Pour répondre, en supposant que la condition initiale $x(0)$ ainsi que les taux de
    croissances $g(t)$ pour $t\in[0,T]$ sont connus, nous devons
    déterminer le niveau de la variable $x(T)$.\newline

  \item Pour simplifier le problème, supposons que le taux de croissance soit constant, c'est-à-dire que $g(t)=g \forall t\in[0,T]$. La fonction du temps $x(t)$ inconnue doit donc satisfaire~:
    \[
      \frac{\dot x(t)}{x(t)} = g \quad \forall x\in [0,T]
    \]

  \end{itemize}

\end{frame}


\begin{frame}
  \frametitle{Formalisme en temps continu}
  \framesubtitle{Facteur et taux de croissance, II}

  \begin{itemize}

  \item Par ailleurs nous savons que $ (\log u)' = \nicefrac{u'}{u}$. Nous pouvons donc réécrire la condition sur la fonction $x(t)$ comme~:
    \[
      \dot{\log x(t)} = g
    \]

  \item En passant, nous obtenons une nouvelle définition du taux de croissance~:

    \begin{block}{Taux de croissance}
      Le taux de croissance d'une variable $x(t)$ est égal à la variation du logarithme de cette variable~:
      \[
        g(t) = \frac{\mathrm d \log x(t)}{\mathrm dt}
      \]
    \end{block}

  \item Pour déterminer $\log x(T)$ il suffit de sommer la condition initiale, $\log x(0)$, et les variation de $\log x(s)$ pour $s\in[0,T]$~:
    \[
      \log x(T) = \log x(0) + \int_0^T \dot{\log x(s)}\mathrm d s
    \]
  \end{itemize}

\end{frame}


\begin{frame}
  \frametitle{Formalisme en temps continu}
  \framesubtitle{Facteur et taux de croissance, III}

  \begin{itemize}

  \item En substituant la contrainte sur la fonction $x(t)$~:
    \[
      \begin{split}
        \log x(T) &= \log x(0) + \int_0^T g\mathrm d s\\
        &= \log x(0) + gT
      \end{split}
    \]

  \item En appliquant la fonction exponentielle (puisque notre problème est de trouver $x(T)$ pas $\log x(T)$)~:
    \[
      x(T) = x(0)e^{gT}
    \]

  \item Le facteur de croissance est donc~:
    \[
      G(0,T) = e^{gT}
    \]

  \item Plus généralement, pour un taux de croissance variable, nous aurions~:
    \[
      x(T) = x(0)e^{\int_0^Tg(s)\mathrm ds}
    \]
    Le facteur de croissance entre 0 et $T$ est l'exponentielle de la somme des taux de croissance entre $0$ et $T$.

  \end{itemize}

\end{frame}


\begin{frame}
  \frametitle{Formalisme en temps continu}
  \framesubtitle{Taux de croissance d'un produit}

  \begin{itemize}

  \item Soit $x(t) = u(t)\cdot v(t)$, avec $g_u(t)$ et $g_v(t)$ les taux de croissance de $u$ et $v$. Quel est le taux de croissance de $x$~?\newline

  \item En appliquant le logarithme, on transforme le produit en somme~:
    \[
      \log x(t) = \log u(t) + \log v(t)
    \]

  \item En dérivant par rapport à $t$. On obtient directement~:
    \[
      g_x(t) = g_u(t) + g_v(t)
    \]
    on retient donc que le taux de croissance d'un produit est la somme des taux de croissance.\newline

  \item[\dbend] En temps discret, c'est un peu plus compliqué~:
    \[
      g_{x,t} = g_{u_t} + g_{v,t} + g_{u,t}\cdot g_{v,t}
    \]
    si on approxime pas le résultat en élmiminant le terme croisé.
  \end{itemize}

\end{frame}


\end{document}


% Local Variables:
% ispell-check-comments: exclusive
% ispell-local-dictionary: "french"
% TeX-master: t
% End: