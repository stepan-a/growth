\documentclass[10pt]{beamer}

\usepackage{etex}

\usepackage{fourier-orns}
\usepackage{ccicons}
\usepackage{amssymb}
\usepackage{amstext}
\usepackage{amsbsy}
\usepackage{amsopn}
\usepackage{amscd}
\usepackage{amsxtra}
\usepackage{amsthm}
\usepackage{float}
\usepackage{color}
\usepackage{mathrsfs}
\usepackage{bm}
\usepackage{lastpage}
\usepackage[nice]{nicefrac}
\usepackage{setspace}
\usepackage{hyperref}
\usepackage{ragged2e}
\usepackage{listings}
\usepackage{algorithms/algorithm}
\usepackage{algorithms/algorithmic}


\usepackage{tikz,pgfplots}
\pgfplotsset{compat=newest}
\usetikzlibrary{patterns, arrows, decorations.pathreplacing, decorations.markings, calc}
\pgfplotsset{plot coordinates/math parser=false}
\newlength\figureheight
\newlength\figurewidth
\usepackage[utf8x]{inputenc}
\usepackage{cancel}
\usepackage{tikz-qtree}
\usepackage{dcolumn}

\newcommand{\trace}{\mathrm{tr}}
\newcommand{\vect}{\mathrm{vec}}
\newcommand{\tracarg}[1]{\mathrm{tr}\left\{#1\right\}}
\newcommand{\vectarg}[1]{\mathrm{vec}\left(#1\right)}
\newcommand{\vecth}[1]{\mathrm{vech}\left(#1\right)}
\newcommand{\iid}[2]{\mathrm{iid}\left(#1,#2\right)}
\newcommand{\normal}[2]{\mathcal N\left(#1,#2\right)}
\newcommand{\dynare}{\href{http://www.dynare.org}{\color{blue}Dynare}}
\newcommand{\sample}{\mathcal Y_T}
\newcommand{\samplet}[1]{\mathcal Y_{#1}}
\newcommand{\slidetitle}[1]{\fancyhead[L]{\textsc{#1}}}

\newcommand{\R}{{\mathbb R}}
\newcommand{\C}{{\mathbb C}}
\newcommand{\N}{{\mathbb N}}
\newcommand{\Z}{{\mathbb Z}}
\newcommand{\binomial}[2]{\begin{pmatrix} #1 \\ #2 \end{pmatrix}}
\newcommand{\bigO}[1]{\mathcal O \left(#1\right)}
\newcommand{\red}{\color{red}}
\newcommand{\blue}{\color{blue}}

\newcolumntype{d}{D{.}{.}{-1}}

\definecolor{gray}{gray}{0.4}

\setbeamertemplate{footline}{
{\hfill\vspace*{1pt}\href{http://creativecommons.org/licenses/by-sa/3.0/legalcode}{\ccbysa}\hspace{.1cm}
\raisebox{-.075cm}{\href{https://mnemosyne.adjemian.eu/stepan-a/growth}{\includegraphics[scale=.1]{../img/gitlab.png}}}
}\hspace{1cm}}

\setbeamertemplate{navigation symbols}{}
\setbeamertemplate{blocks}[rounded][shadow=true]
\newenvironment{notes}
{\bgroup \justifying\bgroup\tiny\begin{spacing}{1.0}}
{\end{spacing}\egroup\egroup}

\newenvironment{exercise}[1]
{\bgroup \small\begin{block}{Ex. #1}}
{\end{block}\egroup}

\begin{document}

\title{Croissance\\\small{Introduction}}
\author[S. Adjemian]{St\'ephane Adjemian}
\institute{\texttt{stephane.adjemian@univ-lemans.fr}}
\date{Septembre 2018}

\begin{frame}
  \titlepage{}
\end{frame}

\section{Introduction}

\begin{frame}
  \frametitle{Deux regards sur la croissance}
  \begin{itemize}

  \item On peut s'intéresser à la croissance économique dans
    deux directions:

    \begin{itemize}

      \item[$\Rightarrow$] L'évolution dans le temps de la croissance.\newline

      Pourquoi le taux de croissance d'une économie est-il significativement différent en différentes périodes de son histoire ?\newline

      \item[$\Rightarrow$] L'hétérogénéité dans la coupe des nations de la croissance.\newline

      Pourquoi le taux de croissance de la Chine est-il plus important de celui des économies occidentales sur la période récente?

    \end{itemize}

    \bigskip

  \item Ces directions ne sont pas forcément interchangeables. Nous
    montrerons néanmoins dans ce cours qu'un modèle développé pour expliquer la
    croissance dans la première direction peut expliquer une bonne
    part de l'hétérogénéité internationale (observée) des taux de
    croissance.

  \end{itemize}

\end{frame}


\begin{frame}
  \frametitle{Séries temporelles}
  \framesubtitle{Les États Unis d'Amérique du Nord, I}
  \begin{figure}[H]
    \centering
    \includegraphics[scale=.3]{usa_logged_rgdp_per_capita}
    \caption{PIB réel par tête en logarithme (USA).}
    \label{fig:maddison-rgdp-usa}
  \end{figure}

\end{frame}


\begin{frame}
  \frametitle{Séries temporelles}
  \framesubtitle{Les États Unis d'Amérique du Nord, II}

  \begin{itemize}
  \item Sur la période 1800-2016 le taux de croissance annuel moyen du PIB par tête est $\bar g \approx 1.53\%$.\newline

  \item Sur la même période, le facteur de croissance, le ratio du PIB de 2016 au PIB de 2800, est $G \approx 26,8$. En un peu plus de deux siècles la production par tête d'un américain a été multpliée par environ 27.\newline

  \item[\textbf{Rappel 1}] Le taux de croissance annuel moyen ($\bar g$) est calculé à partir du facteur de croissance ($G$):
    \[
         \bar g = (G^{1/T}-1)\times 100
    \]
    où $T$ est le nombre d'années sur lequel le facteur de croissance est calculé.\newline

  \item Approximativement, le taux de croissance est aussi donné graphiquement par la pente du PIB par tête en log sur la figure \ref{fig:maddison-rgdp-usa}.

  \end{itemize}

\end{frame}


\begin{frame}
  \frametitle{Séries temporelles}
  \framesubtitle{France, I}
  \begin{figure}[H]
    \centering
    \includegraphics[scale=.3]{fra_logged_rgdp_per_capita1}
    \caption{PIB réel par tête en logarithme (France).}
    \label{fig:maddison-rgdp-fra-1}
  \end{figure}

\end{frame}

\begin{frame}
  \frametitle{Séries temporelles}
  \framesubtitle{France, II}
  \begin{figure}[H]
    \centering
    \includegraphics[scale=.3]{fra_logged_rgdp_per_capita2}
    \caption{PIB réel par tête en logarithme (France).}
    \label{fig:maddison-rgdp-fra-2}
  \end{figure}

\end{frame}


\begin{frame}
  \frametitle{Séries temporelles}
  \framesubtitle{France, III}

  \begin{itemize}

  \item La croissance est un phénomène récent en France, postérieur à la révolution.\newline

  \item On identifie facilement les deux guerres mondiales. On observe que le taux de croissance est temporairement plus élevé après ces épisodes.\newline

  \item On identifie les \textit{trentes glorieuses}, et la baisse notable du taux de croissance à partir des années 70-80.\newline

  \item Sur la période 1280-2016, le taux de croissance annuel moyen est $\bar g \approx 0,49$, mais sur la période 1820-2016 on a un taux de croissance annuel moyen nettement plus important $\bar g \approx 1,69$.\newline

  \item En un peu plus de 7 siècle la France a multiplié son niveau de PIB par tête par un peu plus de 36.

  \end{itemize}

\end{frame}

\begin{frame}
  \frametitle{Séries temporelles}
  \framesubtitle{Grande Bretagne}
  \begin{figure}[H]
    \centering
    \includegraphics[scale=.3]{gbr_logged_rgdp_per_capita1}
    \caption{PIB réel par tête en logarithme (Grande Bretagne). La ligne verticale correspond à l'invention par Watt de la machine à vapeur.}
    \label{fig:maddison-rgdp-gbr}
  \end{figure}

\end{frame}


\begin{frame}
  \frametitle{Contrefactuel}
  \framesubtitle{Le prix d'un point de croissance}

  \begin{itemize}

  \item Un point de croissance annuel moyen sur d'aussi longues périodes vaut très cher un termes de niveau de PIB.\newline

  \item Entre 1870 (après la guerre civile) et 2000 le PIB par tête aux États Unis (\$ 1996) a été multiplié par 10, avec un taux de croissance annuel moyen de 1,79\%. Sur cette période le PIB par tête est passé de  3340\$ à 33330\$.\newline

  \item En imaginant un monde où les États Unis auraient connus un taux de croissance annuel moyen inférieur de 1 point sur ces 130 années, nous observerions en 2000 un PIB par tête de seulement 9234\$, soit un facteur de croissance de 2,765. Moins d'un tiers du niveau effectivement observé en 2000 !…\newline

  \item … À peu près le niveau du Mexique ou de la Pologne en 2000 !

  \end{itemize}

\end{frame}


\begin{frame}
  \frametitle{Comparaisons internationales, I}

\bigskip

  \begin{itemize}

  \item Comment comparer des niveaux de PIB ?\newline

  \item En 1960:

    \begin{itemize}
    \item La Suisse est le pays le plus riche en termes de PIB par tête (14980\$ base 1996).
    \item La Tanzanie est le pays le plus pauvre en termes de PIB par tête (381\$ base 1996).
    \end{itemize}

    $\Rightarrow$ Le PIB par tête est 39$\times$ moindre en Tanzanie !\newline

  \item En 2000:

    \begin{itemize}
    \item Les États Unis sont le pays le plus riche en termes de PIB par tête (33330\$ base 1996).
    \item La Tanzanie est le pays le plus pauvre en termes de PIB par tête (482\$ base 1996).
    \end{itemize}

    $\Rightarrow$ Le PIB par tête est 69$\times$ moindre en Tanzanie !\newline

  \item Les différences sont très importantes (plus que dans la chronologie sur période séculiaire) et augmentent.\newline

  \end{itemize}

\end{frame}


\begin{frame}
  \frametitle{Comparaisons internationales, II}

\bigskip

  \begin{itemize}

  \item Le taux de croissance annuel moyen (dans le temps) moyen (dans la coupe) :
    \[
    \bar{\bar g} = \frac{1}{N}\sum_{i=1}^{N} \bar g_{i}
    \]

  \item Entre 1960 et 2000, $\bar{\bar g} \approx 1,8$ (pour un échantillon de 112 pays).\newline

  \item L'hétérogénéité augmente:
    \begin{itemize}
    \item 1960: $\sigma (log(y)) \approx 0.89$
    \item 2000: $\sigma (log(y)) \approx 1.12$
    \end{itemize}

\bigskip

  \item Le pire: $\bar g_{i} = -2,5\%$ pour la République Démocratique du Congo.\newline

  \item Le meilleur: $\bar g_{i} = 6,4\%$ pour Taiwan.\newline

  \end{itemize}

\end{frame}


\begin{frame}
  \frametitle{Comparaisons internationales, III}

\bigskip

  \begin{itemize}

  \item \textbf{Désastres}: République Démocratique du Congo, l'Irak (qui
    divise son PIB par tête par un facteur supérieur à 2 durant la
    période), de nombreux autres pays ont connu des taux de croissance
    du PIB réel par tête négatifs : Tchad, Madagascar, Mozambique,
    Somalie, Zambie, Ouganda, Guayana, Zaïre, Nicaragua, Bénin,
    République Centre Africaine, Haïti, Burundi, Ghana, Venezuela,
    Mauritanie et le Niger.\newline

  \item \textbf{Miracles}: Taiwan, Corée du Sud (5.9\%), Singapour
    (6.2\%), Hong-Kong (5.4\%), le Botswana (5,1\%)...\newline


  \end{itemize}

\end{frame}

\begin{frame}
  \frametitle{Comparaisons internationales, III}
  \framesubtitle{Dynamique de la distribution}


  \bigskip

  \begin{figure}[H]
    \centering
    \includegraphics[scale=.4]{plt01}
    \caption{Évolution de la distribution du PIB par tête.}
    \label{fig:quah}
  \end{figure}


\end{frame}




\end{document}


% Local Variables:
% ispell-check-comments: exclusive
% ispell-local-dictionary: "french"
% TeX-master: t
% End: