\synctex=1

\documentclass[10pt,notheorems]{beamer}

\usepackage{etex}
\usepackage{fourier-orns}
\usepackage{ccicons}
\usepackage{amssymb}
\usepackage{amstext}
\usepackage{amsbsy}
\usepackage{amsopn}
\usepackage{amscd}
\usepackage{amsxtra}
\usepackage{amsthm}
\usepackage{float}
\usepackage{color, colortbl}
\usepackage{mathrsfs}
\usepackage{bm}
\usepackage{lastpage}
\usepackage[nice]{nicefrac}
\usepackage{setspace}
\usepackage{ragged2e}
\usepackage{listings}
\usepackage{algorithms/algorithm}
\usepackage{algorithms/algorithmic}
\usepackage[frenchb]{babel}
\usepackage{tikz,pgfplots,pgfplotstable}
\pgfplotsset{compat=newest}
\usetikzlibrary{patterns, arrows, decorations.pathreplacing, decorations.markings, calc}
\pgfplotsset{plot coordinates/math parser=false}
\newlength\figureheight
\newlength\figurewidth
\usepackage[utf8x]{inputenc}
\usepackage{cancel}
\usepackage{tikz-qtree}
\usepackage{dcolumn}
\usepackage{adjustbox}
\usepackage{environ}
\usepackage[cal=boondox]{mathalfa}
\usepackage{manfnt}
\usepackage{hyperref}
\hypersetup{
  colorlinks=true,
  linkcolor=blue,
  filecolor=black,
  urlcolor=black,
}
\usepackage{venndiagram}

% Git hash
\usepackage{xstring}
\usepackage{catchfile}
\immediate\write18{git rev-parse HEAD > git.hash}
\CatchFileDef{\HEAD}{git.hash}{\endlinechar=-1}
\newcommand{\gitrevision}{\StrLeft{\HEAD}{7}}

\newcommand{\trace}{\mathrm{tr}}
\newcommand{\vect}{\mathrm{vec}}
\newcommand{\tracarg}[1]{\mathrm{tr}\left\{#1\right\}}
\newcommand{\vectarg}[1]{\mathrm{vec}\left(#1\right)}
\newcommand{\vecth}[1]{\mathrm{vech}\left(#1\right)}
\newcommand{\iid}[2]{\mathrm{iid}\left(#1,#2\right)}
\newcommand{\normal}[2]{\mathcal N\left(#1,#2\right)}
\newcommand{\dynare}{\href{http://www.dynare.org}{\color{blue}Dynare}}
\newcommand{\sample}{\mathcal Y_T}
\newcommand{\samplet}[1]{\mathcal Y_{#1}}
\newcommand{\slidetitle}[1]{\fancyhead[L]{\textsc{#1}}}

\newcommand{\R}{{\mathbb R}}
\newcommand{\C}{{\mathbb C}}
\newcommand{\N}{{\mathbb N}}
\newcommand{\Z}{{\mathbb Z}}
\newcommand{\binomial}[2]{\begin{pmatrix} #1 \\ #2 \end{pmatrix}}
\newcommand{\bigO}[1]{\mathcal O \left(#1\right)}
\newcommand{\red}{\color{red}}
\newcommand{\blue}{\color{blue}}

\renewcommand{\qedsymbol}{C.Q.F.D.}

\newcolumntype{d}{D{.}{.}{-1}}
\definecolor{gray}{gray}{0.9}
\newcolumntype{g}{>{\columncolor{gray}}c}

\setbeamertemplate{theorems}[numbered]

\theoremstyle{plain}
\newtheorem{theorem}{Théorème}

\theoremstyle{definition} % insert bellow all blocks you want in normal text
\newtheorem{definition}{Définition}
\newtheorem{properties}{Propriétés}
\newtheorem{lemma}{Lemme}
\newtheorem{property}[properties]{Propriété}
\newtheorem{example}{Exemple}
\newtheorem*{idea}{Éléments de preuve} % no numbered block

\setbeamertemplate{footline}{
  {\hfill\vspace*{1pt}\href{http://creativecommons.org/licenses/by-sa/3.0/legalcode}{\ccbysa}\hspace{.1cm}
    \raisebox{-.075cm}{\href{https://git.adjemian.eu/University/growth}{\includegraphics[scale=.1]{../img/gitlab.png}}}\enspace
    \href{https://git.adjemian.eu/University/growth/-/blob/\HEAD/cours/solow.tex}{\gitrevision}\enspace\today
  }\hspace{1cm}}

\setbeamertemplate{navigation symbols}{}
\setbeamertemplate{blocks}[rounded][shadow=true]
\setbeamertemplate{caption}[numbered]

\newenvironment{notes}
{\bgroup \justifying\bgroup\tiny\begin{spacing}{1.0}}
  {\end{spacing}\egroup\egroup}

\newenvironment{exercise}[1]
{\bgroup \small\begin{block}{Ex. #1}}
  {\end{block}\egroup}

\newenvironment{defn}[1]
{\bgroup \small\begin{block}{Définition. #1}}
  {\end{block}\egroup}

\newenvironment{exemple}[1]
{\bgroup \small\begin{block}{Exemple. #1}}
  {\end{block}\egroup}


\begin{document}

\title{Croissance\\\small{Le modèle de Solow}}
\author[S. Adjemian]{St\'ephane Adjemian}
\institute{\texttt{stephane.adjemian@univ-lemans.fr}}
\date{Septembre 2020}

\begin{frame}
  \titlepage{}
\end{frame}

\begin{frame}
  \frametitle{Plan}
  \tableofcontents
\end{frame}


\section{Environnement et hypothèses}

\begin{frame}
  \frametitle{Environnement, I}

  \begin{itemize}
  \item Un modèle d'équilibre général.
    \medskip
  \item Les ménages (familles)~:
    \begin{itemize}
    \item[--] possèdent les facteurs de production et les actifs de l'économie,
    \item[--] déterminent la fécondité, la participation au marché du travail \textbf{et} le partage entre consommation présente et future (c'est-à-dire l'épargne).
    \end{itemize}
    \medskip
  \item Les entreprises~:
    \begin{itemize}
    \item[--] louent les facteurs de production (capital et travail) aux ménages,
    \item[--] produisent un bien à partir des services du capital et du travail, vendu aux ménages.
    \end{itemize}
    \medskip
  \item Ménages et firmes se rencontrent sur des marchés~:
    \begin{itemize}
    \item[--] le marché des biens, où les entreprises vendent la production au ménages,
    \item[--] un marché par facteur de production.
    \end{itemize}
    \medskip
  \item \textbf{Simplification:} une robinsonnade\ldots
  \end{itemize}
  
\end{frame}


\begin{frame}
  \frametitle{Robinson schizophrène}

  \begin{center}
    \includegraphics[scale=.2]{../img/crusoe-1.png}
  \end{center}

\end{frame}


\begin{frame}
  \frametitle{Environnement, II}
  
  \begin{itemize}

  \item Économie autarcique.

    \begin{itemize}
      \item[$\Rightarrow$] Pas de différence entre revenu et production d'une économie.
      \item[$\Rightarrow$] Pas de différence entre épargne et investissement.
    \end{itemize}

  \bigskip
    
  \item Hypothèse de concurrence parfaite.\newline

    \begin{itemize}

    \item[$\Rightarrow$] Pas de différence entre l'économie
      centralisée (Robinson) et l'équilibre décentralisé (équilibre
      général avec des marchés),\newline

      \begin{itemize}
        \item[--] On va donc au plus simple\ldots
        \item[--] Mais pas de prix (puisqu'il n'y a pas de marchés). 
      \end{itemize}

    \end{itemize}

  \bigskip

  \item Offre de travail inélastique (pas de chômage).\newline

  \item Démographie exogène.\newline

  \item Comportement d'épargne exogène.
  
  \end{itemize}

\end{frame}


\begin{frame}
  \frametitle{Au c\oe{}ur du modèle de Solow}

  \begin{center}
    \rotatebox{-20}{\includegraphics[scale=.2]{../img/croissant.png}}\rotatebox{-20}{\includegraphics[scale=.15]{../img/croissant.png}}\rotatebox{-20}{\includegraphics[scale=.10]{../img/croissant.png}}\rotatebox{-20}{\includegraphics[scale=.05]{../img/croissant.png}}\quad\rotatebox{-20}{\includegraphics[scale=.01]{../img/croissant.png}}
  \end{center}

\end{frame}


\begin{frame}
  \frametitle{L'hypothèse des rendements (marginaux) décroissants, I}

  \begin{itemize}
    
    \item Une vieille idée chez les économistes (Turgot, Malthus, Ricardo, \ldots) pas toujours très optimistes.\newline

    \item Supposons que, chaque année, Robinson augmente la surface cultivable sur son île\ldots Sans accroître son effort car il travaille déjà beaucoup.\newline
    
    \item Même si les terres suplémentaires sont également fertiles, sa production de blé augmentera de moins en moins.\newline

    \item \emph{Ceteris paribus} en augmentant un unique facteur de production (ici la terre) le \textbf{rendement marginal} est décroissant (l'effort de Robinson sur chaque parcelle diminue).\newline

    \item[\dbend] Si Robinson double la surface cultivée \textbf{et} la quantité de travail (en faisant travailler Vendredi) il se peut qu'il parvienne à doubler la production, on parle alors de \textbf{rendements d'échelle} constants. 

  \end{itemize}

\end{frame}


\begin{frame}
  \frametitle{L'hypothèse des rendements (marginaux) décroissants, II}

  \medskip
  
  \begin{itemize}
    
  \item Cette hypothèse est le c\oe{}ur du modèle de Solow. Sans cette
    hypothèse, qu'il faudra évaluer par rapport aux données, les
    prédictions du modèle changent radicalement.\newline

  \item Elle explique pourquoi une économie croît temporairement plus
    vite après une crise ou une guerre.\newline

  \item Après une guerre, \emph{i.e.} une destruction du capital
    physique, l'investissement dans le capital a un rendement très
    élevé. L'acroissement élevé de production (revenu) permettra
    d'investir demain... Mais au fur et à mesure que le stock de
    capital physique augmente, le rendement de l'investissement
    diminue et les accroissements induits de la production sont de
    plus en plus faibles.\newline

  \item Ce mécanisme conduit à long terme à un épuisement de la croissance.\newline

  \item Pas de croissance à long terme dans ce modèle ($\neq$ Kaldor).\newline

  \end{itemize}

\end{frame}


\begin{frame}
  \frametitle{Progrès technique exogène, I}

  \begin{center}
    \includegraphics[scale=1]{../img/manne-tintoretto.png}
  \end{center}

\end{frame}


\begin{frame}
  \frametitle{Progrès technique exogène, II}

  \begin{itemize}
    
  \item Pour palier l'épuisement de la croissance, il faut une source d'impulsion externe qui améliore l'efficacité des facteurs de production.\newline

  \item On suppose que le progrès tombe du ciel, il s'agit d'une manne.\newline

  \item Tout le monde bénéficie gratuitement d'une technologie qui s'améliore continuement\ldots\newline

  \item \ldots Et on ne cherche pas à expliquer son origine.

  \end{itemize}
\end{frame}


\section{La fonction de production néoclassique}

\begin{frame}
  \frametitle{La fonction de production néoclassique}

\end{frame}


\section{L'équation fondamentale du modèle de Solow}

\begin{frame}
  \frametitle{L'équation fondamentale du modèle de Solow}

\end{frame}


\section{Progrès technique}

\begin{frame}
  \frametitle{Le progrès technique}

\end{frame}


\end{document}


% Local Variables:
% ispell-check-comments: exclusive
% ispell-local-dictionary: "french"
% TeX-master: t
% End: