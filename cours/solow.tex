\synctex=1

\documentclass[10pt,notheorems]{beamer}

\usepackage{etex}
\usepackage{fourier-orns}
\usepackage{ccicons}
\usepackage{amssymb}
\usepackage{amstext}
\usepackage{amsbsy}
\usepackage{amsopn}
\usepackage{amscd}
\usepackage{amsxtra}
\usepackage{amsthm}
\usepackage{float}
\usepackage{color, colortbl}
\usepackage{mathrsfs}
\usepackage{bm}
\usepackage{lastpage}
\usepackage[nice]{nicefrac}
\usepackage{setspace}
\usepackage{ragged2e}
\usepackage{listings}
\usepackage{algorithms/algorithm}
\usepackage{algorithms/algorithmic}
\usepackage[frenchb]{babel}
\usepackage{tikz,pgfplots,pgfplotstable}
\pgfplotsset{compat=newest}
\usetikzlibrary{patterns, arrows, decorations.pathreplacing, decorations.markings, decorations.text, calc}
\pgfplotsset{plot coordinates/math parser=false}
\newlength\figureheight
\newlength\figurewidth
%\usepackage[utf8x]{inputenc}
\usepackage{cancel}
\usepackage{tikz-qtree}
\usepackage{dcolumn}
\usepackage{adjustbox}
\usepackage{environ}
\usepackage[cal=boondox]{mathalfa}
\usepackage{manfnt}
\usepackage{hyperref}
\hypersetup{
  colorlinks=true,
  linkcolor=blue,
  filecolor=black,
  urlcolor=black,
}
\usepackage{venndiagram}
\usepackage{scrextend}

% Git hash
\usepackage{xstring}
\usepackage{catchfile}
\immediate\write18{git rev-parse HEAD > git.hash}
\CatchFileDef{\HEAD}{git.hash}{\endlinechar=-1}
\newcommand{\gitrevision}{\StrLeft{\HEAD}{7}}

\newcommand{\trace}{\mathrm{tr}}
\newcommand{\vect}{\mathrm{vec}}
\newcommand{\tracarg}[1]{\mathrm{tr}\left\{#1\right\}}
\newcommand{\vectarg}[1]{\mathrm{vec}\left(#1\right)}
\newcommand{\vecth}[1]{\mathrm{vech}\left(#1\right)}
\newcommand{\iid}[2]{\mathrm{iid}\left(#1,#2\right)}
\newcommand{\normal}[2]{\mathcal N\left(#1,#2\right)}
\newcommand{\dynare}{\href{http://www.dynare.org}{\color{blue}Dynare}}
\newcommand{\sample}{\mathcal Y_T}
\newcommand{\samplet}[1]{\mathcal Y_{#1}}
\newcommand{\slidetitle}[1]{\fancyhead[L]{\textsc{#1}}}

\newcommand{\R}{{\mathbb R}}
\newcommand{\C}{{\mathbb C}}
\newcommand{\N}{{\mathbb N}}
\newcommand{\Z}{{\mathbb Z}}
\newcommand{\binomial}[2]{\begin{pmatrix} #1 \\ #2 \end{pmatrix}}
\newcommand{\bigO}[1]{\mathcal O \left(#1\right)}
\newcommand{\red}{\color{red}}
\newcommand{\blue}{\color{blue}}

\newcommand*\circled[1]{\tikz[baseline=(char.base)]{
    \node[shape=circle,draw,inner sep=1.0pt] (char) {#1};}}

\renewcommand{\qedsymbol}{C.Q.F.D.}

\newcolumntype{d}{D{.}{.}{-1}}
\definecolor{gray}{gray}{0.9}
\newcolumntype{g}{>{\columncolor{gray}}c}

\setbeamertemplate{theorems}[numbered]

\theoremstyle{plain}
\newtheorem{theorem}{Théorème}

\theoremstyle{definition} % insert bellow all blocks you want in normal text
\newtheorem{definition}{Définition}
\newtheorem{properties}{Propriétés}
\newtheorem{lemma}{Lemme}
\newtheorem{property}[properties]{Propriété}
\newtheorem{example}{Exemple}
\newtheorem*{idea}{Éléments de preuve} % no numbered block

\setbeamertemplate{footline}{
  {\hfill\vspace*{1pt}\href{http://creativecommons.org/licenses/by-sa/3.0/legalcode}{\ccbysa}\hspace{.1cm}
    \raisebox{-.1cm}{\href{https://github.com/stepan-a/growth}{\includegraphics[scale=.015]{../img/git.png}}}\enspace
    \href{https://github.com/stepan-a/growth/blob/\HEAD/cours/solow.tex}{\gitrevision}\enspace--\enspace\today\enspace
  }}

\setbeamertemplate{navigation symbols}{}
\setbeamertemplate{blocks}[rounded][shadow=true]
\setbeamertemplate{caption}[numbered]

\newenvironment{notes}
{\bgroup \justifying\bgroup\tiny\begin{spacing}{1.0}}
  {\end{spacing}\egroup\egroup}

\newenvironment{exercise}[1]
{\bgroup \small\begin{block}{Ex. #1}}
  {\end{block}\egroup}

\newenvironment{defn}[1]
{\bgroup \small\begin{block}{Définition. #1}}
  {\end{block}\egroup}

\newenvironment{exemple}[1]
{\bgroup \small\begin{block}{Exemple. #1}}
  {\end{block}\egroup}


\begin{document}

\title{Croissance\\\small{Le modèle de Solow}}
\author[S. Adjemian]{St\'ephane Adjemian}
\institute{\texttt{stephane.adjemian@univ-lemans.fr}}
\date{Septembre 2020}

\begin{frame}
  \titlepage{}
\end{frame}

\begin{frame}
  \frametitle{Plan}
  \tableofcontents
\end{frame}


\section{Environnement et hypothèses}

\begin{frame}
  \frametitle{Environnement, I}

  \begin{itemize}
  \item Un modèle d'équilibre général.
    \medskip
  \item Les ménages (familles)~:
    \begin{itemize}
    \item[--] possèdent les facteurs de production et les actifs de l'économie,
    \item[--] déterminent la fécondité, la participation au marché du travail \textbf{et} le partage entre consommation présente et future (c'est-à-dire l'épargne).
    \end{itemize}
    \medskip
  \item Les entreprises~:
    \begin{itemize}
    \item[--] louent les facteurs de production (capital et travail) aux ménages,
    \item[--] produisent un bien à partir des services du capital et du travail, vendu aux ménages.
    \end{itemize}
    \medskip
  \item Ménages et firmes se rencontrent sur des marchés~:
    \begin{itemize}
    \item[--] le marché des biens, où les entreprises vendent la production au ménages,
    \item[--] un marché par facteur de production.
    \end{itemize}
    \medskip
  \item \textbf{Simplification:} une robinsonnade\ldots
  \end{itemize}

\end{frame}


\begin{frame}
  \frametitle{Robinson schizophrène}

  \begin{center}
    \includegraphics[scale=.2]{../img/crusoe-1.png}
  \end{center}

\end{frame}


\begin{frame}
  \frametitle{Environnement, II}

  \begin{itemize}

  \item Économie autarcique.

    \begin{itemize}
    \item[$\Rightarrow$] Pas de différence entre revenu et production d'une économie.
    \item[$\Rightarrow$] Pas de différence entre épargne et investissement.
    \end{itemize}

    \bigskip

  \item Hypothèse de concurrence parfaite.\newline

    \begin{itemize}

    \item[$\Rightarrow$] Pas de différence entre l'économie
      centralisée (Robinson) et l'équilibre décentralisé (équilibre
      général avec des marchés),\newline

      \begin{itemize}
      \item[--] On va donc au plus simple\ldots
      \item[--] Mais pas de prix (puisqu'il n'y a pas de marchés).
      \end{itemize}

    \end{itemize}

    \bigskip

  \item Offre de travail inélastique (pas de chômage).\newline

  \item Démographie exogène.\newline

  \item Comportement d'épargne exogène.

  \end{itemize}

\end{frame}


\begin{frame}
  \frametitle{Au c\oe{}ur du modèle de Solow}

  \begin{center}
    \rotatebox{-20}{\includegraphics[scale=.2]{../img/croissant.png}}\rotatebox{-20}{\includegraphics[scale=.15]{../img/croissant.png}}\rotatebox{-20}{\includegraphics[scale=.10]{../img/croissant.png}}\rotatebox{-20}{\includegraphics[scale=.05]{../img/croissant.png}}\quad\rotatebox{-20}{\includegraphics[scale=.01]{../img/croissant.png}}
  \end{center}

\end{frame}


\begin{frame}
  \frametitle{L'hypothèse des rendements (marginaux) décroissants, I}

  \begin{itemize}

  \item Une vieille idée chez les économistes (Turgot, Malthus, Ricardo, \ldots) pas toujours très optimistes.\newline

  \item Supposons que, chaque année, Robinson augmente la surface cultivable sur son île\ldots Sans accroître son effort car il travaille déjà beaucoup.\newline

  \item Même si les terres suplémentaires sont également fertiles, sa production de blé augmentera de moins en moins.\newline

  \item \emph{Ceteris paribus} en augmentant un unique facteur de production (ici la terre) le \textbf{rendement marginal} est décroissant (l'effort de Robinson sur chaque parcelle diminue).\newline

  \item[\dbend] Si Robinson double la surface cultivée \textbf{et} la quantité de travail (en faisant travailler Vendredi) il se peut qu'il parvienne à doubler la production, on parle alors de \textbf{rendements d'échelle} constants.

  \end{itemize}

\end{frame}


\begin{frame}
  \frametitle{L'hypothèse des rendements (marginaux) décroissants, II}

  \medskip

  \begin{itemize}

  \item Cette hypothèse est le c\oe{}ur du modèle de Solow. Sans cette
    hypothèse, qu'il faudra évaluer par rapport aux données, les
    prédictions du modèle changent radicalement.\newline

  \item Elle explique pourquoi une économie croît temporairement plus
    vite après une crise ou une guerre.\newline

  \item Après une guerre, \emph{i.e.} une destruction du capital
    physique, l'investissement dans le capital a un rendement très
    élevé. L'acroissement élevé de production (revenu) permettra
    d'investir demain... Mais au fur et à mesure que le stock de
    capital physique augmente, le rendement de l'investissement
    diminue et les accroissements induits de la production sont de
    plus en plus faibles.\newline

  \item Ce mécanisme conduit à long terme à un épuisement de la croissance.\newline

  \item Pas de croissance à long terme dans ce modèle ($\neq$ Kaldor).\newline

  \end{itemize}

\end{frame}


\begin{frame}
  \frametitle{Progrès technique exogène, I}

  \begin{center}
    \includegraphics[scale=1]{../img/manne-tintoretto.png}
  \end{center}

\end{frame}


\begin{frame}
  \frametitle{Progrès technique exogène, II}

  \begin{itemize}

  \item Pour palier l'épuisement de la croissance, il faut une source d'impulsion externe qui améliore l'efficacité des facteurs de production.\newline

  \item On suppose que le progrès tombe du ciel, il s'agit d'une manne.\newline

  \item Tout le monde bénéficie gratuitement d'une technologie qui s'améliore continuement\ldots\newline

  \item \ldots Et on ne cherche pas à expliquer son origine.

  \end{itemize}
\end{frame}


\section{La fonction de production néoclassique}

\begin{frame}
  \frametitle{Le bien}

  \begin{itemize}

  \item On suppose qu'il existe un unique bien dans l'économie, en quantitée $Y$.\newline

  \item Ce bien homogène est utilisé pour la consommation ($C$) et l'investissement ($I$) dans le capital physique ($K$).\newline

  \item Dans ce chapitre on suppose que la production du bien homogène (en quantitée $Y$) nécessite deux facteurs~: les services du stock de capital physique ($K$) et du travail ($L$).\newline

  \item La technologie de production est caractérisée par une fonction de production, $Y = F(K,L)$, dont les propriétés sont déterminantes dans le modèle de Solow (pour la transition et le long terme).\newline

  \end{itemize}

\end{frame}



\begin{frame}
  \frametitle{La fonction de production néoclassique, I}


  \begin{definition}
    Une fonction de production $F(K,L)$ est dite néoclassique si~:
    \medskip
    \begin{enumerate}

    \item Les productivités marginales sont positives, $F_K(K,L)>0$ et $F_L(K,L)>0$.\newline

    \item Les productivités marginales sont décroissantes, $F_{KK}(K,L)<0$ et $F_{LL}(K,L)<0$.\newline

    \item Les rendements d'échelle sont constants~: $F(\lambda K, \lambda L) = \lambda F(K,L)$ pour tout $\lambda>0$.\newline

    \item Les conditions d'Inada~:
      \[
        \begin{split}
          \lim_{K\rightarrow 0}F_K(K,L) = \infty & \lim_{K\rightarrow \infty}F_K(K,L) = 0 \\
          \lim_{L\rightarrow 0}F_L(K,L) = \infty & \lim_{L\rightarrow \infty}F_L(K,L) = 0
        \end{split}
      \]

    \end{enumerate}
  \end{definition}

\end{frame}


\begin{frame}
  \frametitle{La fonction de production néoclassique, II}
  \framesubtitle{Remarques}

  Une fonction de production néoclassique est donc~:

  \medskip

  \begin{itemize}

  \item Croissante, puisque les productivités marginales sont positives. Quand on augmente la quantité de $K$ ou de $L$ on augmente la production ($Y$).\newline

  \item À rendements décroissants, puisque les productivités marginales sont décroissantes. Plus on augmente la quantité de travail, moins la production augmente.\newline

  \item À rendements d'échelle constants, puisque la fonction deproduction est homogène de degré un. Doubler les quantités de facteurs de production, double la production.\newline

  \end{itemize}

  On verra plus loin que les conditions d'Inada, qui caractérisent le comportement aux bords des productivités marginales, sont essentielles pour définir les propriétés de long terme du modèle.\newline
\end{frame}

\begin{frame}
  \frametitle{La fonction de production néoclassique, III}
  \framesubtitle{Exemple~: La fonction Cobb-Douglas (a)}


  \[
    Y = K^{\alpha}L ^{1-\alpha}
  \]
  avec $0<\alpha<1$.

  \bigskip

  \begin{itemize}

  \item $F_K(K,L) = \alpha K^{\alpha-1}L^{1-\alpha}=\alpha\frac{Y}{K}>0$,
  \item $F_L(K,L) = (1-\alpha) K^{\alpha}L^{-\alpha}=(1-\alpha)\frac{Y}{L}>0$,
  \item $F_{KK}(K,L) = \alpha(\alpha-1)K^{\alpha-2}L^{1-\alpha}=\alpha(\alpha-1)\frac{Y}{K^2}<0$
  \item $F_{LL}(K,L) = -\alpha(1-\alpha)K^{\alpha}L^{-\alpha-1}=-\alpha(1-\alpha)\frac{Y}{L^2}<0$
  \item $\left(\lambda K\right)^{\alpha}\left(\lambda L\right)^{1-\alpha}=\lambda^{\alpha+1-\alpha}K^{\alpha}L ^{1-\alpha} = \lambda K^{\alpha}L ^{1-\alpha}$ $\forall \lambda>0$
  \item En notant que l'on a aussi $F_K(K,L) = \alpha \left(\frac{L}{K}\right)^{1-\alpha}$ et $F_L(K,L) = (1-\alpha) \left(\frac{K}{L}\right)^{\alpha}$, on voit facilement que les conditions d'Inada sont satisfaites.
  \end{itemize}

\end{frame}


\begin{frame}
  \frametitle{La fonction de production néoclassique, III}
  \framesubtitle{Exemple~: La fonction Cobb-Douglas (b)}

  \begin{center}
    \begin{tikzpicture}[scale=1.2]
      \begin{axis}[
        title={$Y = K^{\alpha}L^{1-\alpha}$},
        xlabel=$K$,
        ylabel=$L$,
        xticklabels={,,},
        yticklabels={,,},
        zticklabels={,,},
        3d box=complete,
        enlargelimits=false,
        grid,
        grid style={dashed, gray!60},
        axis line style={thin},
        small,
        ]
        \addplot3[
        mesh,
        draw=black,
        samples=50,
        domain=0:10,
        domain y=0:10,
        ]
        {x^0.333333*y^(1-0.333333)};
      \end{axis}
    \end{tikzpicture}
  \end{center}
\end{frame}

\begin{frame}
  \frametitle{La fonction de production néoclassique, III}
  \framesubtitle{Exemple~: La fonction Cobb-Douglas (c)}

  \begin{itemize}
  \item L'hypothèse de rendements d'échelle constants permet de réécrire la technologie sous forme intensive, c'est à dire en exprimant la production par tête $y=\nicefrac{Y}{L}$ comme une fonction du stock de capital physique par tête $k = \nicefrac{K}{L}$.\newline

  \item On a~:
    \[
      Y = K^{\alpha}L^{1-\alpha}
    \]
    en divisant les deux membres par $L$~:
    \[
      \begin{split}
        \frac{Y}{L} &= \frac{K^{\alpha}L^{1-\alpha}}{L^{\alpha+1-\alpha}}\\
        &= \left(\frac{K}{L}\right)^{\alpha} \left(\frac{L}{L}\right)^{1-\alpha}\\
      \end{split}
    \]
    et donc~:
    \[
      y = k^{\alpha}
    \]
  \end{itemize}

\end{frame}


\begin{frame}
  \frametitle{La fonction de production néoclassique, III}
  \framesubtitle{Exemple~: La fonction Cobb-Douglas (d)}

  \medskip

  La fonction de production sous forme intensive, $y=f(k)=k^{\alpha}$, hérite des propriétés de la fonction de production $F(K,L)=K^{\alpha}L^{1-\alpha}$:

  \medskip

  \begin{enumerate}
  \item Productivité marginal positive~: $f'(k) = \alpha k^{\alpha-1}>0$,
  \item Productivité marginale décroissante~: $f''(k)= -(1-\alpha)\alpha k^{\alpha-2}<0$,
  \item Conditions d'Inada~: $\lim_{k\rightarrow 0} f'(k)=\infty$ et $\lim_{k\rightarrow \infty} f'(k)=0$
  \end{enumerate}

  \medskip

  \begin{columns}
    \begin{column}{0.6\textwidth}
      {\small
        \begin{itemize}
        \item[--] Les pentes des tangentes sont $>0$,
        \item[--] Les pentes des tangentes sont $\searrow$,
        \item[--] La pente de la tangente en 0 est $\infty$,
        \item[--] La pente de la tangente en $\infty$ est 0.
          \bigskip
        \item[--] La fonction est concave.
        \end{itemize}}
    \end{column}
    \begin{column}{0.4\textwidth}
      \begin{tikzpicture}[scale=.75]
        \begin{axis}[
          title={$y = k^{\alpha}$},
          xlabel=$k$,
          ylabel=$y$,
          xticklabels={,,},
          yticklabels={,,},
          enlargelimits=false,
          grid style={dashed, gray!60},
          axis x line = bottom,
          axis y line = left,
          axis line style={thin},
          small,
          ]
          \addplot[
          draw=black,
          thick,
          smooth,
          samples=500,
          domain=0:8,
          ]
          {x^0.333333};
          \addplot[mark=none, red, thin] coordinates {(0,0.66666666) (8,3.3333333)};
          \addplot[mark=none, red, thin] coordinates {(0,1.0582673679787995) (8,2.1165347359575994)};
          \addplot[mark=*] coordinates {(1,1)};
          \addplot[mark=*] coordinates {(4,1.5874010519681994)};
        \end{axis}
      \end{tikzpicture}
    \end{column}
  \end{columns}

\end{frame}


\begin{frame}
  \frametitle{La fonction de production néoclassique, III}
  \framesubtitle{Exemple~: La fonction Cobb-Douglas (e)}

  \medskip

  \begin{itemize}

  \item Le paramètre $\alpha$ s'interprète comme l'élasticité de la production par rapport au stock de capital physique.\newline

  \item En effet, cette élasticité est définie par~:
    \[
      \epsilon(K,L) = \frac{\frac{\partial Y}{\partial K}}{\frac{Y}{K}}
    \]
    soit~:
    \[
      \begin{split}
        \epsilon(K,L) &= \frac{K}{Y}\alpha K^{\alpha-1}L^{1-\alpha}\\
        &= \alpha\frac{K^{\alpha}L^{1-\alpha}}{Y}\\
        &= \alpha \quad \forall (K,L)\in \mathbb R_+^2
      \end{split}
    \]

    \medskip

  \item On note que cette élasticité ne dépend pas de $K$ ou $L$.

  \end{itemize}
\end{frame}


\begin{frame}
  \frametitle{La fonction de production néoclassique, III}
  \framesubtitle{Exemple~: La fonction Cobb-Douglas (e, suite)}

  \medskip

  \begin{itemize}

  \item Ce même paramètre $\alpha$, dans le modèle de Solow, où la concurrence est parfaite, détermine aussi le partage du revenu entre le capital et le travail.\newline

  \item En effet, dans une économie en concurrence parfaite, les facteurs sont rémunérés aux productivités marginales.\newline

  \item Chaque unité de travail est rémunérée au taux de salaire $w=\frac{\partial Y}{\partial L}$.\newline

  \item Chaque unité de capital est rémunérée au taux d'intérêt $r = \frac{\partial Y}{\partial K}$.\newline

  \item La rémunération totale du capital est donc~: $K\frac{\partial Y}{\partial K}$, et la part de la rémunération du capital dans le revenu total~:
    \[
      \frac{K}{Y}\frac{\partial Y}{\partial K} = \alpha
    \]

  \end{itemize}
\end{frame}


\begin{frame}
  \frametitle{La fonction de production néoclassique, III}
  \framesubtitle{Exemple~: La fonction Cobb-Douglas (f)}

  \medskip

  \begin{itemize}

  \item Le Taux Marginal de Substitution Technique du travail au capital, noté $\textrm{TMST}_{L,K}$, est défini comme la variation de travail nécessaire pour compenser une variation infinitésimale du stock de capital physique (de sorte que que $Y$ soit constant).\newline

  \item En considérant la différentielle totale de la fonction de production~:
    \[
      \mathrm dY = F_K(K,L)\mathrm dK + F_L(K,L)\mathrm dL
    \]
    \begin{columns}
      \begin{column}{0.7\textwidth}
        \begin{addmargin}[1,1cm]{0cm}
          en posant $\mathrm dY = 0$, il vient~:
          \[
            \frac{\mathrm d L}{\mathrm d K} = -\frac{F_K(K,L)}{F_L(K,L)} \equiv \textrm{TMST}_{L,K}
          \]
        \end{addmargin}
      \end{column}
      \begin{column}{0.3\textwidth}
        {\tiny Certains imposent la positivité du TMST en le définissant comme le rapport des productivités marginales.}
      \end{column}
    \end{columns}
    \bigskip
  \item Avec la fonction Cobb-Douglas, nous avons~:
    \begin{columns}
      \begin{column}{0.7\textwidth}
        \begin{addmargin}[1,1cm]{0cm}
          \[
            \textrm{TMST}_{L,K} = -\frac{\alpha}{1-\alpha}\frac{L}{K}
          \]
        \end{addmargin}
      \end{column}
      \begin{column}{0.3\textwidth}
        {\tiny Le TMST donne la pente des isoquantes, à l'optimum il doit être égal au rapport des prix.}
      \end{column}
    \end{columns}
  \end{itemize}
\end{frame}


\begin{frame}
  \frametitle{La fonction de production néoclassique, III}
  \framesubtitle{Exemple~: La fonction Cobb-Douglas (g)}

  \medskip

  \begin{itemize}

  \item L'élasticité de substitution entre les facteurs, notée $\sigma(K,L)$, est le taux de croissance du ratio des facteurs de production ($\mathrm d \log \frac{L}{K}$) rapporté au taux de croissance du taux marginal de substitution technique ($\mathrm d \log |TMST_{L,K}|$).\newline

  \item $\sigma(K,L)$ caractérise la courbure des isoquantes.\newline

  \item Dans le cas Cobb-Douglas, le TMST est proportionel au ratio $\nicefrac{L}{K}$.\newline

  \item Les deux de croissance de ces deux termes sont donc identiques, ainsi~:
    \[
      \sigma(K,L) = 1\quad \forall (K,L) \in \mathbb R_+^2
    \]
  \end{itemize}
\end{frame}




\section{L'équation fondamentale du modèle de Solow}

\begin{frame}
  \frametitle{L'équation fondamentale du modèle de Solow}
  \framesubtitle{Encore une histoire de baignoire et de fuite\ldots}

  \begin{block}{Accumulation du stock de capital physique aggrégé}
    \[
      \dot K(t) = I(t) - \delta K(t)
    \]
  \end{block}

  \bigskip

  \begin{itemize}

  \item $0<\delta<1$ est le taux de dépréciation du stock de capital physique.\newline

  \item $\delta K$ est le volume de capital qui s'évapore à chaque instant.\newline

  \item[\dbend] Ne pas confondre le \textbf{taux} de dépréciation ($\delta$) et la dépréciation ($\delta K$, il s'agit d'un volume).\newline

  \item Cette équation nous dit que le stock de capital physique augmente ($\dot K > 0$) ssi l'investissement est supérieur à la dépréciation.
  \end{itemize}

\end{frame}


\begin{frame}
  \frametitle{L'équation fondamentale du modèle de Solow}
  \framesubtitle{Expérience 1: Investissement constant (a)\ldots}

  \bigskip

  \begin{itemize}

  \item Supposons que l'investissement soit constant, \emph{i.e.} $I(t)=\iota\quad\forall t\in\mathbb R_+$. Caractérisons la dynamique du stock de capital.\newline

  \item Nous avons~:
    \[
      \dot K(t) = \iota - \delta K(t)
    \]
    Il s'agit d'une équation différentielle linéaire d'ordre 1 à
    coefficients constants.\newline

  \item Si $\iota=0$ (\emph{i.e.} pas de second membre), on retrouve une équation différentielle analogue à celle que nous avons rencontrée dans le chapitre introductif~:
    \[
      \frac{\dot K(t)}{K(t)} = -\delta
    \]
    S'il n'y a pas d'investissement le stock de capital décroît, le taux de (dé)croissance est le taux de dépréciation du capital. Il n'est pas nécessaire de savoir résoudre une équation différentielle pour comprendre qu'à long terme le capital disparaît\ldots

  \end{itemize}

\end{frame}


\begin{frame}
  \frametitle{L'équation fondamentale du modèle de Solow}
  \framesubtitle{Expérience 1: Investissement constant (b)\ldots}

  \bigskip

  \begin{itemize}

  \item Cherchons une solution de cette équation différentielle, \emph{i.e.} cherchons une fonction $K(t)$ qui vérifie cette équation.\newline

  \item En utilisant la dérivée logarithmique, puis en intégrant les deux membres de cette équation, il vient~:
    \[
      \log K(t) = \tilde A - \delta t
    \]
    où $\tilde A$ est une constante d'intégration (qui apparaît avec le calcul des primitives).\newline

  \item Nous savons donc qu'une solution de l'équation différentielle est~:
    \[
      K(t) = A e^{-\delta t}
    \]
    où $A = e^{\tilde A}\in\mathbb R_+$. Nous avons donc une infinité de solution, une pour chaque valeur  possible de $A$.
  \end{itemize}

\end{frame}


\begin{frame}
  \frametitle{L'équation fondamentale du modèle de Solow}
  \framesubtitle{Expérience 1: Investissement constant (c)\ldots}

  \bigskip

  \begin{itemize}

  \item Puisque $\delta$, le taux de dépréciation est positif, nous avons bien comme annoncé~:
    \[
      \lim_{t\rightarrow\infty}K(t) = 0
    \]
    Puisqu'il n'y a pas d'investissement, la dépréciation épuise complétement le stock de capital à long terme.\newline

  \item Que se passe t-il si $\iota>0$~?\ldots Il s'agit d'une équation différentielle linéaire avec un second membre (l'investissement $\iota$). Pour trouver une solution, on commence par chercher une solution particulère de l'équation différentielle avec second membre~:
    \[
      \dot K(t) = -\delta K(t) + \iota
    \]

  \item La constante $K^{\star}(t) = \nicefrac{\iota}{\delta} \forall t\in\mathbb R_+$ est une solution évidente.\newline

  \item En effet pour cette solution la variation est nulle, $\dot{K^{\star}} = 0$, et $\iota-\delta K^{\star}(t) = 0$.
  \end{itemize}

\end{frame}


\begin{frame}
  \frametitle{L'équation fondamentale du modèle de Solow}
  \framesubtitle{Expérience 1: Investissement constant (d)\ldots}

  \bigskip

  \begin{itemize}

  \item Une solution générale de l'équation avec second membre est la somme de la solution générale de l'équation sans second membre et de la solution particulière de l'équation complète~:
    \[
      K(t) = Ae^{-\delta t} + \frac{\iota}{\delta}
    \]

  \item On peut vérifier en substituant cette solution dans l'équation différentielle complète.
    \begin{itemize}
    \item[--] Pour le membre de gauche on a~:
      \[
        {\color{red}\dot K(t)} = -\delta A e^{-\delta t}
      \]
    \item[--] Pour le membre de droite on a~:
      \[
        \begin{split}
          {\color{red}\iota - \delta K(t)} &= \iota - \delta \left(Ae^{-\delta t} + \frac{\iota}{\delta}\right)\\
          &= -\delta A e^{-\delta t}
        \end{split}
      \]
    \end{itemize}
  \item Les deux membres sont identiques, on a donc bien trouvé la solution générale de l'équation différentielle $\dot K(t) = -\delta K(t) + \iota$.
  \end{itemize}

\end{frame}


\begin{frame}
  \frametitle{L'équation fondamentale du modèle de Solow}
  \framesubtitle{Expérience 1: Investissement constant (e)\ldots}

  \bigskip

  \begin{itemize}

  \item À nouveau, nous disposons d'une infinité de solution~: une pour chaque valeur de la constante d'intégration $A$.\newline

  \item Pour sélectionner une solution, on utilise la condition initiale du stock de capital physique. Suppose qu'à l'instant initial nous ayons $K(0)=K_0>0$.\newline

  \item En évaluant la solution générale en 0, nous devons avoir~:
    \[
      K_0 = A + \frac{\iota}{\delta}
    \]

  \item Ainsi, la solution de l'équation différentielle qui « passe » par la condition initiale $K_0$ est~:
    \[
      K(t) = \left(K_0-\frac{\iota}{\delta}\right)e^{-\delta t} + \frac{\iota}{\delta}
    \]
  \end{itemize}

\end{frame}


\begin{frame}
  \frametitle{L'équation fondamentale du modèle de Solow}
  \framesubtitle{Expérience 1: Investissement constant (f)\ldots}

  \bigskip

  \begin{itemize}

  \item Pour toute condition initiale $K_0$ on a~:
    \[
      \lim_{t\rightarrow\infty} K(t) = \frac{\iota}{\delta} \equiv K^{\star}
    \]

  \item Le niveau de long terme du stock de capital physique ne dépend pas de la condition initiale, et ce niveau est d'autant plus élevé que l'investissement est important ou le taux de dépréciation faible.\newline

  \item Si $K_0>\frac{\iota}{\delta}$, la transition vers $K^{\star}$ est monotone décroissante.\newline
  \item Si $K_0<\frac{\iota}{\delta}$, la transition vers $K^{\star}$ est monotone croissante.\newline

  \item Si $K_0 = K^{\star}$ alors $K(t)=K^{\star}\forall t\in\mathbb R_+$.
  \end{itemize}

\end{frame}


\begin{frame}
  \frametitle{L'équation fondamentale du modèle de Solow}
  \framesubtitle{Expérience 2: Investissement variable (a)\ldots}

  \begin{itemize}
  \item On peut généraliser en considérant un flux d'investissement variable (c'est aussi plus vraisemblable). L'équation différentielle à résoudre devient~:
    \[
      \dot K(t) = -\delta K(t) + I(t)
    \]

    \bigskip

  \item Il s'agit d'une fonction différentielle linéaire à coefficients variables (on a remplacé la constante $\iota$ par une fonction arbitraire $I(t)$).\newline


  \item Si la condition initiale est $K(0)=K_0$ alors on peut montrer qu'il existe une unique solution~:
    \[
      K(t) = e^{-\delta t}\left(K_0 + \int_0^te^{\delta s}I(s)\mathrm ds\right)
    \]
  \end{itemize}

\end{frame}


\begin{notes}
  \begin{itemize}

  \item Pour résoudre l'équation différentielle, multiplions les deux membres par $e^{-\delta t}$~:
    \[
      e^{\delta t}\left(\dot K(t)+\delta K(t)\right) = e^{\delta t} I(t)
    \]

  \item Or, nous avons~:
    \[
      \begin{split}
        \frac{\mathrm d}{\mathrm dt} e^{\delta t} K(t) &= e^{\delta t} \dot K(t) + \delta e^{\delta t}K(t)\\
        &= e^{\delta t}\left(\dot K(t)+\delta K(t)\right)
      \end{split}
    \]

  \item On peut donc réécrire l'équation différentielle sous la forme~:
    \[
      \frac{\mathrm d}{\mathrm ds} e^{\delta s} K(s) = e^{\delta s} I(s)
    \]

  \item En intégrant les deux membres entre 0 et $t$, il vient~:
    \[
      e^{\delta t}K(t) - K(0) = \int_0^te^{\delta s} I(s)\mathrm ds
    \]
    et donc~:
    \[
      K(t) = e^{-\delta t}\left(K(0) + \int_0^te^{\delta s} I(s)\mathrm ds\right)
    \]

  \end{itemize}

\end{notes}


\begin{frame}
  \frametitle{L'équation fondamentale du modèle de Solow}
  \framesubtitle{Expérience 2: Investissement variable (b)\ldots}

  \begin{itemize}

  \item Si le taux de dépréciation est nul, $\delta=0$, on voit que le stock de capital à l'instant $t$ est le stock de capital à l'instant initial ($K_0$) plus la somme des flux d'investissement entre 0 et $t$ ($\int_0^tI(s)\mathrm ds$).\newline

  \item Plus généralement, pour interpréter le résultat, notons que $e^{-\tau \delta}$ est le facteur de dépréciation entre 0 et $\tau$. Si $X$ se déprécie au taux $\delta$ alors $\frac{X(t)}{X(0)}=e^{-\delta t}$ et $\frac{X(t)}{X(s)} = \frac{X(t)}{X(0)}\frac{X(0)}{X(s)} = e^{-(t-s)\delta}$.\newline

  \item En réécrivant l'équation pour $K(t)$ sous la forme~:
    \[
      K(t) = e^{-\delta t}K_0 + \int_0^te^{-(t-s)\delta}I(s)\mathrm ds
    \]
    $K(t)$ est donc la somme de ce qui reste du capital initial, $e^{-\delta t}K_0$, et la somme des flux d'investissement dépréciés~: $e^{-(t-s)\delta}I(s)$ pour $s\in[0,t]$.
  \end{itemize}

\end{frame}


\begin{frame}
  \frametitle{L'équation fondamentale du modèle de Solow}
  \framesubtitle{Expérience 2: Investissement variable (c)\ldots}

  \begin{itemize}

  \item Il est difficile d'en dire plus sur la dynamique du stock de capital physique, comme nous pouvions le faire dans le cas où l'investissement est constant, sans spécifier la fonction d'investissmeent\ldots\newline

  \item Heureusement dans le modèle de Solow, on a une idée plus précise sur $I(t)$, qui va nous permettre d'aller beaucoup plus loin.\newline

  \item L'investissement ne vient pas de nul part\ldots Dans une économie fermée il est déterminé par l'épargne qui elle même dépend du revenu (ou de la production) et donc indirectement de $K(t)$.\newline

  \end{itemize}

\end{frame}


\begin{frame}
  \frametitle{L'équation fondamentale du modèle de Solow}
  \framesubtitle{L'investissement}

  \begin{itemize}

  \item Dans une économie fermée l'investissement doit être égal à l'épargne.\newline

  \item[\circled{$\mathcal H$}] On suppose que l'épargne est une faction constante de la production (du revenu).\newline

  \item On note $s\in[0,1]$ le taux d'épargne, on a donc~: $I(t) = sY(t)$.\newline

  \item La production elle même dépend, via la fonction de production, du stock de capital physique et du travail, on a donc $I(t) = s K(t)^{\alpha}L(t)^{1-\alpha}$ et finalement~:
  \end{itemize}

  \begin{block}{Accumulation du stock de capital physique aggrégé}
    \[
      \dot K(t) =  s K(t)^{\alpha}L(t)^{1-\alpha} -  \delta K(t)
    \]
  \end{block}

\end{frame}


\begin{frame}
  \frametitle{L'équation fondamentale du modèle de Solow}
  \framesubtitle{Sans croissance de la population (a)}

  \begin{itemize}

  \item Supposons que la population soit constante, $L(t) = L_0\forall t\in\mathbb R_+$.\newline

  \item La dynamique du capital est alors caractérisée par~:
    \bigskip
    \[
      \dot K(t) =  \underbrace{s L_0^{1-\alpha}K(t)^{\alpha}}_{\text{Investissement}} -  \underbrace{\delta K(t)}_{\text{Dépréciation}}
    \]

    \bigskip

  \item Il s'agit d'une équation différentielle non linéaire que nous ne savons pas résoudre (on verra en TD qu'il est possible de résoudre cette équation avec un changement de variable).\newline

  \item Sans explicitement résoudre l'équation, on peut se faire une idée assez précise de la dynamique à l'aide d'une analyse graphique.\newline
  \end{itemize}

\end{frame}


\begin{frame}
  \frametitle{L'équation fondamentale du modèle de Solow}
  \framesubtitle{Sans croissance de la population (b)}

  \bigskip

  \begin{itemize}

  \item Pour représenter graphiquement la dynamique nous travaillons dans un plan avec $K$ sur l'axe des abscisses et une quantité de bien homogène sur l'axe des ordonnées.\newline

  \item Dans ce plan on peut représenter~:
    \medskip
    \begin{itemize}
    \item[--] {\color{blue} L'investissement~:} $sL_0^{1-\alpha}K^{\alpha}$, est une courbe monotone croissante concave avec une pente infinie en 0 et nulle en l'infini (Inada) qui a pour origine (0,0).\newline
    \item[--] {\color{blue} La dépréciation~:} $\delta K$, qui est une droite croissante de pente $\delta$ et d'origine en (0,0).\newline
    \end{itemize}

  \item Puisque la pente de l'investissement est infinie en 0 alors que la pente de dépréciation est finie ($\delta$), la courbe d'investissement part au dessus de la droite de dépréciation.\newline

  \item Puisque la pente de la courbe d'investissement décroît de façon monotone et tend vers 0, elle croise nécessairement une unique fois la droite de dépréciation.
  \end{itemize}

\end{frame}


\begin{frame}
  \frametitle{L'équation fondamentale du modèle de Solow}
  \framesubtitle{Sans croissance de la population (c)}

  \bigskip

  \begin{center}
    \begin{tikzpicture}[scale=1.3]
      \begin{axis}[
        title={},
        xlabel= $K$,
        ylabel= {Quantité de bien homogène},
        xticklabels={,,},
        yticklabels={,,},
        enlargelimits=true,
        grid style={dashed, gray!60},
        axis x line = bottom,
        axis y line = left,
        axis line style={thin},
        xmin = 0,
        xmax = 11,
        ymin = 0,
        ymax = 0.8,
        small,
        clip=false,
        ]
        \only<2->{
          \addplot[
          draw=black,
          thick,
          smooth,
          samples=500,
          domain=0:10,
          ]
          {.2*x^0.333333};
          \node[right] at (10.1, 0.43){{\small{\color{blue}$s L_0^{1-\alpha}K^{\alpha}$}}};
        }
        \only<3->{
          \addplot[
          draw=black,
          thick,
          smooth,
          samples=2,
          domain=0:10,
          ]
          {.075*x};
          \node[right] at (10.1, 0.75){{\small{\color{blue}$\delta K$}}};
        }
        \only<4-5>{
          \addplot[draw=red, thick, smooth, <->] coordinates {
            (1,0.075)
            (1,0.2)};
          \node[right] at (1.6, 0.13){{\small{\color{red}$\dot K>0$}}};
        }
        \only<5-5>{
          \addplot[draw=red, thick, smooth, <->] coordinates {
            (7,0.525)
            (7,0.38256142141126714)};
          \node[right] at (7.6, 0.55){{\small{\color{red}$\dot K<0$}}};
        }
        \only<6->{
          \node[right] at (10.1, 0.75){{\small{\color{blue}$\delta K$}}};
          \node at (0.5,0) {\tiny $\blacktriangleright$};
          \node at (1.5,0) {\tiny $\blacktriangleright$};
          \node at (2.5,0) {\tiny $\blacktriangleright$};
          \node at (3.5,0) {\tiny $\blacktriangleright$};
          \node at (5.5,0) {\tiny $\blacktriangleleft$};
          \node at (6.5,0) {\tiny $\blacktriangleleft$};
          \node at (7.5,0) {\tiny $\blacktriangleleft$};
          \node at (8.5,0) {\tiny $\blacktriangleleft$};
          \node at (9.5,0) {\tiny $\blacktriangleleft$};
        }
        %
        % Steady state
        %
        \only<7->{
          \addplot[draw=red, thin] coordinates {(4.35464843161454, 0) (4.35464843161454,0.3265986323710905)};
          \addplot[draw=red, thin] coordinates {(0, 0.3265986323710905) (4.35464843161454,0.3265986323710905)};
          \node[draw=red,circle, fill=red, scale=.3] at (4.35464843161454,0.3265986323710905) {};
          \node[below] at (4.35464843161454,0) {\tiny{\color{red}$K^{\star}$}};
          \node[left] at (0,0.3265986323710905) {\tiny{\color{red}$I^{\star}$}};
        }
      \end{axis}
    \end{tikzpicture}
  \end{center}

\end{frame}


\begin{frame}
  \frametitle{L'équation fondamentale du modèle de Solow}
  \framesubtitle{Sans croissance de la population (d)}

  \bigskip

  \begin{itemize}

  \item Pour toute condition initiale, $K(0)$, l'économie se dirige à long terme vers $K^{\star}$.\newline

  \item La convergence vers $K^{\star}$ est monotone croissante si $K(0)<K^{\star}$ ou décroissante si $K(0)>K^{\star}$.\newline

  \item $K^{\star}$ est ce que l'on appelle un état stationnaire, quand le stock de capital est égal à $K^{\star}$ il ne varie plus ($\dot K = 0$) car l'investissement est exactement égal à la dépréciation.\newline

  \item L'état stationnaire est défini par l'égalité de l'investissement et de la dépréciation~:
    \[
      \delta K^{\star} = s L_0^{1-\alpha}{K^{\star}}^{\alpha}
    \]
    soit de façon équivalente~:
    \[
      K^{\star} = L_0\left(\frac{s}{\delta}\right)^{\frac{1}{1-\alpha}}
    \]
  \end{itemize}

\end{frame}


\begin{frame}
  \frametitle{L'équation fondamentale du modèle de Solow}
  \framesubtitle{Sans croissance de la population (e)}

  \bigskip

  \begin{itemize}

  \item À long terme, le stock de capital atteint un niveau plus élevé si le taux d'épargne ($s$) est plus important, si l'élasticité de la production par rapport au capital ($\alpha$) est plus élevée (en supposant $s>\delta$), ou si le taux de dépréciation ($\delta$) est plus faible\newline

    \begin{center}
      \begin{tikzpicture}[scale=1]
        \begin{axis}[
          title={},
          xlabel= $K$,
          ylabel= {Quantité de bien homogène},
          xticklabels={,,},
          yticklabels={,,},
          enlargelimits=true,
          grid style={dashed, gray!60},
          axis x line = bottom,
          axis y line = left,
          axis line style={thin},
          xmin = 0,
          xmax = 11,
          ymin = 0,
          ymax = 0.8,
          small,
          clip=false,
          ]
          \addplot[
          draw=black,
          thick,
          smooth,
          samples=500,
          domain=0:10,
          ]
          {.2*x^0.333333};
          \node[right] at (10.1, 0.43){{\tiny{\color{blue}$s L_0^{1-\alpha}K^{\alpha}$}}};
          \addplot[
          draw=black,
          thick,
          smooth,
          samples=2,
          domain=0:10,
          ]
          {.075*x};
          \node[right] at (10.1, 0.75){{\tiny{\color{blue}$\delta K$}}};
          %
          % Steady state
          %
          \coordinate (a) at (4.354648431614539, 0.3265986323710904);
          \draw [dashed]
          (a -| 0,0)      node [left] { \tiny $I^{\star}$ }
          -- (a)          node [circle,fill,inner sep=1pt] {}
          -- (a |- 0,0)   node [below] { \tiny $K^{\star}$ };
          %
          % Increase in the elasticity of production w.r.t. capital
          %
          \only<2>{
            \addplot[
            draw=red,
            thick,
            smooth,
            samples=500,
            domain=0:10,
            ]
            {.2*x^0.5};
            \coordinate (a1) at (7.1111111111111125, 0.5333333333333334);
            \draw [dashed, red]
            (a1 -| 0,0)      node [left] { \tiny $I^{\star}$ }
            -- (a1)          node [circle,fill,inner sep=1pt] {}
            -- (a1 |- 0,0)   node [below] { \tiny $K^{\star}$ };
            \node at (4.4,0.70) {\color{red}\small $\mathrm d \alpha >0$};
          }
          %
          % Increase in the saving rate
          %
          \only<4>{
            \addplot[
            draw=red,
            thick,
            smooth,
            samples=500,
            domain=0:10,
            ]
            {.3*x^0.333333333};
            \coordinate (a2) at (7.999999999999997, 0.5999999999999998);
            \draw [dashed, red]
            (a2 -| 0,0)      node [left] { \tiny $I^{\star}$ }
            -- (a2)          node [circle,fill,inner sep=1pt] {}
            -- (a2 |- 0,0)   node [below] { \tiny $K^{\star}$ };
            \node at (4.4,0.70) {\color{red}\small $\mathrm d s >0$};
          }
          %
          % Decrease in the depreciation rate
          %
          \only<6>{
            \addplot[
            draw=red,
            thick,
            smooth,
            samples=500,
            domain=0:10,
            ]
            {.05*x};
            \coordinate (a3) at (7.999999999999997, 0.3999999999999999);
            \draw [dashed, red]
            (a3 -| 0,0)      node [left] { \tiny $I^{\star}$ }
            -- (a3)          node [circle,fill,inner sep=1pt] {}
            -- (a3 |- 0,0)   node [below] { \tiny $K^{\star}$ };
            \node at (4.4,0.70) {\color{red}\small $\mathrm d \delta <0$};
          }
        \end{axis}
      \end{tikzpicture}
    \end{center}

  \end{itemize}

\end{frame}


\begin{frame}
  \frametitle{L'équation fondamentale du modèle de Solow}
  \framesubtitle{Démographie, population en France (log)}

  \begin{center}
    \begin{tikzpicture}

  \begin{axis}[/pgf/number format/1000 sep={},scale=1, enlargelimits=false, color=blue!30!black]
    \addplot[style={black,mark=none}] table[x index=0, y index=1, col sep=space]{../data/fra_logged_population.dat};
  \end{axis}

\end{tikzpicture}
  \end{center}

\end{frame}


\begin{frame}
  \frametitle{L'équation fondamentale du modèle de Solow}
  \framesubtitle{Démographie, taux de croissance de la population en France (\%)}

  \begin{center}
    \begin{tikzpicture}

  \begin{axis}[/pgf/number format/1000 sep={},scale=1, enlargelimits=false, color=blue!30!black]
    \addplot[style={black,mark=none}] table[x index=0, y index=1, col sep=space]{../data/fra_population_growth.dat};
    \addplot[mark=none, red, domain=1821:2009] {0.382753845354837};
  \end{axis}

\end{tikzpicture}
  \end{center}

\end{frame}


\begin{frame}
  \frametitle{L'équation fondamentale du modèle de Solow}
  \framesubtitle{Démographie}

  \begin{itemize}

  \item On ne cherche pas à expliquer les fluctuations de la population ou la tendance démographique.\newline

  \item On suppose que la population augmente régulièrement de façon exogène, et on note $n>0$ le taux de croissance de la population.\newline

  \item Plus formellement, on a donc~:
    \[
      \frac{\dot L(t)}{L(t)} = n \quad \forall t\in\mathbb R_+
    \]
    et on suppose que la condition initiale est $L(0) = L_0>0$.\newline

  \item En résolvant cette équation différentielle, on obtient la population à l'instant $t$ commune une fonction de la condition initiale et du taux de croissance~:
    \[
      L(t) = L_0e^{nt}
    \]

  \end{itemize}

\end{frame}


\begin{frame}
  \frametitle{L'équation fondamentale du modèle de Solow}
  \framesubtitle{Démographie et dynamique du capital (a)}

  \begin{itemize}

  \item La croissance de la population affecte la dynamique du capital~:
    \[
      \dot K(t) =  \underbrace{s K(t)^{\alpha}L(t)^{1-\alpha}}_{\text{Investissement}} -  \underbrace{\delta K(t)}_{\text{Dépréciation}}
    \]

  \item[$\Rightarrow$] Le lien entre la variation du stock de capital
    ($\dot K(t)$) et le niveau du stock de capital ($K(t)$) n'est plus
    constant dans le temps\ldots\newline

  \item Car la croissance de la population augmente la production et donc l'investissement.\newline

  \item On dit que l'équation différentielle n'est pas autonome, car l'équation fonctionnelle dépend de $t$ directement via la population $L(t)=e^{nt}$.\newline

  \end{itemize}

\end{frame}


\begin{frame}
  \frametitle{L'équation fondamentale du modèle de Solow}
  \framesubtitle{Démographie et dynamique du capital (b)}

  \begin{itemize}

  \item Il n'est plus possible d'étudier simplement la dynamique de $K$, car la courbe d'investissement se déplace de façon continue.\newline

  \end{itemize}

  \begin{center}
    \begin{tikzpicture}[scale=1]
      \begin{axis}[
        title={},
        xlabel= $K$,
        ylabel= {Quantité de bien homogène},
        xticklabels={,,},
        yticklabels={,,},
        enlargelimits=true,
        grid style={dashed, gray!60},
        axis x line = bottom,
        axis y line = left,
        axis line style={thin},
        xmin = 0,
        xmax = 11,
        ymin = 0,
        ymax = 0.8,
        small,
        clip=false,
        ]
        \addplot[
        draw=black,
        thick,
        smooth,
        samples=500,
        domain=0:10,
        ]
        {.2*x^0.333333};
        \node[right] at (10.1, 0.43){{\tiny{\color{blue}$s K^{\alpha}L(t)^{1-\alpha}$}}};
        \addplot[
        draw=black,
        thick,
        smooth,
        samples=2,
        domain=0:10,
        ]
        {.075*x};
        \node[right] at (10.1, 0.75){{\tiny{\color{blue}$\delta K$}}};
        %
        % Steady state
        %
        \coordinate (a) at (4.354648431614539, 0.3265986323710904);
        \draw [dashed]
        (a -| 0,0)      node [left] { \tiny $I^{\star}$ }
        -- (a)          node [circle,fill,inner sep=1pt] {}
        -- (a |- 0,0)   node [below] { \tiny $K^{\star}$ };
        %
        % Growing population (1)
        %
        \only<2->{
          \addplot[
          draw=black!60,
          thick,
          smooth,
          samples=500,
          domain=0:10,
          ]
          {.25*x^0.333333333};
          \coordinate (a1) at (6.085806194501845, 0.45643546458763834);
          \draw [dashed, black!60]
          (a1 -| 0,0) node [left] { \tiny $I^{\star}$ }
          -- (a1) node [circle,fill,inner sep=1pt] {}
          -- (a1 |- 0,0) node [below] { \tiny $K^{\star}$ };
        }
        %
        % Growing population (2)
        %
        \only<3->{
          \addplot[
          draw=black!30,
          thick,
          smooth,
          samples=500,
          domain=0:10,
          ]
          {.3*x^0.333333333};
          \coordinate (a2) at (7.999999999999997, 0.5999999999999998);
          \draw [dashed, black!30]
          (a2 -| 0,0)      node [left] { \tiny $I^{\star}$ }
          -- (a2)          node [circle,fill,inner sep=1pt] {}
          -- (a2 |- 0,0)   node [below] { \tiny $K^{\star}$ };
        }
      \end{axis}
    \end{tikzpicture}
  \end{center}

\end{frame}


\begin{frame}
  \frametitle{L'équation fondamentale du modèle de Solow}
  \framesubtitle{Démographie et dynamique du capital (c)}

  \bigskip

  \begin{itemize}

  \item Tout se passe comme si l'équation différentielle définissant
    la dynamique du stock de capital physique changeait à chaque instant.\newline

  \item Le déplacement de la courbe d'investissement et donc le
    déplacement vers $+\infty$ de l'intersection entre les courbes
    d'investissement et de dépréciation ne nous permet plus de définir
    le comportement du capital à long terme.\newline

  \item On a l'intuition que le stock de capital physique va croître indéfiniment, mais on ne sait pas précisément comment.\newline

  \item Il faut que nous trouvions un moyen de stabiliser la forme de
    la dynamique, en éliminant l'effet de la croissance
    démographique.\newline

  \item C'est implicitement ce que nous avons fait plus tôt en posant
    $L(t)=L_0$ pour tout $t\in\mathbb R_+$, mais cette solution
    extrême n'est pas satisfaisante.

  \end{itemize}

\end{frame}


\begin{frame}
  \frametitle{L'équation fondamentale du modèle de Solow}
  \framesubtitle{Stationnariser la dynamique du capital (a)}

  \bigskip

  \begin{itemize}

  \item Nous contournons le problème en étudiant la dynamique du stock de capital par tête, $k(t)=\nicefrac{K(t)}{L(t)}$, plutôt celle du stock de capital aggrégé.\newline

  \item De façon générale, pour un aggrégat $X(t)$ on définit l'aggrégat par tête avec $x(t) = \nicefrac{X(t)}{L(t)}$.\newline

  \item Si nous parvenons à déterminer les propriétés dynamique de $k(t)$ nous pouvons en déduire celle de $K(t)$ car~:
    \[
      K(t) = L(t)k(t) \quad\Rightarrow\quad g_K(t) = g_k(t) + n
    \]
    Il suffit de rajouter une constante (le taux de croissance démographique) au taux de croissance du stock de capital par tête pour obtenir le taux de croissance du stock de capital aggrégé.

  \end{itemize}

\end{frame}


\begin{frame}
  \frametitle{L'équation fondamentale du modèle de Solow}
  \framesubtitle{Stationnariser la dynamique du capital (b)}

  \bigskip

  \begin{itemize}

  \item Par définition et en utilisant les formules usuelles de dérivation, la variation du stock de capital par tête est donnée par~:
    \[
      \begin{split}
        \dot k(t) &= \frac{\mathrm d}{\mathrm dt}\left(\frac{K(t)}{L(t)}\right)\\
        &= \frac{\dot K(t)L(t)-K(t)\dot L(t)}{L(t)^2}\\
        &= \frac{\dot K(t)}{L(t)}-\frac{K(t)}{L(t)}\frac{\dot L(t)}{L(t)}
      \end{split}
    \]
    \[
      \Leftrightarrow\dot k(t) = \frac{\dot K(t)}{L(t)} - nk(t)
    \]

    \bigskip

  \item Sur le membre de droite il nous reste le stock de capital aggrégé rapporté à la population, que nous devons éliminer pour obtenir une expression de $\dot k(t)$ en fonction de $k(t)$.

  \end{itemize}

\end{frame}


\begin{frame}
  \frametitle{L'équation fondamentale du modèle de Solow}
  \framesubtitle{Stationnariser la dynamique du capital (c)}

  \bigskip

  \begin{itemize}

  \item En substituant la loi d'évolution du stock de capital aggrégé, il vient~:
    \[
      \dot k(t) = \frac{s K(t)^{\alpha}L(t)^{1-\alpha}-\delta K(t)}{L(t)} - nk(t)
    \]

    \bigskip

  \item En répartissant le dénominateur, $L(t)$, sous les deux puissances (on exploite les rendements d'échelle constants)~:
    \[
      \dot k(t) = s \left(\frac{K(t)}{L(t)}\right)^{\alpha}\left(\frac{L(t)}{L(t)}\right)^{1-\alpha} - (n+\delta)k(t)
    \]

    \bigskip

  \item Finalement~:
    \[
      \dot k(t) = s k(t)^{\alpha} - (n+\delta)k(t)
    \]

    \medskip

    Une équation différentielle non linéaire d'ordre 1 \textbf{autonome}, la variation du stock de capital par tête ne dépend que de son niveau.\newline
  \end{itemize}

\end{frame}


\begin{frame}
  \frametitle{L'équation fondamentale du modèle de Solow}
  \framesubtitle{Dynamique du capital par tête (a)}

  \bigskip

  \begin{block}{Accumulation du stock de capital physique par tête}
    \[
      \dot k(t) =  s k(t)^{\alpha} -  (n+\delta) k(t)
    \]
  \end{block}

  \bigskip

  \begin{itemize}

  \item $s k(t)^{\alpha}$ est l'investissement par tête (le taux d'épargne fois la production par tête),\newline

  \item $(n+\delta)k(t)$ est la dépréciation du capital par tête. Le terme $(n+\delta)$ s'interprète comme le taux de dépréciation du capital par tête. Celui-ci se déprécie pour deux raisons~:\newline

    \begin{itemize}
    \item[($\delta$)] parce que les machines s'usent, et
    \item[($n$)] parce que le nombre de travailleurs par machine augmente.
    \end{itemize}

    \bigskip

  \item Le stock de capital par tête s'accroît ssi l'investissement
    est supérieur à la dépréciation du stock de capital par tête.
  \end{itemize}

\end{frame}


\begin{frame}
  \frametitle{L'équation fondamentale du modèle de Solow}
  \framesubtitle{Dynamique du capital par tête (b)}

  \begin{center}
    \begin{tikzpicture}[scale=1.3]
      \begin{axis}[
        title={},
        xlabel= $k$,
        ylabel= {},
        xticklabels={,,},
        yticklabels={,,},
        enlargelimits=true,
        grid style={dashed, gray!60},
        axis x line = bottom,
        axis y line = left,
        axis line style={thin},
        xmin = 0,
        xmax = 11,
        ymin = 0,
        ymax = 0.8,
        small,
        clip=false,
        ]
        \addplot[
        draw=black,
        thick,
        smooth,
        samples=500,
        domain=0:10,
        ]
        {.2*x^0.333333};
        \node[right] at (10.1, 0.43){{\tiny{\color{blue}$s k^{\alpha}$}}};
        \addplot[
        draw=black,
        thick,
        smooth,
        samples=2,
        domain=0:10,
        ]
        {.075*x};
        \node[right] at (10.1, 0.75){{\tiny{\color{blue}$(n+\delta) k$}}};
        \node at (0.5,0) {\tiny $\blacktriangleright$};
        \node at (1.5,0) {\tiny $\blacktriangleright$};
        \node at (2.5,0) {\tiny $\blacktriangleright$};
        \node at (3.5,0) {\tiny $\blacktriangleright$};
        \node at (5.5,0) {\tiny $\blacktriangleleft$};
        \node at (6.5,0) {\tiny $\blacktriangleleft$};
        \node at (7.5,0) {\tiny $\blacktriangleleft$};
        \node at (8.5,0) {\tiny $\blacktriangleleft$};
        \node at (9.5,0) {\tiny $\blacktriangleleft$};
        %
        % Steady state
        %
        \addplot[draw=red, thin] coordinates {(4.35464843161454, 0) (4.35464843161454,0.3265986323710905)};
        \addplot[draw=red, thin] coordinates {(0, 0.3265986323710905) (4.35464843161454,0.3265986323710905)};
        \node[draw=red,circle, fill=red, scale=.3] at (4.35464843161454,0.3265986323710905) {};
        \node[below] at (4.35464843161454,0) {\tiny{\color{red}$k^{\star}$}};
        \node[left] at (0,0.3265986323710905) {\tiny{\color{red}$i^{\star}=(n+\delta)k^{\star}$}};
      \end{axis}
    \end{tikzpicture}
  \end{center}

\end{frame}


\begin{frame}
  \frametitle{L'équation fondamentale du modèle de Solow}
  \framesubtitle{État stationnaire (a)}

  \bigskip

  \begin{definition}
    L'état stationnaire du modèle de Solow est le niveau de capital par tête tel que le stock de capital par tête est constant.
  \end{definition}

  \bigskip

  \begin{itemize}

  \item Quand le stock de capital par tête est constant toutes les variables par tête sont constantes, car~:
    $y = k^{\alpha}$, $c = (1-s)k^{\alpha}$ et $i = sk^{\alpha}$.\newline

  \item D'après la loi d'évolution du stock de capital par tête, à l'état stationnaire l'investissement par tête doit être égal à la dépréciation du capital par tête.\newline

  \item Dans le modèle de Solow il existe deux états stationnaires. Le premier est dit « trivial »~: quand le stock de capital par tête est nul, la production et donc l'investissement par tête sont nuls comme la dépréciation. Dans cet état stationnaire toutes les variables par tête sont nulles\ldots\newline

  \end{itemize}

\end{frame}


\begin{frame}
  \frametitle{L'équation fondamentale du modèle de Solow}
  \framesubtitle{État stationnaire (b)}

  \bigskip

  \begin{itemize}

  \item Cet état stationnaire ne semble donc pas très intéressant
    (l'économie n'existe pas), c'est pourquoi on ne le considère pas habituellement.\newline

  \end{itemize}

  \begin{columns}
    \begin{column}{0.5\textwidth}
      \begin{itemize}
      \item Une autre raison d'évacuer l'état stationnaire trivial
        est qu'il est instable.
      \item Si une économie en $k=0$ est perturbée (une quantité
        arbitrairement petite de capital tombe du ciel)\ldots
      \item Elle s'éloigne définitivement de $k=0$.
      \end{itemize}
    \end{column}
    \begin{column}{0.5\textwidth}
      \begin{tikzpicture}[scale=.85]
        \begin{axis}[
          title={},
          xlabel= $k$,
          ylabel= {},
          xticklabels={,,},
          yticklabels={,,},
          enlargelimits=true,
          grid style={dashed, gray!60},
          axis x line = bottom,
          axis y line = left,
          axis line style={thin},
          xmin = 0,
          xmax = 11,
          ymin = 0,
          ymax = 0.8,
          small,
          clip=false,
          ]
          \addplot[
          draw=black,
          thick,
          smooth,
          samples=500,
          domain=0:10,
          ]
          {.2*x^0.333333};
          \node[right] at (10.1, 0.43){{\tiny{\color{blue}$s k^{\alpha}$}}};
          \addplot[
          draw=black,
          thick,
          smooth,
          samples=2,
          domain=0:10,
          ]
          {.075*x};
          \node[right] at (10.1, 0.75){{\tiny{\color{blue}$(n+\delta) k$}}};
          \node at (0.5,0) {\tiny {\color{red}$\blacktriangleright$}};
          \node at (1.5,0) {\tiny {\color{red}$\blacktriangleright$}};
          \node at (2.5,0) {\tiny {\color{red}$\blacktriangleright$}};
          \node at (3.5,0) {\tiny {\color{red}$\blacktriangleright$}};
          %
          % Steady state
          %
          \addplot[draw=black!60, thin, dashed] coordinates {(4.35464843161454, 0) (4.35464843161454,0.3265986323710905)};
          \addplot[draw=black!60, thin, dashed] coordinates {(0, 0.3265986323710905) (4.35464843161454,0.3265986323710905)};
          \node[draw=black,circle, fill=black, scale=.3] at (4.35464843161454,0.3265986323710905) {};
          \node[draw=red,circle, fill=red, scale=.3] at (0,0) {};
          \node[below] at (4.35464843161454,0) {\tiny{$k^{\star}$}};
          \node[left] at (0,0.3265986323710905) {\tiny{$i^{\star}$}};

        \end{axis}
      \end{tikzpicture}
    \end{column}
  \end{columns}

\end{frame}


\begin{frame}
  \frametitle{L'équation fondamentale du modèle de Solow}
  \framesubtitle{État stationnaire (c)}

  \bigskip

  \begin{itemize}

  \item L'état stationnaire non trivial $k^{\star}>0$ doit satisfaire~:
    \[
      s \left. k^{\star} \right.^{\alpha} = (n+\delta) k^{\star}
    \]
    car la variation de $k$ est nulle ssi l'investissement par tête
    est égal à la dépréciation du capital par tête.\newline

  \item En divisant les deux membres par $k^{\star}$ (que nous avons supposé strictement positif) et le taux d'épargne, il vient~:
    \[
      \left. k^{\star} \right.^{\alpha-1} = \frac{n+\delta}{s}
    \]
    la productivité moyenne du capital à l'état stationnaire.\newline

  \item En inversant et prenant la racine $1-\alpha$, on obtient finalement~:
    \[
      k^{\star} = \left(\frac{s}{n+\delta}\right)^{\frac{1}{1-\alpha}}
    \]
  \end{itemize}

\end{frame}


\begin{frame}
  \frametitle{L'équation fondamentale du modèle de Solow}
  \framesubtitle{État stationnaire (c)}

  \bigskip

  \begin{itemize}

  \item L'état stationnaire non trivial $k^{\star}>0$ doit satisfaire~:
    \[
      s \left. k^{\star} \right.^{\alpha} = (n+\delta) k^{\star}
    \]
    car la variation de $k$ est nulle ssi l'investissement par tête
    est égal à la dépréciation du capital par tête.\newline

  \item En divisant les deux membres par $k^{\star}$ (que nous avons supposé strictement positif) et le taux d'épargne, il vient~:
    \[
      \left. k^{\star} \right.^{\alpha-1} = \frac{n+\delta}{s}
    \]
    la productivité moyenne du capital à l'état stationnaire.\newline

  \item En inversant et prenant la racine $1-\alpha$, on obtient finalement~:
    \[
      k^{\star} = \left(\frac{s}{n+\delta}\right)^{\frac{1}{1-\alpha}}
    \]
  \end{itemize}

\end{frame}


\begin{frame}
  \frametitle{L'équation fondamentale du modèle de Solow}
  \framesubtitle{État stationnaire (d)}

  \bigskip

  \begin{itemize}

  \item Comme dans le cas avec population constante, l'état
    stationnaire (qui est aussi le niveau de long terme) est une
    fonction croissante du taux d'épargne ($s$), décroissante du taux de
    dépréciation du capital par tête ($n+\delta$), et croissante de l'élasticité
    $\alpha$.\newline

    \begin{center}
      \begin{tikzpicture}[scale=1]
        \begin{axis}[
          title={},
          xlabel= $k$,
          ylabel= {Quantité de bien par tête},
          xticklabels={,,},
          yticklabels={,,},
          enlargelimits=true,
          grid style={dashed, gray!60},
          axis x line = bottom,
          axis y line = left,
          axis line style={thin},
          xmin = 0,
          xmax = 11,
          ymin = 0,
          ymax = 0.8,
          small,
          clip=false,
          ]
          \addplot[
          draw=black,
          thick,
          smooth,
          samples=500,
          domain=0:10,
          ]
          {.2*x^0.333333};
          \node[right] at (10.1, 0.43){{\tiny{\color{blue}$s k^{\alpha}$}}};
          \addplot[
          draw=black,
          thick,
          smooth,
          samples=2,
          domain=0:10,
          ]
          {.075*x};
          \node[right] at (10.1, 0.75){{\tiny{\color{blue}$(n+\delta) k$}}};
          %
          % Steady state
          %
          \coordinate (a) at (4.354648431614539, 0.3265986323710904);
          \draw [dashed]
          (a -| 0,0)      node [left] { \tiny $i^{\star}$ }
          -- (a)          node [circle,fill,inner sep=1pt] {}
          -- (a |- 0,0)   node [below] { \tiny $k^{\star}$ };
          %
          % Increase in the elasticity of production w.r.t. capital
          %
          \only<2>{
            \addplot[
            draw=red,
            thick,
            smooth,
            samples=500,
            domain=0:10,
            ]
            {.2*x^0.5};
            \coordinate (a1) at (7.1111111111111125, 0.5333333333333334);
            \draw [dashed, red]
            (a1 -| 0,0)      node [left] { \tiny $i^{\star}$ }
            -- (a1)          node [circle,fill,inner sep=1pt] {}
            -- (a1 |- 0,0)   node [below] { \tiny $k^{\star}$ };
            \node at (4.4,0.70) {\color{red}\small $\mathrm d \alpha >0$};
          }
          %
          % Increase in the saving rate
          %
          \only<4>{
            \addplot[
            draw=red,
            thick,
            smooth,
            samples=500,
            domain=0:10,
            ]
            {.3*x^0.333333333};
            \coordinate (a2) at (7.999999999999997, 0.5999999999999998);
            \draw [dashed, red]
            (a2 -| 0,0)      node [left] { \tiny $i^{\star}$ }
            -- (a2)          node [circle,fill,inner sep=1pt] {}
            -- (a2 |- 0,0)   node [below] { \tiny $k^{\star}$ };
            \node at (4.4,0.70) {\color{red}\small $\mathrm d s >0$};
          }
          %
          % Decrease in the depreciation rate
          %
          \only<6>{
            \addplot[
            draw=red,
            thick,
            smooth,
            samples=500,
            domain=0:10,
            ]
            {.05*x};
            \coordinate (a3) at (7.999999999999997, 0.3999999999999999);
            \draw [dashed, red]
            (a3 -| 0,0)      node [left] { \tiny $i^{\star}$ }
            -- (a3)          node [circle,fill,inner sep=1pt] {}
            -- (a3 |- 0,0)   node [below] { \tiny $k^{\star}$ };
            \node at (4.4,0.70) {\color{red}\small $\mathrm d (n+\delta) <0$};
          }
        \end{axis}
      \end{tikzpicture}
    \end{center}

  \end{itemize}

\end{frame}


\begin{frame}
  \frametitle{L'équation fondamentale du modèle de Solow}
  \framesubtitle{État stationnaire (e)}

  \bigskip

  En substituant $k^{\star}$ dans la fonction de production, on déduit l'état stationnaire des autres variables par tête du modèle.

  \bigskip

  \begin{block}{État stationnaire du modèle de Solow}
    À l'état stationnaire le capital par tête, la production par tête, l'investissement par tête et la consommation par tête sont donnés par~:
    \begin{eqnarray*}
      k^{\star} = \left(\frac{s}{n+\delta}\right)^{\frac{1}{1-\alpha}}  &\quad y^{\star} = \left(\frac{s}{n+\delta}\right)^{\frac{\alpha}{1-\alpha}}\\
      i^{\star} =  s\left(\frac{s}{n+\delta}\right)^{\frac{\alpha}{1-\alpha}}  &\quad c^{\star} = (1-s)\left(\frac{s}{n+\delta}\right)^{\frac{\alpha}{1-\alpha}}
    \end{eqnarray*}
  \end{block}

\end{frame}


\begin{frame}
  \frametitle{L'équation fondamentale du modèle de Solow}
  \framesubtitle{État stationnaire (f)}


  \begin{center}
    \begin{tikzpicture}[scale=1.3]
      \begin{axis}[
        title={},
        xlabel= $k$,
        ylabel= {},
        xticklabels={,,},
        yticklabels={,,},
        enlargelimits=true,
        grid style={dashed, gray!60},
        axis x line = bottom,
        axis y line = left,
        axis line style={thin},
        xmin = 0,
        xmax = 11,
        ymin = 0,
        ymax = 0.8,
        small,
        clip=false,
        ]
        %
        % Investment
        %
        \addplot[
        draw=black,
        thick,
        smooth,
        samples=500,
        domain=0:10,
        ]
        {.2*x^0.333333};
        \node[right] at (10.1, 0.43){{\tiny{\color{blue}$s k^{\alpha}$}}};
        %
        % Depreciation
        %
        \addplot[
        draw=black,
        thick,
        smooth,
        samples=2,
        domain=0:10,
        ]
        {.075*x};
        \node[right] at (10.1, 0.75){{\tiny{\color{blue}$(n+\delta) k$}}};
        %
        % Output
        %
        \addplot[
        draw=black,
        thick,
        smooth,
        samples=500,
        domain=0:10,
        ]
        {.3*x^0.333333};
        \node[right] at (10.1, 0.65){{\tiny{\color{blue}$k^{\alpha}$}}};
        %
        % Steady state (k,i)
        %
        \coordinate (a0) at (4.354648431614539, 0.3265986323710904);
        \draw [dashed]
        (a0 -| 0,0)      node [left] { \tiny $i^{\star}$ }
        -- (a0)          node [circle,fill,inner sep=1pt] {}
        -- (a0 |- 0,0)   node [below] { \tiny $k^{\star}$ };
        %
        % Steady state (k,y)
        %
        \coordinate (a1) at (4.354648431614539, 0.48989770830357504);
        \draw [dashed]
        (a1 -| 0,0)      node [left] { \tiny $y^{\star}$ }
        -- (a1)          node [circle,fill,inner sep=1pt] {}
        -- (a1 |- 0,0);
        %
        % Steady state (c)
        %
        \addplot[draw=black!80, thin, <->] coordinates {
          (-1.5, 0.3265986323710904)
          (-1.5, 0.48989770830357504)};
        \node[right] at (-2.5, 0.4){{\tiny{$c^{\star}$}}};
        \addplot[draw=black!40, thin, dotted] coordinates {
          (-1.5, 0.3265986323710904)
          (0, 0.3265986323710904)};
        \addplot[draw=black!40, thin, dotted] coordinates {
          (-1.5, 0.48989770830357504)
          (0, 0.48989770830357504)};
      \end{axis}
    \end{tikzpicture}
  \end{center}

\end{frame}


\begin{frame}
  \frametitle{L'équation fondamentale du modèle de Solow}
  \framesubtitle{État stationnaire (e)}

  \bigskip

  \begin{itemize}

  \item Il existe un unique état stationnaire strictement positif.\newline

  \item Cet état stationnaire est globalement stable. Pour toute
    condition initiale (sauf $y(0)=0$), l'économie se rapproche de cet
    état stationnaire. En particulier~:
    \[
      \lim_{t\rightarrow\infty} y(t) = y^{\star}\quad\forall y(0)
    \]

    \bigskip

  \item L'histoire n'a pas d'importance à long terme, que l'on soit
    initialement riche ou pauvre on rejoint le même état
    stationnaire.\newline

  \item Épuisement de la croissance  à long terme (pourquoi est-ce un problème~?)

  \end{itemize}
\end{frame}


\begin{frame}
  \frametitle{Statique comparative}
  \framesubtitle{Augmentation permanente du taux de croissance démographique}

  \bigskip

  \begin{block}{}
    Si $n$ (ou $\delta$) augmente de façon permanente, alors le niveau de long terme de l'économie ($k^{\star}$, $y^{\star}$, $i^{\star}$ ou $c^{\star}$) baisse.
  \end{block}

  \bigskip

  \begin{itemize}

  \item Relativement intuitif~: s'il y a plus de têtes le niveau des variables par tête diminue (ne pas oublier que le rendement marginal du travail est décroissant).\newline

  \item Pour la production par tête, on a~:
    \[
      \frac{\mathrm dy^{\star}}{\mathrm d n} = -\frac{\alpha}{1-\alpha}\frac{s}{(n+\delta)^2}\left(\frac{s}{n+\delta}\right)^{\frac{\alpha}{1-\alpha}-1}<0
    \]
    car le rendement marginal du capital est décroissant ($\alpha<1$).\newline

  \item On peut aussi le voir graphiquement.
  \end{itemize}

\end{frame}


\begin{frame}
  \frametitle{Statique comparative}
  \framesubtitle{Augmentation permanente de l'élasticité de la production par rapport au capital}

  \bigskip

  \begin{block}{}
    Si $\alpha$ augmente de façon permanente, et si $s>n+\delta$, alors le niveau de long terme de l'économie ($k^{\star}$, $y^{\star}$, $i^{\star}$ ou $c^{\star}$) s'accroît.
  \end{block}

  \bigskip

  \begin{itemize}

  \item Pour la production par tête on montre que~:
    \[
      \frac{\mathrm d y^{\star}}{\mathrm d \alpha} = \frac{1}{(1-\alpha)^2}\log\left(\frac{s}{n+\delta}\right)\left(\frac{s}{n+\delta}\right)^{\frac{\alpha}{1-\alpha}}
    \]
    Cette dérivée est positive si et seulement si $s>n+\delta$.\newline

  \item On peut aussi le voir graphiquement.
  \end{itemize}

\end{frame}


\begin{notes}
  \begin{itemize}

  \item Pour le calcul de la dérivée de la production par tête à l'état stationnaire par rapport à l'élasticité de la production par rapport au capital, notons d'abord que~:
    \[
      \frac{\mathrm d}{\mathrm d\alpha} \frac{\alpha}{1-\alpha} = \frac{1}{(1-\alpha)^2}
    \]
    Ensuite on reconnaît que nous devons dériver une fonction de la forme~:
    \[
      f(x) = a^{u(x)}
    \]
    par rapport à $x$, où $a$ est une constante réelle positive. Pour ce faire on réécrit la fonction sous la forme équivalente suivante~:
    \[
      f(x) = e^{u(x)\log a}
    \]
    En utilisant la dérivée de la fonction exponentielle (et la dérivation en chaîne), il vient après simplification~:
    \[
      f'(x)= u'(x)\log (a) a^{u(x)}
    \]

    \bigskip

  \item Le signe de la dérivée de $y^{\star}$ par rapport à $\alpha$
    dépend d'une condition de sur le taux d'épargne ($s$) et le taux
    de dépréciation du capital par tête ($n+\delta$). La condition est
    généralement satisfaite dans les données. Avec les Penn World
    Tables, on peut trouver des séries pour le taux de dépréciation,
    calculer le taux de croissance de la population et approximer le
    taux d'épargne avec le ratio investissement sur PIB. Une aproche
    plus simple consiste à noter qu'à l'état stationnaire nous
    avons~:
    \[
      \frac{k^{\star}}{y^{\star}} = \frac{s}{n+\delta}
    \]
    l'inverse de la productivité moyenne du capital. Il ne nous
    reste qu'à calculer ce ratio dans les données pour vérifier que
    la condition est généralement satisfaite. Les deux figures
    suivantes représentent une série temporelle de ce ratio en
    France entre 1950 et 2017, puis la moyenne de ce ratio pour une
    large coupe d'économie.
  \end{itemize}

  \begin{center}
    \begin{tikzpicture}[scale=1.3]

  \begin{axis}[
    /pgf/number format/1000 sep={},
    scale=1,
    title= \small{Ratio $\nicefrac{K}{Y}$ en France},
    enlargelimits=false,
    color=blue!30!black
    ]
    \addplot[style={black,mark=none}] table[x index=0, y index=1, col sep=space]{../data/fra_k_over_y_ratio.dat};
  \end{axis}

\end{tikzpicture}
  \end{center}

  \begin{center}
    \begin{tikzpicture}[scale=1.3]

  \begin{axis}[
    enlargelimits=false,
    color=blue!30!black,
    scale=1,
    xmin=6,
    xmax=10.3,
    xlabel style={font=\color{white!15!black}},
    xlabel={Logarithme du PIB par tête en 1960 (US\$ 2011)},
    ymin=0,
    ymax=14,
    title=\small{Ratio $\nicefrac{K}{Y}$ moyen dans le monde (1960-2000)},
    axis background/.style={fill=white}]
    \node[right, align=left]
    at (axis cs:7.956,2.528) {ARG};
    \node[right, align=left]
    at (axis cs:9.563,3.01) {AUS};
    \node[right, align=left]
    at (axis cs:9.103,4.565) {AUT};
    \node[right, align=left]
    at (axis cs:6.517,1.704) {BDI};
    \node[right, align=left]
    at (axis cs:9.116,6.017) {BEL};
    \node[right, align=left]
    at (axis cs:7.388,5.93) {BEN};
    \node[right, align=left]
    at (axis cs:6.585,1.606) {BFA};
    \node[right, align=left]
    at (axis cs:7.265,1.783) {BGD};
    \node[right, align=left]
    at (axis cs:7.479,1.891) {BOL};
    \node[right, align=left]
    at (axis cs:7.656,4.055) {BRA};
    \node[right, align=left]
    at (axis cs:9.025,3.745) {BRB};
    \node[right, align=left]
    at (axis cs:6.003,3.407) {BWA};
    \node[right, align=left]
    at (axis cs:7.243,5.945) {CAF};
    \node[right, align=left]
    at (axis cs:9.5,3.603) {CAN};
    \node[right, align=left]
    at (axis cs:9.908,4.181) {CHE};
    \node[right, align=left]
    at (axis cs:8.463,2.496) {CHL};
    \node[right, align=left]
    at (axis cs:6.922,2.278) {CHN};
    \node[right, align=left]
    at (axis cs:7.377,2.642) {CIV};
    \node[right, align=left]
    at (axis cs:7.261,2.205) {CMR};
    \node[right, align=left]
    at (axis cs:7.904,1.486) {COD};
    \node[right, align=left]
    at (axis cs:7.44,3.517) {COG};
    \node[right, align=left]
    at (axis cs:8.152,3.673) {COL};
    \node[right, align=left]
    at (axis cs:7.523,6.583) {COM};
    \node[right, align=left]
    at (axis cs:7.02,4.542) {CPV};
    \node[right, align=left]
    at (axis cs:8.43,2.372) {CRI};
    \node[right, align=left]
    at (axis cs:8.261,8.595) {CYP};
    \node[right, align=left]
    at (axis cs:9.204,4.248) {DEU};
    \node[right, align=left]
    at (axis cs:9.395,5.052) {DNK};
    \node[right, align=left]
    at (axis cs:7.98,2.572) {DOM};
    \node[right, align=left]
    at (axis cs:9.152,3.592) {DZA};
    \node[right, align=left]
    at (axis cs:8.285,6.94) {ECU};
    \node[right, align=left]
    at (axis cs:6.531,1.036) {EGY};
    \node[right, align=left]
    at (axis cs:8.644,4.128) {ESP};
    \node[right, align=left]
    at (axis cs:6.259,2.492) {ETH};
    \node[right, align=left]
    at (axis cs:9.089,4.548) {FIN};
    \node[right, align=left]
    at (axis cs:7.928,2.318) {FJI};
    \node[right, align=left]
    at (axis cs:9.212,4.395) {FRA};
    \node[right, align=left]
    at (axis cs:8.154,2.1) {GAB};
    \node[right, align=left]
    at (axis cs:9.385,4.493) {GBR};
    \node[right, align=left]
    at (axis cs:8.35,12.621) {GHA};
    \node[right, align=left]
    at (axis cs:8.001,0.854) {GIN};
    \node[right, align=left]
    at (axis cs:7.828,1.558) {GMB};
    \node[right, align=left]
    at (axis cs:6.638,2.19) {GNB};
    \node[right, align=left]
    at (axis cs:6.861,1.237) {GNQ};
    \node[right, align=left]
    at (axis cs:8.507,5.294) {GRC};
    \node[right, align=left]
    at (axis cs:7.691,2.973) {GTM};
    \node[right, align=left]
    at (axis cs:8.434,6.598) {HKG};
    \node[right, align=left]
    at (axis cs:7.608,3.023) {HND};
    \node[right, align=left]
    at (axis cs:7.192,2.497) {HTI};
    \node[right, align=left]
    at (axis cs:7.156,4.029) {IDN};
    \node[right, align=left]
    at (axis cs:6.948,3.254) {IND};
    \node[right, align=left]
    at (axis cs:8.695,4.049) {IRL};
    \node[right, align=left]
    at (axis cs:8.24,2.576) {IRN};
    \node[right, align=left]
    at (axis cs:9.293,4.264) {ISL};
    \node[right, align=left]
    at (axis cs:9.013,4.257) {ISR};
    \node[right, align=left]
    at (axis cs:8.831,4.622) {ITA};
    \node[right, align=left]
    at (axis cs:8.505,7.284) {JAM};
    \node[right, align=left]
    at (axis cs:7.978,2.404) {JOR};
    \node[right, align=left]
    at (axis cs:8.508,3.466) {JPN};
    \node[right, align=left]
    at (axis cs:7.377,2.765) {KEN};
    \node[right, align=left]
    at (axis cs:7.015,2.85) {KOR};
    \node[right, align=left]
    at (axis cs:7.899,3.564) {LKA};
    \node[right, align=left]
    at (axis cs:6.858,4.444) {LSO};
    \node[right, align=left]
    at (axis cs:9.665,8.655) {LUX};
    \node[right, align=left]
    at (axis cs:7.137,4.646) {MAR};
    \node[right, align=left]
    at (axis cs:7.271,2.564) {MDG};
    \node[right, align=left]
    at (axis cs:8.686,3.023) {MEX};
    \node[right, align=left]
    at (axis cs:6.557,1.101) {MLI};
    \node[right, align=left]
    at (axis cs:6.578,2.838) {MLT};
    \node[right, align=left]
    at (axis cs:6.202,1.203) {MOZ};
    \node[right, align=left]
    at (axis cs:7.014,2.032) {MRT};
    \node[right, align=left]
    at (axis cs:8.204,5.635) {MUS};
    \node[right, align=left]
    at (axis cs:6.745,2.701) {MWI};
    \node[right, align=left]
    at (axis cs:7.773,2.834) {MYS};
    \node[right, align=left]
    at (axis cs:8.255,2.8) {NAM};
    \node[right, align=left]
    at (axis cs:7.221,6.798) {NER};
    \node[right, align=left]
    at (axis cs:8.303,3.877) {NGA};
    \node[right, align=left]
    at (axis cs:8.382,3.26) {NIC};
    \node[right, align=left]
    at (axis cs:9.347,4.899) {NLD};
    \node[right, align=left]
    at (axis cs:9.289,3.266) {NOR};
    \node[right, align=left]
    at (axis cs:6.553,1.721) {NPL};
    \node[right, align=left]
    at (axis cs:9.46,2.458) {NZL};
    \node[right, align=left]
    at (axis cs:7.093,2.395) {PAK};
    \node[right, align=left]
    at (axis cs:7.881,2.404) {PAN};
    \node[right, align=left]
    at (axis cs:7.887,2.435) {PER};
    \node[right, align=left]
    at (axis cs:7.525,3.437) {PHL};
    \node[right, align=left]
    at (axis cs:8.314,5.843) {PRT};
    \node[right, align=left]
    at (axis cs:7.485,2.908) {PRY};
    \node[right, align=left]
    at (axis cs:7.218,2.858) {ROU};
    \node[right, align=left]
    at (axis cs:6.869,0.95) {RWA};
    \node[right, align=left]
    at (axis cs:7.87,4.184) {SEN};
    \node[right, align=left]
    at (axis cs:7.88,5.352) {SGP};
    \node[right, align=left]
    at (axis cs:7.495,1.816) {SLV};
    \node[right, align=left]
    at (axis cs:9.373,5.341) {SWE};
    \node[right, align=left]
    at (axis cs:8.626,4.432) {SYC};
    \node[right, align=left]
    at (axis cs:7.779,2.844) {SYR};
    \node[right, align=left]
    at (axis cs:7.181,1.106) {TCD};
    \node[right, align=left]
    at (axis cs:7.017,5.147) {TGO};
    \node[right, align=left]
    at (axis cs:6.979,5.014) {THA};
    \node[right, align=left]
    at (axis cs:9.153,3.994) {TTO};
    \node[right, align=left]
    at (axis cs:7.422,6.049) {TUN};
    \node[right, align=left]
    at (axis cs:8.454,2.467) {TUR};
    \node[right, align=left]
    at (axis cs:7.774,3.222) {TWN};
    \node[right, align=left]
    at (axis cs:7.068,3.181) {TZA};
    \node[right, align=left]
    at (axis cs:6.671,2.454) {UGA};
    \node[right, align=left]
    at (axis cs:8.815,4.094) {URY};
    \node[right, align=left]
    at (axis cs:9.768,3.591) {USA};
    \node[right, align=left]
    at (axis cs:8.817,4.462) {VEN};
    \node[right, align=left]
    at (axis cs:8.685,3.76) {ZAF};
    \node[right, align=left]
    at (axis cs:7.418,15.329) {ZMB};
    \node[right, align=left]
    at (axis cs:7.596,9.03) {ZWE};
    \addplot[draw=red, thin, dashed] coordinates {
      (6, 1)
      (10.3, 1)};
  \end{axis}
\end{tikzpicture}
  \end{center}

\end{notes}


\begin{frame}
  \frametitle{Statique comparative}
  \framesubtitle{Augmentation permanente du taux d'épargne (a)}

  \bigskip

  \begin{block}{}
    Si $s$ augmente de façon permanente, alors $k^{\star}$, $y^{\star}$ et $i^{\star}$ augmentent, mais l'effet sur la consommation par tête à long terme est ambigüe.
  \end{block}

  \bigskip

  \begin{itemize}

  \item Pour la production par tête on montre que~:
    \[
      \frac{\mathrm d y^{\star}}{\mathrm d s} = \frac{\alpha}{1-\alpha}\frac{1}{n+\delta}\left(\frac{s}{n+\delta}\right)^{\frac{\alpha}{1-\alpha}-1}>0
    \]
    car le rendement marginal du capital est décroissant ($\alpha<1$).\newline

  \item Pour l'investissement par tête, on a~:
    \[
      \frac{\mathrm d i^{\star}}{\mathrm d s} = y^{\star} + s\frac{\mathrm d y^{\star}}{\mathrm d s}>0
    \]

  \item On peut aussi le voir graphiquement.
  \end{itemize}

\end{frame}


\begin{frame}
  \frametitle{Statique comparative}
  \framesubtitle{Augmentation permanente du taux d'épargne (b)}

  \bigskip

  \begin{itemize}

  \item Pour la consommation par tête c'est un peu plus compliqué\ldots Rappelons que~:
    \[
      c^{\star} = (1-s) y^{\star}
    \]

    \medskip

  \item Une augmentation de $s$ augmente $y^{\star}$, comme nous venons de le montrer, mais aussi diminue $(1-s)$.\newline

  \item Pour augmenter la taille du gâteau on sacrifie une fraction du gâteau\ldots L'effet total dépend du rendement du sacrifice. On a~:
    \[
      \frac{\mathrm d c^{\star}}{\mathrm d s} = (1-s)\frac{\mathrm d y^{\star}}{\mathrm d s} - y^{\star}
    \]
    ou encore~:
    \[
      \mathrm d c^{\star} = \underbrace{(1-s)\mathrm d y^{\star}}_{\text{Le gain}} - \underbrace{y^{\star}\mathrm d s}_{\text{Le sacrifice}}
    \]

  \end{itemize}

\end{frame}


\begin{frame}
  \frametitle{Statique comparative}
  \framesubtitle{Augmentation permanente du taux d'épargne (c)}

  \bigskip

  \begin{itemize}

  \item À cause des rendements marginaux décroissants, si l'économie
    est pauvrement dotée en capital, le rendement d'une unité de
    capital supplémentaire de capital, c'est à dire du sacrifice de
    consommation, est grand. Si l'économie est richement dotée en
    capital le rendement du même sacrifice est faible.\newline

  \item Comme le stock de capital physique à l'état stationnaire est
    une fonction monotone croissante du taux d'épargne, on peut
    ré-exprimer l'assertion précédente comme~: si le taux d'épargne
    est faible (proche de 0) le rendement du sacrifice de consommation
    est grand, si le taux d'épargne est grand (proche de 1) le
    rendement du même sacrifice est faible.\newline

  \item Intuitivement, on voit que l'effet à long terme du
    augmentation du taux d'épargne dépend du niveau initial du taux
    d'épargne. L'effet sur la consommation par tête sera positif à
    long si et seulement si le taux d'épargne est faible.

  \end{itemize}

\end{frame}


\begin{frame}
  \frametitle{La rêgle d'or}
  \framesubtitle{Motivation}

  \bigskip

  \begin{itemize}

  \item Deux cas polaires~:\newline

    \begin{itemize}
    \item[--] $s=0$ alors $y^{\star}$ et donc $c^{\star}=0$.\newline
    \item[--] $s=1$ alors $y^{\star}$ est maximale mais la part consommée et nulle, donc $c^{\star}=0$.\newline
    \end{itemize}

    \medskip

  \item Perturbation des cas polaires~:\newline

    \begin{itemize}
    \item[--] $s=0+\varepsilon$ alors $y^{\star}$ augmente beaucoup (Inada) mais la part consommée de la production baisse peu $\Rightarrow$ $c^{\star}$ augmente.\newline
    \item[--] $s=1-\varepsilon$ alors $y^{\star}$ baisse peu ($\alpha<1$, Inada) et la part consommée devient positive $\Rightarrow$ $c^{\star}$ augmente.\newline
    \end{itemize}

    \medskip

  \item Entre les deux, pouvons nous trouver un taux d'épargne qui maximise la consommation à long terme~?

  \end{itemize}

\end{frame}


\begin{frame}
  \frametitle{La rêgle d'or}
  \framesubtitle{Stock de capital optimal (a)}

  \bigskip

  \begin{itemize}

  \item Nous avons déjà montré que le niveau du stock de capital par tête à long terme, $k^{\star}$, est une fonction monotone croissante du taux d'épargne $s$.\newline

  \item Ainsi, maximiser le niveau de la consommation par tête à long terme, $c^{\star}$, par rapport à $s$ est équivalent à maximiser $c^{\star}$ par rapport à $k^{\star}$.\newline

  \item Écrivons $c^{\star}$ comme une fonction de $k^{\star}$~:
    \[
      \begin{split}
        c^{\star}(k^{\star}) &= \left(1-s\right)\left.k^{\star}\right.^{\alpha}\\
        &= \left(1-(n+\delta)\left.k^{\star}\right.^{1-\alpha}\right)\left.k^{\star}\right.^{\alpha}\\
        &= \underbrace{\left.k^{\star}\right.^{\alpha}}_{y^{\star}}-\underbrace{(n+\delta)k^{\star}}_{i^{\star}}\\
      \end{split}
    \]

  \end{itemize}

\end{frame}


\begin{frame}
  \frametitle{La rêgle d'or}
  \framesubtitle{Stock de capital optimal (b)}

  \bigskip

  \begin{itemize}

  \item La condition nécessaire d'optimalité est obtenue en annulant $\left.c^{\star}\right.'(k^{\star})$~:
    \[
      \alpha \left. k_{\textrm{or}}^{\star}\right. ^{\alpha-1} = n+\delta
    \]
    \[
      \Leftrightarrow \alpha \left. k_{\textrm{or}}^{\star}\right. ^{\alpha-1}-\delta = n
    \]

    \bigskip

  \item À l'optimum, la productivité marginale du capital nette de la dépréciation (qui doit correspondre à la rémunération du capital dans un environnement parfaitement concurrentiel, $r^{\star}$) doit être égale au taux de croissance de la population.\newline

  \item Si le taux d'intérêt réel est supérieur au taux de croissance de la population, c'est-à-dire si $k^{\star}<k_{\textrm{or}}^{\star}$ on a intérêt à reporter de la consommation vers le futur (augmenter l'épargne).\newline

  \item On vérifie que $\left.c^{\star}\right.''(k^{\star})<0$, nous avons bien un unique optimum.

  \end{itemize}

\end{frame}


\begin{frame}
  \frametitle{La rêgle d'or}
  \framesubtitle{Taux d'épargne optimal}

  \bigskip

  \begin{itemize}

  \item En substituant l'expression de l'état stationnaire du capital par tête, il vient~:
    \[
      \alpha \left(\left(\frac{s_{\mathrm{or}}}{n+\delta}\right)^{\frac{1}{1-\alpha}}\right)^{\alpha-1} = n+\delta
    \]
    à l'optimum.\newline

  \item On a donc~:
    \[
      \alpha \frac{n+\delta}{s_{\mathrm{or}}} = n+\delta
    \]
    et donc~:
    \[
      s_{\mathrm{or}} = \alpha
    \]

    \bigskip

  \item Le taux d'épargne de la règle d'or est égal à la part de la rémunération du capital dans le revenu total.\newline

  \item Si le taux d'épargne effectif est différent de la part de la rémunération du capital, le niveau de la consommation à long terme sera sous optimal.

  \end{itemize}

\end{frame}


\begin{frame}
  \frametitle{La rêgle d'or}
  \framesubtitle{Faut-il suivre les recommendations de la rêgle d'or~? (a)}

  \bigskip

  \begin{itemize}

  \item Si $s\neq s_{\mathrm{or}}$ doit-on inciter les ménages à changer leur comportement d'épargne~?\newline

  \item Sur les considérations de long terme, la réponse est sans aucun doute positive. Mais il faut aussi prendre en compte la transition\ldots\newline

  \item On distingue deux situations~:\newline

    \begin{columns}
      \begin{column}{.5\textwidth}
        \begin{Center}
          \textbf{Sur-accumulation}\\
          $s>s_{\mathrm{or}}$
        \end{Center}
        \begin{itemize}
        \item L'économie épargne trop.\newline
        \item Diminuer le taux d'épargne augmente $c^{\star}$\ldots\newline
        \item Et augmente la consommation à court terme.
        \end{itemize}
      \end{column}
      \begin{column}{.5\textwidth}
        \begin{Center}
          \textbf{Sous-accumulation}\\
          $s<s_{\mathrm{or}}$
        \end{Center}
        \begin{itemize}
        \item L'économie épargne trop peu.\newline
        \item Augmenter le taux d'épargne augmente $c^{\star}$\ldots\newline
        \item Mais diminue la consommation à court terme.
        \end{itemize}
      \end{column}
    \end{columns}

  \end{itemize}

\end{frame}


\begin{frame}
  \frametitle{La rêgle d'or}
  \framesubtitle{Faut-il suivre les recommendations de la rêgle d'or~? (b)}

  \bigskip

  \begin{itemize}

  \item Dans une situation de sous accumulation il y a un arbitrage entre les générations.\newline

  \item La génération présente devrait sacrifier son niveau de consommation pour le bénéfice des générations futures.\newline

  \item La réponse de la génération présente dépendra de sa préférence pour le bien-être des générations futures...\newline

  \item Mais aussi de la durée, le long de la transition, de la période durant laquelle son niveau de consommation sera moindre.\newline

  \item Si l'économie converge très vite vers l'état stationnaire, il n'y a plus d'arbitrage entre les générations.\newline

  \item C'est pourquoi il est important d'avoir les idées plus claires sur la dynamique de transition.

  \end{itemize}

\end{frame}


\begin{frame}
  \frametitle{Croissance à long terme}

  \bigskip

  \begin{itemize}

  \item À long terme les variables par tête sont constantes (on a un état stationnaire). {\color{red}$\neq$ Kaldor}\newline

  \item Mais ce n'est pas le cas des variables aggrégats. La production agrégée à long terme est~:
    \[
      Y^{\star}(t) = y^{\star} L(t)
    \]

    \bigskip

  \item En prenant le logarithme, puis en dérivant par rapport au temps, on obtient le taux de croissance de la production agrégée~:
    \[
      g_{Y^{\star}}(t) = n
    \]

  \item On peut faire la même chose pour les autres aggrégats. À long terme~:
    \[
      g_{Y^{\star}}(t) = g_{K^{\star}}(t) = g_{C^{\star}}(t) = g_{I^{\star}}(t) = n
    \]
    la croissance des aggrégats est déterminée par la démographie (seule source de croissance à long terme).


  \end{itemize}

\end{frame}


\section{Équation fondamentale avec progès technique}


\begin{frame}
  \frametitle{Progrès technique}
  \framesubtitle{Spécification (a)}

  \bigskip

  \begin{itemize}

  \item Afin de réconcilier le modèle de Solow avec les faits stylisés de Kaldor, on ajoute une source de croissance supplémentaire.\newline

  \item On suppose qu'il existe un progès technique exogène qui améliore continûment l'efficacité du travail.\newline

  \item Notons $A(t)$ l'efficacité du travail à l'instant $t$.\newline

  \item On suppose que $A(t)$ croît au taux constant $x>0$, c'est-à-dire que~:
    \[
      \frac{\dot A(t)}{A(t)} = x\quad \forall t\in \mathbb R_+
    \]

    \bigskip

  \item On a donc la même ED que pour la population, et on peut, étant donnée une condition initial $A(0) = A_0>0$, déterminer le niveau de l'efficacité du travail~:
    \[
      A(t) = A_0e^{xt}
    \]

  \end{itemize}

\end{frame}


\begin{frame}
  \frametitle{Progrès technique}
  \framesubtitle{Spécification (b)}

  \bigskip

  \begin{itemize}

  \item Le progrès technique se manifeste dans la fonction de production~:
    \[
      Y(t) = K(t)^\alpha\left(A(t)L(t)\right)^{1-\alpha}
    \]

    \bigskip

  \item Notons que dans le cas d'une technologie Cobb-Douglas,
    considérer un progrès technique qui améliore l'efficacité du
    travail, l'efficacité du capital, ou l'efficacité globale des
    facteurs, ne change pas grand chose. Il suffit de re-définir $A$
    pour obtenir des résultats équivalents\ldots\newline

  \item Mais pour d'autres fonctions de production, cela peut changer
    radicalement les propriétés de long terme du modèle.\newline

  \end{itemize}

\end{frame}


\begin{frame}
  \frametitle{Progrès technique}
  \framesubtitle{Spécification (c)}

  \bigskip

  \begin{itemize}

  \item Nous pouvons réécrire la production par tête~:
    \[
      \begin{split}
        Y(t) &= K(t)^\alpha\left(A(t)L(t)\right)^{1-\alpha}\\
        \Leftrightarrow \frac{Y(t)}{L(t)} &= \frac{K(t)^\alpha\left(A(t)L(t)\right)^{1-\alpha}}{L(t)}\\
        \Leftrightarrow y(t) &= \left(\frac{K(t)}{L(t)}\right)^\alpha\left(\frac{A(t)L(t)}{L(t)}\right)^{1-\alpha}\\
        \Leftrightarrow y(t) &= k(t)^\alpha A(t)^{1-\alpha}
      \end{split}
    \]

    \bigskip

  \item $A(t)$ joue le même rôle que $L(t)$, et va poser les mêmes problèmes dans l'analyse de la dynamique\ldots

  \end{itemize}

\end{frame}


\begin{frame}
  \frametitle{Progrès technique}
  \framesubtitle{Équation fondamentale du modèle de Solow revisitée (a)}

  \bigskip

  \begin{itemize}

  \item La loi d'évolution du stock de capital physique par tête est donnée par~:
    \[
      \dot k(t) = s k(t)^{\alpha}A(t)^{1-\alpha} - (n+\delta)k(t)
    \]

    \bigskip

  \item Nous sommes à nouveau confrontés à une équation différentielle non autonome.\newline

  \end{itemize}

  \begin{center}
    \begin{tikzpicture}[scale=1]
      \begin{axis}[
        title={},
        xlabel= $k$,
        ylabel= {Quantité de bien par tête},
        xticklabels={,,},
        yticklabels={,,},
        enlargelimits=true,
        grid style={dashed, gray!60},
        axis x line = bottom,
        axis y line = left,
        axis line style={thin},
        xmin = 0,
        xmax = 11,
        ymin = 0,
        ymax = 0.8,
        small,
        clip=false,
        ]
        \addplot[
        draw=black,
        thick,
        smooth,
        samples=500,
        domain=0:10,
        ]
        {.2*x^0.333333};
        \node[right] at (10.1, 0.43){{\tiny{\color{blue}$s k^{\alpha}A(t)^{1-\alpha}$}}};
        \addplot[
        draw=black,
        thick,
        smooth,
        samples=2,
        domain=0:10,
        ]
        {.075*x};
        \node[right] at (10.1, 0.75){{\tiny{\color{blue}$(n+\delta) k$}}};
        %
        % Steady state
        %
        \coordinate (a) at (4.354648431614539, 0.3265986323710904);
        \draw [dashed]
        (a -| 0,0)      node [left] { \tiny $i^{\star}$ }
        -- (a)          node [circle,fill,inner sep=1pt] {}
        -- (a |- 0,0)   node [below] { \tiny $k^{\star}$ };
        %
        % Growing labour efficiency (1)
        %
        \only<2->{
          \addplot[
          draw=black!60,
          thick,
          smooth,
          samples=500,
          domain=0:10,
          ]
          {.25*x^0.333333333};
          \coordinate (a1) at (6.085806194501845, 0.45643546458763834);
          \draw [dashed, black!60]
          (a1 -| 0,0) node [left] { \tiny $i^{\star}$ }
          -- (a1) node [circle,fill,inner sep=1pt] {}
          -- (a1 |- 0,0) node [below] { \tiny $k^{\star}$ };
        }
        %
        % Growing labour efficiency (2)
        %
        \only<3->{
          \addplot[
          draw=black!30,
          thick,
          smooth,
          samples=500,
          domain=0:10,
          ]
          {.3*x^0.333333333};
          \coordinate (a2) at (7.999999999999997, 0.5999999999999998);
          \draw [dashed, black!30]
          (a2 -| 0,0)      node [left] { \tiny $i^{\star}$ }
          -- (a2)          node [circle,fill,inner sep=1pt] {}
          -- (a2 |- 0,0)   node [below] { \tiny $k^{\star}$ };
        }
      \end{axis}
    \end{tikzpicture}
  \end{center}

\end{frame}


\begin{frame}
  \frametitle{Progrès technique}
  \framesubtitle{Équation fondamentale du modèle de Solow revisitée (b)}

  \bigskip

  \begin{itemize}

  \item Afin de travailler sur une équation différentielle autonome, c'est-à-dire avec une ED ou le lien entre variation et niveau est invariant, on « élimine » l'effet du progrès technique en étudiant la dynamique du progrès par tête efficace, plutôt que la dynamique des variables par tête.\newline

  \item On définit le stock de capital par tête efficace comme~:
    \[
      \hat k(t) = \frac{K(t)}{A(t)L(t)}
    \]
    le stock de capital agrégé rapporté aux têtes efficaces.\newline

  \item On définit de la même façon $\hat y(t)$, $\hat c(t)$ et $\hat i(t)$.\newline

  \item On cherche à caractériser la dynamique de $\hat k(t)$ et à montrer qu'elle est caractérisée par une équation différentielle autonome.\newline

  \end{itemize}

\end{frame}


\begin{frame}
  \frametitle{Progrès technique}
  \framesubtitle{Équation fondamentale du modèle de Solow revisitée (c)}

  \bigskip

  Par définition du stock de capital par tête efficace, on a~:\newline

  \[
    \begin{split}
      \dot{\hat k}(t) &= \frac{\mathrm d}{\mathrm dt}\left(\frac{K(t)}{A(t)L(t)}\right)\\
      &= \frac{\dot K(t) A(t)L(t) - K(t)\frac{\mathrm d}{\mathrm dt}A(t)L(t)}{\left(A(t)L(t)\right)^2}\\
      &= \frac{\dot K(t) A(t)L(t) - K(t)\left(\dot A(t)L(t) + A(t)\dot L(t)\right)}{\left(A(t)L(t)\right)^2}\\
      &= \frac{\dot K(t) A(t)L(t) - K(t)\dot A(t)L(t) - K(t)A(t)\dot L(t)}{\left(A(t)L(t)\right)^2}\\
      &= \frac{\dot K(t)}{A(t)L(t)} - \frac{K(t)}{A(t)L(t)} (n+x)
    \end{split}
  \]

\end{frame}


\begin{frame}
  \frametitle{Progrès technique}
  \framesubtitle{Équation fondamentale du modèle de Solow revisitée (c)}

  \bigskip

  \begin{itemize}

  \item La variation du stock de capital par tête efficace obéit donc à~:
    \[
      \dot{\hat k}(t) = \frac{\dot K(t)}{A(t)L(t)} -(n+x)\hat k(t)
    \]

    \bigskip

  \item On élmine la variation du stock de capital agrégé sur le membre de droite en substituant la loi d'évolution de $K$~:
    \[
      \dot{\hat k}(t) = \frac{sK(t)^{\alpha}\left(A(t)L(t)\right)^{1-\alpha}-\delta K(t)}{A(t)L(t)} -(n+x)\hat k(t)
    \]

  \item On caractérise alors facilement la loi d'évolution de $\hat k$ en exploitant les rendements d'échelle constants.\newline
  \end{itemize}

  \begin{block}{Accumulation du stock de capital physique par tête efficace}
    \[
      \dot{\hat{k}}(t) =  s \hat k(t)^{\alpha} -  (n+x+\delta) \hat k(t)
    \]
  \end{block}

\end{frame}


\begin{frame}
  \frametitle{Progrès technique}
  \framesubtitle{Équation fondamentale du modèle de Solow revisitée (d)}

  \bigskip

  \begin{block}{Accumulation du stock de capital physique par tête efficace}
    \[
      \dot{\hat{k}}(t) =  \underbrace{s \hat k(t)^{\alpha}}_{\substack{\text{investissement par}\\\text{tête efficace}}} -  \underbrace{(n+x+\delta) \hat k(t)}_{\substack{\text{dépréciation du capital}\\\text{par tête efficace}}}
    \]
  \end{block}

  \bigskip

  \begin{itemize}

  \item $n+x+\delta$ s'interprète comme le taux de dépréciation du stock de capital par tête efficace.\newline

  \item Le stock de capital par tête efficace se déprécie car~:\newline
    \begin{itemize}
    \item[($\delta$)] Les machines s'usent,
    \item[($n$)] Le nombre de têtes augmente, et
    \item[($x$)] La productivité des têtes augmente.
    \end{itemize}

  \end{itemize}

\end{frame}


\begin{frame}
  \frametitle{Progrès technique}
  \framesubtitle{Équation fondamentale du modèle de Solow revisitée (e)}

  \bigskip

  \begin{center}
    \begin{tikzpicture}[scale=1.3]
      \begin{axis}[
        title={},
        xlabel= $\hat k$,
        ylabel= {},
        xticklabels={,,},
        yticklabels={,,},
        enlargelimits=true,
        grid style={dashed, gray!60},
        axis x line = bottom,
        axis y line = left,
        axis line style={thin},
        xmin = 0,
        xmax = 11,
        ymin = 0,
        ymax = 0.8,
        small,
        clip=false,
        ]
        \addplot[
        draw=black,
        thick,
        smooth,
        samples=500,
        domain=0:10,
        ]
        {.2*x^0.333333};
        \node[right] at (10.1, 0.43){{\tiny{\color{blue}$s \hat k^{\alpha}$}}};
        \addplot[
        draw=black,
        thick,
        smooth,
        samples=2,
        domain=0:10,
        ]
        {.075*x};
        \node[right] at (10.1, 0.75){{\tiny{\color{blue}$(n+x+\delta) \hat k$}}};
        \addplot[
        draw=black,
        thick,
        smooth,
        samples=500,
        domain=0:10,
        ]
        {.3*x^0.333333};
        \node[right] at (10.1, 0.65){{\tiny{\color{blue}$\hat y = \hat k^{\alpha}$}}};
        \node at (0.5,0) {\tiny $\blacktriangleright$};
        \node at (1.5,0) {\tiny $\blacktriangleright$};
        \node at (2.5,0) {\tiny $\blacktriangleright$};
        \node at (3.5,0) {\tiny $\blacktriangleright$};
        \node at (5.5,0) {\tiny $\blacktriangleleft$};
        \node at (6.5,0) {\tiny $\blacktriangleleft$};
        \node at (7.5,0) {\tiny $\blacktriangleleft$};
        \node at (8.5,0) {\tiny $\blacktriangleleft$};
        \node at (9.5,0) {\tiny $\blacktriangleleft$};
        %
        % Steady state
        %
        \addplot[dashed, thin] coordinates {(0, 0.3265986323710905) (4.35464843161454,0.3265986323710905)};
        \node[draw=black,circle, fill=black, scale=.3] at (4.35464843161454,0.3265986323710905) {};
        \node[below] at (4.35464843161454,0) {\tiny{$\hat k^{\star}$}};
        \node[left] at (0,0.3265986323710905) {\tiny{$\hat i^{\star}$}};
        \coordinate (a1) at (4.354648431614539, 0.48989770830357504);
        \draw [dashed]
        (a1 -| 0,0)      node [left] { \tiny $\hat y^{\star}$ }
        -- (a1)          node [circle,fill,inner sep=1pt] {}
        -- (a1 |- 0,0);
        \addplot[draw=black!80, thin, <->] coordinates {
          (-1.5, 0.3265986323710904)
          (-1.5, 0.48989770830357504)};
        \node[right] at (-2.5, 0.4){{\tiny{$\hat c^{\star}$}}};
        \addplot[draw=black!40, thin, dotted] coordinates {
          (-1.5, 0.3265986323710904)
          (0, 0.3265986323710904)};
        \addplot[draw=black!40, thin, dotted] coordinates {
          (-1.5, 0.48989770830357504)
          (0, 0.48989770830357504)};
      \end{axis}
    \end{tikzpicture}
  \end{center}

\end{frame}


\begin{frame}
  \frametitle{Progrès technique}
  \framesubtitle{État stationnaire du modèle de Solow (a)}

  \bigskip

  De la même façon que dans le modèle sans progrès technique (pour les variables par tête), on peut calculer l'état stationnaire des variables par tête efficace~:

  \bigskip

  \begin{block}{État stationnaire du modèle de Solow}
    À l'état stationnaire le capital par tête, la production par tête,
    l'investissement par tête et la consommation par tête sont donnés
    par~:
    \begin{eqnarray*}
      \hat k^{\star} = \left(\frac{s}{n+x+\delta}\right)^{\frac{1}{1-\alpha}}  &\quad \hat y^{\star} = \left(\frac{s}{n+x+\delta}\right)^{\frac{\alpha}{1-\alpha}}\\
      \hat i^{\star} =  s\left(\frac{s}{n+x+\delta}\right)^{\frac{\alpha}{1-\alpha}}  &\quad \hat c^{\star} = (1-s)\left(\frac{s}{n+x+\delta}\right)^{\frac{\alpha}{1-\alpha}}
    \end{eqnarray*}
  \end{block}

\end{frame}


\begin{frame}
  \frametitle{Progrès technique}
  \framesubtitle{État stationnaire du modèle de Solow (b)}

  \bigskip

  On montre aussi, en suivant la même démarche que dans le modèle sans progrès technique, que l'état stationnaire à les mêmes propriétés~:\newline

  \begin{itemize}

  \item $\nearrow n+x+\delta \,\Rightarrow\, \searrow\hat k^{\star},\, \searrow\hat y^{\star},\, \searrow\hat i^{\star},\, \text{ et }\searrow\hat c^{\star}$\newline

  \item $\nearrow \alpha \,\Rightarrow\, \nearrow\hat k^{\star},\, \nearrow\hat y^{\star},\, \nearrow\hat i^{\star},\, \text{ et }\nearrow\hat c^{\star}$ dès lors que $s>n+x+\delta$\newline

  \item $\nearrow s \,\Rightarrow\, \nearrow\hat k^{\star},\, \nearrow\hat y^{\star},\, \nearrow\hat i^{\star},\, \text{ mais l'effet sur } \hat c^{\star}$ est ambigüe\newline

  \item Le taux d'épargne de la rêgle d'or est toujours $s_{or} = \alpha$, mais la condition pour déterminer le stock de capital physique par tête efficace devient $f'(\hat k_{or}) - \delta  = n + x$ (à l'optimum le rendement du capital net de la dépréciation doit être égal au taux de croissance).
  \end{itemize}

\end{frame}


\begin{frame}
  \frametitle{Progrès technique}
  \framesubtitle{État stationnaire du modèle de Solow (c)}

  \bigskip

  \begin{itemize}

  \item À long terme, le long du sentier de croissance équilibrée, le
    taux de croissance des variables par tête est donné par le taux
    de croissance de l'efficacité du travail~:
    \[
      y^{\star}(t) = \hat y^{\star} A(t)\, \Rightarrow\, g_{y^{\star}}(t) = x
    \]
    cela est valable pour toutes les variables par tête.\newline

  \item À long terme, le long du sentier ce croissance équilibrée, le taux de croissance des aggrégats est égal à $n+x$~:
    \[
      Y^{\star}(t) = \hat y^{\star} A(t)L(t)\, \Rightarrow\, g_{Y^{\star}}(t) = n+x
    \]

  \end{itemize}

\end{frame}


\begin{frame}
  \frametitle{Progrès technique}
  \framesubtitle{Dynamique de transition (a)}

  \bigskip

  \begin{itemize}

  \item Caractérisons plus précisement la dynamique de transition en calculant le taux de croissance du stock de capital par tête efficace à l'instant $t$.\newline

  \item On a par définition~:
    \[
      g_{\hat k}(t) = \frac{\dot{\hat{k}}(t)}{\hat{k}(t)}
    \]

    \bigskip

  \item En substituant la loi d'évolution du stock de capital par tête efficace, il vient directement~:
    \[
      g_{\hat k}(t) = s\hat k(t)^{\alpha-1} - (n+x+\delta)
    \]
    Notons que $\hat k^{\alpha-1}$ peut se lire comme $\nicefrac{\hat y}{\hat k}$ la productivité moyenne du capital.

  \end{itemize}

\end{frame}


\begin{frame}
  \frametitle{Progrès technique}
  \framesubtitle{Dynamique de transition (b)}

  \begin{block}{Taux de croissance du capital par tête efficace}
    \[
      g_{\hat k}(t) = \underbrace{\frac{s\hat y}{\hat k}}_{\substack{\text{L'investissement par}\\\text{unité de capital}}} - \underbrace{(n+x+\delta)}_{\substack{\text{Le taux de dépréciation du capital}\\\text{par tête efficace}}}
    \]
  \end{block}

  \begin{itemize}

  \item Le taux de croissance du stock de capital physique par tête efficace est égal à l'investissemenent net (de la dépréciation) par unité de capital.\newline

  \item On peut représenter graphiqement ce taux de croissance en notant que:\newline
    \begin{itemize}

    \item $\nicefrac{\hat y}{\hat k}$ est une fonction monotone décroissante de $\hat k$ ($\alpha<1$),
    \item $\nicefrac{\hat y}{\hat k}$ tend vers $\infty$ quand $\hat k$ tend vers 0 (Inada)
    \item $\nicefrac{\hat y}{\hat k}$ tend vers $0$ quand $\hat k$ tend vers $\infty$ (Inada)
    \end{itemize}

  \end{itemize}

\end{frame}


\begin{frame}
  \frametitle{Progrès technique}
  \framesubtitle{Dynamique de transition (c)}

  \begin{center}
    \begin{tikzpicture}[scale=1.3]
      \begin{axis}[
        title={},
        xlabel= $\hat k$,
        ylabel= {$g_{\hat k}$},
        xticklabels={,,},
        yticklabels={,,},
        enlargelimits=true,
        grid style={dashed, gray!60},
        axis x line = bottom,
        axis y line = left,
        axis lines = middle,
        axis line style={thin},
        xmin = -1,
        xmax = 11,
        ymin = -0.2,
        ymax = 0.8,
        small,
        clip=false,
        ]
        \addplot[
        draw=black,
        thick,
        smooth,
        samples=500,
        domain=.175:10,
        ]
        {.267*x^(0.333333-1)-0.1} ;
        \addplot[
        draw=black,
        dashed,
        samples=2,
        domain=0:10,
        ] coordinates { (0, -.08) (10, -.08)};
        \node[right] at (10.1, -.05) {\tiny \color{blue} $s\hat k^{\alpha-1}-(n+x+\delta)$};
        \node at (0.5,0) {\tiny $\blacktriangleright$};
        \node at (1.5,0) {\tiny $\blacktriangleright$};
        \node at (2.5,0) {\tiny $\blacktriangleright$};
        \node at (3.5,0) {\tiny $\blacktriangleright$};
        \node at (5.5,0) {\tiny $\blacktriangleleft$};
        \node at (6.5,0) {\tiny $\blacktriangleleft$};
        \node at (7.5,0) {\tiny $\blacktriangleleft$};
        \node at (8.5,0) {\tiny $\blacktriangleleft$};
        \node at (9.5,0) {\tiny $\blacktriangleleft$};
        \node[left] at (-.5,-.08) {\tiny $-(n+x+\delta)$};
        %
        % Steady state
        %
        \node[below] at (4.35464843161454,0) {\tiny{\color{red}$\hat k^{\star}$}};
      \end{axis}
    \end{tikzpicture}
  \end{center}


\end{frame}

\end{document}

% Local Variables:
% ispell-check-comments: exclusive
% ispell-local-dictionary: "french"
% TeX-master: t
% End: