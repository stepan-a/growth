\synctex=1

\documentclass[10pt,notheorems]{beamer}

\usepackage{etex}
\usepackage{fourier-orns}
\usepackage{ccicons}
\usepackage{amssymb}
\usepackage{amstext}
\usepackage{amsbsy}
\usepackage{amsopn}
\usepackage{amscd}
\usepackage{amsxtra}
\usepackage{amsthm}
\usepackage{float}
\usepackage{color, colortbl}
\usepackage{mathrsfs}
\usepackage{bm}
\usepackage{lastpage}
\usepackage[nice]{nicefrac}
\usepackage{setspace}
\usepackage{ragged2e}
\usepackage{listings}
\usepackage{algorithms/algorithm}
\usepackage{algorithms/algorithmic}
\usepackage[frenchb]{babel}
\usepackage{tikz,pgfplots,pgfplotstable}
\pgfplotsset{compat=newest}
\usetikzlibrary{patterns, arrows, decorations.pathreplacing, decorations.markings, decorations.text, calc}
\pgfplotsset{plot coordinates/math parser=false}
\newlength\figureheight
\newlength\figurewidth
\usepackage[utf8x]{inputenc}
\usepackage{cancel}
\usepackage{tikz-qtree}
\usepackage{dcolumn}
\usepackage{adjustbox}
\usepackage{environ}
\usepackage[cal=boondox]{mathalfa}
\usepackage{manfnt}
\usepackage{hyperref}
\hypersetup{
  colorlinks=true,
  linkcolor=blue,
  filecolor=black,
  urlcolor=black,
}
\usepackage{venndiagram}
\usepackage{scrextend}

% Git hash
\usepackage{xstring}
\usepackage{catchfile}
\immediate\write18{git rev-parse HEAD > git.hash}
\CatchFileDef{\HEAD}{git.hash}{\endlinechar=-1}
\newcommand{\gitrevision}{\StrLeft{\HEAD}{7}}

\newcommand{\trace}{\mathrm{tr}}
\newcommand{\vect}{\mathrm{vec}}
\newcommand{\tracarg}[1]{\mathrm{tr}\left\{#1\right\}}
\newcommand{\vectarg}[1]{\mathrm{vec}\left(#1\right)}
\newcommand{\vecth}[1]{\mathrm{vech}\left(#1\right)}
\newcommand{\iid}[2]{\mathrm{iid}\left(#1,#2\right)}
\newcommand{\normal}[2]{\mathcal N\left(#1,#2\right)}
\newcommand{\dynare}{\href{http://www.dynare.org}{\color{blue}Dynare}}
\newcommand{\sample}{\mathcal Y_T}
\newcommand{\samplet}[1]{\mathcal Y_{#1}}
\newcommand{\slidetitle}[1]{\fancyhead[L]{\textsc{#1}}}

\newcommand{\R}{{\mathbb R}}
\newcommand{\C}{{\mathbb C}}
\newcommand{\N}{{\mathbb N}}
\newcommand{\Z}{{\mathbb Z}}
\newcommand{\binomial}[2]{\begin{pmatrix} #1 \\ #2 \end{pmatrix}}
\newcommand{\bigO}[1]{\mathcal O \left(#1\right)}
\newcommand{\red}{\color{red}}
\newcommand{\blue}{\color{blue}}

\newcommand*\circled[1]{\tikz[baseline=(char.base)]{
    \node[shape=circle,draw,inner sep=1.0pt] (char) {#1};}}

\renewcommand{\qedsymbol}{C.Q.F.D.}

\newcolumntype{d}{D{.}{.}{-1}}
\definecolor{gray}{gray}{0.9}
\newcolumntype{g}{>{\columncolor{gray}}c}

\setbeamertemplate{theorems}[numbered]

\theoremstyle{plain}
\newtheorem{theorem}{Théorème}

\theoremstyle{definition} % insert bellow all blocks you want in normal text
\newtheorem{definition}{Définition}
\newtheorem{properties}{Propriétés}
\newtheorem{lemma}{Lemme}
\newtheorem{property}[properties]{Propriété}
\newtheorem{example}{Exemple}
\newtheorem*{idea}{Éléments de preuve} % no numbered block

\setbeamertemplate{footline}{
  {\hfill\vspace*{1pt}\href{http://creativecommons.org/licenses/by-sa/3.0/legalcode}{\ccbysa}\hspace{.1cm}
    \raisebox{-.075cm}{\href{https://git.adjemian.eu/University/growth}{\includegraphics[scale=.1]{../img/gitlab.png}}}\enspace
    \href{https://git.adjemian.eu/University/growth/-/blob/\HEAD/cours/solow.tex}{\gitrevision}\enspace\today
  }\hspace{1cm}}

\setbeamertemplate{navigation symbols}{}
\setbeamertemplate{blocks}[rounded][shadow=true]
\setbeamertemplate{caption}[numbered]

\newenvironment{notes}
{\bgroup \justifying\bgroup\tiny\begin{spacing}{1.0}}
  {\end{spacing}\egroup\egroup}

\newenvironment{exercise}[1]
{\bgroup \small\begin{block}{Ex. #1}}
  {\end{block}\egroup}

\newenvironment{defn}[1]
{\bgroup \small\begin{block}{Définition. #1}}
  {\end{block}\egroup}

\newenvironment{exemple}[1]
{\bgroup \small\begin{block}{Exemple. #1}}
  {\end{block}\egroup}


\begin{document}

\title{Croissance\\\small{Le modèle de Solow}}
\author[S. Adjemian]{St\'ephane Adjemian}
\institute{\texttt{stephane.adjemian@univ-lemans.fr}}
\date{Septembre 2020}

\begin{frame}
  \titlepage{}
\end{frame}

\begin{frame}
  \frametitle{Plan}
  \tableofcontents
\end{frame}


\section{Environnement et hypothèses}

\begin{frame}
  \frametitle{Environnement, I}

  \begin{itemize}
  \item Un modèle d'équilibre général.
    \medskip
  \item Les ménages (familles)~:
    \begin{itemize}
    \item[--] possèdent les facteurs de production et les actifs de l'économie,
    \item[--] déterminent la fécondité, la participation au marché du travail \textbf{et} le partage entre consommation présente et future (c'est-à-dire l'épargne).
    \end{itemize}
    \medskip
  \item Les entreprises~:
    \begin{itemize}
    \item[--] louent les facteurs de production (capital et travail) aux ménages,
    \item[--] produisent un bien à partir des services du capital et du travail, vendu aux ménages.
    \end{itemize}
    \medskip
  \item Ménages et firmes se rencontrent sur des marchés~:
    \begin{itemize}
    \item[--] le marché des biens, où les entreprises vendent la production au ménages,
    \item[--] un marché par facteur de production.
    \end{itemize}
    \medskip
  \item \textbf{Simplification:} une robinsonnade\ldots
  \end{itemize}

\end{frame}


\begin{frame}
  \frametitle{Robinson schizophrène}

  \begin{center}
    \includegraphics[scale=.2]{../img/crusoe-1.png}
  \end{center}

\end{frame}


\begin{frame}
  \frametitle{Environnement, II}

  \begin{itemize}

  \item Économie autarcique.

    \begin{itemize}
    \item[$\Rightarrow$] Pas de différence entre revenu et production d'une économie.
    \item[$\Rightarrow$] Pas de différence entre épargne et investissement.
    \end{itemize}

    \bigskip

  \item Hypothèse de concurrence parfaite.\newline

    \begin{itemize}

    \item[$\Rightarrow$] Pas de différence entre l'économie
      centralisée (Robinson) et l'équilibre décentralisé (équilibre
      général avec des marchés),\newline

      \begin{itemize}
      \item[--] On va donc au plus simple\ldots
      \item[--] Mais pas de prix (puisqu'il n'y a pas de marchés).
      \end{itemize}

    \end{itemize}

    \bigskip

  \item Offre de travail inélastique (pas de chômage).\newline

  \item Démographie exogène.\newline

  \item Comportement d'épargne exogène.

  \end{itemize}

\end{frame}


\begin{frame}
  \frametitle{Au c\oe{}ur du modèle de Solow}

  \begin{center}
    \rotatebox{-20}{\includegraphics[scale=.2]{../img/croissant.png}}\rotatebox{-20}{\includegraphics[scale=.15]{../img/croissant.png}}\rotatebox{-20}{\includegraphics[scale=.10]{../img/croissant.png}}\rotatebox{-20}{\includegraphics[scale=.05]{../img/croissant.png}}\quad\rotatebox{-20}{\includegraphics[scale=.01]{../img/croissant.png}}
  \end{center}

\end{frame}


\begin{frame}
  \frametitle{L'hypothèse des rendements (marginaux) décroissants, I}

  \begin{itemize}

  \item Une vieille idée chez les économistes (Turgot, Malthus, Ricardo, \ldots) pas toujours très optimistes.\newline

  \item Supposons que, chaque année, Robinson augmente la surface cultivable sur son île\ldots Sans accroître son effort car il travaille déjà beaucoup.\newline

  \item Même si les terres suplémentaires sont également fertiles, sa production de blé augmentera de moins en moins.\newline

  \item \emph{Ceteris paribus} en augmentant un unique facteur de production (ici la terre) le \textbf{rendement marginal} est décroissant (l'effort de Robinson sur chaque parcelle diminue).\newline

  \item[\dbend] Si Robinson double la surface cultivée \textbf{et} la quantité de travail (en faisant travailler Vendredi) il se peut qu'il parvienne à doubler la production, on parle alors de \textbf{rendements d'échelle} constants.

  \end{itemize}

\end{frame}


\begin{frame}
  \frametitle{L'hypothèse des rendements (marginaux) décroissants, II}

  \medskip

  \begin{itemize}

  \item Cette hypothèse est le c\oe{}ur du modèle de Solow. Sans cette
    hypothèse, qu'il faudra évaluer par rapport aux données, les
    prédictions du modèle changent radicalement.\newline

  \item Elle explique pourquoi une économie croît temporairement plus
    vite après une crise ou une guerre.\newline

  \item Après une guerre, \emph{i.e.} une destruction du capital
    physique, l'investissement dans le capital a un rendement très
    élevé. L'acroissement élevé de production (revenu) permettra
    d'investir demain... Mais au fur et à mesure que le stock de
    capital physique augmente, le rendement de l'investissement
    diminue et les accroissements induits de la production sont de
    plus en plus faibles.\newline

  \item Ce mécanisme conduit à long terme à un épuisement de la croissance.\newline

  \item Pas de croissance à long terme dans ce modèle ($\neq$ Kaldor).\newline

  \end{itemize}

\end{frame}


\begin{frame}
  \frametitle{Progrès technique exogène, I}

  \begin{center}
    \includegraphics[scale=1]{../img/manne-tintoretto.png}
  \end{center}

\end{frame}


\begin{frame}
  \frametitle{Progrès technique exogène, II}

  \begin{itemize}

  \item Pour palier l'épuisement de la croissance, il faut une source d'impulsion externe qui améliore l'efficacité des facteurs de production.\newline

  \item On suppose que le progrès tombe du ciel, il s'agit d'une manne.\newline

  \item Tout le monde bénéficie gratuitement d'une technologie qui s'améliore continuement\ldots\newline

  \item \ldots Et on ne cherche pas à expliquer son origine.

  \end{itemize}
\end{frame}


\section{La fonction de production néoclassique}

\begin{frame}
  \frametitle{Le bien}

  \begin{itemize}

  \item On suppose qu'il existe un unique bien dans l'économie, en quantitée $Y$.\newline

  \item Ce bien homogène est utilisé pour la consommation ($C$) et l'investissement ($I$) dans le capital physique ($K$).\newline

  \item Dans ce chapitre on suppose que la production du bien homogène (en quantitée $Y$) nécessite deux facteurs~: les services du stock de capital physique ($K$) et du travail ($L$).\newline

  \item La technologie de production est caractérisée par une fonction de production, $Y = F(K,L)$, dont les propriétés sont déterminantes dans le modèle de Solow (pour la transition et le long terme).\newline

  \end{itemize}

\end{frame}



\begin{frame}
  \frametitle{La fonction de production néoclassique, I}


  \begin{definition}
    Une fonction de production $F(K,L)$ est dite néoclassique si~:
    \medskip
    \begin{enumerate}

    \item Les productivités marginales sont positives, $F_K(K,L)>0$ et $F_K(K,L)>0$.\newline

    \item Les productivités marginales sont décroissantes, $F_{KK}(K,L)<0$ et $F_{LL}(K,L)<0$.\newline

    \item Les rendements d'échelle sont constants~: $F(\lambda K, \lambda L) = \lambda F(K,L)$ pour tout $\lambda>0$.\newline

    \item Les conditions d'Inada~:
      \[
        \begin{split}
          \lim_{K\rightarrow 0}F_K(K,L) = \infty & \lim_{K\rightarrow \infty}F_K(K,L) = 0 \\
          \lim_{L\rightarrow 0}F_L(K,L) = \infty & \lim_{L\rightarrow \infty}F_L(K,L) = 0
        \end{split}
      \]

    \end{enumerate}
  \end{definition}

\end{frame}


\begin{frame}
  \frametitle{La fonction de production néoclassique, II}
  \framesubtitle{Remarques}

  Une fonction de production néoclassique est donc~:

  \medskip

  \begin{itemize}

  \item Croissante, puisque les productivités marginales sont positives. Quand on augmente la quantité de $K$ ou de $L$ on augmente la production ($Y$).\newline

  \item À rendements décroissants, puisque les productivités marginales sont décroissantes. Plus on augmente la quantité de travail, moins la production augmente.\newline

  \item À rendements d'échelle constants, puisque la fonction deproduction est homogène de degré un. Doubler les quantités de facteurs de production, double la production.\newline

  \end{itemize}

  On verra plus loin que les conditions d'Inada, qui caractérisent le comportement aux bords des productivités marginales, sont essentielles pour définir les propriétés de long terme du modèle.\newline
\end{frame}

\begin{frame}
  \frametitle{La fonction de production néoclassique, III}
  \framesubtitle{Exemple~: La fonction Cobb-Douglas (a)}


  \[
    Y = K^{\alpha}L ^{1-\alpha}
  \]
  avec $0<\alpha<1$.

  \bigskip

  \begin{itemize}

  \item $F_K(K,L) = \alpha K^{\alpha-1}L^{1-\alpha}=\alpha\frac{Y}{K}>0$,
  \item $F_L(K,L) = (1-\alpha) K^{\alpha}L^{-\alpha}=(1-\alpha)\frac{Y}{L}>0$,
  \item $F_{KK}(K,L) = \alpha(\alpha-1)K^{\alpha-2}L^{1-\alpha}=\alpha(\alpha-1)\frac{Y}{K^2}<0$
  \item $F_{LL}(K,L) = -\alpha(1-\alpha)K^{\alpha}L^{-\alpha-1}=-\alpha(1-\alpha)\frac{Y}{L^2}<0$
  \item $\left(\lambda K\right)^{\alpha}\left(\lambda L\right)^{1-\alpha}=\lambda^{\alpha+1-\alpha}K^{\alpha}L ^{1-\alpha} = \lambda K^{\alpha}L ^{1-\alpha}$ $\forall \lambda>0$
  \item En notant que l'on a aussi $F_K(K,L) = \alpha \left(\frac{L}{K}\right)^{1-\alpha}$ et $F_L(K,L) = (1-\alpha) \left(\frac{K}{L}\right)^{\alpha}$, on voit facilement que les conditions d'Inada sont satisfaites.
  \end{itemize}

\end{frame}


\begin{frame}
  \frametitle{La fonction de production néoclassique, III}
  \framesubtitle{Exemple~: La fonction Cobb-Douglas (b)}

  \begin{center}
    \begin{tikzpicture}[scale=1.2]
      \begin{axis}[
        title={$Y = K^{\alpha}L^{1-\alpha}$},
        xlabel=$K$,
        ylabel=$L$,
        xticklabels={,,},
        yticklabels={,,},
        zticklabels={,,},
        3d box=complete,
        enlargelimits=false,
        grid,
        grid style={dashed, gray!60},
        axis line style={thin},
        small,
        ]
        \addplot3[
        mesh,
        draw=black,
        samples=50,
        domain=0:10,
        domain y=0:10,
        ]
        {x^0.333333*y^(1-0.333333)};
      \end{axis}
    \end{tikzpicture}
  \end{center}
\end{frame}

\begin{frame}
  \frametitle{La fonction de production néoclassique, III}
  \framesubtitle{Exemple~: La fonction Cobb-Douglas (c)}

  \begin{itemize}
  \item L'hypothèse de rendements d'échelle constants permet de réécrire la technologie sous forme intensive, c'est à dire en exprimant la production par tête $y=\nicefrac{Y}{L}$ comme une fonction du stock de capital physique par tête $k = \nicefrac{K}{L}$.\newline

  \item On a~:
    \[
      Y = K^{\alpha}L^{1-\alpha}
    \]
    en divisant les deux membres par $L$~:
    \[
      \begin{split}
        \frac{Y}{L} &= \frac{K^{\alpha}L^{1-\alpha}}{L^{\alpha+1-\alpha}}\\
        &= \left(\frac{K}{L}\right)^{\alpha} \left(\frac{L}{L}\right)^{1-\alpha}\\
      \end{split}
    \]
    et donc~:
    \[
      y = k^{\alpha}
    \]
  \end{itemize}

\end{frame}


\begin{frame}
  \frametitle{La fonction de production néoclassique, III}
  \framesubtitle{Exemple~: La fonction Cobb-Douglas (d)}

  \medskip

  La fonction de production sous forme intensive, $y=f(k)=k^{\alpha}$, hérite des propriétés de la fonction de production $F(K,L)=K^{\alpha}L^{1-\alpha}$:

  \medskip

  \begin{enumerate}
  \item Productivité marginal positive~: $f'(k) = \alpha k^{\alpha-1}>0$,
  \item Productivité marginale décroissante~: $f''(k)= -(1-\alpha)\alpha k^{\alpha-2}<0$,
  \item Conditions d'Inada~: $\lim_{k\rightarrow 0} f'(k)=\infty$ et $\lim_{k\rightarrow \infty} f'(k)=0$
  \end{enumerate}

  \medskip

  \begin{columns}
    \begin{column}{0.6\textwidth}
      {\small
        \begin{itemize}
        \item[--] Les pentes des tangentes sont $>0$,
        \item[--] Les pentes des tangentes sont $\searrow$,
        \item[--] La pente de la tangente en 0 est $\infty$,
        \item[--] La pente de la tangente en $\infty$ est 0.
          \bigskip
        \item[--] La fonction est concave.
        \end{itemize}}
    \end{column}
    \begin{column}{0.4\textwidth}
      \begin{tikzpicture}[scale=.75]
        \begin{axis}[
          title={$y = k^{\alpha}$},
          xlabel=$k$,
          ylabel=$y$,
          xticklabels={,,},
          yticklabels={,,},
          enlargelimits=false,
          grid style={dashed, gray!60},
          axis x line = bottom,
          axis y line = left,
          axis line style={thin},
          small,
          ]
          \addplot[
          draw=black,
          thick,
          smooth,
          samples=500,
          domain=0:8,
          ]
          {x^0.333333};
          \addplot[mark=none, red, thin] coordinates {(0,0.66666666) (8,3.3333333)};
          \addplot[mark=none, red, thin] coordinates {(0,1.0582673679787995) (8,2.1165347359575994)};
          \addplot[mark=*] coordinates {(1,1)};
          \addplot[mark=*] coordinates {(4,1.5874010519681994)};
        \end{axis}
      \end{tikzpicture}
    \end{column}
  \end{columns}

\end{frame}


\begin{frame}
  \frametitle{La fonction de production néoclassique, III}
  \framesubtitle{Exemple~: La fonction Cobb-Douglas (e)}

  \medskip

  \begin{itemize}

  \item Le paramètre $\alpha$ s'interprète comme l'élasticité de la production par rapport au stock de capital physique.\newline

  \item En effet, cette élasticité est définie par~:
    \[
      \epsilon(K,L) = \frac{\frac{\partial Y}{\partial K}}{\frac{Y}{K}}
    \]
    soit~:
    \[
      \begin{split}
        \epsilon(K,L) &= \frac{K}{Y}\alpha K^{\alpha-1}L^{1-\alpha}\\
        &= \alpha\frac{K^{\alpha}L^{1-\alpha}}{Y}\\
        &= \alpha \quad \forall (K,L)\in \mathbb R_+^2
      \end{split}
    \]

    \medskip

  \item On note que cette élasticité ne dépend pas de $K$ ou $L$.

  \end{itemize}
\end{frame}


\begin{frame}
  \frametitle{La fonction de production néoclassique, III}
  \framesubtitle{Exemple~: La fonction Cobb-Douglas (e, suite)}

  \medskip

  \begin{itemize}

  \item Ce même paramètre $\alpha$, dans le modèle de Solow, où la concurrence est parfaite, détermine aussi le partage du revenu entre le capital et le travail.\newline

  \item En effet dans une économie en concurrence parfaiteles facteurs sont rémunérés aux productivités marginales.\newline

  \item Chaque unité de travail est rémunérée au taux de salaire $w=\frac{\partial Y}{\partial L}$.\newline

  \item Chaque unité de capital est rémunérée au taux d'intérêt $r = \frac{\partial Y}{\partial K}$.\newline

  \item La rémunération totale du capital est donc~: $K\frac{\partial Y}{\partial K}$, et la part de la rémunération du capital dans le revenu total~:
    \[
      \frac{K}{Y}\frac{\partial Y}{\partial K} = \alpha
    \]

  \end{itemize}
\end{frame}


\begin{frame}
  \frametitle{La fonction de production néoclassique, III}
  \framesubtitle{Exemple~: La fonction Cobb-Douglas (f)}

  \medskip

  \begin{itemize}

  \item Le Taux Marginal de Substitution Technique du travail au capital, noté $\textrm{TMST}_{L,K}$, est défini comme la variation de travail nécessaire pour compenser une variation infinitésimale du stock de capital physique (de sorte que que $Y$ soit constant).\newline

  \item En considérant la différentielle totale de la fonction de production~:
    \[
      \mathrm dY = F_K(K,L)\mathrm dK + F_L(K,L)\mathrm dL
    \]
    \begin{columns}
      \begin{column}{0.7\textwidth}
        \begin{addmargin}[1,1cm]{0cm}
          en posant $\mathrm dY = 0$, il vient~:
          \[
            \frac{\mathrm d L}{\mathrm d K} = -\frac{F_K(K,L)}{F_L(K,L)} \equiv \textrm{TMST}_{L,K}
          \]
        \end{addmargin}
      \end{column}
      \begin{column}{0.3\textwidth}
        {\tiny Certains imposent la positivité du TMST en le définissant comme le rapport des productivités marginales.}
      \end{column}
    \end{columns}
    \bigskip
  \item Avec la fonction Cobb-Douglas, nous avons~:
    \begin{columns}
      \begin{column}{0.7\textwidth}
        \begin{addmargin}[1,1cm]{0cm}
          \[
            \textrm{TMST}_{L,K} = -\frac{\alpha}{1-\alpha}\frac{L}{K}
          \]
        \end{addmargin}
      \end{column}
      \begin{column}{0.3\textwidth}
        {\tiny Le TMST donne la pente des isoquantes, à l'optimum il doit être égal au rapport des prix.}
      \end{column}
    \end{columns}
  \end{itemize}
\end{frame}


\begin{frame}
  \frametitle{La fonction de production néoclassique, III}
  \framesubtitle{Exemple~: La fonction Cobb-Douglas (g)}

  \medskip

  \begin{itemize}

  \item L'élasticité de substitution entre les facteurs, notée $\sigma(K,L)$, est le taux de croissance du ratio des facteurs de production ($\mathrm d \log \frac{L}{K}$) rapporté au taux de croissance du taux marginal de substitution technique ($\mathrm d \log |TMST_{L,K}|$).\newline

  \item $\sigma(K,L)$ caractérise la courbure des isoquantes.\newline

  \item Dans le cas Cobb-Douglas, le TMST est proportionel au ratio $\nicefrac{L}{K}$.\newline

  \item Les deux de croissance de ces deux termes sont donc identiques, ainsi~:
    \[
      \sigma(K,L) = 1\quad \forall (K,L) \in \mathbb R_+^2
    \]
  \end{itemize}
\end{frame}




\section{L'équation fondamentale du modèle de Solow}

\begin{frame}
  \frametitle{L'équation fondamentale du modèle de Solow}
  \framesubtitle{Encore une histoire de baignoire et de fuite\ldots}

  \begin{block}{Accumulation du stock de capital physique aggrégé}
    \[
      \dot K(t) = I(t) - \delta K(t)
    \]
  \end{block}

  \bigskip

  \begin{itemize}

  \item $0<\delta<1$ est le taux de dépréciation du stock de capital physique.\newline

  \item $\delta K$ est le volume de capital qui s'évapore à chaque instant.\newline

  \item[\dbend] Ne pas confondre le \textbf{taux} de dépréciation ($\delta$) et la dépréciation ($\delta K$, il s'agit d'un volume).\newline

  \item Cette équation nous dit que le stock de capital physique augmente ($\dot K > 0$) ssi l'investissement est supérieur à la dépréciation.
  \end{itemize}

\end{frame}


\begin{frame}
  \frametitle{L'équation fondamentale du modèle de Solow}
  \framesubtitle{Expérience 1: Investissement constant (a)\ldots}

  \bigskip

  \begin{itemize}

  \item Supposons que l'investissement soit constant, \emph{i.e.} $I(t)=\iota\quad\forall t\in\mathbb R_+$. Caractérisons la dynamique du stock de capital.\newline

  \item Nous avons~:
    \[
      \dot K(t) = \iota - \delta K(t)
    \]
    Il s'agit d'une équation différentielle linéaire d'ordre 1 à
    coefficients constants.\newline

  \item Si $\iota=0$ (\emph{i.e.} pas de second membre), on retrouve une équation différentielle analogue à celle que nous avons rencontrée dans le chapitre introductif~:
    \[
      \frac{\dot K(t)}{K(t)} = -\delta
    \]
    S'il n'y a pas d'investissement le stock de capital décroît, le taux de (dé)croissance est le taux de dépréciation du capital. Il n'est pas nécessaire de savoir résoudre une équation différentielle pour comprendre qu'à long terme le capital disparaît\ldots

  \end{itemize}

\end{frame}


\begin{frame}
  \frametitle{L'équation fondamentale du modèle de Solow}
  \framesubtitle{Expérience 1: Investissement constant (b)\ldots}

  \bigskip

  \begin{itemize}

  \item Cherchons une solution de cette équation différentielle, \emph{i.e.} cherchons une fonction $K(t)$ qui vérifie cette équation.\newline

  \item En utilisant la dérivée logarithmique, puis en intégrant les deux membres de cette équation, il vient~:
    \[
      \log K(t) = \tilde A - \delta t
    \]
    où $\tilde A$ est une constante d'intégration (qui apparaît avec le calcul des primitives).\newline

  \item Nous savons donc qu'une solution de l'équation différentielle est~:
    \[
      K(t) = A e^{-\delta t}
    \]
    où $A = e^{\tilde A}\in\mathbb R_+$. Nous avons donc une infinité de solution, une pour chaque valeur  possible de $A$.
  \end{itemize}

\end{frame}


\begin{frame}
  \frametitle{L'équation fondamentale du modèle de Solow}
  \framesubtitle{Expérience 1: Investissement constant (c)\ldots}

  \bigskip

  \begin{itemize}

  \item Puisque $\delta$, le taux de dépréciation est positif, nous avons bien comme annoncé~:
    \[
      \lim_{t\rightarrow\infty}K(t) = 0
    \]
    Puisqu'il n'y a pas d'investissement, la dépréciation épuise complétement le stock de capital à long terme.\newline

  \item Que se passe t-il si $\iota>0$~?\ldots Il s'agit d'une équation différentielle linéaire avec un second membre (l'investissement $\iota$). Pour trouver une solution, on commence par chercher une solution particulère de l'équation différentielle avec second membre~:
    \[
      \dot K(t) = -\delta K(t) + \iota
    \]

  \item La constante $K^{\star}(t) = \nicefrac{\iota}{\delta} \forall t\in\mathbb R_+$ est une solution évidente.\newline

  \item En effet pour cette solution la variation est nulle, $\dot{K^{\star}} = 0$, et $\iota-\delta K^{\star}(t) = 0$.
  \end{itemize}

\end{frame}


\begin{frame}
  \frametitle{L'équation fondamentale du modèle de Solow}
  \framesubtitle{Expérience 1: Investissement constant (d)\ldots}

  \bigskip

  \begin{itemize}

  \item Une solution générale de l'équation avec second membre est la somme de la solution générale de l'équation sans second membre et de la solution particulière de l'équation complète~:
    \[
      K(t) = Ae^{-\delta t} + \frac{\iota}{\delta}
    \]

  \item On peut vérifier en substituant cette solution dans l'équation différentielle complète.
    \begin{itemize}
    \item[--] Pour le membre de gauche on a~:
      \[
        {\color{red}\dot K(t)} = -\delta A e^{-\delta t}
      \]
    \item[--] Pour le membre de droite on a~:
      \[
        \begin{split}
          {\color{red}\iota - \delta K(t)} &= \iota - \delta \left(Ae^{-\delta t} + \frac{\iota}{\delta}\right)\\
          &= -\delta A e^{-\delta t}
        \end{split}
      \]
    \end{itemize}
  \item Les deux membres sont identiques, on a donc bien trouvé la solution générale de l'équation différentielle $\dot K(t) = -\delta K(t) + \iota$.
  \end{itemize}

\end{frame}


\begin{frame}
  \frametitle{L'équation fondamentale du modèle de Solow}
  \framesubtitle{Expérience 1: Investissement constant (e)\ldots}

  \bigskip

  \begin{itemize}

  \item À nouveau, nous disposons d'une infinité de solution~: une pour chaque valeur de la constante d'intégration $A$.\newline

  \item Pour sélectionner une solution, on utilise la condition initiale du stock de capital physique. Suppose qu'à l'instant initial nous ayons $K(0)=K_0>0$.\newline

  \item En évaluant la solution générale en 0, nous devons avoir~:
    \[
      K_0 = A + \frac{\iota}{\delta}
    \]

  \item Ainsi, la solution de l'équation différentielle qui « passe » par la condition initiale $K_0$ est~:
    \[
      K(t) = \left(K_0-\frac{\iota}{\delta}\right)e^{-\delta t} + \frac{\iota}{\delta}
    \]
  \end{itemize}

\end{frame}


\begin{frame}
  \frametitle{L'équation fondamentale du modèle de Solow}
  \framesubtitle{Expérience 1: Investissement constant (f)\ldots}

  \bigskip

  \begin{itemize}

  \item Pour toute condition initiale $K_0$ on a~:
    \[
      \lim_{t\rightarrow\infty} K(t) = \frac{\iota}{\delta} \equiv K^{\star}
    \]

  \item Le niveau de long terme du stock de capital physique ne dépend pas de la condition initiale, et ce niveau est d'autant plus élevé que l'investissement est important ou le taux de dépréciation faible.\newline

  \item Si $K_0>\frac{\iota}{\delta}$, la transition vers $K^{\star}$ est monotone décroissante.\newline
  \item Si $K_0<\frac{\iota}{\delta}$, la transition vers $K^{\star}$ est monotone croissante.\newline

  \item Si $K_0 = K^{\star}$ alors $K(t)=K^{\star}\forall t\in\mathbb R_+$.
  \end{itemize}

\end{frame}


\begin{frame}
  \frametitle{L'équation fondamentale du modèle de Solow}
  \framesubtitle{Expérience 2: Investissement variable (a)\ldots}

  \begin{itemize}
  \item On peut généraliser en considérant un flux d'investissement variable (c'est plus vraisemblable). L'équation différentielle à résoudre devient~:
    \[
      \dot K(t) = -\delta K(t) + I(t)
    \]

    \bigskip

  \item Il s'agit d'une fonction différentielle linéaire à coefficients variables (on a remplacé la constante $\iota$ par une fonction arbitraire $I(t)$).\newline


  \item Si la condition initiale est $K(0)=K_0$ alors on peut montrer qu'il existe une unique solution~:
    \[
      K(t) = e^{-\delta t}\left(K_0 + \int_0^te^{\delta s}I(s)\mathrm ds\right)
    \]
  \end{itemize}

\end{frame}


\begin{frame}
  \frametitle{L'équation fondamentale du modèle de Solow}
  \framesubtitle{Expérience 2: Investissement variable (b)\ldots}

  \begin{itemize}

  \item Si le taux de dépréciation est nul, $\delta=0$, on voit que le stock de capital à l'instant $t$ est le stock de capital à l'instant initial ($K_0$) plus la somme des flux d'investissement entre 0 et $t$ ($\int_0^tI(s)\mathrm ds$).\newline

  \item Plus généralement, pour interpréter le résultat, notons que $e^{-\tau \delta}$ est le facteur de dépréciation entre 0 et $\tau$. Si $X$ se déprécie au taux $\delta$ alors $\frac{X(t)}{X(0)}=e^{-\delta t}$ et $\frac{X(t)}{X(s)} = \frac{X(t)}{X(0)}\frac{X(0)}{X(s)} = e^{-(t-s)\delta}$.\newline

  \item En réécrivant l'équation pour $K(t)$ sous la forme~:
    \[
      K(t) = e^{-\delta t}K_0 + \int_0^te^{-(t-s)\delta}I(s)\mathrm ds
    \]
    $K(t)$ est donc la somme de ce qui reste du capital initial, $e^{-\delta t}K_0$, et la somme des flux d'investissement dépréciés~: $e^{-(t-s)\delta}I(s)$ pour $s\in[0,t]$.
  \end{itemize}

\end{frame}


\begin{frame}
  \frametitle{L'équation fondamentale du modèle de Solow}
  \framesubtitle{Expérience 2: Investissement variable (c)\ldots}

  \begin{itemize}

  \item Il est difficile d'en dire plus sur la dynamique du stock de capital physique, comme nous pouvions le faire dans le cas où l'investissement est constant, sans spécifier la fonction d'investissmeent\ldots\newline

  \item Heureusement dans le modèle de Solow, on a une idée plus précise sur $I(t)$, qui va nous permettre d'aller beaucoup plus loin.\newline

  \item L'investissement ne vient pas de nul part\ldots Dans une économie fermée il est déterminé par l'épargne qui elle même dépend du revenu (ou de la production) et donc indirectement de $K(t)$.\newline

  \end{itemize}

\end{frame}


\begin{frame}
  \frametitle{L'équation fondamentale du modèle de Solow}
  \framesubtitle{L'investissement}

  \begin{itemize}

  \item Dans une économie fermée l'investissement doit être égal à l'épargne.\newline

  \item[\circled{$\mathcal H$}] On suppose que l'épargne est une faction constante de la production (du revenu).\newline

  \item On note $s\in[0,1]$ le taux d'épargne, on a donc~: $I(t) = sY(t)$.\newline

  \item La production elle même dépend, via la fonction de production, du stock de capital physique et du travail, on a donc $I(t) = s K(t)^{\alpha}L(t)^{1-\alpha}$ et finalement~:
  \end{itemize}

  \begin{block}{Accumulation du stock de capital physique aggrégé}
    \[
      \dot K(t) =  s K(t)^{\alpha}L(t)^{1-\alpha} -  \delta K(t)
    \]
  \end{block}

\end{frame}


\begin{frame}
  \frametitle{L'équation fondamentale du modèle de Solow}
  \framesubtitle{Sans croissance de la population (a)}

  \begin{itemize}

  \item Supposons que la population soit constante, $L(t) = L_0\forall t\in\mathbb R_+$.\newline

  \item La dynamique du capital est alors caractérisée par~:
    \bigskip
    \[
      \dot K(t) =  \underbrace{s L_0^{1-\alpha}K(t)^{\alpha}}_{\text{Investissement}} -  \underbrace{\delta K(t)}_{\text{Dépréciation}}
    \]

    \bigskip

  \item Il s'agit d'une équation différentielle non linéaire que nous ne savons pas résoudre (on verra en TD qu'il est possible de résoudre cette équation avec un changement de variable).\newline

  \item Sans explicitement résoudre l'équation, on peut se faire une idée assez précise de la dynamique à l'aide d'une analyse graphique.\newline
  \end{itemize}

\end{frame}


\begin{frame}
  \frametitle{L'équation fondamentale du modèle de Solow}
  \framesubtitle{Sans croissance de la population (b)}

  \bigskip

  \begin{itemize}

  \item Pour représenter graphiquement la dynamique nous travaillons dans un plan avec $K$ sur l'axe des abscisses et une quantité de bien homogène sur l'axe des ordonnées.\newline

  \item Dans ce plan on peut représenter~:
    \medskip
    \begin{itemize}
    \item[--] {\color{blue} L'investissement~:} $sL_0^{1-\alpha}K^{\alpha}$, est une courbe monotone croissante concave avec une pente infinie en 0 et nulle en l'infini (Inada) qui a pour origine (0,0).\newline
    \item[--] {\color{blue} La dépréciation~:} $\delta K$, qui est une droite croissante de pente $\delta$ et d'origine en (0,0).\newline
    \end{itemize}

  \item Puisque la pente de l'investissement est infinie en 0 alors que la pente de dépréciation est finie ($\delta$), la courbe d'investissement part au dessus de la droite de dépréciation.\newline

  \item Puisque la pente de la courbe d'investissement décroît de façon monotone et tend vers 0, elle croise nécessairement une unique fois la droite de dépréciation.
  \end{itemize}

\end{frame}


\begin{frame}
  \frametitle{L'équation fondamentale du modèle de Solow}
  \framesubtitle{Sans croissance de la population (c)}

  \bigskip

  \begin{center}
    \begin{tikzpicture}[scale=1.3]
      \begin{axis}[
        title={},
        xlabel= $K$,
        ylabel= {Quantité de bien homogène},
        xticklabels={,,},
        yticklabels={,,},
        enlargelimits=true,
        grid style={dashed, gray!60},
        axis x line = bottom,
        axis y line = left,
        axis line style={thin},
        xmin = 0,
        xmax = 15,
        ymin = 0,
        ymax = 0.8,
        small,
        ]
        \addplot[
        draw=black,
        thick,
        smooth,
        samples=500,
        domain=0:10,
        ]
        {.2*x^0.333333};
        \node[right] at (10.1, 0.43){{\color{blue}$s L_0^{1-\alpha}K^{\alpha}$}};
        \addplot[
        draw=black,
        thick,
        smooth,
        samples=2,
        domain=0:10,
        ]
        {.075*x};
        \node[right] at (10.1, 0.75){{\color{blue}$\delta K$}};
      \end{axis}
    \end{tikzpicture}
  \end{center}

\end{frame}


\begin{frame}
  \frametitle{L'équation fondamentale du modèle de Solow}
  \framesubtitle{Sans croissance de la population (c)}

  \bigskip

  \begin{center}
    \begin{tikzpicture}[scale=1.3]
      \begin{axis}[
        title={},
        xlabel= $K$,
        ylabel= {Quantité de bien homogène},
        xticklabels={,,},
        yticklabels={,,},
        enlargelimits=true,
        grid style={dashed, gray!60},
        axis x line = bottom,
        axis y line = left,
        axis line style={thin},
        xmin = 0,
        xmax = 15,
        ymin = 0,
        ymax = 0.8,
        small,
        ]
        \addplot[
        draw=black,
        thick,
        smooth,
        samples=500,
        domain=0:10,
        ]
        {.2*x^0.333333};
        \node[right] at (10.1, 0.43){{\color{blue}$s L_0^{1-\alpha}K^{\alpha}$}};
        \addplot[
        draw=black,
        thick,
        smooth,
        samples=2,
        domain=0:10,
        ]
        {.075*x};
        \node[right] at (10.1, 0.75){{\color{blue}$\delta K$}};
        \addplot[draw=red, thick, smooth, <->] coordinates {
          (1,0.075)
          (1,0.2)};
        \node[right] at (1.6, 0.13){{\color{red}$\dot K>0$}};
        \addplot[draw=red, thick, smooth, <->] coordinates {
          (7,0.525)
          (7,0.38256142141126714)};
        \node[right] at (7.6, 0.55){{\color{red}$\dot K<0$}};
      \end{axis}
    \end{tikzpicture}
  \end{center}

\end{frame}


\end{document}


% Local Variables:
% ispell-check-comments: exclusive
% ispell-local-dictionary: "french"
% TeX-master: t
% End: