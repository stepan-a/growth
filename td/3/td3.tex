\documentclass[12pt,a4paper,notitlepage,twocolumn]{article}

\usepackage{amsmath}
\usepackage{amssymb}
\usepackage{amsbsy}
\usepackage{float}
\usepackage[french]{babel}
\usepackage{graphicx}

\usepackage[utf8x]{inputenc}
\usepackage[T1]{fontenc}
\usepackage{palatino}

 \usepackage[active]{srcltx}
\usepackage{scrtime}

\newcommand{\exercice}[1]{\textsc{\textbf{Exercice}} #1}
\newcommand{\question}[1]{\textbf{(#1)}}
\setlength{\parindent}{0cm}


\title{\textsc{Croissance\\ \small{(Fiche de TD n°3)}}}
\author{Stéphane Adjemian\thanks{Université du Maine, Gains. \texttt{stephane DOT adjemian AT univ DASH lemans DOT fr}}}
\date{Le \today\ à \thistime}




\begin{document}

\maketitle

\noindent \textbf{Exercice 1 [Retour sur la règle d'or]} Soit $K(t)$ le stock de capital physique d'une économie,
$L(t)$ la population qui croît au taux constant $n>0$, $\alpha \in
]0,1[$ la part du capital dans le produit, $s \in ]0,1[$ le taux
d'épargne et $\delta>0$ le taux de dépréciation du capital physique.
La dynamique du capital physique agrégé est décrite par~:
\[
  \dot{K}(t) = s Y(t) - \delta K(t)
\]
et la technologie de production est de type Cobb - Douglas :
\[
  Y(t) = K(t)^{\alpha}L(t)^{1-\alpha}
\]
\textbf{(1)} Écrivez l'équation différentielle qui décrit la loi
d'évolution du capital physique par tête. Interprétez cette équation.
\textbf{(2)} Déterminez l'état stationnaire de l'économie intensive (on notera $k^{\ast}$,
$y^{\ast}$ et $c^{\ast}$ les niveaux de long terme du capital par
tête, du produit par tête et de la consommation par tête).
\textbf{(3)} Montrez que le taux d'épargne qui maximise le niveau de
la consommation par tête à long terme, le taux d'épargne dit de la
règle d'or, est : $s_{\mathrm{or}} = \alpha$. Montrez que, dans ce
cas, le rendement du capital par tête net de la dépréciation
effective du capital par tête doit être nul à long terme. On notera
$k^{\ast}_{\mathrm{or}}$, $y^{\ast}_{\mathrm{or}}$ et
$c^{\ast}_{\mathrm{or}}$ les niveaux de long terme du capital par
tête, du produit par tête et de la consommation par tête dans le cas
de la règle d'or. En particulier on a~:
\[
c^{\ast}_{\mathrm{or}} =
(1-\alpha)\left[\frac{\alpha}{n+\delta}\right]^{\frac{\alpha}{1-\alpha}}
\]
\textbf{(4)} On suppose l'existence d'un gouvernement qui taxe
proportionnellement les revenus des ménages et alloue le montant de
l'impôt à l'accumulation du capital. L'équation d'évolution du
capital agrégé devient~:
\[
  \dot{K}(t) = s(1-\tau) Y(t) + \tau Y(t) - \delta K(t)
\]
où $\tau \in [0,1]$ est le taux d'imposition. Montrez que l'équation
d'évolution du capital par tête devient~:
\[
  \dot{k}(t) = (s + \tau(1-s)) k(t)^{\alpha} - (n+\delta) k(t)
\]
Interprétez cette équation. \textbf{(5)} Calculez le nouvel état
stationnaire de l'économie intensive. En particulier vous établirez
que~:
\[
c^{\ast} = (1-s)(1-\tau)\left[\frac{s +
\tau(1-s)}{n+\delta}\right]^{\frac{\alpha}{1-\alpha}}
\]
\textbf{(6)} On suppose que, dans le monde sans politique fiscale
($\tau = 0$), l'épargne des ménages est insuffisante pour atteindre
le niveau de consommation à long terme le plus important possible
($s<\alpha$). Montrez qu'il existe un taux de taxe sur les revenus,
$\bar{\tau}$, qui maximise le niveau de la consommation à long
terme. \textbf{(7)} Le taux $\bar{\tau}$ permet-il d'atteindre le
niveau de consommation de la règle d'or, $c^{\ast}_{\mathrm{or}}$, à
long terme~?\newline


\noindent \textbf{Exercice 2 [Vitesse de convergence].} Soit $K(t)$
le stock de capital physique agrégé dans une économie, $L(t)$ la
population qui croît au taux constant $n>0$, $s \in ]0,1[$ et
$\delta>0$. La dynamique du capital physique est décrite par :
$\dot{K}(t)=sY-\delta K(t)$ où le produit est déterminé par la
fonctions de production Cobb-Douglas $B
K(t)^{\alpha}L(t)^{1-\alpha}$, avec $B>0$ le niveau constant de la
technologie et $\alpha \in ]0,1[$ la part du capital dans le
produit. \textbf{(1)} Après avoir montré que la dynamique du capital
physique par tête $k(t)=K(t)/L(t)$ est caractérisé par l'équation
différentielle non linéaire~:
\[
\frac{\dot{k}}{k} = s B k^{-(1-\alpha)}-(n+\delta)
\]
calculez l'état stationnaire du modèle. Le niveau de la technologie
affecte t-il le niveau de long terme de l'économie~? \textbf{(2)} En
utilisant une approximation de Taylor à l'ordre un autour de l'état
stationnaire, montrez que la dynamique du capital par tête est
localement décrite par l'équation différentielle linéaire suivante~:
\[
\frac{\dot{k}}{k} \simeq (1-\alpha)
(n+\delta)\left\{1-\frac{k}{k^{\ast}}\right\}
\]
On posera $\beta = (1-\alpha)(n+\delta)$ la vitesse de convergence.
Commentez cette équation. \textbf{(3)} En utilisant la formule du
développement limité de la fonction $\log$, montrez qu'on peut
finalement approximer la dynamique de l'économie dans un voisinage
de l'état stationnaire de la façon suivante~:
\[
\frac{\mathrm d}{\mathrm dt}\left(\log \frac{k}{k^\ast}\right) =
-\beta \log \frac{k}{k^{\ast}}
\]
\textbf{(4)} En donnant des valeurs raisonnables aux paramètres du
modèle, calculez la vitesse de convergence prédite par le modèle.
Combien de temps faut-il pour réduire de moitié la distance à l'état
stationnaire~? Quel est le rôle du taux d'épargne sur la
transition~? \textbf{(5)} A votre avis, le modèle est-il satisfaisant
du point de vu de ses prédictions sur la transition de l'économie~?

\bigskip
\bigskip

\noindent \textbf{Exercice 3 [Terre dans le modèle de Solow].} Soit
$K(t)$ le stock de capital physique agrégé dans une économie, $L(t)$
la population qui croît au taux constant $\dot{L}/L=n>0$, $T(t)=T$
la quantité de terre constante dans l'économie, $s \in ]0,1[$ le
taux d'épargne en capital physique et $\delta>0$ le taux de
dépréciation du stock de capital physique. La dynamique du capital
physique est décrite par : $\dot{K}(t)=sY-\delta K(t)$ où le produit
est déterminé par la fonction de production Cobb-Douglas $
Y(t)K(t)^{\alpha}\left(A(t)L(t)\right)^{\beta}T^{1-\alpha-\beta}$,
avec $\alpha$ et $\beta$ positifs et inférieurs à un, $A(t)>0$ le
niveau de la technologie à l'instant $t$. La technologie croît au
taux constant $\dot{A}/A = x$. On remarque que nous retrouvons le
modèle de Solow habituel, c'est-à-dire sans la terre, si
$\beta=1-\alpha$. \textbf{(1)} Exprimez le taux de croissance du
produit en fonction des taux de croissance du capital physique, de
la technologie et de la population. \textbf{(2)} Quels sont les
conséquences pour le produit si le taux de croissance du capital est
constant~? \textbf{(3)} Comment se traduit une augmentation du ratio
capital-travail ($K/Y$) sur le taux de croissance du stock de
capital physique~? Quelle condition ce ratio doit-il satisfaire pour
que le taux de croissance du stock capital physique soit constant,
c'est-à-dire pour que nous puissions définir un sentier de
croissance équilibré dans ce modèle~? \textbf{(4)} Calculez le taux
de croissance le long du sentier de croissance équilibré dans ce
modèle (ie, calculez le taux de croissance du produit). Comparez
avec le taux de croissance, à long terme, dans le modèle de Solow
habituel. \textbf{(5.1)} \'Etablissez l'équation suivante pour la
variation du taux de croissance du stock de capital physique~:
\[
\dot{g}_K \equiv \frac{\mathrm d}{\mathrm dt}\left(
\frac{\dot{K}}{K}\right) = s\frac{Y}{K}\left(g_Y-g_K\right)
\]
\textbf{(5.2)} \`A partir de la réponse à la première question
montrez que~:
\[
g_Y-g_K = (1-\alpha)\left(g_K^*-g_K\right)
\]
\textbf{(5.3)} Utilisez les deux derniers résultats pour étudier la
stabilité du sentier de croissance équilibré.



\textbf{Exercice 5 [Pétrole dans le modèle de Solow].} On
suppose que la production du bien final nécessite l'utilisation
d'une ressource naturelle, le pétrole, dans la quelle on ne peut
investir. Cette ressource constitue un stock qui se déprécie à
chaque instant. On note $\dot{P} = - v P$, avec $v$ une constante
positive (le taux de décroissance du stock de pétrole). La
technologie est de type Cobb-Douglas~:
\[
Y(t) = K(t)^{\alpha}(A(t)L(t))^{\beta}P(t)^{1-\alpha-\beta}
\]
avec $\alpha$ et $\beta$ deux constantes positives inférieures à un,
$A(t)$ le niveau de la technologie qui croît au taux constant $x>0$,
$L(t)$ la population qui croît au taux constant $n>0$. La dynamique
du stock de capital physique agrégé est caractérisée par l'équation
différentielle habituelle~: $\dot{K} = sY - \delta K$. \textbf{(1)}
Exprimez le taux de croissance du produit en fonction des taux de
croissance du capital physique, de la technologie, de la population
et du taux de dépréciation de la ressource naturelle. \textbf{(2)}
Quels sont les conséquences pour le produit si le taux de croissance
du capital est constant~? \textbf{(3)} Comment se traduit une
augmentation du ratio capital-travail ($K/Y$) sur le taux de
croissance du stock de capital physique~? Quelle condition ce ratio
doit-il satisfaire pour que le taux de croissance du stock capital
physique soit constant, c'est-à-dire pour que nous puissions définir
un sentier de croissance équilibré dans ce modèle~? \textbf{(4)}
Calculez le taux de croissance le long du sentier de croissance
équilibré dans ce modèle (ie, calculez le taux de croissance du
produit). Comparez avec le taux de croissance, à long terme, dans le
modèle de Solow habituel. Le taux de croissance du produit réel le
long du sentier de croissance équilibré est-il forcément positif ?
\textbf{(5)} \'Etablissez l'équation suivante pour la variation du
taux de croissance du stock de capital physique~:
\[
\dot{g}_K \equiv \frac{\mathrm d}{\mathrm dt}\left(
\frac{\dot{K}}{K}\right) = s\frac{Y}{K}\left(g_Y-g_K\right)
\]
\`A partir de la réponse à la première question montrez que~:
\[
g_Y-g_K = (1-\alpha)\left(g_K^*-g_K\right)
\]
Utilisez les deux derniers résultats pour discuter la stabilité du
sentier de croissance équilibré. \textbf{(7)} Le taux de croissance
du produit par tête, le long du sentier de croissance équilibré,
est-il forcément positif dans ce modèle~? \textbf{(8)} Normalisez
les variables de ce modèle de façon à pouvoir définir un état
stationnaire. Caractérisez cet état stationnaire.



\end{document}
