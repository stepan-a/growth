\documentclass[10pt,a4paper,notitlepage,onecolumn]{article}

\usepackage{amsmath}
\usepackage{amssymb}
\usepackage{amsbsy}
\usepackage{float}
\usepackage[french]{babel}
\usepackage{graphicx}

\usepackage[utf8x]{inputenc}
\usepackage[T1]{fontenc}
\usepackage{palatino}
\usepackage{nicefrac}
\usepackage[active]{srcltx}
\usepackage{scrtime}

\newcommand{\exercice}[2]{\textsc{\textbf{Exercice}} #1 \textbf{[#2]}}
\newcommand{\question}[1]{\textbf{(#1)}}
\newcommand{\growth}[1]{\frac{\dot{#1}}{#1}}
\setlength{\parindent}{0cm}

\title{\textsc{Croissance\\ \small{(Correction de la fiche de TD n°3)}}}
\author{Stéphane Adjemian\thanks{Université du Maine, Gains. \texttt{stephane DOT adjemian AT univ DASH lemans DOT fr}}}
\date{Le \today\ à \thistime}





\begin{document}

\maketitle

\exercice{1}{Retour sur la règle d'or} Écrivons la loi d'évolution du capital par
tête $k(t) = K(t)/L(t)$. Commençons par diviser les deux membres de
l'équation décrivant l'évolution du stock de capital agrégé dans
l'économie par le nombre de tête $L(t)$~:
\[
\frac{\dot{K}}{L} = s \frac{Y}{L} - \delta \frac{K}{L}
\]
En substituant la définition de la technologie et en notant que
$\alpha+1-\alpha = 1$, il vient~:
\[
\frac{\dot{K}}{L} = s
\frac{K^{\alpha}L^{1-\alpha}}{L^{\alpha+(1-\alpha)}} - \delta k
\]
soit de façon équivalente~:
\[
\frac{\dot{K}}{L} = s
\left(\frac{K}{L}\right)^{\alpha}\left(\frac{L}{L}\right)^{1-\alpha}
- \delta k
\]
\begin{equation}\label{equ:01}
\Leftrightarrow \frac{\dot{K}}{L} = s k^{\alpha} - \delta k
\end{equation}
Nous cherchons à décrire la loi d'évolution de $k$. Dans la dernière
équation, le capital intensif apparaît bien en niveau sur le membre
de droite, mais dans le membre de gauche nous avons la variation du
stock de capital agrégé rapporté à la taille de la population alors
que nous aimerions avoir la variation du capital par tête. Par
définition du capital par tête, nous avons~:
\[
\dot{k} = \frac{\mathrm d}{\mathrm dt}\left(\frac{K}{L}\right)
\]
en vertu des formules usuelles de dérivation, il vient~:
\[
\dot{k} = \frac{\dot{K}L-K\dot{L}}{L^2}
\]
\[
\Leftrightarrow \dot{k} = \frac{\dot{K}L}{L^2}-\frac{K\dot{L}}{L^2}
\]
\[
\Leftrightarrow \dot{k} =
\frac{\dot{K}}{L}-\frac{K}{L}\frac{\dot{L}}{L}
\]
\[
\Leftrightarrow \dot{k} = \frac{\dot{K}}{L}-nk
\]
car la population croît au taux constant $n$. Nous pouvons donc
exprimer la variation du stock de capital agrégé rapporté à la
taille de la population en fonction de la variation du stock de
capital intensif et du niveau du stock de capital intensif. Par
substitution dans l'équation \ref{equ:01} il vient~:
\[
\dot{k} = s k^{\alpha} - (n+\delta) k
\]
Cette équation différentielle non linéaire décrit l'évolution du
stock de capital par tête dans l'économie. Le stock de capital par
tête s'accroît si et seulement si l'investissement par tête est
suffisant pour << couvrir >> la dépréciation du capital
intensif.\newline 

\question{2} L'état de cette économie est complètement
décrit par le capital intensif (si on connaît $k$ on peut déterminer
le produit par tête, $y$, via la fonction de production, la
consommation par tête, $c$, via la règle $c=(1-s)y$ et par
complémentarité l'investissement par tête, $i = sy$). L'économie est
à son état stationnaire lorsque les variables intensives ne varient
plus, c'est-à-dire lorsque le capital par tête ne varie plus
(puisque cette variable << résume >> l'ensemble de l'économie).
Formellement, on cherche le niveau de capital par tête $k^{\ast}$
tel que $\dot{k}$ soit nul, c'est-à-dire tel que~:
\[
s\left[k^{\ast}\right]^{\alpha} = (n+\delta)k^{\ast}
\]
À l'état stationnaire, le volume de l'investissement en capital physique par tête doit 
égaliser la dépréciation du capital physique par tête.  
Une première solution triviale est $k^{\ast}=0$. On ne considère pas
cette solution car il s'agit d'une situation dégénérée, dans le sens
où l'économie n'existe pas à l'état stationnaire (toutes les
variables sont nulles). De plus on peut montrer que cet état
stationnaire, qualifié de trivial, n'est pas stable dans ce modèle.
Après une variation infinitésimale du stock de capital physique par
tête, on ne revient jamais à l'état stationnaire
trivial\footnote{Vous pouvez établir ce point graphiquement.}. Si
$k^{\ast}>0$, en divisant les deux membres de l'équation par
$\left[k^{\ast}\right]^{\alpha}$, il vient~:
\[
k^{\ast} = \left(\frac{s}{n+\delta}\right)^{\frac{1}{1-\alpha}}
\]
En substituant dans la technologie intensive Cobb-Douglas, on
obtient directement le niveau d'état stationnaire du produit par
tête~:
\[
y^{\ast} = \left(\frac{s}{n+\delta}\right)^{\frac{\alpha}{1-\alpha}}
\]
Sachant que le niveau de la consommation est une fraction constante,
$1-s$, du produit, on obtient l'état stationnaire de la
consommation~:
\[
c^{\ast} =
(1-s)\left(\frac{s}{n+\delta}\right)^{\frac{\alpha}{1-\alpha}}
\]
On voit qu'une augmentation permanente du taux d'épargne déplace
vers le haut le niveau de long terme\footnote{Ici on parle
indifféremment de niveau de long terme et d'état stationnaire car
celui-ci est globalement stable (ce que vous pouvez également
établir graphiquement).} du capital physique et du produit
intensifs. Dans une économie où les ménages accumulent plus de
capital, on atteint un niveau de long terme plus élevé.\newline


\question{3} Une augmentation du taux d'épargne a un effet
ambigu sur le niveau de long terme de la consommation. Considérons
deux cas polaires. Si $s=1$ alors la consommation est toujours
nulle. A fortiori elle doit être nulle à l'état stationnaire. Si
$s=0$, alors le stock de capital intensif décroît à chaque instant
car l'investissement est toujours nul. Le taux de décroissance de
$k$ est constant et déterminé par le taux de dépréciation du capital
par tête ($n+\delta$). Pendant la transition le niveau de la
consommation est positif, mais à long terme la consommation devient
nulle car le le capital et produit intensifs tendent vers zéro.\\
Imaginons que nous dévions marginalement de ces deux cas polaires.
Pla\-çons nous dans le cas où la totalité du revenu est épargnée
($s=1$). Si on diminue de façon infinitésimale le taux d'épargne
alors, même si cela va contribuer à diminuer le niveau de long terme
de $k$ et $y$, cela va induire une augmentation (infinitésimale) de
la consommation à long terme. Dans ce cas une diminution du taux
d'épargne induit une augmentation de la consommation à long terme.
Envisageons le cas symétrique où la totalité du revenu est consommée
à chaque instant ($s=0$). Une augmentation infinitésimale du taux
d'épargne nous permet de définir un état stationnaire non trivial,
où le produit sera strictement positif, vers lequel l'économie va se
diriger. Ainsi en augmentant le taux d'épargne on crée un <<~gâteau~>>
à long terme, que le ménage pourra partager entre épargne et
consommation. Une augmentation infinitésimale du taux d'épargne
induit une augmentation infinitésimale de la consommation à long
terme.\\ Au total on comprend pourquoi l'effet d'une variation du
taux d'épargne induit un effet ambigu sur la consommation à long
terme. Quand $s$ est proche de 1, le modèle prédit une relation
décroissante entre le taux d'épargne et la consommation à long
terme. Quand $s$ est proche de 0, le modèle prédit une relation
croissante entre le taux d'épargne et la consommation à long terme.
Tout ceci suggère qu'il pourrait exister un taux d'épargne optimal,
au sens de la maximisation du niveau de la consommation par tête à
long terme, entre 0 et 1. C'est ce que nous nous proposons d'établir
ici. Posons la consommation intensive à l'état stationnaire comme
une fonction du taux d'épargne~:
\[
c^{\ast}(s) =
(1-s)\left(\frac{s}{n+\delta}\right)^{\frac{\alpha}{1-\alpha}}
\]
On définit formellement le taux d'épargne de la règle d'or,
$s_{\mathrm{or}}$, de la façon suivante~:
\[
s_{\mathrm{or}} = \arg\max_{s} c^{\ast}(s)
\]
La condition nécessaire d'optimalité associée à ce programme est~:
\[
-\left(\frac{s_{\mathrm{or}}}{n+\delta}\right)^{\frac{\alpha}{1-\alpha}}+
\frac{\alpha}{1-\alpha}\frac{(1-s_{\mathrm{or}})}{n+\delta}\left(\frac{s_{\mathrm{or}}}{n+\delta}\right)^{\frac{\alpha}{1-\alpha}-1}
= 0
\]
En excluant le cas où $s_{\mathrm{or}}=0$ (il s'agit d'un des deux
cas polaires, évoqués plus haut, pour lequel on atteint le plus
faible niveau possible de consommation à long terme), on peut
diviser les deux membres de cette équation par le premier terme du
membre de gauche. On obtient~:
\[
\frac{\alpha}{1-\alpha}\frac{(1-s_{\mathrm{or}})}{n+\delta}\left(\frac{s_{\mathrm{or}}}{n+\delta}\right)^{-1}
= 1
\]
\[
\Leftrightarrow\frac{(1-s_{\mathrm{or}})}{s_{\mathrm{or}}}
 = \frac{1-\alpha}{\alpha}
\]
\[
\Leftrightarrow s_{\mathrm{or}}=\alpha
\]
notre candidat pour le taux d'épargne de la règle d'or. Nous venons
de montrer qu'il existe une unique valeur de $s$ pour laquelle la
dérivée de $c^{\ast}(s)$ est nulle. Il nous reste à montrer que nous
avons bien maximisé une fonction. La preuve est directe. Il suffit
de noter que nous pouvons reprendre les quatres équations
précédentes en remplaçant ''='' par ''>'' ou ''<''. On voit alors
directement que la dérivée est positive si et seulement si le taux
d'épargne est inférieur à la part du capital dans la valeur ajoutée.
Autrement dit la courbe représentative de la fonction $c^{\ast}(s)$
est nulle en zéro et en un, pour le reste elle a la forme d'un bol
inversé. Ces signes de la dérivée sont conformes avec l'analyse à
proximité des cas polaires évoqués plus haut. Pour conclure,
$s_{\mathrm{or}}=\alpha$ est bien le taux d'épargne de la règle
d'or. On obtient les niveaux de long terme des variables intensives
en remplaçant $s$ par $s_{\mathrm{or}}$ dans les expressions de
$k^{\ast}$, $y^{\ast}$ et $c^{\ast}$. En particulier, nous
obtenons~:
\[
c^{\ast}_{\mathrm{or}} =
(1-\alpha)\left(\frac{\alpha}{n+\delta}\right)^{\frac{\alpha}{1-\alpha}}
\]
Pour interpréter notre résultat, on peut réécrire la définition de
l'état stationnaire du capital physique intensif de la règle d'or de
la façon suivante\footnote{En remplaçant $s$ par $\alpha$.}~:
\[
\alpha \left[k^{\ast}_{\mathrm{or}}\right]^{1-\alpha} = n+\delta
\]
Le terme sur le membre de droite s'interprète comme le rendement du
capital par tête (la productivité marginale) à l'état stationnaire
de la règle d'or. De façon équivalente, on a~:
\[
\alpha \left[k^{\ast}_{\mathrm{or}}\right]^{1-\alpha} - (n+\delta) =
0
\]
Notre résultat devient transparent. Le taux d'épargne de la règle
d'or, c'est le taux qui permet d'annuler le rendement du capital
intensif net de sa dépréciation. Imaginons que le ménage
représentatif puisse choisir son comportement d'accumulation,
\textit{ie} la valeur de $s$, sur des considérations de long terme.
Supposons que le rendement net du capital soit négatif à long terme.
Cette situation apparaît lorsque $s>\alpha$. En effet, dans ce cas
$k^{\ast}>k^{\ast}_{\mathrm{or}}$ et, en vertu de l'hypothèse de
rendements décroissants du capital, on a immédiatement~:
\[
s \left[k^{\ast}\right]^{1-\alpha} - (n+\delta)<\alpha
\left[k^{\ast}_{\mathrm{or}}\right]^{1-\alpha} - (n+\delta) = 0
\]
Si le ménage ne se préoccupe seulement que de ce qu'il peut gagner
ou perdre à long terme, il est alors incité à réduire son taux
d'épargne. Cela se traduit par une baisse du niveau de long terme du
capital intensif et, à nouveau par l'hypothèse de rendements
décroissants du capital\footnote{Le terme $\alpha
\left[k^*\right]^{1-\alpha}$ est une fonction décroissante de
$k^*$.}, une augmentation du rendement net du capital intensif à
long terme. Il va réduire son taux d'épargne tant que le rendement
net du capital intensif à long terme est négatif. Par symétrie, si
le rendement net du capital intensif est positif à long terme, alors
le ménage est incité à augmenter son taux d'épargne, puisqu'il obtient alors 
un bénéfice. Il va augmenter son taux d'épargne tant que le
rendement net du capital intensif est strictement positif. Ainsi, on
comprend bien que le meilleur choix pour le ménage est d'accorder
son comportement d'accumulation de façon à annuler le rendement net
du capital intensif.\newline

\question{4} On suppose l'existence d'un gouvernement qui
taxe proportionnellement les revenus des ménages et alloue le
montant de l'impôt à l'accumulation du capital. L'équation
d'évolution du capital agrégé devient~:
\[
  \dot{K}(t) = s(1-\tau) Y(t) + \tau Y(t) - \delta K(t)
\]
où $\tau \in [0,1]$ est le taux d'imposition. En regroupant les
termes en $Y(t)$, il vient~:
\[
  \dot{K} = (s+\tau(1-s)) Y  - \delta K
\]
Le stock de capital agrégé augmente si et seulement si
l'investissement effectif est supérieur à la dépréciation du
capital. L'investissement effectif est composé de l'investissement
<<~choisi~>>\footnote{Dans le modèle de Solow, le comportement
d'épargne est exogène ; le taux d'épargne n'est pas une variable de
choix du ménage représentatif.} par le ménage représentatif sur la
base de son revenu disponible, $s(1-\tau)Y$, et de l'investissement
induit par la politique fiscale de l'état, $\tau Y$\footnote{Pour le
dire autrement, en suivant la dernière équation, l'investissement
effectif est composé de l'investissement que choisirait le ménage
dans un monde sans politique fiscale, $s Y$, et de l'investissement
induit par la politique fiscale, $\tau (1-s) Y$, une partie de la
consommation souhaitée et détournée par l'état vers l'accumulation
de capital physique.}. En posant $s_{\tau} = s+\tau(1-s)$ le taux
d'épargne effectif (c'est-à-dire incluant l'effet <<~redistribution~>>
de la politique fiscale), on peut poursuivre comme dans le modèle de
Solow habituel~:
\[
  \dot{K} = s_{\tau} Y  - \delta K
\]
Nous retrouvons donc directement le résultat habituel~:
\[
  \dot{k}(t) = (s + \tau(1-s)) k(t)^{\alpha} - (n+\delta) k(t)
\]
Le stock de capital intensif augmente si et seulement si
l'investissement (effectif) par tête couvre la dépréciation du
capital intensif.\newline

\question{5} Pour caractériser l'économie à l'état
stationnaire il nous suffit de déterminer  l'état stationnaire du
capital physique intensif. En utilisant $s_{\tau}$, le taux
d'épargne effectif, au lieu de $s$ on voit immédiatement que l'état
stationnaire de $k$ devient~:
\[
k^{\ast} = \left(\frac{s +
\tau(1-s)}{n+\delta}\right)^{\frac{1}{1-\alpha}}
\]
et en substituant dans la fonction de production~:
\[
y^{\ast} = \left(\frac{s +
\tau(1-s)}{n+\delta}\right)^{\frac{\alpha}{1-\alpha}}
\]
La consommation à long terme est définie par la part non épargnée,
$1-s$, du revenu disponible à long terme $(1-\tau)y^{\ast}$. \`A
l'état stationnaire, nous devons avoir~:
\[
c^{\ast} = (1-s)(1-\tau)\left(\frac{s +
\tau(1-s)}{n+\delta}\right)^{\frac{\alpha}{1-\alpha}}
\]
On obtient des résultats intuitifs par rapport à ce que nous avons
l'habitude de voir. Une augmentation de la taxe, c'est-à-dire une
augmentation du taux d'épargne effectif, induit une augmentation du
niveau de long terme du capital et du produit intensifs. L'effet sur
le niveau de long terme de la consommation est plus ambigu. Une
augmentation de la taxe augmente certes le produit (ou le revenu) à
long terme mais dans le même temps elle diminue la fraction du
revenu que le ménage peut partager entre épargne et consommation.
Selon les tailles respectives de ces deux effets le revenu
disponible des ménages peut baisser ou augmenter. La consommation
étant proportionnelle au revenu disponible, celle-ci peut baisser ou
augmenter à long terme suite à une augmentation permanente du taux
de taxe.\newline

\question{6} On suppose que dans le monde sans politique fiscale, le ménage sous accumule le capital~:
$s<s_{\mathrm{or}}=\alpha$. On  a vu dans notre réponse  à la question
précédente que l'effet d'une variation  permanente du taux de taxe sur
le niveau de la consommation par tête est ambigu. On veut montrer, par
analogie avec le taux d'épargne de la règle d'or, qu'il existe un taux
de  taxe, strictement  positif, qui  maximise la  consommation  à long
terme. Celui-ci est défini de la façon suivante~:
\[
\overline{\tau} = \arg\max_{\tau} c^{\ast}(\tau)
\]
La condition nécessaire d'optimalité associée à ce programme est~:
\[
\begin{split}
(1-s)\left[\frac{s +
\overline{\tau}(1-s)}{n+\delta}\right]^{\frac{\alpha}{1-\alpha}} &=\\
(1-s)&(1-\overline{\tau})\left[\frac{s +
\overline{\tau}(1-s)}{n+\delta}\right]^{\frac{\alpha}{1-\alpha}-1}\frac{1-s}{n+\delta}\frac{\alpha}{1-\alpha}
\end{split}
\]
En supposant que $\overline{\tau}$ est strictement positif (sinon il
ne s'agit plus d'une taxe mais d'une subvention)
$s+\overline{\tau}(1-s)$ est nécessairement positif et donc nous
pouvons diviser les deux membre de cette égalité par le membre de
gauche~:
\[
 1 = (1-\overline{\tau})\left[\frac{s +
\overline{\tau}(1-s)}{n+\delta}\right]^{-1}\frac{1-s}{n+\delta}\frac{\alpha}{1-\alpha}
\]
\[
\Leftrightarrow \frac{1-\alpha}{\alpha} =
\frac{(1-\overline{\tau})(1-s)}{s + \overline{\tau}(1-s)}
\]
\[
\Leftrightarrow (1-\alpha)(s + \overline{\tau}(1-s)) =
\alpha(1-\overline{\tau})(1-s)
\]
\[
\Leftrightarrow s + \overline{\tau}(1-s) - \alpha s - \alpha
\overline{\tau}(1-s) = (\alpha-\alpha\overline{\tau})(1-s)
\]
\[
\Leftrightarrow s + \overline{\tau}(1-s) - \alpha s - \alpha
\overline{\tau}(1-s) = \alpha-\alpha\overline{\tau}-s\alpha +
s\alpha\overline{\tau}
\]
\[
\Leftrightarrow s + \overline{\tau}-\overline{\tau}s - \alpha s -
\alpha \overline{\tau}+s\alpha \overline{\tau} =
\alpha-\alpha\overline{\tau}-s\alpha + s\alpha\overline{\tau}
\]
\[
\Leftrightarrow s + \overline{\tau}(1-s)  = \alpha
\]
D'où finalement~:
\[
\overline{\tau} = \frac{\alpha-s}{1-s}>0
\]
On voit directement\footnote{En remplaçant ``='' par ``>'' dans les
équivalences qui précèdent.} qu'il s'agit bien d'un maximum car la
dérivée de $c^{\ast}$ par rapport à $\tau$ est positive si et
seulement si $\tau<\overline{\tau}$. Notons que nous parvenons à
exhiber une valeur de $\tau$ optimale strictement positive car nous
avons postulé une situation de sous accumulation ($s<\alpha$). Le
meilleur choix consiste à ajuster le taux d'imposition, $\tau$, de
façon à égaliser le taux d'épargne effectif, $s_{\tau}$, avec le
taux d'épargne de la règle d'or ($s_{\mathrm{or}}=\alpha$).\newline 


\question{7} On peut montrer que le taux de taxe
$\overline{\tau}$ permet d'obtenir un niveau de consommation à long
terme identique à celui que nous obtiendrions en suivant la règle
d'or. Pour établir ce point il suffit de substituer l'expression
obtenue de $\overline{\tau}$ dans l'expression analytique de
$c^{\ast}$ obtenue plus haut. On a~:
\[
c^{\ast}(\overline{\tau}) = (1-s)(1-\overline{\tau})\left(\frac{s +
\overline{\tau}(1-s)}{n+\delta}\right)^{\frac{\alpha}{1-\alpha}}
\]
\[
\Leftrightarrow c^{\ast}(\overline{\tau}) =
(1-s)\frac{1-\alpha}{1-s}\left(\frac{s +
\alpha-s}{n+\delta}\right)^{\frac{\alpha}{1-\alpha}}
\]
\[
\Leftrightarrow c^{\ast}(\overline{\tau}) =
(1-\alpha)\left(\frac{\alpha}{n+\delta}\right)^{\frac{\alpha}{1-\alpha}}
\]
\[
\Leftrightarrow c^{\ast}(\overline{\tau}) = c_{\mathrm{or}}^{\ast}
\]
Ainsi, en détournant (par l'intermédiaire d'une politique fiscale)
une partie du revenu destiné à la consommation vers l'accumulation,
l'état parvient à répliquer le taux d'épargne de la règle
d'or.\newline

\textbf{\textsc{Un peu plus loin...}} L'état peut mimer la règle d'or, en
dirigeant de façon autoritaire une partie de la consommation des
ménages vers l'accumulation, car l'économie est dans une situation
de sur accumulation. Dans le cas contraire, \textit{ie} dans une
situation de sous accumulation, l'état ne peut atteindre cet
objectif à l'aide de la politique fiscale que nous venons de
décrire. En effet, la politique fiscale considérée ici revient
simplement à faire en sorte que le taux d'épargne
effectif,$s_{\tau}$, soit supérieur au taux d'épargne $s$. Cela va
dans le bon sens car l'économie est dans une situation de sous
accumulation. Quand, au contraire, les ménages sur accumulent le
capital physique, on ne peut se rapprocher de la règle d'or en
faisant en sorte que le taux d'épargne effectif soit supérieur au
taux d'épargne. Il faut amender la redistribution implicite (entre
consommation et investissement) mise en oeuvre à l'aide de la
politique fiscale. Pour cela on pourrait simplement supposer que
l'état redistribue ses gains liés aux taxes sur les revenus sous la
forme d'unités de consommation au ménage. Dans ce cas la loi
d'évolution du stock de capital agrégé devrait s'écrire~:
\[
\dot{K} = s(1-\tau)Y - \delta K
\]
Le taux d'épargne effectif serait alors $s_{\tau}=s(1-\tau)$. Et le
niveau de consommation serait donné par~: $c(t) =
[(1-s)(1-\tau)+\tau]y(t)$. En suivant la même démarche, on peut
montrer que dans une situation de sur accumulation ($s>\alpha$) il
est alors possible de trouver un taux de taxe positif qui maximise
le niveau de consommation à long terme. On montre que ce taux de
taxe doit être tel que le taux d'épargne effectif égalise le taux
d'épargne de la règle d'or~: $\overline{\tau} = (s-\alpha)/s$. Enfin
on montre qu'avec cette politique fiscale l'état peut mimer le taux
d'épargne de la règle d'or ($c^{\ast}(\overline{\tau})$) en
détournant de façon autoritaire l'épargne vers la
consommation.\newline

Pour finir, il convient de noter que l'effet de la
politique fiscale dans le modèle de Solow demeure simple car le
comportement d'épargne est exogène. Un changement de politique
fiscale n'affecte pas le comportement d'accumulation des ménages.\newline

\bigskip


\exercice{2}{Vitesse de convergence} Le niveau de la technologie ($B$)
affecte l'état  stationnaire de  la même façon  que le  taux d'épargne
$s$.  Pour s'en convaincre, il suffit  de noter que $B$ n'apparaît que
comme un  facteur de $s$ dans  l'équation différentielle caractérisant
la  dynamique du  stock  de  capital par  tête.   On  remarque que  le
paramètre $B$, de la même manière que le taux d'épargne, n'affecte pas
la vitesse de transition vers l'état stationnaire.\newline

\question{1} Trivial. L'état stationnaire est :
\[
k^{\star} = \left(\frac{sB}{n+\delta}\right)^{\frac{1}{1-\alpha}}
\]
Une  augmentation  du  niveau  de  la  technologie,  $B$,  induit  une
élévation  du  niveau de  long  terme  de  l'économie (comme  le  taux
d'épargne en capital physique). \question{2} Notons
\[
\varphi(k) = sBk^{\alpha-1} - (n+\delta) 
\]
la fonction définissant  le taux de croissance de  $k$. En considérant
une approximation  de Taylor à l'ordre  1 dans un voisinage  de l'état
stationnaire, nous avons :
\[
\growth{k} \simeq \varphi (k^\star) + \varphi '(k^{\star}) \left(k-k^{\star}\right)
\]
Puisque par définition de l'état stationnaire nous avons $\varphi (k^\star)=0$, il vient :
\[
\growth{k} \simeq \varphi '(k^{\star}) \left(k-k^{\star}\right)
\]
Par ailleurs, nous avons :
\[
\varphi '(k) = -(1-\alpha)sB\frac{k^{\alpha-1}}{k}
\]
et donc :
\[
\begin{split}
  \varphi '(k^{\star}) &= -(1-\alpha)sB\frac{\frac{n+\delta}{sB}}{k^{\star}}\\
  &= -(1-\alpha)(n+\delta)\frac{1}{k^{\star}}
\end{split}
\]
Nous obtenons l'approximation  suivante du taux de  croissance dans un
voisinage de l'état stationnaire :
\[
\growth{k} \simeq -(1-\alpha)(n+\delta)\frac{k-k^{\star}}{k^{\star}}
\] 
\[
\Leftrightarrow \growth{k} \simeq (1-\alpha)(n+\delta)\left(1-\frac{k}{k^{\star}}\right)
\]
Le  taux de  croissance est  positif si  et seulement  si le  stock de
capital physique  par tête est supérieur  à son niveau de  long terme.
\question{3}  Notons  que dans  un  voisinage  de l'état  stationnaire
$1-\frac{k}{k^{\star}}$ est proche  de zéro. Par ailleurs  on sait que
$\log (1+x)\simeq x$ pour $x$ dans un voisinage de 0. Ainsi :
\[
\log \left(1-\left(1-\frac{k}{k^{\star}}\right)\right) \simeq -\left(1-\frac{k}{k^{\star}}\right) 
\]
ou de façon équivalente :
\[
\log \frac{k}{k^{\star}} \simeq -\left(1-\frac{k}{k^{\star}}\right) 
\]
Nous avons donc :
\[
\Leftrightarrow \growth{k} \simeq -(1-\alpha)(n+\delta)\log \frac{k}{k^{\star}}
\]
On notera  $\beta =  (1-\alpha)(n+\delta)$ la vitesse  de convergence.
\question{4} En posant  $\alpha=\nicefrac{1}{3}$ et $n+\delta=4\%$, on
obtient une  vitesse d'ajustement vers l'état  stationnaire de 2,66\%.
Notons que cette vitesse est ``artificiellement'' basse car nous avons
omis le  progrès technique  dans ce  modèle.  Il  faut $\nicefrac{\log
  2}{\beta}\simeq 26$ ans pour réduire  de moitié la distance à l'état
stationnaire.   Le  taux d'épargne  n'a  aucun  effet sur  la  vitesse
d'ajustement vers l'état  stationnaire. \question{5} Quantitativement,
la transition dans  ce modèle semble plutôt  satisfaisante. La vitesse
de convergence théorique n'est pas très éloignée de ce que suggère les
données.   Mais  cela est  la  conséquence  de l'omission  du  progrès
technique  qui pose  par  ailleurs de  gros  problèmes...  Puisque  la
prédiction de  ce modèle est que  le taux de croissance  des variables
par tête est nul à long terme, ce qui semble en contradiction avec les
données.

\bigskip

\exercice{3}{Terre dans le modèle de Solow} \question{1} En appliquant
la fonction  logarithme népérien à  la fonction de production :
\[
\log Y(t) = \alpha \log K(t) + \beta \log A(t) + \beta \log L(t) + (1-\alpha-\beta) \log T
\]
puis en dérivant par rapport à $t$ on obtient directement :
\[
\growth{Y} = \alpha\growth{K} + \beta (n+x)
\]
\question{2} Puisque les taux de  croissance de la population ($n$) et
de technologie ($x$) sont constants,  le taux de croissance du produit
est  constant si  et seulement  si le  taux de  croissance du  capital
physique est constant. \question{3} Le  taux de croissance du stock de
capital physique est donné par :
\[
\growth{K} = s\frac{Y}{K} - \delta
\]
l'investissement  brut  par   unité  de  capital  moins   le  taux  de
dépréciation du capital physique. On voit que le taux de croissance du
capital  physique  est  une  fonction croissante  de  la  productivité
moyenne du  capital.  Ainsi  une augmentation de  $\nicefrac{K}{Y}$ se
traduit par  une baisse du  taux de croissance  de $K$. On  voit aussi
qu'une condition nécessaire  et suffisante pour le  taux de croissance
de $K$ soit  constant est que le ratio  \nicefrac{K}{Y} soit constant,
c'est-à-dire  que  les  taux  de  croissance  de  $K$  et  $Y$  soient
identiques. \question{4}  Le long  du sentier de  croissance équilibré
nous avons $g_K=g_Y$ et donc :
\[
g_{Y} = \alpha g_{Y} + \beta (n+x)
\]
soit de façon équivalente :
\[
g_Y = \frac{\beta}{1-\alpha}(n+x)
\]
Dans le modèle de Solow habituel  (sans le facteur fixe Terre) le taux
de  croissance de  long  terme de  la production  est  $n+x$. On  peut
montrer que $\nicefrac{\beta}{1-\alpha}$ est inférieur à 1 et que donc
le  taux  de croissance  est  plus  faible  en présence  d'un  facteur
fixe. En effet :
\[
\frac{\beta}{1-\alpha}<1
\]
\[
\Leftrightarrow \beta < 1-\alpha
\]
\[
\Leftrightarrow \alpha + \beta < 1
\]
ce qui est  vrai par hypothèse (rendements décroissants  par rapport à
$K$ et $L$). \question{5} Nous savons que :
\[
g_K = s\frac{Y}{K} - \delta
\]
En dérivant les deux membres par rapport au temps, il vient :
\[
\dot g_K = s \dot{\left(\frac{Y}{K}\right)}
\]
\[
\Leftrightarrow \dot g_K = s \frac{\dot Y K - Y \dot K}{\dot K}
\]
\[
\Leftrightarrow \dot g_K = s \frac{Y}{K}\left(g_Y-g_K\right)
\]
Si  le  taux  de  croissance  du produit  est  supérieur  au  taux  de
croissance du  capital, alors  le taux de  croissance du  capital doit
augmenter.   Ceci suggère  une propriété  de stabilité  du sentier  de
croissance  équilibré (comme  dans  le modèle  de  Solow sans  facteur
fixe). D'après la réponse à la première question, nous avons :
\[
g_Y = \alpha g_K + \beta (n+x)
\]
\[
\Leftrightarrow g_Y-g_K = -(1-\alpha) g_K + \beta (n+x)
\]
\[
\Leftrightarrow g_Y-g_K = (1-\alpha) \left(\frac{\beta}{1-\alpha}(n+x)-g_K\right)
\]
En notant :
\[
g_K^{\star} = \frac{\beta}{1-\alpha}((n+x))
\]
le taux de croissance à long terme du capital, il vient :
\[
g_Y - g_k = (1-\alpha)\left(g_K^{\star}-g_K\right)
\]
Ainsi nous avons :
\[
\dot g_K = s\frac{Y}{K}(1-\alpha)\left(g_K^{\star}-g_K\right)
\]
Le  taux de  croissance  de  $K$ augmente  si  et  seulement s'il  est
inférieur à $g_K^{\star}$, le sentier de croissance équilibré est donc
stable.\newline

À long  terme le  taux de croissance  du produit est  égal au  taux de
croissance du capital physique et le taux de croissance du produit par
tête est donné par :
\[
g_y^{\star} = \frac{\beta}{1-\alpha}((n+x)) - n
\]
où nous  savons que  $\nicefrac{\beta}{1-\alpha}<1$, pour  des petites
valeurs du taux de croissance de  la technologie ($x$) il se peut donc
que  le  taux  de  croissance   du  produit  par  tête  soit  négatif.\newline

\bigskip

\exercice{4}{Pétrole dans le modèle  de Solow} \question{1} En prenant
le logarithme népérien  de la fonction de production  puis en dérivant
par rapport à $t$, on obtient :
\[
\growth{Y} = \alpha\growth{K} + \beta (n+x) + (1-\alpha-\beta)\growth{P}
\]
\question{2} Puisque les taux de croissance de la population ($n$), de
la  technologie  ($x$)  et  de  la  ressource  naturelle  ($-v$)  sont
constants, si le  taux de croissance du capital est  constant alors le
taux   de    croissance   de   la   production    est   nécessairement
constants. \question{3} D'après la loi d'évolution du stock de capital
physique nous avons :
\[
\growth{K} = s\frac{Y}{K} - \delta
\]
Ainsi  une augmentation  du ratio  $\nicefrac{K}{Y}$, l'inverse  de la
productivité moyenne du capital, se traduit  par une baisse de taux de
croissance  du  stock  de  capital  physique.  Pour  que  le  taux  de
croissance de  $K$ soit  constant il  faut et il  suffit que  le ratio
$\nicefrac{K}{Y}$ soit constant (son  inverse, la productivité moyenne
du capital, doit  être constante). Sous cette même  condition, le taux
de croissance de $Y$ est constant.  \question{4} Le long du sentier de
croissance équilibré,  le taux  de croissance du  produit est  égal au
taux  de   croissance  du   stock  de   capital  physique   (le  ratio
$\nicefrac{K}{Y}$  est alors  constant). Nous  pouvons donc  écrire la
première équation de la façon suivante :
\[
(1-\alpha)\growth{Y} = \beta (n+x) + (1-\alpha-\beta)\growth{P}
\]
puis en divisant par $1-\alpha$ :
\[
\growth{Y} = \frac{\beta}{1-\alpha} (n+x) + \frac{1-\alpha-\beta}{1-\alpha}\growth{P}
\]
le taux  de croissance  du produit  le long  du sentier  de croissance
équilibré. Dans le modèle de Solow de base, c'est-à-dire sans pétrole,
le taux  de croissance  du produit  le long  du sentier  de croissance
équilibré      est       $n+x$.       On      sait       déjà      que
$\nicefrac{\beta(n+x)}{1-\alpha}$ est inférieur à $n+x$, en notant que
le taux de croissance de la ressource épuisable est $-v<0$, on conclut
que taux de croissance le long  du sentier de croissance équilibré est
plus  faible  lorsque  l'on  augmente  le  modèle  de  Solow  avec  le
pétrole. Le taux de croissance est alors :
\[
\growth{Y} = \frac{\beta}{1-\alpha} (n+x) - \frac{1-\alpha-\beta}{1-\alpha}v
\]
Le produit peut décroître si la dépréciation de la ressource naturelle
est assez importante (supérieure à $(n+x)\frac{\beta}{1-\alpha-\beta}$). \question{5} Nous savons que le taux de croissance du stock de capital est :
\[
g_K = s \frac{Y}{K} -\delta K
\]
en dérivant par rapport au temps, il vient :
\[
\dot g_K = s \frac{Y}{K}\left(g_Y - g_K\right)
\]
Par ailleurs nous avons :
\[
g_K^{\star} = \frac{\beta}{1-\alpha} (n+x) - \frac{1-\alpha-\beta}{1-\alpha}v
\]
et
\[
\begin{split}
  g_Y &= \alpha g_K  + \beta (n+x) - (1-\alpha-\beta)v\\
\Leftrightarrow g_Y-g_K  &= -(1-\alpha) g_K + \beta (n+x) - (1-\alpha-\beta)v\\
\Leftrightarrow g_Y-g_K  &= -(1-\alpha)\left(g_K - g_K^{\star}\right)
\end{split}
\]
En substituant dans l'équation de la variation du taux de croissance de $K$ on obtient :
\[
\dot g_K = -s \frac{Y}{K}(1-\alpha)\left(g_K - g_K^{\star}\right)
\]
Cette équation nous dit que le taux  de croissance de K augmente si et
seulement si le taux de croissance est inférieur au taux de croissance
le long du  sentier de croissance équilibré  ($g_K^{\star}$). Ainsi on
voit que le  sentier de croissance équilibré est stable,  dans le sens
où si l'économie quitte le  sentier de croissance équilibré elle finit
forcément par le  rejoindre ; $g_K^{\star}$ est donc aussi  le taux de
croissance à long terme. \question{7} Le taux de croissance du produit
par tête est :
\[
g_y = g_Y - n = \alpha g_K  + \beta (n+x) - (1-\alpha-\beta)v -n
\]
Ce  taux   de  croissance  n'est   pas  forcément  positif,   même  si
$g_Y>0$. \question{8} Pour toute variable $X(t)$ on pose :
\[
\hat x (t) = \frac{X(t)}{A(t)L(t)P(t)}
\]
En particulier on montre facilement que :
\[
\hat y(t) = \hat k(t) ^{\alpha}
\]
Puis en suivant la démarche habituelle, on obtient la variation de $\hat k$ :
\[
\dot{\hat{k}}(t) = s \hat{k}^{\alpha} - (n+x+\delta-v)\hat{k}(t)
\]
où $n+x+\delta-v$  est le taux de  dépréciation de $\hat k$.  À l'état
stationnaire l'investissement  en $\hat k$  doit être égal au  taux de
dépréciation de $\hat k$. On trouve l'état stationnaire non trivial :
\[
\hat k ^{\star} = \left(\frac{s}{n+x+\delta-v}\right)^{\frac{1}{1-\alpha}}
\]
\end{document}