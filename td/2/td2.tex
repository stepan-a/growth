\documentclass[10pt,a4paper,notitlepage]{article}
\usepackage{amsmath}
\usepackage{amssymb}
\usepackage{amsbsy}
\usepackage{float}
\usepackage[french]{babel}
\usepackage{graphicx}

\usepackage[utf8x]{inputenc}
\usepackage[T1]{fontenc}
\usepackage{palatino}

 \usepackage[active]{srcltx}
\usepackage{scrtime}

\newcommand{\exercice}[1]{\textsc{\textbf{Exercice}} #1}
\newcommand{\question}[1]{\textbf{(#1)}}
\setlength{\parindent}{0cm}

\begin{document}

\title{\textsc{Croissance\\ \small{(Fiche de TD n°2)}}}
\author{Stéphane Adjemian\thanks{Université du Maine, Gains. \texttt{stephane DOT adjemian AT univ DASH lemans DOT fr}}}
\date{Le \today\ à \thistime}

\maketitle

\exercice{1}  Soit $K(t)$  le  stock de  capital  physique, $L(t)$  la
population qui  croît au taux  constant $n>0$, $\alpha \in  ]0,1[$, $s
\in  ]0,1[$  et $\delta>0$.   La  dynamique  du capital  physique  est
décrite par :  $\dot{K}(t)=sK(t)^{\alpha}L(t)^{1-\alpha}-\delta K(t)$.
\question{1} Après  avoir déterminé  la dynamique du  capital physique
par  tête,  $k(t)=K(t)/L(t)$,   caractérisez  l'état  stationnaire  du
modèle.  \question{2}   Si  l'économie   est  initialement   à  l'état
stationnaire,  quel sera  l'effet  d'un changement  permanent du  taux
d'épargne  sur la  dynamique  du  capital physique  par  tête, sur  la
dynamique du produit  par tête et sur la dynamique  de la consommation
par  tête.   \question{3}  Montrez  qu'il  existe  un  taux  d'épargne
$\bar{s}$  qui  induit  une   consommation  de  long  terme  optimale.
\question{4} Si le  taux d'épargne effectif $s$ est  différent du taux
d'épargne $\bar{s}$,  l'état (hors modèle ici)  doit-il nécessairement
inciter les ménages à choisir un taux d'épargne $\bar{s}$ ?\newline

\exercice{2}  Soit $K(t)$  le  stock de  capital  physique, $L(t)$  la
population qui  croît au taux  constant $n>0$, $\alpha \in  ]0,1[$, $s
\in ]0,1[$ et $\delta>0$. La dynamique du capital physique est décrite
par :  $\dot{K}(t)=sY(t)-\delta K(t)$.   La technologie  de production
est de type  Cobb-Douglas : $Y =  K^{\alpha}(AL)^{1-\alpha}$ où $A(t)$
est   un   indice   technologique   qui   croît   au   taux   constant
$x$. \question{1} Déterminez l'équation différentielle qui caractérise
l'évolution  de  la  technologie  $A(t)$.  Pourquoi  est-il  pertinent
d'introduire un progrès technique (exogène)  dans le modèle de Solow ?
\question{2}  Déterminez la  dynamique du  stock de  capital par  tête
efficace,   \textit{ie}   écrivez    l'équation   différentielle   qui
caractérise  le mouvement  de $\hat{k}=K/(AL)$.  \question{3} Calculez
l'état  stationnaire  du modèle.  Pourquoi  travaillons  nous sur  des
variables   par  tête   efficace   (alors  que   cette  variable   est
inobservable) ?  Quel  est l'impact  d'une augmentation  permanente du
taux d'épargne sur le long  terme de l'économie ? \question{4} Montrez
que, si le niveau initial d'une économie est inférieur à son niveau de
long  terme  ($\hat{k}(0)<\hat{k}^*$)  alors son  taux  de  croissance
positif  diminue  le  long  de la  transition.  Quel  est  l'hypothèse
essentielle  sur  la technologie  qui  explique  ce résultat ?  Donnez
l'intuition.   \question{5}  Caractérisez   la  dynamique  de  $K(t)$,
$k(t)$, $\hat y(t)$ et  $y(t)$ le long de la transition  et le long du
sentier de croissance équilibré. Le  taux de croissance du produit par
tête est-il  nécessairement positif lorsque  le taux de  croissance du
produit par tête efficace est positif (c'est-à-dire lorsque le taux de
croissance  du  stock  de  capital  physique  par  tête  efficace  est
positif) ?  Le  taux   de  croissance  du  produit   par  tête  est-il
nécessairement négatif  lorsque le taux  de croissance du  produit par
tête efficace est négatif ?\newline


\end{document}
