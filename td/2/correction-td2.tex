\documentclass[10pt,a4paper,notitlepage]{report}
\usepackage{amsmath}
\usepackage{amssymb}
\usepackage{amsbsy}
\usepackage{float}
\usepackage[french]{babel}
\usepackage{graphicx}

\usepackage[utf8x]{inputenc}
\usepackage[T1]{fontenc}
\usepackage{palatino}
\usepackage{wrapfig}

\usepackage[active]{srcltx}
\usepackage{scrtime}
\usepackage{nicefrac}

 \usepackage{pgf,tikz}
 \usepackage{pgfplots}
 \pgfplotsset{compat=newest}
 \pgfplotsset{ticks=none}
\pgfplotsset{plot coordinates/math parser=false}
 \usetikzlibrary{arrows}
 \usetikzlibrary{plotmarks}


\newcommand{\exercice}[1]{\textsc{\textbf{Exercice}} #1}
\newcommand{\question}[1]{\textbf{(#1)}}
\setlength{\parindent}{0cm}


\newlength\figureheight
\newlength\figurewidth
\setlength\figureheight{4cm}
\setlength\figurewidth{4cm}



\begin{document}



\title{\textsc{Croissance\\ \small{(Correction de la fiche de TD n°2)}}}
\author{Stéphane Adjemian\thanks{Université du Maine, Gains. \texttt{stephane DOT adjemian AT univ DASH lemans DOT fr}}}
\date{Le \today\ à \thistime}

\maketitle


\exercice{1}  \question{1} Nous  avons déjà  montré dans  la fiche  de
travaux dirigés n°1 que la dynamique  du stock de capital par tête est
caractérisée par l'équation différentielle suivante :
\[
\dot k(t) = sk(t)^{\alpha} - (n+\delta)k(t)
\]
À l'état stationnaire  les variables par tête  sont constantes. Notons
$k^{\star}$  le niveau  d'état stationnaire  du stock  de capital  par
tête. L'état stationnaire est solution de l'équation suivante :
\[
s\left. k^{\star}\right.^{\alpha} = (n+\delta)k^{\star}
\]
À l'état  stationnaire l'investissement par  tête doit être égal  à la
dépréciation du capital par tête.  Cette équation admet deux solutions
en $k^{\star}$. La solution la  plus évidente est $k^{\star}=0$. Si le
stock  de capital  est nul  alors sa  dépréciation est  nécessairement
nulle, par ailleurs, lorsque le stock de capital est nul la production
et donc l'investissement par  tête sont nuls (puisque l'investissement
est une fraction de la  production). Nous écartons cette solution, car
l'économie  serait  alors  réduite  à néant,  et  nous  cherchons  une
solution  strictement positive  pour  $k^{\star}$. En  divisant les  deux
membres de la dernière égalité par $k^{\star}$, nous obtenons :
\[
s\left. k^{\star}\right.^{\alpha-1} = (n+\delta)
\]
À l'état stationnaire, l'investissement par unité de capital doit être
égal au taux de dépréciation. Nous obtenons finalement :
\[
k^{\star}= \left(\frac{s}{n+\delta}\right)^{\frac{1}{1-\alpha}}
\]
On déduit directement l'état stationnaire de la production par tête, en substituant ce résultat dans la fonction de production intensive\footnote{Celle-ci exprime la production par tête en fonction du stock de capital par tête. En effet nous avons :
\[
Y = K^{\alpha}L^{1-\alpha}
\]
En divisant les deux membres par la population :
\[
\frac{Y}{L} = K^{\alpha}L^{-\alpha}
\]
en utilisant les définitions des variables par tête, on a de façon équivalente :
\[
y = k^{\alpha}
\]
} :
\[
y^{\star} = \left.k^{\star}\right.^{\alpha}= \left(\frac{s}{n+\delta}\right)^{\frac{\alpha}{1-\alpha}}
\]
Enfin la consommation par tête à l'état stationnaire est donnée par :
\[
c^{\star} = (1-s)y^{\star} = (1-s)\left(\frac{s}{n+\delta}\right)^{\frac{\alpha}{1-\alpha}}
\]
\question{2} Une augmentation permanente  du taux d'épargne induit une
augmentation de l'état  stationnaire du stock de  capital physique par
tête. En effet, nous avons :
\[
\begin{split}
\frac{\mathrm d k^{\star}}{\mathrm d s} &= \frac{1}{1-\alpha}\frac{1}{n+\delta}\left(\frac{s}{n+\delta}\right)^{\frac{1}{1-\alpha}-1}\\
\Leftrightarrow \frac{\mathrm d k^{\star}}{\mathrm d s} &= \frac{1}{1-\alpha}\frac{1}{n+\delta}\left(\frac{s}{n+\delta}\right)^{\frac{\alpha}{1-\alpha}} > 0
\end{split}
\]
puisque, par l'hypothèse de  rendement marginal positif et décroissant
du capital  physique, nous  avons $0<\alpha<1$. Puisque  la production
par tête est une fonction monotone  croissante du stock de capital par
tête,   l'augmentation    permanente   du   taux    d'épargne   induit
nécessairement une  augmentation de  la production  par tête  à l'état
stationnaire.  En effet,  en utilisant  les dérivées  en chaîne,  nous
obtenons :
\[
\frac{\mathrm d y^{\star}}{\mathrm d s} = \alpha \left. k^{\star} \right. ^{\alpha-1} \frac{\mathrm d k^{\star}}{\mathrm d s} > 0 
\]
Les  conséquences   sur  le  niveau   à  l'état  stationnaire   de  la
consommation sont moins évidentes. En effet, l'augmentation induite de
la production par tête,  $y^{\star}$, s'accompagne d'une diminution de
la  part consommée  du revenu,  $(1-s)$. Pour  conclure il  faut peser
l'augmentation  de la  production avec  la baisse  de la  propension à
consommer. Nous devons donc comparer deux pentes : la pente de la part
de la production consommée, constante et  égale à -1, avec la pente de
la  production, qui  dépend du  niveau  de $s$  et est  donnée par  la
dernière  équation.  Une  augmentation  du taux  d'épargne induit  une
augmentation de la consommation à  l'état stationnaire si et seulement
si la  pente de  la production  est supérieure  à un  (c'est-à-dire la
valeur   absolue  de   la   pente   de  la   part   consommée  de   la
production). Nous donnerons une réponse plus précise en répondant à la
question  suivante.\newline  

\begin{wrapfigure}{r}{0.5\textwidth}
  \begin{center}
    % This file was created by matlab2tikz v0.3.1.
% Copyright (c) 2008--2013, Nico Schlömer <nico.schloemer@gmail.com>
% All rights reserved.
% 
% The latest updates can be retrieved from
%   http://www.mathworks.com/matlabcentral/fileexchange/22022-matlab2tikz
% where you can also make suggestions and rate matlab2tikz.
% 
% 
% 
\begin{tikzpicture}
\begin{axis}[%
width=\figurewidth,
height=\figureheight,
clip=false,
clip=false,
scale only axis,
xmin=0, xmax=30,
ymin=0, ymax=1.4,
axis lines=left
]
\addplot [
color=black,
solid,
forget plot
]
table{
0 0
28.7622646138341 1.15049058455336
};
\addplot [
color=black,
solid,
line width=2.0pt,
forget plot
]
table{
0 0
0.0576398088453589 0.05849647477156
0.115279617690718 0.0735308511502296
0.172919426536077 0.0840581269783267
0.230559235381435 0.0924292633358124
0.288199044226794 0.0994923745172508
0.345838853072153 0.105662189849011
0.403478661917512 0.11117625682918
0.461118470762871 0.116184820210316
0.51875827960823 0.120789649952372
0.576398088453589 0.125063245431119
0.634037897298947 0.129059292704436
0.691677706144306 0.132818785821471
0.749317514989665 0.136373818440655
0.806957323835024 0.139750041763553
0.864597132680383 0.142968318744095
0.922236941525742 0.146045872920782
0.9798767503711 0.148997107893363
1.03751655921646 0.151834205493942
1.09515636806182 0.154567571271115
1.15279617690718 0.157206172168023
1.21043598575254 0.159757796517095
1.26807579459789 0.162229257035804
1.32571560344325 0.164626551318079
1.38335541228861 0.166954990165339
1.44099522113397 0.169219301261099
1.49863502997933 0.171423713714515
1.55627483882469 0.17357202759679
1.61391464767005 0.17566767158677
1.67155445651541 0.177713751107618
1.72919426536077 0.179713088794227
1.78683407420612 0.181668258726159
1.84447388305148 0.183581615555138
1.90211369189684 0.185455319423058
1.9597535007422 0.187291357387029
2.01739330958756 0.189091561928662
2.07503311843292 0.19085762701578
2.13267292727828 0.192591122098734
2.19031273612364 0.194293504355148
2.24795254496899 0.195966129442336
2.30559235381435 0.197610260972586
2.36323216265971 0.199227078890945
2.42087197150507 0.200817686906125
2.47851178035043 0.202383119101385
2.53615158919579 0.203924345832744
2.59379139804115 0.205442279005705
2.65143120688651 0.20693777680823
2.70907101573187 0.208411647966552
2.76671082457722 0.209864655580992
2.82435063342258 0.211297520591054
2.88199044226794 0.21271092491244
2.9396302511133 0.214105514282938
2.99727005995866 0.215481900849347
3.05490986880402 0.216840665523516
3.11254967764938 0.218182360132036
3.17018948649474 0.21950750938113
3.2278292953401 0.220816612655678
3.28546910418545 0.222110145669082
3.34310891303081 0.223388561978719
3.40074872187617 0.224652294380046
3.45838853072153 0.225901756190973
3.51602833956689 0.22713734243679
3.57366814841225 0.228359430944874
3.63130795725761 0.22956838335736
3.68894776610297 0.230764546069133
3.74658757494833 0.23194825109772
3.80422738379368 0.233119816891002
3.86186719263904 0.234279549078042
3.9195070014844 0.235427741167843
3.97714681032976 0.236564675200343
4.03478661917512 0.23769062235355
4.09242642802048 0.238805843510375
4.15006623686584 0.239910589788334
4.2077060457112 0.241005103035066
4.26534585455655 0.242089616292275
4.32298566340191 0.243164354230535
4.38062547224727 0.244229533557136
4.43826528109263 0.245285363398968
4.49590508993799 0.246332045662292
4.55354489878335 0.247369775371055
4.61118470762871 0.248398740985289
4.66882451647407 0.249419124700999
4.72646432531943 0.250431102732833
4.78410413416479 0.251434845580716
4.84174394301014 0.252430518281536
4.8993837518555 0.253418280646903
4.95702356070086 0.254398287487885
5.01466336954622 0.255370688827587
5.07230317839158 0.256335630102374
5.12994298723694 0.257293252352439
5.1875827960823 0.258243692402416
5.24522260492765 0.25918708303266
5.30286241377301 0.260123553141758
5.36050222261837 0.261053227900827
5.41814203146373 0.261976228900091
5.47578184030909 0.262892674288201
5.53342164915445 0.263802678904724
5.59106145799981 0.264706354406219
5.64870126684517 0.265603809386256
5.70634107569053 0.266495149489738
5.76398088453589 0.267380477521846
5.82162069338124 0.268259893551916
5.8792605022266 0.269133495012527
5.93690031107196 0.270001376794065
5.99454011991732 0.270863631335015
6.05217992876268 0.271720348708215
6.10981973760804 0.272571616703274
6.1674595464534 0.273417520905388
6.22509935529876 0.274258144770708
6.28273916414411 0.275093569698474
6.34037897298947 0.275923875100056
6.39801878183483 0.276749138465084
6.45565859068019 0.277569435424793
6.51329839952555 0.278384839812742
6.57093820837091 0.279195423723028
6.62857801721627 0.280001257566122
6.68621782606163 0.280802410122448
6.74385763490699 0.281598948593801
6.80149744375234 0.282390938652733
6.8591372525977 0.283178444489977
6.91677706144306 0.283961528860022
6.97441687028842 0.284740253124917
7.03205667913378 0.285514677296389
7.08969648797914 0.286284860076357
7.1473362968245 0.287050858895901
7.20497610566986 0.287812729952783
7.26261591451521 0.288570528247556
7.32025572336057 0.289324307618347
7.37789553220593 0.290074120774358
7.43553534105129 0.290820019328155
7.49317514989665 0.291562053826782
7.55081495874201 0.292300273781769
7.60845476758737 0.293034727698069
7.66609457643273 0.293765463101971
7.72373438527809 0.294492526568042
7.78137419412344 0.295215963745124
7.8390140029688 0.295935819381442
7.89665381181416 0.296652137348849
7.95429362065952 0.297364960666243
8.01193342950488 0.298074331522196
8.06957323835024 0.29878029129683
8.1272130471956 0.299482880582952
8.18485285604096 0.300182139206508
8.24249266488632 0.300878106246347
8.30013247373167 0.301570820053356
8.35777228257703 0.302260318268971
8.41541209142239 0.302946637843089
8.47305190026775 0.303629815051413
8.53069170911311 0.304309885512247
8.58833151795847 0.304986884202761
8.64597132680383 0.305660845474747
8.70361113564918 0.306331803069887
8.76125094449455 0.306999790134548
8.8188907533399 0.307664839234126
8.87653056218526 0.30832698236695
8.93417037103062 0.308986250977767
8.99181017987598 0.309642675970823
9.04944998872134 0.31029628772255
9.1070897975667 0.310947116093877
9.16472960641206 0.311595190442183
9.22236941525742 0.312240539632892
9.28000922410278 0.312883192050734
9.33764903294813 0.313523175610683
9.39528884179349 0.314160517768575
9.45292865063885 0.314795245531422
9.51056845948421 0.315427385467441
9.56820826832957 0.316056963715791
9.62584807717493 0.316684005996039
9.68348788602029 0.317308537617368
9.74112769486565 0.317930583487516
9.79876750371101 0.318550168121484
9.85640731255636 0.319167315649993
9.91404712140172 0.31978204982772
9.97168693024708 0.320394394041301
10.0293267390924 0.321004371317129
10.0869665479378 0.321612004328934
10.1446063567832 0.322217315405165
10.2022461656285 0.322820326536179
10.2598859744739 0.323421059381236
10.3175257833192 0.324019535275312
10.3751655921646 0.324615775235739
10.43280540101 0.325209799968664
10.4904452098553 0.325801629875354
10.5480850187007 0.326391285058325
10.605724827546 0.326978785327329
10.6633646363914 0.32756415020518
10.7210044452367 0.328147398933434
10.7786442540821 0.328728550477934
10.8362840629275 0.329307623534207
10.8939238717728 0.329884636532732
10.9515636806182 0.330459607644079
11.0092034894635 0.331032554783922
11.0668432983089 0.331603495617925
11.1244831071543 0.332172447566514
11.1821229159996 0.332739427809532
11.239762724845 0.333304453290784
11.2974025336903 0.33386754072247
11.3550423425357 0.334428706589515
11.4126821513811 0.334987967153794
11.4703219602264 0.335545338458265
11.5279617690718 0.336100836330993
11.5856015779171 0.33665447638909
11.6432413867625 0.33720627404256
11.7008811956078 0.337756244498057
11.7585210044532 0.33830440276255
11.8161608132986 0.338850763646918
11.8738006221439 0.339395341769447
11.9314404309893 0.33993815155926
11.9890802398346 0.340479207259664
12.04672004868 0.341018522931424
12.1043598575254 0.341556112455961
12.1619996663707 0.342091989538485
12.2196394752161 0.342626167711053
12.2772792840614 0.343158660335561
12.3349190929068 0.343689480606674
12.3925589017522 0.344218641554686
12.4501987105975 0.344746156048324
12.5078385194429 0.345272036797488
12.5654783282882 0.345796296355934
12.6231181371336 0.346318947123897
12.6807579459789 0.346840001350661
12.7383977548243 0.347359471137073
12.7960375636697 0.347877368438004
12.853677372515 0.34839370506476
12.9113171813604 0.348908492687438
12.9689569902057 0.34942174283724
13.0265967990511 0.349933466908734
13.0842366078965 0.350443676162066
13.1418764167418 0.350952381725135
13.1995162255872 0.351459594595718
13.2571560344325 0.351965325643547
13.3147958432779 0.352469585612358
13.3724356521233 0.35297238512188
13.4300754609686 0.353473734669804
13.487715269814 0.353973644633693
13.5453550786593 0.354472125272869
13.6029948875047 0.354969186730259
13.66063469635 0.355464839034193
13.7182745051954 0.355959092100188
13.7759143140408 0.356451955732677
13.8335541228861 0.356943439626714
13.8911939317315 0.357433553369646
13.9488337405768 0.357922306442751
14.0064735494222 0.358409708222842
14.0641133582676 0.358895767983845
14.1217531671129 0.359380494898343
14.1793929759583 0.35986389803909
14.2370327848036 0.360345986380506
14.294672593649 0.360826768800126
14.3523124024944 0.361306254080038
14.4099522113397 0.361784450908287
14.4675920201851 0.362261367880254
14.5252318290304 0.362737013500007
14.5828716378758 0.363211396181631
14.6405114467211 0.36368452425053
14.6981512555665 0.364156405944709
14.7557910644119 0.364627049416029
14.8134308732572 0.365096462731438
14.8710706821026 0.365564653874187
14.9287104909479 0.366031630745015
14.9863502997933 0.366497401163318
15.0439901086387 0.366961972868295
15.101629917484 0.367425353520076
15.1592697263294 0.367887550700826
15.2169095351747 0.368348571915835
15.2745493440201 0.368808424594579
15.3321891528655 0.369267116091776
15.3898289617108 0.369724653688412
15.4474687705562 0.370181044592753
15.5051085794015 0.370636295941341
15.5627483882469 0.371090414799971
15.6203881970922 0.371543408164652
15.6780280059376 0.371995282962549
15.735667814783 0.372446046052913
15.7933076236283 0.37289570422799
15.8509474324737 0.373344264213922
15.908587241319 0.373791732671625
15.9662270501644 0.374238116197653
16.0238668590098 0.374683421325054
16.0815066678551 0.375127654524207
16.1391464767005 0.375570822203643
16.1967862855458 0.376012930710854
16.2544260943912 0.376453986333095
16.3120659032366 0.376893995298159
16.3697057120819 0.377332963775154
16.4273455209273 0.377770897875256
16.4849853297726 0.378207803652457
16.542625138618 0.378643687104297
16.6002649474633 0.379078554172584
16.6579047563087 0.379512410744105
16.7155445651541 0.379945262651322
16.7731843739994 0.38037711567306
16.8308241828448 0.380807975535184
16.8884639916901 0.38123784791126
16.9461038005355 0.381666738423215
17.0037436093809 0.382094652641974
17.0613834182262 0.3825215960881
17.1190232270716 0.382947574232414
17.1766630359169 0.38337259249661
17.2343028447623 0.38379665625386
17.2919426536077 0.384219770829406
17.349582462453 0.384641941501152
17.4072222712984 0.385063173500234
17.4648620801437 0.385483472011591
17.5225018889891 0.385902842174521
17.5801416978344 0.386321289083238
17.6377815066798 0.386738817787405
17.6954213155252 0.387155433292675
17.7530611243705 0.387571140561212
17.8107009332159 0.387985944512213
17.8683407420612 0.388399850022412
17.9259805509066 0.388812861926587
17.983620359752 0.389224985018054
18.0412601685973 0.38963622404915
18.0988999774427 0.390046583731718
18.156539786288 0.390456068737579
18.2141795951334 0.390864683698995
18.2718194039788 0.391272433209129
18.3294592128241 0.3916793218225
18.3870990216695 0.392085354055424
18.4447388305148 0.392490534386458
18.5023786393602 0.392894867256827
18.5600184482056 0.393298357070855
18.6176582570509 0.393701008196383
18.6752980658963 0.394102824965185
18.7329378747416 0.394503811673373
18.790577683587 0.394903972581801
18.8482174924323 0.395303311916463
18.9058573012777 0.39570183386888
18.9634971101231 0.39609954259649
19.0211369189684 0.396496442223025
19.0787767278138 0.396892536838885
19.1364165366591 0.397287830501509
19.1940563455045 0.397682327235737
19.2516961543499 0.398076031034172
19.3093359631952 0.398468945857531
19.3669757720406 0.398861075634994
19.4246155808859 0.399252424264551
19.4822553897313 0.399642995613337
19.5398951985766 0.40003279351797
19.597535007422 0.400421821784882
19.6551748162674 0.40081008419064
19.7128146251127 0.401197584482272
19.7704544339581 0.401584326377578
19.8280942428034 0.40197031356545
19.8857340516488 0.402355549706173
19.9433738604942 0.402740038431733
20.0010136693395 0.403123783346115
20.0586534781849 0.403506788025603
20.1162932870302 0.403889056019063
20.1739330958756 0.404270590848242
20.231572904721 0.404651396008043
20.2892127135663 0.405031474966811
20.3468525224117 0.405410831166606
20.404492331257 0.405789468023481
20.4621321401024 0.406167388927744
20.5197719489477 0.406544597244232
20.5774117577931 0.406921096312565
20.6350515666385 0.407296889447415
20.6926913754838 0.407671979938749
20.7503311843292 0.408046371052095
20.8079709931745 0.40842006602878
20.8656108020199 0.408793068086179
20.9232506108653 0.409165380417963
20.9808904197106 0.40953700619433
21.038530228556 0.409907948562247
21.0961700374013 0.410278210645681
21.1538098462467 0.410647795545831
21.2114496550921 0.411016706341355
21.2690894639374 0.411384946088592
21.3267292727828 0.411752517821788
21.3843690816281 0.412119424553313
21.4420088904735 0.412485669273877
21.4996486993189 0.412851254952742
21.5572885081642 0.413216184537937
21.6149283170096 0.413580460956463
21.6725681258549 0.413944087114497
21.7302079347003 0.414307065897601
21.7878477435456 0.414669400170916
21.845487552391 0.415031092779364
21.9031273612364 0.415392146547843
21.9607671700817 0.41575256428142
22.0184069789271 0.41611234876552
22.0760467877724 0.416471502766118
22.1336865966178 0.416830029029923
22.1913264054632 0.417187930284561
22.2489662143085 0.417545209238759
22.3066060231539 0.417901868582525
22.3642458319992 0.418257910987323
22.4218856408446 0.418613339106248
22.47952544969 0.418968155574202
22.5371652585353 0.419322363008064
22.5948050673807 0.419675964006856
22.652444876226 0.420028961151916
22.7100846850714 0.420381357007056
22.7677244939167 0.420733154118732
22.8253643027621 0.421084355016198
22.8830041116075 0.421434962211674
22.9406439204528 0.421784978200493
22.9982837292982 0.422134405461266
23.0559235381435 0.42248324645603
23.1135633469889 0.422831503630403
23.1712031558343 0.423179179413729
23.2288429646796 0.423526276219236
23.286482773525 0.423872796444171
23.3441225823703 0.424218742469956
23.4017623912157 0.424564116662322
23.4594022000611 0.424908921371458
23.5170420089064 0.425253158932145
23.5746818177518 0.425596831663901
23.6323216265971 0.425939941871111
23.6899614354425 0.426282491843168
23.7476012442878 0.426624483854602
23.8052410531332 0.426965920165217
23.8628808619786 0.427306803020218
23.9205206708239 0.427647134650343
23.9781604796693 0.427986917271988
24.0358002885146 0.428326153087335
24.09344009736 0.428664844284478
24.1510799062054 0.429002993037546
24.2087197150507 0.429340601506824
24.2663595238961 0.429677671838874
24.3239993327414 0.430014206166657
24.3816391415868 0.430350206609647
24.4392789504322 0.430685675273953
24.4969187592775 0.431020614252429
24.5545585681229 0.431355025624794
24.6121983769682 0.431688911457741
24.6698381858136 0.43202227380505
24.7274779946589 0.432355114707699
24.7851178035043 0.432687436193975
24.8427576123497 0.433019240279578
24.900397421195 0.433350528967732
24.9580372300404 0.433681304249289
25.0156770388857 0.434011568102835
25.0733168477311 0.434341322494793
25.1309566565765 0.434670569379523
25.1885964654218 0.43499931069943
25.2462362742672 0.435327548385057
25.3038760831125 0.435655284355189
25.3615158919579 0.43598252051695
25.4191557008033 0.4363092587659
25.4767955096486 0.43663550098613
25.534435318494 0.43696124905036
25.5920751273393 0.437286504820029
25.6497149361847 0.437611270145394
25.70735474503 0.437935546865616
25.7649945538754 0.438259336808856
25.8226343627208 0.438582641792361
25.8802741715661 0.438905463622558
25.9379139804115 0.439227804095138
25.9955537892568 0.439549664995147
26.0531935981022 0.43987104809707
26.1108334069476 0.440191955164918
26.1684732157929 0.440512387952313
26.2261130246383 0.440832348202571
26.2837528334836 0.441151837648785
26.341392642329 0.441470858013909
26.3990324511744 0.441789411010837
26.4566722600197 0.442107498342484
26.5143120688651 0.442425121701869
26.5719518777104 0.442742282772189
26.6295916865558 0.4430589832269
26.6872314954011 0.443375224729793
26.7448713042465 0.443691008935073
26.8025111130919 0.444006337487432
26.8601509219372 0.444321212022123
26.9177907307826 0.444635634165041
26.9754305396279 0.444949605532789
27.0330703484733 0.445263127732752
27.0907101573187 0.445576202363174
27.148349966164 0.445888831013225
27.2059897750094 0.446201015263071
27.2636295838547 0.446512756683949
27.3212693927001 0.44682405683823
27.3789092015455 0.447134917279491
27.4365490103908 0.447445339552584
27.4941888192362 0.4477553251937
27.5518286280815 0.448064875730437
27.6094684369269 0.448373992681869
27.6671082457722 0.448682677558604
27.7247480546176 0.448990931862856
27.782387863463 0.449298757088507
27.8400276723083 0.449606154721166
27.8976674811537 0.449913126238239
27.955307289999 0.450219673108984
28.0129470988444 0.450525796794579
28.0705869076898 0.450831498748179
28.1282267165351 0.451136780414976
28.1858665253805 0.451441643232264
28.2435063342258 0.45174608862949
28.3011461430712 0.452050118028321
28.3587859519166 0.452353732842696
28.4164257607619 0.452656934478888
28.4740655696073 0.452959724335557
28.5317053784526 0.453262103803809
28.589345187298 0.453564074267253
28.6469849961433 0.453865637102055
28.7046248049887 0.454166793676989
28.7622646138341 0.454467545353501
};
\addplot [
color=red,
solid,
line width=2.0pt,
forget plot
]
table{
0 0
0.0576398088453589 0.11699294954312
0.115279617690718 0.147061702300459
0.172919426536077 0.168116253956653
0.230559235381435 0.184858526671625
0.288199044226794 0.198984749034502
0.345838853072153 0.211324379698022
0.403478661917512 0.22235251365836
0.461118470762871 0.232369640420633
0.51875827960823 0.241579299904744
0.576398088453589 0.250126490862238
0.634037897298947 0.258118585408872
0.691677706144306 0.265637571642942
0.749317514989665 0.272747636881311
0.806957323835024 0.279500083527106
0.864597132680383 0.28593663748819
0.922236941525742 0.292091745841563
0.9798767503711 0.297994215786726
1.03751655921646 0.303668410987883
1.09515636806182 0.309135142542231
1.15279617690718 0.314412344336045
1.21043598575254 0.31951559303419
1.26807579459789 0.324458514071607
1.32571560344325 0.329253102636159
1.38335541228861 0.333909980330679
1.44099522113397 0.338438602522198
1.49863502997933 0.342847427429031
1.55627483882469 0.347144055193579
1.61391464767005 0.35133534317354
1.67155445651541 0.355427502215235
1.72919426536077 0.359426177588455
1.78683407420612 0.363336517452319
1.84447388305148 0.367163231110276
1.90211369189684 0.370910638846116
1.9597535007422 0.374582714774058
2.01739330958756 0.378183123857323
2.07503311843292 0.38171525403156
2.13267292727828 0.385182244197467
2.19031273612364 0.388587008710296
2.24795254496899 0.391932258884672
2.30559235381435 0.395220521945171
2.36323216265971 0.39845415778189
2.42087197150507 0.40163537381225
2.47851178035043 0.404766238202769
2.53615158919579 0.407848691665489
2.59379139804115 0.41088455801141
2.65143120688651 0.41387555361646
2.70907101573187 0.416823295933105
2.76671082457722 0.419729311161984
2.82435063342258 0.422595041182108
2.88199044226794 0.42542184982488
2.9396302511133 0.428211028565876
2.99727005995866 0.430963801698694
3.05490986880402 0.433681331047032
3.11254967764938 0.436364720264073
3.17018948649474 0.43901501876226
3.2278292953401 0.441633225311356
3.28546910418545 0.444220291338165
3.34310891303081 0.446777123957437
3.40074872187617 0.449304588760092
3.45838853072153 0.451803512381945
3.51602833956689 0.45427468487358
3.57366814841225 0.456718861889749
3.63130795725761 0.459136766714721
3.68894776610297 0.461529092138265
3.74658757494833 0.46389650219544
3.80422738379368 0.466239633782004
3.86186719263904 0.468559098156084
3.9195070014844 0.470855482335687
3.97714681032976 0.473129350400685
4.03478661917512 0.4753812447071
4.09242642802048 0.477611687020749
4.15006623686584 0.479821179576669
4.2077060457112 0.482010206070132
4.26534585455655 0.484179232584549
4.32298566340191 0.486328708461071
4.38062547224727 0.488459067114273
4.43826528109263 0.490570726797937
4.49590508993799 0.492664091324585
4.55354489878335 0.494739550742111
4.61118470762871 0.496797481970578
4.66882451647407 0.498838249401998
4.72646432531943 0.500862205465667
4.78410413416479 0.502869691161432
4.84174394301014 0.504861036563072
4.8993837518555 0.506836561293807
4.95702356070086 0.508796574975769
5.01466336954622 0.510741377655175
5.07230317839158 0.512671260204749
5.12994298723694 0.514586504704878
5.1875827960823 0.516487384804833
5.24522260492765 0.518374166065321
5.30286241377301 0.520247106283516
5.36050222261837 0.522106455801654
5.41814203146373 0.523952457800183
5.47578184030909 0.525785348576402
5.53342164915445 0.527605357809448
5.59106145799981 0.529412708812438
5.64870126684517 0.531207618772512
5.70634107569053 0.532990298979475
5.76398088453589 0.534760955043691
5.82162069338124 0.536519787103832
5.8792605022266 0.538266990025054
5.93690031107196 0.54000275358813
5.99454011991732 0.541727262670031
6.05217992876268 0.543440697416429
6.10981973760804 0.545143233406548
6.1674595464534 0.546835041810775
6.22509935529876 0.548516289541416
6.28273916414411 0.550187139396947
6.34037897298947 0.551847750200112
6.39801878183483 0.553498276930169
6.45565859068019 0.555138870849587
6.51329839952555 0.556769679625484
6.57093820837091 0.558390847446056
6.62857801721627 0.560002515132245
6.68621782606163 0.561604820244895
6.74385763490699 0.563197897187602
6.80149744375234 0.564781877305466
6.8591372525977 0.566356888979954
6.91677706144306 0.567923057720044
6.97441687028842 0.569480506249833
7.03205667913378 0.571029354592779
7.08969648797914 0.572569720152713
7.1473362968245 0.574101717791802
7.20497610566986 0.575625459905566
7.26261591451521 0.577141056495112
7.32025572336057 0.578648615236694
7.37789553220593 0.580148241548717
7.43553534105129 0.58164003865631
7.49317514989665 0.583124107653563
7.55081495874201 0.584600547563538
7.60845476758737 0.586069455396138
7.66609457643273 0.587530926203942
7.72373438527809 0.588985053136083
7.78137419412344 0.590431927490247
7.8390140029688 0.591871638762885
7.89665381181416 0.593304274697699
7.95429362065952 0.594729921332486
8.01193342950488 0.596148663044393
8.06957323835024 0.597560582593659
8.1272130471956 0.598965761165905
8.18485285604096 0.600364278413015
8.24249266488632 0.601756212492693
8.30013247373167 0.603141640106713
8.35777228257703 0.604520636537943
8.41541209142239 0.605893275686178
8.47305190026775 0.607259630102826
8.53069170911311 0.608619771024493
8.58833151795847 0.609973768405521
8.64597132680383 0.611321690949493
8.70361113564918 0.612663606139773
8.76125094449455 0.613999580269096
8.8188907533399 0.615329678468252
8.87653056218526 0.6166539647339
8.93417037103062 0.617972501955535
8.99181017987598 0.619285351941647
9.04944998872134 0.620592575445099
9.1070897975667 0.621894232187754
9.16472960641206 0.623190380884366
9.22236941525742 0.624481079265783
9.28000922410278 0.625766384101468
9.33764903294813 0.627046351221367
9.39528884179349 0.628321035537149
9.45292865063885 0.629590491062843
9.51056845948421 0.630854770934881
9.56820826832957 0.632113927431581
9.62584807717493 0.633368011992079
9.68348788602029 0.634617075234736
9.74112769486565 0.635861166975032
9.79876750371101 0.637100336242968
9.85640731255636 0.638334631299987
9.91404712140172 0.63956409965544
9.97168693024708 0.640788788082603
10.0293267390924 0.642008742634259
10.0869665479378 0.643224008657868
10.1446063567832 0.64443463081033
10.2022461656285 0.645640653072358
10.2598859744739 0.646842118762471
10.3175257833192 0.648039070550624
10.3751655921646 0.649231550471477
10.43280540101 0.650419599937328
10.4904452098553 0.651603259750707
10.5480850187007 0.65278257011665
10.605724827546 0.653957570654658
10.6633646363914 0.65512830041036
10.7210044452367 0.656294797866868
10.7786442540821 0.657457100955869
10.8362840629275 0.658615247068414
10.8939238717728 0.659769273065463
10.9515636806182 0.660919215288159
11.0092034894635 0.662065109567845
11.0668432983089 0.663206991235851
11.1244831071543 0.664344895133028
11.1821229159996 0.665478855619065
11.239762724845 0.666608906581569
11.2974025336903 0.66773508144494
11.3550423425357 0.668857413179029
11.4126821513811 0.669975934307589
11.4703219602264 0.67109067691653
11.5279617690718 0.672201672661986
11.5856015779171 0.67330895277818
11.6432413867625 0.674412548085121
11.7008811956078 0.675512488996114
11.7585210044532 0.676608805525101
11.8161608132986 0.677701527293836
11.8738006221439 0.678790683538894
11.9314404309893 0.679876303118521
11.9890802398346 0.680958414519328
12.04672004868 0.682037045862847
12.1043598575254 0.683112224911921
12.1619996663707 0.68418397907697
12.2196394752161 0.685252335422106
12.2772792840614 0.686317320671122
12.3349190929068 0.687378961213348
12.3925589017522 0.688437283109372
12.4501987105975 0.689492312096648
12.5078385194429 0.690544073594976
12.5654783282882 0.691592592711868
12.6231181371336 0.692637894247794
12.6807579459789 0.693680002701322
12.7383977548243 0.694718942274146
12.7960375636697 0.695754736876009
12.853677372515 0.69678741012952
12.9113171813604 0.697816985374876
12.9689569902057 0.698843485674481
13.0265967990511 0.699866933817467
13.0842366078965 0.700887352324131
13.1418764167418 0.70190476345027
13.1995162255872 0.702919189191435
13.2571560344325 0.703930651287094
13.3147958432779 0.704939171224715
13.3724356521233 0.705944770243761
13.4300754609686 0.706947469339608
13.487715269814 0.707947289267385
13.5453550786593 0.708944250545739
13.6029948875047 0.709938373460517
13.66063469635 0.710929678068386
13.7182745051954 0.711918184200376
13.7759143140408 0.712903911465353
13.8335541228861 0.713886879253427
13.8911939317315 0.714867106739292
13.9488337405768 0.715844612885502
14.0064735494222 0.716819416445685
14.0641133582676 0.71779153596769
14.1217531671129 0.718760989796685
14.1793929759583 0.719727796078181
14.2370327848036 0.720691972761012
14.294672593649 0.721653537600251
14.3523124024944 0.722612508160076
14.4099522113397 0.723568901816575
14.4675920201851 0.724522735760509
14.5252318290304 0.725474027000015
14.5828716378758 0.726422792363262
14.6405114467211 0.72736904850106
14.6981512555665 0.728312811889418
14.7557910644119 0.729254098832057
14.8134308732572 0.730192925462876
14.8710706821026 0.731129307748374
14.9287104909479 0.73206326149003
14.9863502997933 0.732994802326635
15.0439901086387 0.733923945736589
15.101629917484 0.734850707040151
15.1592697263294 0.735775101401653
15.2169095351747 0.736697143831669
15.2745493440201 0.737616849189158
15.3321891528655 0.738534232183553
15.3898289617108 0.739449307376825
15.4474687705562 0.740362089185506
15.5051085794015 0.741272591882682
15.5627483882469 0.742180829599943
15.6203881970922 0.743086816329304
15.6780280059376 0.743990565925098
15.735667814783 0.744892092105825
15.7933076236283 0.74579140845598
15.8509474324737 0.746688528427845
15.908587241319 0.747583465343249
15.9662270501644 0.748476232395305
16.0238668590098 0.749366842650108
16.0815066678551 0.750255309048414
16.1391464767005 0.751141644407285
16.1967862855458 0.752025861421708
16.2544260943912 0.752907972666189
16.3120659032366 0.753787990596317
16.3697057120819 0.754665927550307
16.4273455209273 0.755541795750512
16.4849853297726 0.756415607304915
16.542625138618 0.757287374208595
16.6002649474633 0.758157108345169
16.6579047563087 0.75902482148821
16.7155445651541 0.759890525302643
16.7731843739994 0.76075423134612
16.8308241828448 0.761615951070367
16.8884639916901 0.762475695822521
16.9461038005355 0.76333347684643
17.0037436093809 0.764189305283949
17.0613834182262 0.765043192176201
17.1190232270716 0.765895148464829
17.1766630359169 0.76674518499322
17.2343028447623 0.767593312507719
17.2919426536077 0.768439541658813
17.349582462453 0.769283883002305
17.4072222712984 0.770126347000469
17.4648620801437 0.770966944023181
17.5225018889891 0.771805684349043
17.5801416978344 0.772642578166475
17.6377815066798 0.77347763557481
17.6954213155252 0.774310866585349
17.7530611243705 0.775142281122425
17.8107009332159 0.775971889024426
17.8683407420612 0.776799700044824
17.9259805509066 0.777625723853175
17.983620359752 0.778449970036107
18.0412601685973 0.7792724480983
18.0988999774427 0.780093167463437
18.156539786288 0.780912137475159
18.2141795951334 0.78172936739799
18.2718194039788 0.782544866418259
18.3294592128241 0.783358643645
18.3870990216695 0.784170708110849
18.4447388305148 0.784981068772916
18.5023786393602 0.785789734513653
18.5600184482056 0.786596714141709
18.6176582570509 0.787402016392767
18.6752980658963 0.78820564993037
18.7329378747416 0.789007623346746
18.790577683587 0.789807945163602
18.8482174924323 0.790606623832925
18.9058573012777 0.79140366773776
18.9634971101231 0.79219908519298
19.0211369189684 0.79299288444605
19.0787767278138 0.79378507367777
19.1364165366591 0.794575661003018
19.1940563455045 0.795364654471475
19.2516961543499 0.796152062068345
19.3093359631952 0.796937891715062
19.3669757720406 0.797722151269988
19.4246155808859 0.798504848529101
19.4822553897313 0.799285991226673
19.5398951985766 0.80006558703594
19.597535007422 0.800843643569764
19.6551748162674 0.801620168381281
19.7128146251127 0.802395168964543
19.7704544339581 0.803168652755156
19.8280942428034 0.803940627130899
19.8857340516488 0.804711099412345
19.9433738604942 0.805480076863465
20.0010136693395 0.806247566692231
20.0586534781849 0.807013576051205
20.1162932870302 0.807778112038127
20.1739330958756 0.808541181696484
20.231572904721 0.809302792016086
20.2892127135663 0.810062949933621
20.3468525224117 0.810821662333213
20.404492331257 0.811578936046962
20.4621321401024 0.812334777855488
20.5197719489477 0.813089194488463
20.5774117577931 0.813842192625131
20.6350515666385 0.814593778894829
20.6926913754838 0.815343959877499
20.7503311843292 0.81609274210419
20.8079709931745 0.816840132057559
20.8656108020199 0.817586136172359
20.9232506108653 0.818330760835926
20.9808904197106 0.81907401238866
21.038530228556 0.819815897124494
21.0961700374013 0.820556421291362
21.1538098462467 0.821295591091662
21.2114496550921 0.822033412682709
21.2690894639374 0.822769892177183
21.3267292727828 0.823505035643576
21.3843690816281 0.824238849106626
21.4420088904735 0.824971338547754
21.4996486993189 0.825702509905485
21.5572885081642 0.826432369075875
21.6149283170096 0.827160921912925
21.6725681258549 0.827888174228994
21.7302079347003 0.828614131795202
21.7878477435456 0.829338800341832
21.845487552391 0.830062185558728
21.9031273612364 0.830784293095687
21.9607671700817 0.83150512856284
22.0184069789271 0.832224697531041
22.0760467877724 0.832943005532237
22.1336865966178 0.833660058059845
22.1913264054632 0.834375860569121
22.2489662143085 0.835090418477518
22.3066060231539 0.83580373716505
22.3642458319992 0.836515821974645
22.4218856408446 0.837226678212495
22.47952544969 0.837936311148404
22.5371652585353 0.838644726016127
22.5948050673807 0.839351928013712
22.652444876226 0.840057922303831
22.7100846850714 0.840762714014112
22.7677244939167 0.841466308237463
22.8253643027621 0.842168710032397
22.8830041116075 0.842869924423347
22.9406439204528 0.843569956400986
22.9982837292982 0.844268810922532
23.0559235381435 0.844966492912061
23.1135633469889 0.845663007260805
23.1712031558343 0.846358358827459
23.2288429646796 0.847052552438472
23.286482773525 0.847745592888343
23.3441225823703 0.848437484939912
23.4017623912157 0.849128233324644
23.4594022000611 0.849817842742915
23.5170420089064 0.850506317864291
23.5746818177518 0.851193663327802
23.6323216265971 0.851879883742222
23.6899614354425 0.852564983686335
23.7476012442878 0.853248967709204
23.8052410531332 0.853931840330434
23.8628808619786 0.854613606040437
23.9205206708239 0.855294269300686
23.9781604796693 0.855973834543975
24.0358002885146 0.856652306174669
24.09344009736 0.857329688568957
24.1510799062054 0.858005986075093
24.2087197150507 0.858681203013648
24.2663595238961 0.859355343677749
24.3239993327414 0.860028412333314
24.3816391415868 0.860700413219294
24.4392789504322 0.861371350547905
24.4969187592775 0.862041228504858
24.5545585681229 0.862710051249588
24.6121983769682 0.863377822915481
24.6698381858136 0.8640445476101
24.7274779946589 0.864710229415399
24.7851178035043 0.865374872387951
24.8427576123497 0.866038480559157
24.900397421195 0.866701057935465
24.9580372300404 0.867362608498579
25.0156770388857 0.868023136205671
25.0733168477311 0.868682644989585
25.1309566565765 0.869341138759046
25.1885964654218 0.869998621398859
25.2462362742672 0.870655096770113
25.3038760831125 0.871310568710378
25.3615158919579 0.8719650410339
25.4191557008033 0.8726185175318
25.4767955096486 0.87327100197226
25.534435318494 0.873922498100719
25.5920751273393 0.874573009640059
25.6497149361847 0.875222540290788
25.70735474503 0.875871093731233
25.7649945538754 0.876518673617712
25.8226343627208 0.877165283584723
25.8802741715661 0.877810927245116
25.9379139804115 0.878455608190276
25.9955537892568 0.879099329990294
26.0531935981022 0.87974209619414
26.1108334069476 0.880383910329836
26.1684732157929 0.881024775904626
26.2261130246383 0.881664696405142
26.2837528334836 0.88230367529757
26.341392642329 0.882941716027818
26.3990324511744 0.883578822021673
26.4566722600197 0.884214996684969
26.5143120688651 0.884850243403739
26.5719518777104 0.885484565544379
26.6295916865558 0.8861179664538
26.6872314954011 0.886750449459587
26.7448713042465 0.887382017870147
26.8025111130919 0.888012674974863
26.8601509219372 0.888642424044247
26.9177907307826 0.889271268330083
26.9754305396279 0.889899211065577
27.0330703484733 0.890526255465504
27.0907101573187 0.891152404726348
27.148349966164 0.891777662026449
27.2059897750094 0.892402030526143
27.2636295838547 0.893025513367898
27.3212693927001 0.893648113676459
27.3789092015455 0.894269834558982
27.4365490103908 0.894890679105168
27.4941888192362 0.895510650387399
27.5518286280815 0.896129751460875
27.6094684369269 0.896747985363737
27.6671082457722 0.897365355117207
27.7247480546176 0.897981863725713
27.782387863463 0.898597514177014
27.8400276723083 0.899212309442332
27.8976674811537 0.899826252476477
27.955307289999 0.900439346217968
28.0129470988444 0.901051593589158
28.0705869076898 0.901662997496357
28.1282267165351 0.902273560829953
28.1858665253805 0.902883286464528
28.2435063342258 0.90349217725898
28.3011461430712 0.904100236056642
28.3587859519166 0.904707465685393
28.4164257607619 0.905313868957775
28.4740655696073 0.905919448671113
28.5317053784526 0.906524207607618
28.589345187298 0.907128148534507
28.6469849961433 0.907731274204109
28.7046248049887 0.908333587353979
28.7622646138341 0.908935090707002
};
\addplot [
color=black,
dashed,
forget plot
]
table{
7.19056615345852 0
7.19056615345852 0.287622646138341
};
\addplot [
color=black,
dashed,
forget plot
]
table{
20.2330608267779 0
20.2330608267779 0.809322433071114
};
\addplot [
color=black,
only marks,
mark=triangle,
mark options={solid,,rotate=270},
forget plot
]
table{
8.19056615345852 0
};
\addplot [
color=black,
only marks,
mark=triangle,
mark options={solid,,rotate=270},
forget plot
]
table{
9.19056615345852 0
};
\addplot [
color=black,
only marks,
mark=triangle,
mark options={solid,,rotate=270},
forget plot
]
table{
10.1905661534585 0
};
\addplot [
color=black,
only marks,
mark=triangle,
mark options={solid,,rotate=270},
forget plot
]
table{
11.1905661534585 0
};
\addplot [
color=black,
only marks,
mark=triangle,
mark options={solid,,rotate=270},
forget plot
]
table{
12.1905661534585 0
};
\addplot [
color=black,
only marks,
mark=triangle,
mark options={solid,,rotate=270},
forget plot
]
table{
13.1905661534585 0
};
\addplot [
color=black,
only marks,
mark=triangle,
mark options={solid,,rotate=270},
forget plot
]
table{
14.1905661534585 0
};
\addplot [
color=black,
only marks,
mark=triangle,
mark options={solid,,rotate=270},
forget plot
]
table{
15.1905661534585 0
};
\addplot [
color=black,
only marks,
mark=triangle,
mark options={solid,,rotate=270},
forget plot
]
table{
16.1905661534585 0
};
\addplot [
color=black,
only marks,
mark=triangle,
mark options={solid,,rotate=270},
forget plot
]
table{
17.1905661534585 0
};
\addplot [
color=black,
only marks,
mark=triangle,
mark options={solid,,rotate=270},
forget plot
]
table{
18.1905661534585 0
};
\addplot [
color=black,
only marks,
mark=triangle,
mark options={solid,,rotate=270},
forget plot
]
table{
19.1905661534585 0
};
\node[right, inner sep=0mm, text=black]
at (axis cs:7.19056615345852, -0.1, 0) {$k(0)$};
\node[right, inner sep=0mm, text=black]
at (axis cs:20.2330608267779, -0.1, 0) {$k^{\star}$};
\node[right, inner sep=0mm, text=black]
at (axis cs:28.8622646138341, 0.454467545353501, 0) {$\underline{s}k^{\alpha}$};
\node[right, inner sep=0mm, text=black]
at (axis cs:28.8622646138341, 0.908935090707002, 0) {$\overline{s}k^{\alpha}$};
\end{axis}

\end{tikzpicture}%
  \end{center}
  \vspace{-20pt}
  \caption{\textbf{Transition suite à une augmentation du taux d'épargne}}
  \vspace{-10pt}
\end{wrapfigure}
L'augmentation permanente du taux d'épargne induit une augmentation de
$k^{\star}$ et  de $y^{\star}$.   Si l'économie se  situe initialement
sur  un état  stationnaire associé  à un  taux d'épargne  plus faible,
alors l'accroissement  du taux d'épargne  va générer une  dynamique de
transition   pour    amener   l'économie    vers   le    nouvel   état
stationnaire. Durant cette transition le  stock de capital par tête et
la  production  augmentent.  Au   moment  de  l'augmentation  du  taux
d'épargne, le  niveau de la  consommation par tête chute,  puisque les
ménages épargnent plus  alors que la production commence  tout juste à
s'ajuster,  certes à  la  hausse, vers  le  nouvel état  stationnaire.
Puis, en suivant  la croissance de la production,  la consommation par
tête  augmente  de  façon  continue pour  rejoindre  son  nouvel  état
stationnaire, dont  on ne  peut dire  pour l'instant  si il  sera plus
élevé ou moins élevé que la condition initiale de la consommation.\newline


\question{3} La consommation à l'état stationnaire est une fonction du taux d'épargne :
\[
c^{\star}(s) = (1-s)\left(\frac{s}{n+\delta}\right)^{\frac{\alpha}{1-\alpha}}
\]
On vérifie  facilement que  la consommation de  long terme  (le niveau
d'état stationnaire) est nulle si le taux d'épargne est nul. En effet,
si le taux  d'épargne est nul, alors le stock  de capital physique par
tête à  long terme  est nul (en  l'absence d'investissement,  le stock
s'évapore,  du  fait de  la  dépréciation,  jusqu'à  ce que  le  stock
disparaisse) et donc la production à  long terme est nulle. Puisque la
consommation  est   une  fraction  constante  de   la  production,  la
consommation est forcément nulle à long terme dans ce cas. Dans le cas
\begin{wrapfigure}{r}{0.5\textwidth}
  \begin{center}
    % This file was created by matlab2tikz v0.3.1.
% Copyright (c) 2008--2013, Nico Schlömer <nico.schloemer@gmail.com>
% All rights reserved.
% 
% The latest updates can be retrieved from
%   http://www.mathworks.com/matlabcentral/fileexchange/22022-matlab2tikz
% where you can also make suggestions and rate matlab2tikz.
% 
% 
% 
\begin{tikzpicture}

\begin{axis}[%
width=\figurewidth,
height=\figureheight,
clip=false,
scale only axis,
xmin=0, xmax=1.1,
ymin=0, ymax=2,
axis lines=left
]
\addplot [
color=black,
solid,
line width=2.0pt,
forget plot
]
table{
0 0
0.0050251256281407 0.358161427099632
0.0100502512562814 0.501358466196396
0.0150753768844221 0.609073513444695
0.0201005025125628 0.698208680011529
0.0251256281407035 0.775325663728448
0.0301507537688442 0.84379987534319
0.0351758793969849 0.905643279058187
0.0402010050251256 0.962171535583536
0.0452261306532663 1.01430195949721
0.050251256281407 1.062705037312
0.0552763819095477 1.10788893836283
0.0603015075376884 1.15025004715823
0.0653266331658292 1.19010486859178
0.0703517587939698 1.22771107485754
0.0753768844221105 1.26328189985327
0.0804020100502513 1.29699628642506
0.085427135678392 1.32900622636406
0.0904522613065327 1.3594421890503
0.0954773869346734 1.38841721491915
0.100502512562814 1.41603005510117
0.105527638190955 1.44236761606971
0.110552763819095 1.46750688892605
0.115577889447236 1.49151649047692
0.120603015075377 1.51445790772598
0.125628140703518 1.53638651286172
0.130653266331658 1.55735239857545
0.135678391959799 1.57740107122137
0.14070351758794 1.59657403039889
0.14572864321608 1.6149092569751
0.150753768844221 1.63244162668398
0.155778894472362 1.64920326276669
0.160804020100503 1.66522383832441
0.165829145728643 1.68053083691166
0.170854271356784 1.69514977823646
0.175879396984925 1.70910441453518
0.180904522613065 1.72241690216763
0.185929648241206 1.73510795216598
0.190954773869347 1.74719696282259
0.195979899497487 1.75870213687991
0.201005025125628 1.76964058546338
0.206030150753769 1.78002842055392
0.21105527638191 1.7898808375154
0.21608040201005 1.79921218896039
0.221105527638191 1.8080360510457
0.226130653266332 1.81636528313004
0.231155778894472 1.82421208159278
0.236180904522613 1.83158802850141
0.241206030150754 1.83850413572127
0.246231155778894 1.84497088498154
0.251256281407035 1.85099826434449
0.256281407035176 1.85659580146713
0.261306532663317 1.86177259399572
0.266331658291457 1.86653733739155
0.271356783919598 1.87089835045004
0.276381909547739 1.87486359874428
0.281407035175879 1.87844071619714
0.28643216080402 1.88163702496247
0.291457286432161 1.88445955377597
0.296482412060302 1.88691505491816
0.301507537688442 1.88901001991663
0.306532663316583 1.89075069410109
0.311557788944724 1.89214309011281
0.316582914572864 1.8931930004595
0.321608040201005 1.8939060091975
0.326633165829146 1.8942875028148
0.331658291457286 1.89434268038131
0.336683417085427 1.89407656302633
0.341708542713568 1.8934940027973
0.346733668341709 1.89259969094893
0.351758793969849 1.89139816570736
0.35678391959799 1.88989381954956
0.361809045226131 1.88809090603501
0.366834170854271 1.88599354622298
0.371859296482412 1.88360573470615
0.376884422110553 1.88093134528836
0.381909547738693 1.87797413633226
0.386934673366834 1.87473775579993
0.391959798994975 1.87122574600828
0.396984924623116 1.86744154811868
0.402010050251256 1.86338850637903
0.407035175879397 1.85906987213488
0.412060301507538 1.85448880762496
0.417085427135678 1.84964838957525
0.422110552763819 1.84455161260465
0.42713567839196 1.83920139245432
0.4321608040201 1.8336005690518
0.437185929648241 1.82775190942026
0.442211055276382 1.82165811044247
0.447236180904523 1.81532180148832
0.452261306532663 1.80874554691413
0.457286432160804 1.80193184844139
0.462311557788945 1.79488314742215
0.467336683417085 1.78760182699744
0.472361809045226 1.78009021415516
0.477386934673367 1.77235058169297
0.482412060301508 1.76438515009171
0.487437185929648 1.75619608930415
0.492462311557789 1.74778552046402
0.49748743718593 1.73915551751934
0.50251256281407 1.73030810879452
0.507537688442211 1.72124527848471
0.512562814070352 1.71196896808629
0.517587939698492 1.70248107776653
0.522613065326633 1.69278346767595
0.527638190954774 1.682877959206
0.532663316582915 1.67276633619509
0.537688442211055 1.66245034608537
0.542713567839196 1.65193170103293
0.547738693467337 1.64121207897359
0.552763819095477 1.63029312464647
0.557788944723618 1.61917645057746
0.562814070351759 1.60786363802445
0.5678391959799 1.5963562378862
0.57286432160804 1.58465577157652
0.577889447236181 1.57276373186542
0.582914572864322 1.56068158368878
0.587939698492462 1.54841076492793
0.592964824120603 1.53595268716055
0.597989949748744 1.52330873638421
0.603015075376884 1.51048027371375
0.608040201005025 1.4974686360536
0.613065326633166 1.48427513674637
0.618090452261307 1.47090106619836
0.623115577889447 1.45734769248344
0.628140703517588 1.44361626192588
0.633165829145729 1.42970799966318
0.638190954773869 1.41562411018975
0.64321608040201 1.40136577788221
0.648241206030151 1.38693416750705
0.653266331658292 1.37233042471145
0.658291457286432 1.35755567649791
0.663316582914573 1.3426110316833
0.668341708542714 1.32749758134306
0.673366834170854 1.31221639924109
0.678391959798995 1.29676854224584
0.683417085427136 1.28115505073329
0.688442211055276 1.26537694897715
0.693467336683417 1.24943524552698
0.698492462311558 1.23333093357451
0.703517587939699 1.21706499130868
0.708542713567839 1.20063838225992
0.71356783919598 1.18405205563391
0.718592964824121 1.16730694663532
0.723618090452261 1.15040397678194
0.728643216080402 1.13334405420938
0.733668341708543 1.11612807396686
0.738693467336683 1.09875691830443
0.743718592964824 1.08123145695167
0.748743718592965 1.06355254738854
0.753768844221106 1.04572103510836
0.758793969849246 1.02773775387336
0.763819095477387 1.00960352596303
0.768844221105528 0.991319162415481
0.773869346733668 0.972885463262068
0.778894472361809 0.954303217755588
0.78391959798995 0.935573204592157
0.78894472361809 0.916696192127047
0.793969849246231 0.897672938584681
0.798994974874372 0.878504192262966
0.804020100502513 0.859190691732162
0.809045226130653 0.839733166028471
0.814070351758794 0.820132334842507
0.819095477386935 0.800388908702832
0.824120603015075 0.780503589154699
0.829145728643216 0.760477068934174
0.834170854271357 0.740310032137773
0.839195979899497 0.72000315438776
0.844221105527638 0.699557102993249
0.849246231155779 0.678972537107235
0.85427135678392 0.658250107879687
0.85929648241206 0.637390458606813
0.864321608040201 0.616394224876639
0.869346733668342 0.595262034710988
0.874371859296482 0.573994508703993
0.879396984924623 0.552592260157225
0.884422110552764 0.531055895211565
0.889447236180904 0.509386012975897
0.894472361809045 0.487583205652718
0.899497487437186 0.465648058660778
0.904522613065327 0.443581150754801
0.909547738693467 0.421383054142415
0.914572864321608 0.39905433459833
0.919597989949749 0.37659555157588
0.924623115577889 0.35400725831598
0.92964824120603 0.331290001953586
0.934673366834171 0.308444323621721
0.939698492462312 0.285470758553144
0.944723618090452 0.262369836179715
0.949748743718593 0.239142080229542
0.954773869346734 0.215788008821948
0.959798994974874 0.192308134560335
0.964824120603015 0.168702964622997
0.969849246231156 0.144973000851935
0.974874371859296 0.12111873983974
0.979899497487437 0.0971406730145832
0.984924623115578 0.0730392867233666
0.989949748743719 0.0488150623130964
0.994974874371859 0.024468476210507
1 0
};
\addplot [
color=black,
dashed,
forget plot
]
table{
0.33 0
0.33 1.89434268038131
};
\node[right, inner sep=0mm, text=black]
at (axis cs:0.33, -0.1, 0) {$s_{\mathrm{or}}=\alpha$};
\node[inner sep=0mm, text=black]%left, 
at (axis cs:0, -0.1, 0) {$0$};
\node[inner sep=0mm, text=black]%left, 
at (axis cs:1, -0.1, 0) {$1$};


\node[right, inner sep=0mm, text=black]
at (axis cs:1.12, 0, 0) {$s$};
\node[left, inner sep=0mm, text=black]
at (axis cs:0, 2.05, 0) {$c^{\star}(s)$};


\end{axis}
\end{tikzpicture}%
  \end{center}
  \vspace{-20pt}
  \caption{\textbf{Taux d'épargne de la règle d'or}}
  \vspace{-10pt}
\end{wrapfigure}
opposé d'un taux  d'épargne égal à un, on atteint  certes le niveau de
long terme  le plus élevé possible  pour le stock de  capital physique
par tête  ou la production par  tête (voir les réponses  à la question
2), mais il  s'agit d'une économie où les ménages  consomment une part
nulle de la  production! Encore une fois la consommation  à long terme
est nulle. On  peut tout aussi facilement vérifier, dans  ces deux cas
polaires,  que si  on augmente  (en partant  de zéro)  ou diminue  (en
partant  de  un) marginalement  le  taux  d'épargne alors  on  obtient
nécessairement une  augmentation de  la consommation  par tête  à long
terme. Entre ces deux cas polaires,  nous cherchons une valeur du taux
d'épargne telle  que le niveau  de long  terme de la  consommation par
tête est maximal. Si ce niveau  optimal existe, alors il doit être tel
qu'en ce point la dérivée de la consommation à l'état stationnaire par
rapport à $s$  est nulle. La dérivée de $c^{\star}$  par rapport à $s$
est donnée par :
\[
\begin{split}
\frac{\mathrm dc^{\star}}{\mathrm d s} &= -\left(\frac{s}{n+\delta}\right)^{\frac{\alpha}{1-\alpha}}
+ (1-s)\frac{\alpha}{1-\alpha}\frac{1}{n+\delta}\left(\frac{s}{n+\delta}\right)^{\frac{\alpha}{1-\alpha}-1}\\  
&= \left(\frac{s}{n+\delta}\right)^{\frac{\alpha}{1-\alpha}}
\left[\frac{\alpha}{1-\alpha}\frac{1-s}{s}-1\right]
\end{split}
\]
Le taux d'épargne de la  règle d'or $s_{\mathrm{or}}$, qui maximise la
consommation par tête à long terme doit être tel que :
\[
\frac{\mathrm dc^{\star}(s_{\mathrm{or}})}{\mathrm d s} = 0
\]
c'est-à-dire, en simplifiant :
\[
\frac{\alpha}{1-\alpha}\frac{1-s_{\mathrm{or}}}{s_{\mathrm{or}}}=1
\]
On trouve finalement :
\[
s_{\mathrm{or}} = \alpha
\]
Le  taux d'épargne  de la  règle d'or  est égal  à l'élasticité  de la
production par rapport au capital physique. Montrons qu'il s'agit bien
de l'unique maximum. La dérivée de  $c^{\star}$ par rapport à $s$ peut
s'écrire sous la forme :
\[
\frac{\mathrm dc^{\star}}{\mathrm d s} = y^{\star}(s)
\left[\frac{\alpha}{1-\alpha}\frac{1-s}{s}-1\right]
\]
en utilisant la définition de l'état stationnaire de la production par
tête.       De      façon       équivalente,      en       factorisant
$\nicefrac{1}{s(1-\alpha)}$, on a :
\[
\begin{split}
\frac{\mathrm dc^{\star}}{\mathrm d s} &= \frac{y^{\star}(s)}{s(1-\alpha)}
\left[(1-s)\alpha-s(1-\alpha)\right] \\
 &= \frac{y^{\star}(s)}{s(1-\alpha)}
\left[\alpha-s\right]
\end{split}
\]
soit par définition de $s_{\mathrm{or}}$ :
\[
\begin{split}
\frac{\mathrm dc^{\star}}{\mathrm d s} = \frac{y^{\star}(s)}{s(1-\alpha)}
\left[s_{\mathrm{or}}-s\right]
\end{split}
\]
Puisque  la dérivée  de $c^{\star}$  par  rapport à  $s$ est  positive
(négative)  si  et  seulement si  $s<s_{\mathrm{or}}$  (respectivement
$s>s_{\mathrm{or}}$),   $s_{\mathrm{or}}=\alpha$  est   bien  l'unique
valeur du taux  d'épargne qui maximise le niveau de  la consommation à
l'état stationnaire.\newline

\question{4}  Si le  taux  d'épargne effectif  est  différent du  taux
d'épargne de la règle d'or, alors on sait qu'en incitant les ménages à
épargner une  fraction $s_{\mathrm{or}}$  de la production  on amènera
l'économie dans  une situation  où la consommation  par tête  est plus
importante à  long terme. On  pourrait donc penser,  si le but  est de
maximiser le niveau de la consommation par tête à long terme, que l'on
aurait  toujours   intérêt  à  choisir   un  taux  d'épargne   égal  à
l'élasticité de la production par rapport  au capital. Mais il ne faut
pas omettre les conséquences sur le  niveau de consommation par tête à
court terme,  pendant la transition  qui amène l'économie  vers l'état
stationnaire de la règle d'or. On distinguera deux situations : le cas
où l'économie  est initialement en \textit{sur  accumulation} (dans le
sens où le  taux d'épargne effectif est supérieur à  celui de la règle
d'or)  et  le cas  où  l'économie  est  initialement en  situation  de
\textit{sous accumulation} (dans le sens où le taux d'épargne effectif
est inférieur à celui de la règle d'or). Dans les deux cas, on suppose
que l'économie  est initialement à  l'état stationnaire (associé  à un
taux d'épargne non optimal).\newline

\begin{wrapfigure}{r}{0.5\textwidth}
  \vspace{-20pt}
  \begin{center}
    % This file was created by matlab2tikz v0.3.1.
% Copyright (c) 2008--2013, Nico Schlömer <nico.schloemer@gmail.com>
% All rights reserved.
% 
% The latest updates can be retrieved from
%   http://www.mathworks.com/matlabcentral/fileexchange/22022-matlab2tikz
% where you can also make suggestions and rate matlab2tikz.
% 
% 
% 
\begin{tikzpicture}

\begin{axis}[%
width=\figurewidth,
height=\figureheight,
clip=false,
clip=false,
scale only axis,
xmin=0, xmax=5,
ymin=0, ymax=2.5,
axis lines=left
]
\addplot [
color=black,
solid,
line width=2.0pt,
forget plot
]
table{
0 2
0.0505050505050505 1.98745311419893
0.101010101010101 1.97506365274116
0.151515151515152 1.96282964044144
0.202020202020202 1.95074912689693
0.252525252525253 1.93882018617629
0.303030303030303 1.92704091651259
0.353535353535354 1.91540944000019
0.404040404040404 1.90392390229528
0.454545454545455 1.89258247232032
0.505050505050505 1.88138334197208
0.555555555555556 1.87032472583339
0.606060606060606 1.85940486088851
0.656565656565657 1.84862200624205
0.707070707070707 1.83797444284146
0.757575757575758 1.82746047320291
0.808080808080808 1.81707842114073
0.858585858585859 1.80682663150016
0.909090909090909 1.79670346989346
0.95959595959596 1.78670732243939
1.01010101010101 1.77683659550588
1.06060606060606 1.76708971545597
1.11111111111111 1.75746512839697
1.16161616161616 1.74796129993268
1.21212121212121 1.7385767149188
1.26262626262626 1.72930987722138
1.31313131313131 1.72015930947829
1.36363636363636 1.71112355286369
1.41414141414141 1.70220116685545
1.46464646464646 1.69339072900554
1.51515151515152 1.68469083471318
1.56565656565657 1.676100097001
1.61616161616162 1.66761714629383
1.66666666666667 1.65924063020044
1.71717171717172 1.65096921329789
1.76767676767677 1.64280157691863
1.81818181818182 1.63473641894028
1.86868686868687 1.62677245357806
1.91919191919192 1.61890841117975
1.96969696969697 1.61114303802336
2.02020202020202 1.60347509611716
2.07070707070707 1.59590336300239
2.12121212121212 1.58842663155832
2.17171717171717 1.58104370980985
2.22222222222222 1.57375342073743
2.27272727272727 1.56655460208947
2.32323232323232 1.55944610619698
2.37373737373737 1.55242679979067
2.42424242424242 1.54549556382024
2.47474747474747 1.538651293276
2.52525252525253 1.53189289701267
2.57575757575758 1.52521929757545
2.62626262626263 1.51862943102825
2.67676767676768 1.51212224678406
2.72727272727273 1.50569670743747
2.77777777777778 1.49935178859928
2.82828282828283 1.49308647873316
2.87878787878788 1.48689977899444
2.92929292929293 1.48079070307083
2.97979797979798 1.47475827702518
3.03030303030303 1.46880153914024
3.08080808080808 1.46291953976528
3.13131313131313 1.45711134116476
3.18181818181818 1.45137601736879
3.23232323232323 1.44571265402552
3.28282828282828 1.44012034825537
3.33333333333333 1.43459820850708
3.38383838383838 1.42914535441559
3.43434343434343 1.42376091666168
3.48484848484848 1.41844403683336
3.53535353535354 1.41319386728908
3.58585858585859 1.4080095710225
3.63636363636364 1.40289032152913
3.68686868686869 1.39783530267455
3.73737373737374 1.39284370856426
3.78787878787879 1.38791474341523
3.83838383838384 1.38304762142905
3.88888888888889 1.37824156666661
3.93939393939394 1.37349581292442
3.98989898989899 1.36880960361248
4.04040404040404 1.36418219163361
4.09090909090909 1.3596128392644
4.14141414141414 1.35510081803755
4.19191919191919 1.35064540862577
4.24242424242424 1.34624590072707
4.29292929292929 1.34190159295156
4.34343434343434 1.33761179270959
4.39393939393939 1.33337581610137
4.44444444444444 1.32919298780791
4.49494949494949 1.32506264098337
4.54545454545455 1.32098411714875
4.5959595959596 1.31695676608693
4.64646464646465 1.31297994573896
4.6969696969697 1.30905302210175
4.74747474747475 1.30517536912696
4.7979797979798 1.30134636862122
4.84848484848485 1.29756541014757
4.8989898989899 1.29383189092809
4.94949494949495 1.29014521574791
5 1.28650479686019
} node[right] {$c(t)$};
\addplot [
color=black,
dashed,
forget plot
]
table{
0 0.5
5 0.5
};
\addplot [
color=black,
dashed,
forget plot
]
table{
0 1.2
5 1.2
};
\node[right, inner sep=0mm, text=black]
at (axis cs:-1, 0.5, 0) {$c(0)$};
\node[right, inner sep=0mm, text=black]
at (axis cs:-1, 1.2, 0) {$c^{\star}_{\mathrm{or}}$};
\node[right, inner sep=0mm, text=black]
at (axis cs:5, 0, 0) {$t$};
\node[inner sep=0mm, text=black]
at (axis cs:0, 0.5, 0) {$\bullet$};

\end{axis}
\end{tikzpicture}%
  \end{center}
  \vspace{-10pt}
  \caption{\textbf{Transition de la consommation (sur accumulation)}}
  \vspace{-10pt}
\end{wrapfigure}
\textbf{Sur accumulation} Il s'agit du cas le plus simple. En effet si
le taux  d'épargne est  initialement trop élevé,  il faut  le diminuer
pour atteindre  le taux d'épargne  de la  règle d'or. Puisque  la part
épargnée  de   la  production  baisse,   dès  le  passage  de   $s$  à
$s_{\mathrm{or}}$ la  consommation augmente (elle saute).   Ensuite le
niveau  de la  consommation  par  tête s'ajuste  à  la baisse  jusqu'à
atteindre son état stationnaire. En  effet, puisque le passage au taux
d'épargne  de la  règle d'or  exige une  baisse du  taux d'épargne  il
entraîne aussi une  baisse de l'état stationnaire du  capital par tête
et donc de la production par tête. L'économie doit rejoindre un nouvel
état  stationnaire  où  les  ménages  consomment  plus  alors  que  la
production est  plus faible. Le long  de la transition vers  ce nouvel
état stationnaire,  le stock  de capital par  tête, la  production par
tête et donc  la consommation par tête baissent.  Ainsi  le passage de
$s$ à  $s_{\mathrm{or}}$ est toujours  profitable, à court terme  et à
long terme le niveau de consommation  obtenu suite à la baisse du taux
d'épargne  est  toujours  supérieur  à  ce  qu'il  était  initialement : $c(t)>c(0)=c^{\star}(s) \forall t >0$.\newline


\begin{wrapfigure}{r}{0.5\textwidth}
  \vspace{-20pt}
  \begin{center}
    % This file was created by matlab2tikz v0.3.1.
% Copyright (c) 2008--2013, Nico Schlömer <nico.schloemer@gmail.com>
% All rights reserved.
% 
% The latest updates can be retrieved from
%   http://www.mathworks.com/matlabcentral/fileexchange/22022-matlab2tikz
% where you can also make suggestions and rate matlab2tikz.
% 
% 
% 
\begin{tikzpicture}

\begin{axis}[%
width=\figurewidth,
height=\figureheight,
clip=false,
clip=false,
clip=false,
scale only axis,
xmin=0, xmax=5,
ymin=0, ymax=1.3,
axis lines=left
]
\addplot [
color=black,
solid,
line width=2.0pt,
forget plot
]
table{
0 0.0820849986238988
0.0505050505050505 0.189934913601021
0.101010101010101 0.288310114137231
0.151515151515152 0.370102270931525
0.202020202020202 0.437027482291776
0.252525252525253 0.492052896803261
0.303030303030303 0.537782674120448
0.353535353535354 0.576243511663542
0.404040404040404 0.608966522548503
0.454545454545455 0.637105719736532
0.505050505050505 0.661537473502288
0.555555555555556 0.682934767878335
0.606060606060606 0.701820747349274
0.656565656565657 0.718607054425038
0.707070707070707 0.733621388629927
0.757575757575758 0.747127493316243
0.808080808080808 0.759339809110322
0.858585858585859 0.770434341734907
0.909090909090909 0.780556815661001
0.95959595959596 0.789828860305615
1.01010101010101 0.798352754010688
1.06060606060606 0.806215099106599
1.11111111111111 0.813489696253445
1.16161616161616 0.820239812820305
1.21212121212121 0.8265199882284
1.26262626262626 0.83237748221245
1.31313131313131 0.83785344531182
1.36363636363636 0.842983871510853
1.41414141414141 0.847800378696529
1.46464646464646 0.85233085203129
1.51515151515152 0.856599977429252
1.56565656565657 0.860629686355549
1.61616161616162 0.864439528628415
1.66666666666667 0.86804698642375
1.71717171717172 0.871467739995141
1.76767676767677 0.874715893533583
1.81818181818182 0.877804167956594
1.86868686868687 0.880744066129222
1.91919191919192 0.883546014999773
1.96969696969697 0.88621948832064
2.02020202020202 0.88877311297382
2.07070707070707 0.891214761396624
2.12121212121212 0.893551632178974
2.17171717171717 0.895790320558844
2.22222222222222 0.897936880260682
2.27272727272727 0.899996877890562
2.32323232323232 0.901975440911427
2.37373737373737 0.903877300064266
2.42424242424242 0.905706826970339
2.47474747474747 0.907468067540521
2.52525252525253 0.909164771726704
2.57575757575758 0.910800420073666
2.62626262626263 0.91237824746539
2.67676767676768 0.913901264405376
2.72727272727273 0.915372276124375
2.77777777777778 0.916793899769767
2.82828282828283 0.918168579897404
2.87878787878788 0.919498602458175
2.92929292929293 0.920786107447088
2.97979797979798 0.922033100361634
3.03030303030303 0.923241462598089
3.08080808080808 0.924412960898783
3.13131313131313 0.925549255949827
3.18181818181818 0.926651910217051
3.23232323232323 0.92772239509771
3.28282828282828 0.928762097456618
3.33333333333333 0.929772325607622
3.38383838383838 0.930754314794531
3.43434343434343 0.931709232219669
3.48484848484848 0.932638181662997
3.53535353535354 0.933542207730171
3.58585858585859 0.934422299763813
3.63636363636364 0.93527939544874
3.68686868686869 0.936114384138726
3.73737373737374 0.936928109929527
3.78787878787879 0.937721374500497
3.83838383838384 0.938494939744829
3.88888888888889 0.939249530206558
3.93939393939394 0.93998583534066
3.98989898989899 0.94070451161107
4.04040404040404 0.941406184439987
4.09090909090909 0.942091450020638
4.14141414141414 0.942760877004516
4.19191919191919 0.943415008073111
4.24242424242424 0.944054361403266
4.29292929292929 0.944679432034441
4.34343434343434 0.945290693145478
4.39393939393939 0.945888597247768
4.44444444444444 0.946473577301129
4.49494949494949 0.947046047758187
4.54545454545455 0.947606405542532
4.5959595959596 0.948155030965508
4.64646464646465 0.94869228858607
4.6969696969697 0.949218528017791
4.74747474747475 0.949734084686785
4.7979797979798 0.950239280543964
4.84848484848485 0.950734424734839
4.8989898989899 0.951219814229757
4.94949494949495 0.9516957344173
5 0.952162459663316
} node[right] {$c(t)$};
\addplot [
color=black,
dashed,
forget plot
]
table{
0 0.7
5 0.7
};
\addplot [
color=black,
dashed,
forget plot
]
table{
0 1
5 1
};
\addplot [
color=red,
dashed,
forget plot
]
table{
0.606060606060606 0
0.606060606060606 1.3
};
\node[right, inner sep=0mm, text=black]
at (axis cs:-1, 0.7, 0) {$c(0)$};
\node[inner sep=0mm, text=black]
at (axis cs:0, 0.7, 0) {$\bullet$};
\node[right, inner sep=0mm, text=black]
at (axis cs:-1, 1, 0) {$c^{\star}_{\mathrm{or}}$};
\node[inner sep=0mm, text=black]
at (axis cs:0.606060606060606, -0.1, 0) {$T$};
\node[inner sep=0mm, text=black]
at (axis cs:0, -0.1, 0) {$0$};
\node[right, inner sep=0mm, text=black]
at (axis cs:5, 0, 0) {$t$};

\end{axis}
\end{tikzpicture}%
  \end{center}
  \vspace{-10pt}
  \caption{\textbf{Transition de la consommation (sous accumulation)}}
  \vspace{-10pt}
\end{wrapfigure}
\textbf{Sous  accumulation} Si  taux d'épargne  est initialement  trop
faible,  par  rapport  à  l'optimum,  alors  nous  devrions  augmenter
celui-ci. Cela assure un niveau de  consommation par tête plus élevé à
l'état stationnaire, mais cela entraîne mécaniquement une baisse de la
consommation initialement. Au même  instant que l'augmentation du taux
d'épargne, la consommation chute puis  adopte un profil croissant pour
rejoindre   l'état  stationnaire   de  la   règle  d'or.    En  effet,
l'augmentation du taux d'épargne augmente l'état stationnaire du stock
de capital  par tête. Le  long de la  dynamique de transition  vers le
nouvel état stationnaire, le stock  de capital par tête, la production
par tête  et donc  la consommation par  tête augmentent.   Pendant une
première période,  $\forall t  \in [0,T[$,  suite à  l'augmentation du
taux d'épargne,  le niveau  de consommation par  tête demeure  sous sa
condition  initiale   ($c(0)=c^{\star}(s)$),  avant  de   le  dépasser
définitivement.  Il  faut donc  mettre en balance  le bénéfice  à long
terme avec  le coût à  court terme  pour décider s'il  est intéressant
d'adopter le taux d'épargne de la  règle d'or. Le choix dépendra de la
préférence  pour le  présent des  ménages.  Si  les ménages  accordent
beaucoup plus  de poids au  présent qu'au futur, c'est-à-dire  plus de
poids à  la perte en  termes de consommation  entre $0$ et  $T$ qu'aux
gains entre $T$ et $\infty$, ils trouveront peu intéressant le passage
de $s$ à $s_{\mathrm{or}}$.\newline

\exercice{2} \question{1} L'indice  technologique, $A(t)$, est supposé
croître à taux constant. Nous avons donc :
\[
\frac{\dot A(t)}{A(t)} = x \quad \forall\, t
\]
L'introduction du progrès  technique est nécessaire dans  le modèle de
Solow,  car  sans  cette  source  exogène de  croissance  le  taux  de
croissance des variables  par tête est nul à long  terme. Par exemple,
le modèle  de Solow sans  progrès technique prédit que  l'hypothèse de
rendement marginal du capital physique résulte à long terme en un taux
de  croissance nul  pour  la production  par  tête.  Cette  prédiction
contredit  l'observation.  \question{2}  Par définition  du  stock  de
capital par tête efficace, $\hat k$, nous avons :
\[
\begin{split}
  \dot{\hat{k}} &= \frac{\mathrm d}{\mathrm d t}\frac{K}{AL}\\
&= \frac{\dot K (AL) - K \dot{(AL)}}{(AL)^2}\\
&= \frac{\dot K (AL) - K \dot A L - K A \dot L}{(AL)^2}\\
&= \frac{\dot K}{AL} - \hat k \frac{\dot A}{A} - \hat k \frac{\dot L}{L}\\
&= \frac{\dot K}{AL} - (x+n)\hat k
\end{split}
\]
En substituant la loi d'évolution du stock de capital physique agrégé,
il vient :
\[
\dot{\hat{k}} = \frac{s K^{\alpha}(AL)^{1-\alpha}-\delta K}{AL} - (x+n)\hat k 
\]
En exploitant le  fait que la fonction de production  soit homogène de
degré un (rendements d'échelle constants), on a finalement :
\[
\dot{\hat{k}} = s \hat k^{\alpha} - (x+n+\delta)\hat k 
\]
\end{document}