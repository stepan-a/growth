\documentclass[10pt,a4paper,notitlepage]{article}
\usepackage{amsmath}
\usepackage{amssymb}
\usepackage{amsbsy}
\usepackage{float}
\usepackage[french]{babel}
\usepackage{graphicx}

\usepackage[utf8x]{inputenc}
\usepackage[T1]{fontenc}
\usepackage{palatino}

 \usepackage[active]{srcltx}
\usepackage{scrtime}

\newcommand{\exercice}[1]{\textsc{\textbf{Exercice}} #1}
\newcommand{\question}[1]{\textbf{(#1)}}
\setlength{\parindent}{0cm}

\begin{document}

\title{\textsc{Croissance\\ \small{(Fiche de TD n°1)}}}
\author{Stéphane Adjemian\thanks{Université du Maine, Gains. \texttt{stephane DOT adjemian AT univ DASH lemans DOT fr}}}
\date{Le \today\ à \thistime}

\maketitle

\exercice{1}  Soit une  économie dont  le PIB  réel par  tête est,  en
l'année $T$, $y_T=1000$.  \question{1} Si le taux de croissance annuel
moyen de  $y$ est 0.1\% quel  sera le niveau  du PIB réel par  tête de
cette économie  150 ans  plus tard ?  \question{2}  Donnez le  taux de
croissance annuel  moyen si $y_{T+100}=1200$. \question{3}  Si le taux
de croissance  annuel moyen est  1.5\%, combien faut-il  d'années pour
que l'économie double son PIB par tête ?\newline


\exercice{2}  Le tableau  suivant donne  les niveaux  de PIB  par tête
réels en  1950 et en  2000 (exprimés en \$  de l'année 2000)  pour les
états Unis et l'Éthiopie~:
\begin{center}
\begin{tabular}{l|cc}\hline
 & 1950 & 2000 \\ \hline 
États Unis & 11233,41 & 34364,50 \\ 
Éthiopie & 329,27 & 725,37\\\hline\hline
\end{tabular}
\end{center}
\question{1} Calculez les  taux de croissance annuels  moyens pour les
deux pays. \question{2} Caculez, pour 1950  et 2000, les ratios du PIB
par tête réel.  Quel est le taux de croissance  annnuel moyen du ratio
des PIB ? \question{3} Combien de  temps faudrait-il à l'Éthiopie pour
rattraper  le niveau  des États  Unis  en 2000,  si à  partir de  2000
l'Étiopie bénéficie du taux de  croissance annuel moyen des États Unis
sur la  période 1950-2000 et si  le taux de croissance  des États Unis
est nul à partir de 2000 ?\newline


\exercice{3} Soit $I(t)$ le niveau de l'investissement à l'instant $t$
dans  une  économie.   On  suppose  qu'initialement  on  a  $I(0)=I_0=
1/3$.  \question{1}  écrivez  l'équation différentielle  décrivant  le
mouvement  de  l'investissement si  celui-ci  croît  au taux  constant
$\rho$.  \question{2}  Résolvez cette équation en  exprimant le niveau
de l'investissement  à chaque instant $t$.  \question{3} Supposons que
la  population, $L(t)$  augmente  à taux  constant  $n$.  Décrivez  la
dynamique de l'investissement par tête $i(t)=I(t)/L(t)$.  \question{4}
Supposons que la  population, $L$, augmente à taux  $n(t)$ à l'instant
$t$ (non  constant). Supposons que  le niveau  de la population  en 0,
$L(0)$ soit  connue. Exprimez le niveau  de la population en  $T>0$ en
fonction de  la population  initiale et  de la  chronique des  taux de
croissance entre 0 et $T$. Déterminez le facteur de croissance entre 0
et $T$.\newline

\exercice{4} Une fonction de production est dite néo-classique si elle
vérifie les conditions suivantes :
\begin{enumerate}
\item La fonction est homogène de degré 1.
\item La fonction  est croissante dans ses arguments  (les facteurs de
  production),  autrement   dit  les  productivités   marginales  sont
  positives.
\item Les productivités marginales sont décroissantes.
\item Les conditions d'Inada : 
\begin{itemize}
\item Lorsque l'usage  d'un facteur de production  tend vers l'infini,
  alors  la productivité  marginale associée  à ce  facteur tend  vers
  zéro.
\item Lorsque l'usage d'un facteur de production tend vers zéro, alors
  la productivité marginale associée à ce facteur tend vers l'infini.
\end{itemize}
\end{enumerate}
\question{1}  La fonction  $Y=K^{\alpha}L^{\beta}$,  avec $\alpha$  et
$\beta$ deux réels positifs tels  que $\alpha+\beta > 1$, est-elle une
fonction   de   production   néo-classique ?    \question{2}   Comment
s'interprète  le coefficient  $\alpha$ ? \question{3}  La fonction  de
production  $Y=K^{\alpha}L^{1-\alpha}+BK$,  avec   $B$  une  constante
positive et  $\alpha \in ]0,1[$,  est-elle une fonction  de production
néo-classique ?    \question{4}   Le  coefficient   $\alpha$   peut-il
s'interprêter de la même façon que dans le cas de la première fonction
de    production ?    \question{5}    La   fonction    de   production
$Y=\left(\alpha
  K^{\gamma}+(1-\alpha)L^{1-\gamma}\right)^{\frac{1}{\gamma}}$
est-elle une fonction de production néo-classique ?\newline

\exercice{5}  Soit $K(t)$  le  stock de  capital  physique, $L(t)$  la
population qui  croît au taux  constant $n>0$, $\alpha \in  ]0,1[$, $s
\in  ]0,1[$  et $\delta>0$.   La  dynamique  du capital  physique  est
décrite par  : $\dot{K}(t)=BK(t)^{\alpha}L(t)^{1-\alpha}-\delta K(t)$.
Déterminez la dynamique du capital physique par tête $k(t)=K(t)/L(t)$.


\end{document}
