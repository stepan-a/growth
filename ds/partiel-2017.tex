\documentclass[11pt,a4paper,notitlepage,twocolumn]{article}
\usepackage{amsmath,amssymb,amsbsy}
\usepackage{float}
\usepackage[french]{babel}
\usepackage{graphicx}

\usepackage[utf8x]{inputenc}
\usepackage[T1]{fontenc}
\usepackage{palatino}
\usepackage{manfnt}
\usepackage{hyperref}
\usepackage{nicefrac}

\usepackage[active]{srcltx}
\usepackage{scrtime}

\newcommand{\exercice}[1]{\textsc{\textbf{Exercice}} #1}
\newcommand{\question}[1]{\textbf{(#1)}}
\setlength{\parindent}{0cm}

\begin{document}

\title{\textsc{Éducation, Formation et Croissance\\ \small{Partiel}}}
\date{Le \today\ à \thistime}

\maketitle
\thispagestyle{empty}

Soient $K(t)$ le stock de capital physique d'une économie à l'instant
$t$, $L(t)$ la population et $Y(t)$ la production à l'instant $t$. \textbf{(1)} On suppose que la dynamique de la population est caractérisée par :
\[
\dot L(t) = n L(t)
\]
pour tout $t\in\mathbb R_+$, avec $n>0$. Interpréter cette équation. \textbf{(2)} On suppose que la population à l'instant initial $t=0$ est connue, on note $L(0) = L_0$. Déterminer le niveau de la population à un instant $t$ quelconque. \textbf{(3)} On suppose que la production est définie par la fonction de production :
\[
Y(t) = K(t)^{\alpha}L^{1-\alpha}
\]
avec $\alpha\in]0,1[$ un paramètre technologique. Donner une interprétation au paramètre $\alpha$. \textbf{(4)} Cette fonction de production est-elle de type néo-classique ? Justifier votre réponse. \textbf{(5)} Le stock de capital évolue de la façon suivante :
\[
\dot K(t) = sY(t)-\delta K(t)
\]
où $s\in[0,1]$ est le taux d'épargne et $\delta>0$ le taux de dépréciation du capital physique. Définir, à l'aide d'une équation, la dynamique du stock de capital par tête, $k(t)=\nicefrac{K(t)}{L(t)}$. Donner une interprétation de cette équation. \textbf{(6)} Calculer le taux de croissance du stock de capital par tête, $g_k(t)$, et représenter graphiquement ce taux de croissance (en n'oubliant pas de justifier la construction du graphique). \textbf{(7)} Calculer l'état stationnaire des variables par tête dans ce modèle, on notera $k^{\star}$ et $y^{\star}$. \textbf{(8)} Quelles sont les propriétés remarquables de cet état stationnaire ? Quelle(s) hypothèse(s) est (sont) à l'origine de ces propriétés ? \textbf{(9)} On suppose maintenant que la fonction de production est de la forme :
\[
Y(t) = K(t)^{\alpha}L^{1-\alpha} + BK(t)
\]
où $\alpha\in]0,1[$ et $B>0$. Cette fonction de production est-elle de type néoclassique ? Justifier votre réponse. \textbf{(10)} Caractériser la dynamique du capital par tête avec cette nouvelle fonction de production. \textbf{(11)} Déterminer sous quelle condition le capital par tête, $k$, croît à long terme. Interpréter. Sous la même condition, quel est le taux de croissance à long terme de la production par tête ? Sous la même condition, quel est le taux de croissance à long terme de de la production ? \textbf{(12)} S'il est possible de définir un état stationnaire strictement positif pour les variables par tête (sous quelle condition ?), donner la vitesse d'ajustement vers l'état stationnaire (en procédant par analogie avec le résultat obtenu en cours avec la première fonction de production, sans refaire les calculs). L'augmentation de la fonction de production avec un terme linéaire en $K$ contribue t-elle à réduire ou augmenter la vitesse de convergence ?

\end{document}