\documentclass[10pt,a4paper,notitlepage]{article}
\usepackage{amsmath,amssymb,amsbsy}
\usepackage{float}
\usepackage[french]{babel}
\usepackage{graphicx}

\usepackage[utf8x]{inputenc}
\usepackage[T1]{fontenc}
\usepackage{palatino}
\usepackage{manfnt}
\usepackage{hyperref}
\usepackage{nicefrac}

\usepackage[active]{srcltx}
\usepackage{scrtime}

\newcommand{\exercice}[1]{\textsc{\textbf{Exercice}} #1}
\newcommand{\question}[1]{\textbf{(#1)}}
\setlength{\parindent}{0cm}

\begin{document}

\title{\textsc{Éducation, Formation et Croissance\\ \small{Partiel}}}
\date{Le \today\ à \thistime}

\maketitle

Soient $k(t)$ le stock de capital physique d'une économie à l'instant
$t$, $L(t)$ la population qui croît au taux constant $n>0$,
$\alpha\in]0,1[$ un paramètre technologique, $s\in]0,1[$ le taux
d'épargne et $\delta\in[0,1]$ le taux de dépréciation du capital
physique. La dynamique du stock de capital physique est décrite par
l'équation suivante :
\[
\dot K(t) = sY(t)-\delta K(t)
\]
et la technologie de production est :
\[
Y(t) = K(t)^\alpha L(t)^{1-\alpha} + BK(t)
\]
avec $B>0$. \question{1} S'agit-il d'une fonction de production
néoclassique ? Argumentez votre réponse. \question{2} Calculez
l'élasticité de la production par rapport au stock de capital
physique. Commentez vos résultats. \question{3.a}
Interprétez la loi d'évolution du stock de capital physique (la
première équation). \question{3.b} Décrivez la dynamique du capital
par tête $k(t)=\nicefrac{K(t)}{L(t)}$, en montrant que :
\[
\dot k(t) = sk(t)^{\alpha}+sBk(t)-(n+\delta)k(t)
\]
\question{3.c} Interprétez cette équation. \question{4} Donnez une
expression analytique du taux de croissance du stock de capital
physique par tête. \question{5} Représentez graphiquement le taux de
croissance du stock de capital physique par tête (en expliquant la
construction du graphique) et commentez en fonction du signe de
$sB-n-\delta$. \question{6} Calculez l'état stationnaire du modèle
(pour le capital par tête, la production par tête et la consommation
par tête) s'il existe. \question{7} Quel est le rapport entre l'état
stationnaire du stock de capital par tête et le niveau de long terme
du capital par tête ? Justifiez votre réponse. \question{8} Si l'état
stationnaire n'existe pas, que pouvez-vous dire du long terme de cette
économie. \question{9} Donnez une expression du taux de croissance du
produit par tête en fonction du taux de croissance du capital physique
par tête. Commentez. \question{10} Montrez
que ce modèle prédit une relation décroissante entre le taux de
croissance du capital physique par tête et son niveau. Quelle est
l'hypothèse au centre de cette prédiction ? Comment la forme
particulière de la fonction de production vient atténuer cette
hypothèse. \question{11} En procédant par analogie, \emph{ie} je ne
vous demande pas de refaire les calculs vus en cours, donnez la
vitesse d'ajustement vers l'état stationnaire (s'il existe) du stock
de capital physique par tête. Commentez.

\end{document}