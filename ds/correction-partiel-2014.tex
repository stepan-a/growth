\documentclass[10pt,a4paper,notitlepage]{article}
\usepackage{amsmath,amssymb,amsbsy}
\usepackage{float}
\usepackage[french]{babel}
\usepackage{graphicx}

\usepackage[utf8x]{inputenc}
\usepackage[T1]{fontenc}
\usepackage{palatino}
\usepackage{manfnt}
\usepackage{hyperref}
\usepackage{nicefrac}

\usepackage[active]{srcltx}
\usepackage{scrtime}

\usepackage{tikz,pgfplots}
\pgfplotsset{compat=newest}
\usetikzlibrary{patterns, arrows, decorations.pathreplacing, calc}


\newcommand{\exercice}[1]{\textsc{\textbf{Exercice}} #1}
\newcommand{\question}[1]{\textbf{(#1)}}
\setlength{\parindent}{0cm}

\begin{document}

\title{\textsc{Éducation, Formation et Croissance\\ \small{Correction
      du partiel}}}
\date{Le \today\ à \thistime}

\maketitle

Soient $k(t)$ le stock de capital physique d'une économie à l'instant
$t$, $L(t)$ la population qui croît au taux constant $n>0$,
$\alpha\in]0,1[$ un paramètre technologique, $s\in]0,1[$ le taux
d'épargne et $\delta\in[0,1]$ le taux de dépréciation du capital
physique. La dynamique du stock de capital physique est décrite par
l'équation suivante :
\[
\dot K(t) = sY(t)-\delta K(t)
\]
et la technologie de production est :
\[
Y(t) = K(t)^\alpha L(t)^{1-\alpha} + BK(t)
\]
avec $B>0$.\newline

\bigskip
\bigskip

\question{1} Il ne s'agit pas d'une fonction de production
néoclassique. Les rendements d'échelle sont bien constants, les
productivités marginales sont bien positives et décroissantes, mais
toutes les conditions d'Inada ne sont pas satisfaites. Lorsque le
stock de capital tend vers l'infini, la productivité marginale du
capital tend vers $B>0$ et non vers zéro. Cette violation des
conditions d'Inada ne nous permet pas de garantir l'existence de
l'état stationnaire. Nous verrons plus loin que si $B$ est assez
grand, l'état stationnaire n'existe pas et on a de la croissance
endogène (l'économie croît indéfiniment).\newline

\question{2} L'élasticité est définie par:
\[
\epsilon_{Y/K} = \frac{\frac{\partial Y}{\partial K}}{\frac{Y}{K}}
\]
En calculant la dérivée de la fonction de production par rapport à
$K$, il vient:
\[
\begin{split}
  \epsilon_{Y/K} &= \left(\alpha
    K^{\alpha-1}L^{1-\alpha}+B\right)\frac{K}{Y}\\
  &= \frac{\alpha K^{\alpha}L^{1-\alpha}+BK}{Y}\\
  &= \frac{\alpha Y +(1-\alpha)BK}{Y}\\
  &= \alpha + (1-\alpha)B \frac{K}{Y}
\end{split}
\]
On remarque que l'élasticité est plus grande que dans le cas
Cobb-Douglas (\emph{ie} sans le terme $BK$ dans la fonction de
production). A priori, l'élasticité varie durant la transition vers
l'état stationnaire (s'il existe) ou le sentier de croissance
équilibrée.\newline

\question{3.a} Le stock de capital physique agrégé augmente si et
seulement si l'investissement brut est supérieur à la dépréciation du
stock de capital physique.\newline

\question{3.b} On procède exactement comme dans le cours. Par
définition, la variation du stock de capital par tête est donnée par:
\[
\begin{split}
  \dot k(t) &= \frac{\dot K(t)L(t)-K(t)\dot L(t)}{L(t)^2}\\
  &= \frac{\dot K(t)}{L(t)} - nk(t)
\end{split}
\]
En substituant la loi d'évolution du stock de capital physique agrégé
et la fonction de production, il vient:
\[
\begin{split}
  \dot k(t) &= \frac{sK(t)^\alpha L(t)^{1-\alpha} + sBK(t)-\delta K(t)}{L(t)} - nk(t)\\
  &= sk(t)^{\alpha} + sBk(t) - (n+\delta)k(t)
\end{split}
\]
car la fonction de production est homogène de degré un.\newline

\question{3.c} Le stock de capital physique par tête s'accroît si et
seulement si l'investissement par tête domine la dépréciation du stock
de capital par tête ($n+\delta$ est le taux de dépréciation du stock
de capital physique par tête).\newline


\question{4} Le taux de croissance est obtenu en rapportant la
variation de $k$ à son niveau:
\[
\begin{split}
  g_k(t) &= \frac{\dot k(t)}{k(t)}\\
  &= \frac{sk(t)^{\alpha} + sBk(t) - (n+\delta)k(t)}{k(t)}\\
  &= sk(t)^{\alpha-1} - (n+\delta-sB)
\end{split}
\]
Le taux de croissance est positif si et seulement si l'investissement
brut par unité de capital, $sk(t)^{\alpha-1}+sB$, est supérieur au
taux de dépréciation du capital physique par tête, $n+\delta$. Notons,
avec la dernière expression du taux de croissance du stock de capital
physique par tête, que nous obtenons une expression très proche de ce
que nous obtiendrions avec une fonction de production Cobb Douglas: la
seule différence formelle tient à la constante retranchée à
$sk^{\alpha-1}$. Attention cette comparaison n'est que formelle, la
constante $n+\delta-sB$ ne doit pas s'interpréter comme un taux de
dépréciation et $sk(t)^{\alpha-1}$ n'est pas l'investissement brut par
unité de capital. Néanmoins, se rapprochement formel pourra servir
plus loin à vérifier l'exactitude de nos résultats (pour ceux qui ont
en tête les résultats avec une fonction de production Cobb Douglas).\newline

\question{5} On cherche à représenter graphiquement, dans le plan
$(k,g_k)$, la dynamique du modèle. On distinguera deux cas selon le
signe de $n+\delta-sB$. Clairement, le premier terme $sk^{\alpha-1}$
est une fonction monotone décroissante, car le paramètre positif
$\alpha$ est strictement inférieur à l'unité. Ainsi, le taux de
croissance est une fonction monotone décroissante du niveau de capital
par tête. On sait aussi que
$\lim_{k\rightarrow\infty}sk^{\alpha-1}=0$. Ainsi, si
$n+\delta-sB$ est positif alors la courbe représentative de $g_k$ croisera une
unique fois l'axe des abscisses et si le même terme est négatif la
courbe représentative de $g_k$ ne croisera jamais l'axe des
abscisses. Dans ce dernier cas, le taux de croissance est toujours
strictement positif.

\begin{tikzpicture}[scale=.7]
\draw[->] (-.5, 0) -- (12, 0) node [right] {$k$} ;
\draw[->] (0, 0) -- (0, 8) node [above] {$g_k$} ;
\draw[->] (0, 0) -- (0, -2) ;
% Case 1.
\draw[dashed] (0,-1.5) node[left] {$-(n+\delta-sB)$} -- (11.5,-1.5) ;
\draw[thick] (.5,6) to [out=-80,in=150] (5,0) to [out=-30, in=175] (11,-1.45) ;
\node at (5, 0) {$\bullet$} ;
\node at (5, 0) [below] {$k^{\star}$} ;
% Case 2.
\draw[dashed, blue] (0,.5) node[left] {$-(n+\delta-sB)$} -- (11.5,.5) ;
\draw[thick, blue] (.5,8) to [out=-80,in=150] (5,2) to [out=-30, in=175] (11,2-1.45) ;
\end{tikzpicture}

\bigskip

\question{6} L'état stationnaire n'existe que si $n+\delta-sB$ est
positif, car cette condition assure l'existence d'un (unique) niveau
$k>0$ tel que le taux de croissance est nul (voir la question
précédente). Il faut donc que le taux de $sB$ ne soit pas trop grand
dans le sens où $sB<n+\delta$. Sous cette condition l'état
stationnaire est donné par:
\[
k^{\star} = \left(\frac{s}{n+\delta-sB}\right)^{\frac{1}{1-\alpha}}
\]
On obtient directement ce résultat en cherchant le niveau de $k$ tel
que l'investissement brut par unité de capital est égal au taux de
dépréciation du stock de capital physique par tête. On note en
passant, que même si le taux d'épargne apparaît au dénominateur,
l'état stationnaire est toujours (par rapport au cas d'une technologie
Cobb-Douglas) une fonction croissante du taux d'épargne. En
substituant dans la production par tête on obtient:
\[
y^{\star} = \left(\frac{s}{n+\delta-sB}\right)^{\frac{\alpha}{1-\alpha}}+B\left(\frac{s}{n+\delta-sB}\right)^{\frac{1}{1-\alpha}}
\]
En rappelant que la consommation par tête est le complémentaire de
l'épargne par tête:
\[
c^{\star} = (1-s)\left[\left(\frac{s}{n+\delta-sB}\right)^{\frac{\alpha}{1-\alpha}}+B\left(\frac{s}{n+\delta-sB}\right)^{\frac{1}{1-\alpha}}\right]
\]
Une augmentation permanente du taux d'épargne induit une augmentation
permanente de l'état stationnaire de la production par tête et du
stock de capital physique par tête. L'effet sur l'état stationnaire de
la consommation par tête est moins trivial et dépend du niveau initial
du taux d'épargne.\newline

\question{7} Si l'état stationnaire existe, c'est-à-dire si
$sB<n+\delta$, est le niveau de long terme de l'économie
intensive. Pour comprendre ce résultat, il suffit de noter que le
taux de croissance est positif lorsque $k<k^{^{\star}}$ est négatif
lorsque $k>k^{\star}$ (voir le graphique de la question 5). Dès lors
la transition ramène toujours l'économie vers l'état
stationnaire.\newline

\question{8} Si l'état stationnaire n'existe pas, c'est-à-dire si
$sB>n+\delta$, le taux de croissance est positif quel que soit le
niveau du stock de capital par tête. Dans ce cas l'économie croît
perpétuellement, et on parle alors de croissance endogène. Notons que
dans ce cas, même si les variables croissent à long terme comme dans
le modèle avec une technologie $Ak$ (voir le cours), nous ne
sacrifions pas pour autant la transition: ici le taux de croissance
est toujours une fonction (décroissante de $k$) et varie dans le temps.\newline

\question{9} Calculons le taux de croissance de la production par
tête. Par définition, on a:
\[
\begin{split}
  g_y(t) &= \frac{\dot y(t)}{y(t)}\\
  &= \frac{\alpha k(t)^{\alpha-1}\dot k(t)+B\dot k(t)}{y(t)}\\
  &= \frac{\alpha k(t)^{\alpha} g_k(t) + B\dot k(t)}{y(t)}\\
  &= \frac{\alpha y(t) g_k(t) + B\dot k(t) - \alpha B k(t)g_k(t)}{y(t)}\\
  &= \alpha g_k(t) + (1-\alpha)B\frac{k(t)}{y(t)}g_k(t)\\
  &= \underset{\epsilon_{y/k}}{\underbrace{\left(\alpha + (1-\alpha)B\frac{k(t)}{y(t)}\right)}}g_k(t)
\end{split}
\]
Le taux de croissance du PIB par tête est égal au taux de croissance
du stock de capital physique par tête multiplié par l'élasticité de la
production par rapport au capital physique. Ce résultat était attendu
(par définition d'une élasticité), seule l'expression de l'élasticité
est changée (par rapport au cas Cobb-Douglas habituel).\newline

\question{10} Nous le voyons déjà sur le graphique donné plus
haut. Plus formellement nous avons:
\[
\frac{\partial g_k}{\partial k} = -(1-\alpha) s k(t)^{\alpha-2}
\]
Cette dérivée est positive car $\alpha\in(0,1)$. C'est parce que le
rendement du capital physique est décroissant que le taux de
croissance du stock de capital physique par tête dépend négativement
de son niveau. Le long de la transition\footnote{On suppose ici que la
  dotation en capital physique est faible.}, lorsque le stock de
capital physique par tête augmente, le rendement de l'investissement
diminue. Ainsi, le long de la transition, chaque unité supplémentaire
consacrée à l'investissement en capital donnera moins de production
demain et donc aussi moins d'investissement puisque (dans ce modèle)
une part constante, $s$, de la production est investie. Comme le taux
de croissance, nous l'avons vu plus haut, est égal au taux
d'investissement net (investissement brut moins la dépréciation), le
taux de croissance diminue le long de la transition (vers l'état
stationnaire ou le sentier de croissance équilibrée).\newline La
technologie adoptée dans cet exercice vérifie bien l'hypothèse de
rendements décroissants (la productivité marginale du capital physique
est bien décroissante) mais cette productivité marginale ne tend pas
vers zéro lorsque $k$ tend vers l'infini. Cette violation des
conditions d'Inada, qui explique pourquoi sous certaines conditions on
peut observer de la croissance endogène, se manifeste aussi sur la
valeur de l'élasticité de la production par rapport au capital (qui
mesure le rendement du capital et caractérise le lien entre $g_k$ et
$g_y$). Si $B$ est assez grand de sorte que l'économie croît
indéfiniment alors on a:
\[
\lim_{k\rightarrow\infty}\frac{y}{k} =\lim_{k\rightarrow\infty} k^{\alpha-1}+B=B
\]
La productivité moyenne, comme la productivité marginale, tend vers
$B$ lorsque $t$, et donc $k$, tend vers l'infini. En substituant dans
l'expression de l'élasticité de $y$ par rapport à $k$ on obtient:
\[
\lim_{k\rightarrow\infty}\epsilon_{\frac{y}{k}} = \alpha +
(1-\alpha)B\lim_{k\rightarrow\infty}\frac{k}{y} = 1
\]
Si $B$ est assez grand, au sens où $sB>n+\delta$, l'élasticité tend
vers 1 lorsque $t$ tend vers l'infini. Autrement dit, asymptotiquement
la technologie devient linéaire (\emph{ie} équivalente à une
technologie $Ak$) et le rendement du capital constant. Même si le taux
de croissance de $k$ est une fonction décroissante de $k$ (pour
$k<\infty$), il ne tend pas nécessairement vers zéro (dés lors que le
rendement asymptotique de l'investissement par tête est supérieur au taux de
dépréciation constant du capital par tête).\newline

\question{11} Si l'état stationnaire existe, c'est-à-dire si
$sB<n+\delta$, la vitesse de convergence vers l'état stationnaire est
donnée par:
\[
\beta = (1-\alpha)(n+\delta-sB)
\]
Pour obtenir ce résultat il suffit de comparer l'expression de $g_k$
dans le cas Cobb-Douglas avec celle donnée ici et de se rappeler (voir
le cours) de la vitesse de convergence dans le cas
Cobb-Douglas. Clairement, il suffit de remplacer $n+\delta$ par
$n+\delta-sB$. On remarque qu'ici la vitesse d'ajustement vers l'état
stationnaire dépend du comportement d'épargne. Plus une économie
épargne plus longue sera sa transition vers l'état stationnaire
(réduction de la vitesse de convergence). La vitesse d'ajustement
dépend aussi du poids de la partie linéaire en $k$ dans la fonction de
production. Plus la fonction de production est linéaire ($B$ grand)
plus faible est la vitesse de convergence.

\end{document}