\documentclass[11pt,a4paper,notitlepage, twocolumn]{article}
\usepackage{amsmath}
\usepackage{amssymb}
\usepackage{amsbsy}
\usepackage{float}
\usepackage[french]{babel}
\usepackage{graphicx}
\usepackage{enumerate}

\usepackage{palatino}

 \usepackage[active]{srcltx}
\usepackage{scrtime}

\newcounter{qnumber}
\setcounter{qnumber}{0}


\newcommand{\question}{\textbf{(\addtocounter{qnumber}{1}\theqnumber)}\,}
\setlength{\parindent}{0cm}


\begin{document}

\title{\textsc{Croissance}}
\date{Mardi 12 décembre 2023}

\maketitle

\thispagestyle{empty}

\begin{quote}
  \textit{Les réponses non commentées ou insuffisamment détaillées ne seront pas considérées. Prenez le temps de faire des phrases.}
\end{quote}

\bigskip

\textbf{\textsc{Exercice I.}} Soient $K(t)$ le stock de capital physique d'une économie à l'instant
$t$, $L(t)$ la population qui croît au taux constant $n>0$,
$\alpha\in]0,1[$ un paramètre technologique, $s\in]0,1[$ le taux
d'épargne et $\delta\in[0,1]$ le taux de dépréciation du capital
physique. La dynamique du stock de capital physique est décrite par
l'équation suivante~:
\[
\dot K(t) = sY(t)-\delta K(t)
\]
et la technologie de production est~:
\[
Y(t) = K(t)^\alpha L(t)^{1-\alpha}
\]
\question Donner une interprétation au paramètre $\alpha$. \question
Caractériser la dynamique du stock de capital physique par tête. \question
Représenter graphiquement le taux de croissance du capital physique par tête (en
prenant soin de justifier la construction du graphique). \question Calculer
l'état stationnaire non trivial du modèle (pour le capital par tête, la
production par tête et la consommation par tête). \question Expliquer pourquoi
l'état stationnaire est aussi le niveau de long terme de l'économie. Quelle est
l'hypothèse au c\oe ur de cette propriété~? \question Déterminez le taux de
croissance de la production par tête et montrer qu'il est décroissant par
rapport au niveau de la production par tête. \question Déterminer le taux de
croissance de la production à court et à long terme.\newline

\setcounter{qnumber}{0}

\textbf{\textsc{Exercice II.}} On reprend le même modèle que dans l'exercice I, mais avec la fonction de production suivante :
\[
Y(t) = \beta K(t)^\alpha L(t)^{1-\alpha} + (1-\beta)K
\]
où $\beta$ est un paramètre dans l'intervalle $[0,1]$. \question Déterminer la
loi d'évolution du stock de capital par tête. \question Déterminer une condition
sur $\beta$ qui assure l'existence d'un unique état stationnaire non trivial
pour le stock de capital par tête. \question Représenter graphiquement le taux
de croissance du stock de capital par tête selon les valeurs de $\beta$.
\question Que se passe t-il si l'état stationnaire non trivial n'existe pas~?
\question Comment expliquer ce résultat~?

\end{document}

%%% Local Variables:
%%% mode: latex
%%% TeX-master: t
%%% End:
