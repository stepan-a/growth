\documentclass[10pt,a4paper,notitlepage]{article}
\usepackage{amsmath,amssymb,amsbsy}
\usepackage{float}
\usepackage[french]{babel}
\usepackage{graphicx}

\usepackage[utf8x]{inputenc}
\usepackage[T1]{fontenc}
\usepackage{palatino}
\usepackage{manfnt}
\usepackage{hyperref}
\usepackage{nicefrac}

\usepackage[active]{srcltx}
\usepackage{scrtime}

\newcommand{\exercice}[1]{\textsc{\textbf{Exercice}} #1}
\newcommand{\question}[1]{\textbf{(#1)}}
\setlength{\parindent}{0cm}

\begin{document}

\title{\textsc{Éducation, Formation et Croissance\\ \small{Partiel}}}
\date{Le \today\ à \thistime}

\maketitle
\thispagestyle{empty}

Soient $K(t)$ le stock de capital physique d'une économie à l'instant
$t$, $L(t)$ la population qui croît au taux constant $n>0$,
$\alpha\in]0,1[$ un paramètre technologique, $s\in]0,1[$ le taux
d'épargne et $\delta\in[0,1]$ le taux de dépréciation du capital
physique. La dynamique du stock de capital physique est décrite par
l'équation suivante :
\[
\dot K(t) = sY(t)-\delta K(t)
\]
et la technologie de production est :
\[
Y(t) = K(t)^\alpha L(t)^{1-\alpha} + BK(t)
\]
avec $B\geq 0$. \question{1} S'agit-il d'une fonction de production
néoclassique ? Argumentez votre réponse en distinguant suivant les
valeurs possibles du paramètre $B$. \question{2} Supposons que $B$
soit nul, comment s'interprète alors le coefficient $\alpha$ ? Calculez
l'élasticité de la production par rapport au stock de capital physique
quand $B>0$. \question{3} Décrivez la
dynamique du capital par tête $k(t)=\nicefrac{K(t)}{L(t)}$, en
montrant que :
\[
\dot k(t) = sk(t)^{\alpha}+sBk(t)-(n+\delta)k(t)
\]
Interprétez cette équation. \question{4} Déterminez sous quelle
condition (sur les paramètres du modèle) un état stationnaire non
trivial unique, $k^{\star}>0$, existe. Calculez l'état stationnaire
lorsqu'il existe (donnez les expressions pour $k^{\star}$, $y^{\star}$
et $c^{\star}$). \question{5} Représentez graphiquement le taux de
croissance du stock de capital physique par tête, en expliquant la
construction du graphique et en distinguant suivant l'existence ou non
de l'état stationnaire. \question{6} Quel est le rapport entre l'état
stationnaire du stock de capital par tête, s'il existe, et le niveau de long terme
du capital par tête ? \question{7} Si l'état
stationnaire n'existe pas, que pouvez-vous dire du long terme de cette
économie ? \question{8} Montrez que ce modèle prédit une relation
décroissante entre le taux de croissance du capital physique par tête
et son niveau. Quelle est l'hypothèse au centre de cette prédiction ?
Comment la forme particulière de la fonction de production vient
atténuer cette hypothèse ? \question{9} Quel est la vitesse de
convergence vers l'état stationnaire lorsque $B=0$ ?  \question{10} En
procédant par analogie, \emph{ie} je ne vous demande pas de refaire
les calculs vus en cours, donnez la vitesse d'ajustement vers l'état
stationnaire (s'il existe) du stock de capital physique par
tête. Commentez en soulignant en quoi le résultat est qualitativement
différent de celui obtenu à la question précédante (avec la fonction
de production Cobb-Douglas, $B=0$).

\end{document}