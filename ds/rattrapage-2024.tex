\documentclass[12pt,a4paper,notitlepage]{article}
\synctex=1


\usepackage[margin=3cm]{geometry}
\usepackage{amsmath,amssymb,amsbsy}
\usepackage{float}
\usepackage[french]{babel}
\usepackage{graphicx}

\usepackage[utf8x]{inputenc}
\usepackage[T1]{fontenc}
\usepackage{palatino}
\usepackage{manfnt}
\usepackage{hyperref}
\usepackage{nicefrac}

\usepackage[active]{srcltx}
\usepackage{scrtime}

\newcommand{\exercice}[1]{\textsc{\textbf{Exercice}} #1}
\newcommand{\question}[1]{\textbf{(#1)}}
\setlength{\parindent}{0cm}

\begin{document}

\title{\textsc{Croissance et Développement}}
\author{Stéphane Adjemian}
\date{Lundi 17 juin 2024}

\maketitle
\thispagestyle{empty}



On considère un modèle de Solow
augmenté d'une seconde variable d'état: le stock de capital humain.
Soit la fonction de production :
\[
Y(t) = K(t)^{\alpha}K(t)^{\lambda}L(t)^{1-\alpha-\lambda}
\]
Les variables $Y$,  $K$, et $L$ ont  les interprétations usuelles.
$H$ est le niveau de capital humain. Les paramètres $\alpha$ et
$\lambda$ sont  positifs et  vérifient $\alpha+\lambda<1$. \question{1} La fonction de production est-elle néoclassique~?\newline

Le  stock de
capital physique se  déprécie au taux $\delta\in]0,1[$.   La loi d'évolution
du stock de capital physique est donnée par :
\[
\dot{K}(t) = s_{K} Y(t) - \delta K(t)
\]
où $s_{K}\in]0,1[$ est le taux d'épargne en capital physique. La loi
d'évolution du stock de capital humain est donnée par :
\[
\dot{H}(t) = s_{H} Y(t) - \delta H(t)
\]
où $s_{H}\in]0,1[$ est le taux d'épargne en capital humain. On suppose
ici que les taux de dépréciation du capital physique et humain sont
identiques. \question{2} Caractériser la dynamique jointe du stock de
capital physique par tête, $k(t)$, et du stock de capital humain par
tête, $h(t)$. \question{3} calculer l'état stationnaire non trivial
pour ces deux stocks, puis déterminer l'état stationnaire de la
consommation par tête, $c^{\star}$.  \question{4} Décrire l'arbitrage,
relatif au niveau de la consommation par tête à l'état stationnaire,
dans les choix d'épargne (en capital humain et physique). Calculer les
taux d'épargne de la rêgle d'or, on notera $s_{K}^{\mathrm{or}}$
et $s_{H}^{\mathrm{or}}$, c'est-à-dire les valeurs de $s_{K}$
et $s_{H}$ qui maximisent le niveau de la consommation par tête à
l'état stationnaire. \question{5} Supposons que le taux d'épargne en
capital humain soit fixé, égal
à $\bar{s}_{H}<s_{H}^{\mathrm{or}}$. Calculer le taux d'épargne en
capital physique qui maximise $c^{\star}$, on notera $s_{K}^{\star}$
le taux d'épargne en capital physique optimal. \question{6} Montrer
que $s_{K}^{\star}>s_{K}^{\mathrm{or}}$ mais
que
$s_{K}^{\star}+\bar{s}_{H}<s_{K}^{\mathrm{or}}+s_{H}^{\mathrm{or}}$. Commenter. Le
niveau de la consommation à l'état stationnaire est-il égal au niveau
de la consommation de la rêgle d'or~?

\end{document}

%%% Local Variables:
%%% mode: latex
%%% TeX-master: t
%%% End:
