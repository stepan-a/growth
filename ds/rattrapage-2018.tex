\documentclass[10pt,a4paper,notitlepage]{article}
\synctex=1
\usepackage{amsmath,amssymb,amsbsy}
\usepackage{float}
\usepackage[french]{babel}
\usepackage{graphicx}
\usepackage[utf8x]{inputenc}
\usepackage[T1]{fontenc}
\usepackage{palatino}
\usepackage{manfnt}
\usepackage{hyperref}
\usepackage{nicefrac}

\usepackage[active]{srcltx}
\usepackage{scrtime}

\newcommand{\exercice}[1]{\textsc{\textbf{Exercice}} #1}
\newcommand{\question}[1]{\textbf{(#1)}}
\setlength{\parindent}{0cm}

\begin{document}

\title{\textsc{Éducation, Formation et Croissance\\ \small{Partiel}}}
\date{Le \today\ à \thistime}

\maketitle

Soient $K(t)$ le stock de capital physique d'une économie à l'instant
$t$, $Y(t)$ la production à l'instant $t$ dans cette économie, $L(t)$ la population qui
croît au taux constant $n>0$, $s\in]0,1[$ le taux d'épargne et
$\delta\in[0,1]$ le taux de dépréciation du stock de capital
physique. \question{0} Posez l'équation décrivant la dynamique du
stock de capital physique dans l'économie en expliquant sa
forme.\newline

On suppose que la technologie de production est représentée par la fonction :
\[
Y(t) = K(t)^\alpha L(t)^{1-\alpha} + BK(t)
\]
où $B$ est une constante positive et $\alpha\in]0,1[$. \question{1}
S'agit-il d'une fonction de production néoclassique ? Argumentez votre
réponse. \question{2} Calculez l'élasticité de la production par
rapport au stock de capital physique. Commentez vos
résultats. \question{3} Décrivez la dynamique du stock de capital physique par
tête $k(t)=\nicefrac{K(t)}{L(t)}$. Interprétez cette
équation. \question{4} Représentez graphiquement le taux de croissance
du stock de capital physique par tête (en expliquant la construction
du graphique) et commentez en fonction des valeurs de
$B$. \question{5} Calculez l'état stationnaire du modèle (pour le
capital par tête, la production par tête et la consommation par tête)
s'il existe. Vous donnerez les conditions d'existence de l'état
stationnaire. \question{6} Lorsque l'état stationnaire existe, quel
est le rapport entre l'état stationnaire du stock de capital physique par tête
et le niveau de long terme du stock de capital physique par tête ? Justifiez votre
réponse. \question{7} Si l'état stationnaire n'existe pas, que
pouvez-vous dire du long terme de cette économie. \question{8} Que
pouvez-vous dire de la relation entre le taux de croissance du capital
physique par tête et le niveau du capital physique par tête dans cette économie ? Cette
relation est-elle affectée par l'existence ou la non existence de l'état
stationnaire ? Quelle est l'hypothèse au c\oe ur de cette prédiction ?
\question{9}  En procédant par analogie, \emph{ie} je ne
vous demande pas de refaire les calculs vus en cours, donnez la
vitesse d'ajustement vers l'état stationnaire (lorsqu'il existe) du stock
de capital physique par tête. Commentez.

\end{document}