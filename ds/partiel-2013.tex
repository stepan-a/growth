\documentclass[10pt,a4paper,notitlepage]{article}
\usepackage{amsmath,amssymb,amsbsy}
\usepackage{float}
\usepackage[french]{babel}
\usepackage{graphicx}

\usepackage[utf8x]{inputenc}
\usepackage[T1]{fontenc}
\usepackage{palatino}
\usepackage{manfnt}
\usepackage{hyperref}
\usepackage{nicefrac}

\usepackage[active]{srcltx}
\usepackage{scrtime}

\newcommand{\exercice}[1]{\textsc{\textbf{Exercice}} #1}
\newcommand{\question}[1]{\textbf{(#1)}}
\setlength{\parindent}{0cm}

\begin{document}

\title{\textsc{Éducation, Formation et Croissance}}
\author{\textsc{Université du Maine (Partiel, L2)}}
\date{}

\maketitle

\exercice{1}  Dans  cet  exercice  on considère  un  modèle  de  Solow
augmenté de dépenses publiques productives (dans le sens où la dépense
de l'état accroît la production).  Soit la fonction de production :
\[
Y(t) = K(t)^{\alpha}G(t)^{\beta}(A(t)L(t))^{1-\alpha-\beta}
\]
Les variables $Y$,  $K$, $L$ et $A$ ont  les interprétations usuelles.
$G$ est le niveau de la  dépense publique.  Les paramètres $\alpha$ et
$\beta$ sont  positifs et  vérifient $\alpha+\lambda<1$.  Le  stock de
capital physique se  déprécie au taux $\delta>0$.   La loi d'évolution
du stock de capital physique est données par :
\[
\dot{K} = s (1-\tau)Y - \delta K
\]
où  $s\in]0,1[$  est   le  taux  d'épargne  en   capital  physique  et
$\tau\in]0,1[$ une taxe sur le  produit.  Cette taxe sur la production
finance les dépenses publiques, $G = \tau Y$. \question{1} Montrez que
la loi  d'évolution du capital  physique par tête efficace  est donnée
par :
\[
\dot{\hat{k}} = s(1-\tau)\hat{k}^{\alpha}\hat{g}^{\beta} - (\delta+n+x)\hat{k}
\]
où $n>0$  et $x>0$ sont  respectivement les  taux de croissance  de la
population,    $L$,     et    de     la    technologie,     $A$,    et
$\hat{k}^{\alpha}\hat{g}^{\beta}=\hat{y}$ est  la production  par tête
efficace.  Commentez cette  équation. \question{2}  Montrez qu'il  est
possible  d'écrire la  loi d'évolution  su stock  de capital  par tête
efficace sous la forme :
\[
\dot{\hat{k}} = s(1-\tau)\tau^{\frac{\beta}{1-\beta}}\hat{k}^{\frac{\alpha}{1-\beta}} - (\delta+n+x)\hat{k}
\]
en   éliminant  la   variable  $\hat   g$.   \question{3}   Déterminez
l'élasticité de la production par tête efficace rapport au capital par
tête  efficace.    Commentez.   \question{4}  Donnez   une  expression
analytique  du  taux  de  croissance  du stock  de  capital  par  tête
efficace.       Représentez     graphiquement      ce     taux      de
croissance. \question{5}  Montrez qu'il  existe un  taux d'imposition,
$\tau^{\star}$, qui  maximise ce taux  de croissance à  chaque instant
(pour  tout  niveau  de   $\hat  k$).   \question{6}  Calculez  l'état
stationnaire  du   modèle.   Commentez   la  stabilité  de   cet  état
stationnaire.    \question{7}  Montrer   que   le  taux   d'imposition
$\tau^{\star}$ qui  maximise le  taux de  croissance à  chaque instant
maximise  le  niveau  de  long  terme  de  la  consommation  par  tête
efficace. \question{8} Par analogie avec  le résultat obtenu en cours,
déterminez  la  vitesse  d'ajustement  de  l'économie  vers  son  état
stationnaire.  Commentez les  différences. Cela  va t  il dans  le bon
sens ? \question{9} Que deviennent les prédictions du modèle si $\beta = 1-\alpha$ ?

\end{document}