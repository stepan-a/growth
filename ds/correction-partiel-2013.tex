\documentclass[10pt,a4paper,notitlepage]{article}
\usepackage{amsmath,amssymb,amsbsy}
\usepackage{float}
\usepackage[french]{babel}
\usepackage{graphicx}

\usepackage[utf8x]{inputenc}
\usepackage[T1]{fontenc}
\usepackage{palatino}
\usepackage{manfnt}
\usepackage{hyperref}
\usepackage{nicefrac}

\usepackage[active]{srcltx}
\usepackage{scrtime}

\newcommand{\exercice}[1]{\textsc{\textbf{Exercice}} #1}
\newcommand{\question}[1]{\textbf{(#1)}}
\setlength{\parindent}{0cm}

\begin{document}

\title{\textsc{Éducation, Formation et Croissance}}
\author{\textsc{Université du Maine (Correction du partiel, L2)}}
\date{}

\maketitle

\exercice{1} \question{1} Nous avons par définition de $\hat{k}$ :
\[
\dot{\hat{k}} = \frac{\dot K A L - K (\dot A L + A \dot L)}{(AL)^2}
\]
c'est-à-dire :
\[
\dot{\hat{k}} = \frac{\dot K}{A L} - \frac{K}{AL}\frac{\dot A}{A} -  \frac{K}{AL}\frac{\dot L}{L}
\]
soit encore :
\[
\dot{\hat{k}} = \frac{\dot K}{A L} - (n+x)\hat{k}
\]
En substituant la loi d'évolution de K dans le premier terme, il vient :
\[
\dot{\hat{k}} = \frac{s (1-\tau)Y - \delta K}{A L} - (n+x)\hat{k}
\]
En substituant la fonction de production et en développant, on obtient :
\[
\dot{\hat{k}} = s (1-\tau)\frac{K^{\alpha}G^{\beta}(AL)^{1-\alpha-\beta}}{A L} - (\delta+n+x)\hat{k}
\]
En exploitant le fait que la technologie soit à rendement d'échelle constant, on obtient finalement :
\[
\dot{\hat{k}} = s (1-\tau)\hat{k}^{\alpha}\hat{g}^{\beta} - (\delta+n+x)\hat{k}
\]
où $\hat{g}$ est la dépense publique par tête efficace. On retrouve le
résultat habituel, le stock de  capital par tête efficace s'accroît si
et seulement si l'investissement par  tête efficace est supérieur à la
dépréciation du stock  de capital par tête  efficace. \question{2} Par
définition on a $G  = \tau Y$ et donc aussi $\hat g  = \tau \hat y$ en
divisant les deux membres de  l'égalité par $AL$. En substituant cette
dernière égalité dans la fonction  de production par tête efficace, il
vient :
\[
\hat y = \hat{k}^{\alpha}\left(\tau \hat{y}\right)^{\beta}
\]
De façon équivalente, en, divisant les deux membres par $\hat{y}^{\beta}$ :
\[
\hat{y}^{1-\beta} = \tau^{\beta}\hat{k}^{\alpha}
\]
soit finalement :
\[
\hat{y} = \tau^{\frac{\beta}{1-\beta}}\hat{k}^{\frac{\alpha}{1-\beta}}
\]
en substituant  dans la loi d'évolution  du stock de capital  par tête
efficace, nous obtenons finalement :
\[
\dot{\hat{k}} = s(1-\tau)\tau^{\frac{\beta}{1-\beta}}\hat{k}^{\frac{\alpha}{1-\beta}} - (\delta+n+x)\hat{k}
\]
\question{3}  L'élasticité  de la  production  par  tête efficace  par
rapport au stock de capital par tête efficace est :
\[
\epsilon_{\nicefrac{\hat y}{\hat k}} = \frac{\alpha}{1-\beta}
\] 
On   remarque    que   cette   élasticité   est    plus   grande   que
$\alpha$. L'introduction de dépenses  publiques productives accroît la
sensibilité de  $\hat y$ aux  variations de $\hat k$.  \question{4} Le
taux de croissance d'une variable  est obtenu en divisant la variation
de cette variable avec son niveau. Dans le cas du stock de capital par
tête efficace, nous avons donc :
\[
g_{\hat k} = s(1-\tau)\tau^{\frac{\beta}{1-\beta}}\hat{k}^{\frac{\alpha}{1-\beta}-1} - (\delta+n+x)
\]
\question{5} Pour  un niveau quelconque  du stock de  capital physique
par tête efficace, le taux de croissance de $\hat k$ est d'autant plus
grand  que  $(1-\tau)\tau^{\frac{\beta}{1-\beta}}$  est  grand.  Cette
expression  peut  être décomposée  comme  le  produit de  deux  termes
$1-\tau$  et  $\tau^{\frac{\beta}{1-\beta}}$ ;  le premier  terme  est
décroissant par rapport à $\tau$ alors que le second est croissant par
rapport à $\tau$. Il  y a donc un arbitrage dans  le choix de $\tau$ :
augmenter $\tau$  augmente la croissance  car en augmentant  $\tau$ on
augmente la  dépense publique et  donc la production, mais  cette même
augmentation  de $\tau$  s'accompagne d'un  effet défavorable  dans la
mesure on  elle réduit le  taux d'épargne effectif (les  revenus taxés
par l'état ne sont pas épargnés). Si elle existe la taxe optimale doit satisfaire la condition suivante :
\[
\frac{\mathrm d}{\mathrm d\tau} (1-\tau)\tau^{\frac{\beta}{1-\beta}}\,  _{\biggl|_{\tau=\tau^{\star}}} = 0
\]
Nous avons :
\[
\frac{\mathrm d}{\mathrm d\tau} (1-\tau)\tau^{\frac{\beta}{1-\beta}} = (1-\tau)\frac{\beta}{1-\beta}\tau^{\frac{\beta}{1-\beta}-1}-\tau^{\frac{\beta}{1-\beta}} 
\]
En annulant cette dérivée évaluée en $\tau^{\star}$ et en simplifiant, nous obtenons la condition suivante
\[
\frac{1-\tau^{\star}}{\tau^{\star}}\frac{\beta}{1-\beta}=1
\]
Le taxe optimale est donc $\tau^{\star} = \beta$. \question{6} On obtient directement l'état stationnaire par analogie avec les résultats obtenus dans le cours/td en remarquant que la dynamique du stock de capital par tête peut s'écrire :
\[
g_{\hat k} = \bar s\hat{k}^{\bar \alpha-1} - (\delta+n+x)
\]
avec $\bar s = s(1-\tau)\tau^{\frac{\beta}{1-\beta}}$ et $\bar \alpha = \frac{\alpha}{1-\beta}$. Le niveau de capital par tête efficace qui annule le taux de croissance du capital par tête efficace est :
\[
\hat{k}^{\star} = \left(\frac{\bar s}{\delta+n+x}\right)^{\frac{1}{1-\bar\alpha}}
\]
c'est-à-dire en substituant les définitions de $\bar s$ et $\bar \alpha$ :
\[
\hat{k}^{\star} = \left(\frac{s(1-\tau)\tau^{\frac{\beta}{1-\beta}}}{\delta+n+x}\right)^{\frac{1-\beta}{1-\alpha-\beta}}
\]
En rappelant que 
\[
\hat y = \tau^{\frac{\beta}{1-\beta}}\hat{k}^{\frac{\alpha}{1-\beta}}
\]
il vient :
\[
\hat{y}^{\star} = \tau^{\frac{\beta}{1-\beta}}\left(\frac{s(1-\tau)\tau^{\frac{\beta}{1-\beta}}}{\delta+n+x}\right)^{\frac{\alpha}{1-\alpha-\beta}}
\]
La consommation par tête efficace est donnée par la part des revenus net des impôts qui n'est pas épargnée, nous avons donc :
\[
\hat c = (1-s)(1-\tau)\hat y
\]
et donc à l'état stationnaire :
\[
\hat{c}^{\star} = (1-s)(1-\tau)\tau^{\frac{\beta}{1-\beta}}\left(\frac{s(1-\tau)\tau^{\frac{\beta}{1-\beta}}}{\delta+n+x}\right)^{\frac{\alpha}{1-\alpha-\beta}}
\]
que nous pouvons réécrire sous la forme :
\[
\hat{c}^{\star} = (1-s)\left[(1-\tau)\tau^{\frac{\beta}{1-\beta}}\right]^{1+\frac{\alpha}{1-\alpha-\beta}}\left(\frac{s}{\delta+n+x}\right)^{\frac{\alpha}{1-\alpha-\beta}}
\]
ou encore :
\[
\hat{c}^{\star} = \left[(1-\tau)\tau^{\frac{\beta}{1-\beta}}\right]^{\frac{1-\beta}{1-\alpha-\beta}}(1-s)\left(\frac{s}{\delta+n+x}\right)^{\frac{\alpha}{1-\alpha-\beta}}
\]
Cet  état   stationnaire  est   stable.   Nous   avons  vu,   dans  la
représentation graphique de $g_{\hat k}$, que le taux de croissance de
$\hat  k$ est  positif si  et seulement  si $\hat  k$ est  inférieur à
l'état stationnaire. Ainsi  pour toute condition initiale  du stock de
capital  par tête  efficace,  l'économie converge  à  long terme  vers
l'état stationnaire $\hat k^{\star}$.  \question{7} Notons que dans la
dernière  expression de  l'état stationnaire  de $\hat  c$, l'exposant
$\frac{1-\beta}{1-\alpha-\beta}$  est strictement  positif. Ainsi,  le
niveau de  long terme  de la  consommation par  tête efficace  est une
fonction              monotone              croissante              de
$(1-\tau)\tau^{\frac{\beta}{1-\beta}}$.    Nous    savons   que,   par
construction, $\tau^{\star}$ maximise ce terme. Ainsi, $\tau^{\star}$,
en plus d'être  le taux de taxe  qui maximise le taux  de croissance à
tout   instant,   maximise   le   niveau   de   long   terme   de   la
consommation. \question{8} Par  analogie avec le cours,  on trouve que
la vitesse d'ajustement vers l'état stationnaire est :
\[
\lambda = (1-\bar \alpha)(\delta+n+x)
\]
c'est-à-dire :
\[
\lambda = \left(1-\frac{\alpha}{1-\beta}\right)(\delta+n+x)
\]
On  remarque  que  l'introduction de  dépenses  publiques  productives
($\beta>0$)    réduit   la    vitesse    d'ajustement   vers    l'état
stationnaire. Cela va  dans le bon sens, puisque nous  avons montré en
cours que  la vitesse de convergence  théorique du modèle de  Solow de
base  est   trop  importante  par   rapport  à  ce  que   suggère  les
données. \question{9} Si  $\beta = 1-\alpha$, alors  la production par
tête  efficace  devient une  fonction  linéaire  du stock  de  capital
physique  par  tête  efficace.   On  abandonne  ainsi  l'hypothèse  de
productivité marginale décroissante du  capital. Le taux de croissance
du stock de capital par tête efficace s'écrit alors :
\[
g_{\hat k} = s(1-\tau)\tau^{\frac{\beta}{1-\beta}} - (\delta+n+x)
\]
Le taux  de croissance de  $\hat k$ ne dépend  plus de son  niveau. Le
taux  de croissance  est constant :  si  le taux  d'épargne est  assez
important  l'économie croît  indéfiniment  à taux  constant. Ainsi  le
stock  de capital  par tête  tend vers  l'infini, l'état  stationnaire
n'existe plus.
\end{document}