\documentclass[10pt,a4paper,notitlepage]{article}
\usepackage{amsmath,amssymb,amsbsy}
\usepackage{float}
\usepackage[french]{babel}
\usepackage{graphicx}

\usepackage[utf8x]{inputenc}
\usepackage[T1]{fontenc}
\usepackage{palatino}
\usepackage{manfnt}
\usepackage{hyperref}
\usepackage{nicefrac}
\usepackage{color}

\usepackage[active]{srcltx}
\usepackage{scrtime}

\newcommand{\exercice}[1]{\textsc{\textbf{Exercice}} #1}
\newcommand{\question}[1]{\textbf{(#1)}}
\setlength{\parindent}{0cm}

\usepackage{tikz}
\usepackage{pgfplots}
\pgfplotsset{compat=newest}
%\pgfplotsset{ticks=none}
%\pgfplotsset{plot coordinates/math parser=false}
\usetikzlibrary{arrows}
\usetikzlibrary{plotmarks}


\begin{document}

\title{\textsc{Éducation, Formation et Croissance\\ \small{(Correction du devoir surveillé n°1)}}}
\date{Le \today\ à \thistime}

\maketitle

Soient $k(t)$ le stock de  capital physique d'une économie à l'instant
$t$,  $L(t)$  la   population  qui  croît  au   taux  constant  $n>0$,
$\alpha\in]0,1[$  un  paramètre  technologique,  $s\in]0,1[$  le  taux
d'épargne  et  $\delta\in[0,1]$ le  taux  de  dépréciation du  capital
physique. La  dynamique du stock  de capital physique est  décrite par
l'équation suivante :
\[
\dot K(t) = sY(t)-\delta K(t)
\]
et la technologie de production est de type Cobb-Douglas :
\[
Y(t) = K(t)^\alpha L(t)^{1-\alpha}
\]
\question{1}  Donnez une  interprétation du  paramètre $\alpha$  de la
technologie  de production.  {\color{red}  Le  paramètre $\alpha$  est
  l'élasticité  de  la production  par  rapport  au stock  de  capital
  physique. En effet, nous avons :
\[
\begin{split}
\frac{\partial Y}{\partial K}\frac{K}{Y} &= \alpha K^{\alpha-1}L^{1-\alpha}\frac{K}{Y}\\
&= \alpha K^{\alpha}L^{1-\alpha}\frac{1}{K^{\alpha}L^{1-\alpha}}\\
&= \alpha
\end{split}
\]
Dans  le   modèle  de   Solow,  où   les  marchés   sont  parfaitement
concurrentiels, le paramètre $\alpha$ s'interprète aussi comme la part
des revenus du capital  physique dans les revenus totaux.}\question{2}
Interprétez  la  loi d'évolution  du  stock  de capital  physique  (la
première  équation).   {\color{red} Cette  équation  nous  dit que  la
  variation du stock de capital  physique est positive si et seulement
  si  l'investissement est  supérieur à  la dépréciation  du stock  de
  capital physique.}  \question{3.a} Décrivez la dynamique  du capital
par tête $k(t)=\nicefrac{K(t)}{L(t)}$, en montrant que :
\[
\dot k(t) = sk(t)^{\alpha}-(n+\delta)k(t)
\]  {\color{red} En  partant  de  la définition  du  stock de  capital
  physique par tête, nous avons :
\[
\dot k = \frac{\dot K L - K\dot L}{L^2}
\]
en simplifiant  et en  remplaçant $\nicefrac{\dot  L}{L}$ par  $n$, le
taux de croissance constant de la population, il vient :
\[
\dot k = \frac{\dot K}{L} - n
\]
en  substituant  la  loi  d'évolution  du  capital  physique  dans  le
numérateur du premier terme du membre de droite, on obtient :
\[
\dot k = \frac{sK^{\alpha}L^{1-\alpha}-\delta K}{L} - n k
\]
soit encore :
\[
\dot k = \frac{sK^{\alpha}L^{1-\alpha}}{L} - (n+\delta) k
\]
ou de façon équivalente :
\[
\dot k = sK^{\alpha}L^{-\alpha} - (n+\delta) k
\]
soit finalement, par définition de $k$ :
\[
\dot k = sk^{\alpha} - (n+\delta) k
\]
} \question{3.b} Interprétez cette  équation. {\color{red} Le stock de
  capital par tête  s'accroît si et seulement  si l'investissement par
  tête domine la dépréciation du capital par tête (le stock de capital
  par tête  se déprécie pour  deux raisons : l'usure des  machines, au
  taux   $\delta$,  l'augmentation   du  nombre   de  tête,   au  taux
  $n$).}\question{4}  Donnez  une  expression analytique  du  taux  de
croissance  du stock  de capital  physique par  tête. {\color{red}  On
  obtient une expression du taux de croissance de $k$ en rapportant la
  variation  de  $k$ au  niveau  de  $k$.  En  supposant que  $k$  est
  strictement  positif, on  obtient cette  expression en  divisant les
  deux membres de la dernière équation par $k$ :
\[
g_k \equiv \frac{\dot k}{k} = sk^{\alpha-1} - (n+\delta)
\]
Le  taux  de croissance  du  stock  de capital  par  tête  est égal  à
l'investissement net (de  la dépréciation) par unité  de capital. Dans
le  modèle de  Solow,  la croissance  est  entièrement déterminée  par
l'offre.}    \question{5}  Représentez   graphiquement   le  taux   de
croissance du  stock de  capital physique par  tête (en  expliquant la
construction  du graphique)  et commentez.   {\color{red} Nous  venons
  d'établir que  le taux de croissance  du capital par tête  est donné
  par la différence  entre l'investissement brut par  unité de capital
  et le taux de dépréciation du capital par tête. L'investissement par
  unité  de  capital,  $sk^{\alpha-1}$,   est  une  fonction  monotone
  décroissante  de  $k$ à  cause  de  l'hypothèse sur  les  rendements
  décroissants  du  capital.   On  montre  facilement  que  la  courbe
  représentative de  l'investissement par unité de  capital admet deux
  asymptotes,   verticale   et   horizontale,    en   zéro   et   plus
  l'infini\footnote{On    vérifie     que    $\lim_{k\rightarrow    0}
    sk^{\alpha-1}   =    \infty$   et    $\lim_{k\rightarrow   \infty}
    sk^{\alpha-1} = 0$. Ces  comportements aux bords sont conséquences
    des conditions d'Inada.}. Par ailleurs le taux de dépréciation est
  constant,  il  est  donc  graphiquement représenté  par  une  droite
  horizontale. La figure~\ref{fig:question-5} donne une représentation
  graphique  du  taux  de  croissance.   On observe  que  le  taux  de
  croissance  est,  en valeur  absolue,  d'autant  plus important  que
  l'économie   est  éloignée   de  l'état   stationnaire.}\question{6}
Calculez l'état stationnaire  du modèle (pour le capital  par tête, la
production  par tête  et  la consommation  par  tête). {\color{red}  À
  l'état  stationnaire  l'investissement  par   tête  est  égal  à  la
  dépréciation du capital par tête (de sorte que la variation du stock
  est  nulle). L'état  stationnaire, $k^{\star}$,  doit donc  être une
  solution de :
\[
s\left.k^{\star}\right.^{\alpha} = (n+\delta)k^{\star}
\]
En excluant la solution nulle, on trouve l'unique solution strictement positive :
\[
k^{\star} = \left(\frac{s}{n+\delta}\right)^{\frac{1}{1-\alpha}}
\]
En substituant  dans la  fonction de production  par tête,  on obtient
l'état stationnaire de la production par tête :
\[
y^{\star} = \left(\frac{s}{n+\delta}\right)^{\frac{\alpha}{1-\alpha}}
\]
Enfin, sachant que les ménages consomment une fraction $1-s$ du revenu
(de la production), il vient :
\[
c^{\star} = (1-s)\left(\frac{s}{n+\delta}\right)^{\frac{\alpha}{1-\alpha}}
\]
}\question{7} Quel est  le rapport entre l'état  stationnaire du stock
de capital par tête  et le niveau de long terme  du capital par tête ?
Justifiez votre réponse. {\color{red}  L'état stationnaire est le long
  terme du  modèle.  En  effet, nous voyons  clairement sur  la figure
  \ref{fig:question-5} que  pour toute condition initiale  du stock de
  capital  par tête  celui-ci se  dirige vers  $k^{\star}$ puisque  le
  stock  de  capital s'accroît  si  et  seulement  si son  niveau  est
  inférieur  à  $k^{\star}$.}   \question{8.a}  Quel est  le  taux  de
croissance du produit par tête à  court et à long terme ? {\color{red}
  On montre  facilement (voir le cours)  que le taux de  croissance du
  produit par tête est donné par :
\[
g_y = \alpha g_k
\]
Ce  résultat  n'est  valable  que   pour  la  fonction  de  production
Cobb-Douglas. Plus généralement, le taux  de croissance du produit par
tête est égal au taux de  croissance du capital par tête multiplié par
l'élasticité de la production par rapport au stock de capital physique
(il  se trouve  que  dans  le cas  Cobb-Douglas  cette élasticité  est
constante et égale à $\alpha$).  Comme à long terme l'économie atteint
son état stationnaire,  le taux de croissance du produit  par tête est
alors nécessairement  nul. Pour obtenir  le taux de croissance  de $y$
pendant la transition on peut substituer l'expression de $g_k$ dans la
dernière équation :
\[
g_y = \alpha \left(sk^{\alpha-1} - (n+\delta)\right)
\]
soit encore :
\[
g_y = \alpha \left(sy^{\frac{\alpha-1}{\alpha}} - (n+\delta)\right)
\]
À  nouveau, on  retrouve, pour  la  production par  tête, la  relation
décroissante  entre  croissance et  niveau  (si  on n'oublie  pas  que
$\alpha<1$).}   \question{8.b}  Quel  est  le taux  de  croissance  du
produit à court et à long terme ?  {\color{red} On sait que le taux de
  croissance d'un produit de variable est  égal à la somme des taux de
  croissance  des deux  variables. Puisque  $Y =  y L$,  la production
  totale est  égale à la production  par tête fois le  nombre de tête,
  nous avons :
\[
g_Y = g_y + n
\]
À  long terme  le  taux de  croissance  de $Y$  est  déterminé par  la
croissance démographique ($n$), à court terme nous avons :
\[
g_Y = \alpha \left(sy^{\frac{\alpha-1}{\alpha}} - (n+\delta)\right) + n
\]
}\question{9}
Montrez  que ce modèle prédit  une relation décroissante
entre  le taux  de croissance  du  produit par  tête et  le niveau  du
produit  par  tête.   Quelle  est   l'hypothèse  au  centre  de  cette
prédiction ?  {\color{red}  On a montré plus haut que :
\[
g_y = \alpha \left(sy^{\frac{\alpha-1}{\alpha}} - (n+\delta)\right)
\]
En notant que, par définition de l'état stationnaire, nous pouvons réécrire $n+\delta$ comme :
\[
n+\delta =s \left.y^{\star}\right.^{\frac{\alpha-1}{\alpha}}
\]
on peut réécrire le taux de croissance de $y$ de la façon suivante :
\[
g_y = \alpha s\left(y^{\frac{\alpha-1}{\alpha}} - \left.y^{\star}\right.^{\frac{\alpha-1}{\alpha}}\right)
\]
ou encore :
\[
g_y = \alpha s \left.y^{\star}\right.^{\frac{\alpha-1}{\alpha}} \left(\left(\frac{y}{y^{\star}}\right)^{\frac{\alpha-1}{\alpha}} - 1\right)
\]
ou plus simplement :
\[
g_y = \alpha (n+\delta) \left(\left(\frac{y}{y^{\star}}\right)^{\frac{\alpha-1}{\alpha}} - 1\right)
\]
Dans  cette  dernière  expression  du   taux  de  croissance  on  voit
clairement que celui-ci est une fonction décroissante de la distance à
l'état stationnaire, $\nicefrac{y}{y^{\star}}$, puisque $\alpha<1$. Ce
résultât  est  donc  la   conséquence  de  l'hypothèse  de  rendements
décroissants  du  stock  de   capital  physique.   Dans  une  économie
initialement   pauvre   (en   capital  physique)   le   rendement   de
l'investissement est plus important que dans une économie initialement
riche.   Ainsi   le  surcroît  de   production  induit  par   un  même
investissement, et  donc la  croissance, sera  plus important  dans un
pays pauvre que dans un  pays riche.  }  \question{10} Que pouvez-vous
dire  de la  dynamique du  capital  par tête  ou du  produit par  tête
lorsque  $\alpha=1$ ?   {\color{red}   Avec  $\alpha=1$  on  abandonne
  l'hypothèse de rendements décroissants du capital, la production par
  tête est linéaire par rapport au  stock de capital par tête. Dans ce
  cas, le taux de croissance du capital est simplement donné par :
\[
g_k(t) = s - (n+\delta) 
\]
En  abandonnant l'hypothèse  de  rendements décroissants,  on perd  la
relation décroissante entre  le taux de croissance et  le niveau. Dans
ce modèle, le  taux de croissance est même constant  (il ne dépend que
des   taux  d'épargne,   de  croissance   de  la   population  et   de
dépréciation) !  Cela  a plusieurs  conséquences.  Puisque le  taux de
croissance est constant  et positif (on suppose que  le taux d'épargne
est assez  important dans le  sens où $s>n+\delta$) le  modèle n'admet
plus  d'état  stationnaire  pour  les variables  par  tête.   Ici  les
variables  par tête  croissent  à  long terme,  même  en l'absence  de
progrès  technique.  Si  toutes  les  économies sont  structurellement
homogènes alors  elles ont  toutes le même  taux de  croissance. Ainsi
nous perdons aussi la propriété  de convergence des économies ; il n'y
a pas  de rattrapage,  si les économies  ont des  conditions initiales
différentes les écarts persistent indéfiniment.}\newline

\bigskip
\bigskip

\begin{figure}[H]
  \centering
  \input{ds1-question-5.tikz}
  \caption{Taux de croissance du capital physique par tête.}
  \label{fig:question-5}
\end{figure}

\end{document}