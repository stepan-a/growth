\documentclass[10pt,a4paper,notitlepage]{article}
\synctex=1


\usepackage[margin=3cm]{geometry}
\usepackage{amsmath,amssymb,amsbsy}
\usepackage{float}
\usepackage[french]{babel}
\usepackage{graphicx}

\usepackage[utf8x]{inputenc}
\usepackage[T1]{fontenc}
\usepackage{palatino}
\usepackage{manfnt}
\usepackage{hyperref}
\usepackage{nicefrac}

\usepackage[active]{srcltx}
\usepackage{scrtime}

\newcommand{\exercice}[1]{\textsc{\textbf{Exercice}} #1}
\newcommand{\question}[1]{\textbf{(#1)}}
\setlength{\parindent}{0cm}



\begin{document}

\title{\textsc{Croissance et Développement}}
\author{\textsc{Université du Maine (Rattrapage, L2)}}
\date{Mardi 11 juin 2019}


\maketitle
\thispagestyle{empty}


\begin{quote}
  Les réponses non commentées ou non suffisament détaillées ne seront
  pas considérées.
\end{quote}

\bigskip
\bigskip

\exercice{1} Soient $K(t)$ le stock de capital physique d'une économie
à l'instant $t$, $L(t)$ la population qui croît au taux constant
$n>0$, $\alpha\in]0,1[$ un paramètre technologique, $s\in]0,1[$ le
taux d'épargne, $\delta\in[0,1]$ le taux de dépréciation du capital
physique, et $B$ un paramètre positif. La dynamique du stock de
capital physique est décrite par l'équation suivante :
\[
\dot K(t) = sY(t)-\delta K(t)
\]
et la technologie de production est définie par :
\[
  Y(t) = 
  \begin{cases}
    K(t)^\alpha L(t)^{1-\alpha} - BL(t), \text{ si }K(t)^\alpha L(t)^{1-\alpha} \geq BL(t)\\
    0, \text{ sinon.}
  \end{cases}
\]
ou de façon équivalente par :
\[
Y(t) = \max \left(K(t)^\alpha L(t)^{1-\alpha} - BL(t), 0\right)
\]
\question{1} La fonction de production est-elle néoclassique ?
Pourquoi ? \question{2} Écrire la fonction de production intensive,
c'est-à-dire exprimer la production par tête en fonction du stock de
capital par tête. \question{3} Représenter graphiquement la fonction
de production intensive, en justifiant la construction du
graphique. \question{4} Écrire la dynamique du stock de cpaital par
tête. Interpréter. \question{5} Représenter graphiquement les
variations du stock de capital. \question{6} Montrer que, selon la
valeur du paramètre $B$, il peut exister zéro, un ou deux états
stationnaires strictment positif. Donner explicitement la condition
sur $B$. Notons qu'il existe toujours un état stationnaire (avec un
stock de capital nul). \question{7} Dans le cas où il n'existe pas
d'état stationnaire, quelle est la prédiction du modèle sur le long
terme de l'économie ? \question{8} Dans le cas où il existe deux états
stationnaires positifs, quelle est la prédiction de modèle sur le long
terme ? Discuter. \question{9} Notons
$\underline{k}^{\star}<\overline{k}^{\star}$ les deux états
stationnaires. Quelles sont les conséquences, sur
$\underline{k}^{\star}$ et $\overline{k}^{\star}$, d'une augmentation
permanente du taux d'épargne ?\newline


\end{document}

%%% Local Variables:
%%% mode: latex
%%% TeX-master: t
%%% End:
