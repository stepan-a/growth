\documentclass[10pt,a4paper,notitlepage,twocolumn]{article}
\synctex=1


\usepackage[margin=3cm]{geometry}
\usepackage{amsmath,amssymb,amsbsy}
\usepackage{float}
\usepackage[french]{babel}
\usepackage{graphicx}

\usepackage[utf8x]{inputenc}
\usepackage[T1]{fontenc}
\usepackage{palatino}
\usepackage{manfnt}
\usepackage{hyperref}
\usepackage{nicefrac}

\usepackage[active]{srcltx}
\usepackage{scrtime}

\newcommand{\exercice}[1]{\textsc{\textbf{Exercice}} #1}
\newcommand{\question}[1]{\textbf{(#1)}}
\setlength{\parindent}{0cm}

\begin{document}

\title{\textsc{Croissance et Développement}}
\author{\textsc{Université du Maine (Partiel, L2)}}
\date{Mardi 11 décembre 2018}

\maketitle
\thispagestyle{empty}


\exercice{1} Soient $K(t)$ le stock de capital physique d'une économie
à l'instant $t$, $L(t)$ la population qui croît au taux constant
$n>0$, $\alpha\in]0,1[$ un paramètre technologique, $s\in]0,1[$ le
taux d'épargne et $\delta\in[0,1]$ le taux de dépréciation du capital
physique. La dynamique du stock de capital physique est décrite par
l'équation suivante :
\[
\dot K(t) = sY(t)-\delta K(t)
\]
et la technologie de production est de type Cobb-Douglas :
\[
Y(t) = K(t)^\alpha L(t)^{1-\alpha}
\]
\question{1} Donnez une interprétation du paramètre $\alpha$ de la
technologie de production. \question{2} Décrivez et interprétez la
dynamique du capital par tête
$k(t)=\nicefrac{K(t)}{L(t)}$. \question{3} Donnez une expression
analytique du taux de croissance du stock de capital physique par
tête. \question{4} Représentez graphiquement le taux de croissance du
stock de capital physique par tête (en expliquant la construction du
graphique) et commentez. \question{5} Calculez l'état stationnaire du
modèle (pour le capital par tête, la production par tête et la
consommation par tête). \question{6} Quel est le rapport entre l'état
stationnaire du stock de capital par tête et le niveau de long terme
du capital par tête ? Justifiez votre réponse.\newline

\exercice{2} Dans cet exercice on considère un modèle de Solow
augmenté d'une seconde variable d'état: le stock de capital humain.
Soit la fonction de production :
\[
Y(t) = K(t)^{\alpha}K(t)^{\lambda}L(t)^{1-\alpha-\lambda}
\]
Les variables $Y$,  $K$, et $L$ ont  les interprétations usuelles.
$H$ est le niveau de capital humain.  Les paramètres $\alpha$ et
$\lambda$ sont  positifs et  vérifient $\alpha+\lambda<1$.  Le  stock de
capital physique se  déprécie au taux $\delta\in]0,1[$.   La loi d'évolution
du stock de capital physique est donnée par :
\[
\dot{K}(t) = s_{K} Y(t) - \delta K(t)
\]
où $s_{K}\in]0,1[$ est le taux d'épargne en capital physique. La loi
d'évolution du stock de capital humain est donnée par :
\[
\dot{H}(t) = s_{H} Y(t) - \delta H(t)
\]
où $s_{H}\in]0,1[$ est le taux d'épargne en capital
humain. \question{1} Montrez que la dynamique jointe des variables par
tête est caractérisée par :
\[
  \begin{cases}
    \dot{k}(t) &= s_{K}k(t)^{\alpha}h^{\lambda} - (n+\delta)k(t)\\
    \dot{h}(t) &= s_{H}k(t)^{\alpha}h^{\lambda} - (n+\delta)h(t)\\
  \end{cases}
\]
Commentez. \question{2} Posez le système d'équations qui permet de
calculer l'état stationnaire. \question{3} Montrez qu'à l'état
stationnaire on doit avoir :
\[
\frac{k^{\star}}{h^{\star}} = \frac{s_{K}}{s_{H}}
\]
Commentez. \question{4} Calculez l'état stationnaire pour le capital
physique par tête, le capital humain par tête, la production par tête
et la consommation par tête. Vous devriez trouver :
\[
c^{\star} = (1-s_{K}-s_{H})\left(\frac{s_{K}}{n+\delta}\right)^{\frac{\alpha}{1-\alpha-\lambda}}\left(\frac{s_{H}}{n+\delta}\right)^{\frac{\lambda}{1-\alpha-\lambda}}
\]
\question{5} Décrivez l'arbitrage, relatif au niveau de la
consommation par tête à l'état stationnaire, dans les choix
d'épargne. Calculez les taux d'épargne de la rêgle d'or, on notera
$s_{K}^{\mathrm{or}}$ et $s_{H}^{\mathrm{or}}$, c'est-à-dire les
valeurs de $s_{K}$ et $s_{H}$ qui maximisent le niveau de la
consommation par tête à l'état stationnaire. \question{6} Supposons
que le taux d'épargne en capital humain soit fixé, égal à
$\bar{s}_{H}<s_{H}^{\mathrm{or}}$. Calculez le taux d'épargne en
capital physique qui maximise $c^{\star}$, on notera $s_{K}^{\star}$
le taux d'épargne optimal. \question{7} Montrez que
$s_{K}^{\star}>s_{K}^{\mathrm{or}}$ mais que
$s_{K}^{\star}+\bar{s}_{H}<s_{K}^{\mathrm{or}}+s_{H}^{\mathrm{or}}$. Commentez.

\end{document}

%%% Local Variables:
%%% mode: latex
%%% TeX-master: t
%%% End:
