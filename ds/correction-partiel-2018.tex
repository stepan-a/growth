\documentclass[10pt,a4paper,notitlepage,twocolumn]{article}
\synctex=1


\usepackage[margin=3cm]{geometry}
\usepackage{amsmath,amssymb,amsbsy}
\usepackage{float}
\usepackage[french]{babel}
\usepackage{graphicx}

\usepackage[utf8x]{inputenc}
\usepackage[T1]{fontenc}
\usepackage{palatino}
\usepackage{manfnt}
\usepackage{hyperref}
\usepackage{nicefrac}

\usepackage[active]{srcltx}
\usepackage{scrtime}

\newcommand{\exercice}[1]{\textsc{\textbf{Exercice}} #1}
\newcommand{\question}[1]{\textbf{(#1)}}
\setlength{\parindent}{0cm}

\begin{document}

\title{\textsc{Croissance et Développement}}
\author{\textsc{Université du Maine (Éléments de correction du partiel, L2)}}
\date{Mardi 11 décembre 2018}

\maketitle
\thispagestyle{empty}


\exercice{1} Soient $K(t)$ le stock de capital physique d'une économie
à l'instant $t$, $L(t)$ la population qui croît au taux constant
$n>0$, $\alpha\in]0,1[$ un paramètre technologique, $s\in]0,1[$ le
taux d'épargne et $\delta\in[0,1]$ le taux de dépréciation du capital
physique. La dynamique du stock de capital physique est décrite par
l'équation suivante :
\[
\dot K(t) = sY(t)-\delta K(t)
\]
et la technologie de production est de type Cobb-Douglas :
\[
Y(t) = K(t)^\alpha L(t)^{1-\alpha}
\]
\question{1} Le paramètre $\alpha$ s'interprète comme l'élasticité de
la production par rapport au stock de capital physique. En effeft,
nous avons :
\[
  \begin{split}
    \epsilon_{Y/K} &= \frac{F_{K}(K,L)}{\frac{F(K,L)}{K}}\\
  &= \frac{\alpha K^{\alpha-1}L^{1-\alpha}}{K^{\alpha-1}L^{1-\alpha}}\\
  &= \alpha
\end{split}
\]
où $F(K,L)$ est la fonction de production et $F_{K}(K,L)$ sa dérivée
partielle par rapport au stock de capital physique $K$. \question{2}
Nous allons caractériser la dynamique du stock de capital par tête en
calculant les variations de $k$ à chaque instant. Nous avons :
\[
\dot k(t) = \frac{\mathrm d}{\mathrm dt} \left(\frac{K(t)}{L(t)}\right)
\]
En appliquant la formule usuelle de dérivation d'un ratio de
fonctions, nous obtenons :
\[
  \begin{split}
    \dot k(t) &= \frac{\dot K(t)L(t)-K(t)\dot L(t)}{L(t)^{2}}\\
    &= \frac{\dot K(t)}{L(t)} - \frac{K(t)}{L(t)}\frac{\dot L(t)}{L(t)}\\
    &= \frac{\dot K(t)}{L(t)} - k(t) n \\
  \end{split}
\]
puisque le taux de croissance de la population est supposé constant
est égal à $n$. En substituant la loi d'évolution du stock de capital
physique agrégé il vient :
\[
\dot k(t) =  \frac{sY(t)-\delta K(t)}{L(t)} - n k(t)
\]
soit encore :
\[
\dot k(t) =  \frac{sY(t)}{L(t)} - (n+\delta) k(t)
\]
En notant que la production par tête, $y(t)=Y(t)/L(t)$, peut s'exprimer en fonction du
stock de capital physique par tête :
\[
y(t) = k(t)^{\alpha}
\]
nous avons finalement :
\[
\dot k(t) = s k(t)^{\alpha} - (n+\delta)k(t)
\]
Le stock de capital physique par tête augmente, c'est-à-dire sa
variation est positive, si et seulement si l'investissement par tête,
$sk(t)^{\alpha}$ est supérieur à la dépréciation du stock de capital
physique par tête, $(n+\delta)k(t)$. \question{3} Le taux de
croissance d'une variable est le rapport de la variation et du
niveau. Ainsi, nous obtenons le taux de croissance du stock de capital
physique par tête en divisant les deux membres de la dernière équation
par le niveau du stock de capital physique par tête :
\[
g_{k}(t) = s k(t)^{\alpha-1} - (n+\delta)
\]
Le taux de croissance est positif si et seulement si l'investissement
brut par unité de capital physique est supérieur au taux de
dépréciation du stock de capital physique par tête,
$n+\delta$. \question{4} La représentation graphique du taux de
croissance est décrite en plusieurs endroits dans le cours. Il ne faut
pas oublier d'expliquer pourquoi la courbe du taux de croissance est
monotone décroissante, pourquoi le taux de croissance diverge vers
l'infini lorsque le stock de capital par tête tend vers zéro et
pourquoi ce même taux de croissance tend vers $-(n+\delta)$ lorsque le
stock de capital tend vers l'infini. \question{5} À l'état
stationnaire, s'il existe, les variables par têtes sont
constantes. Pour que la variation du stock de capital physique par
tête soit nulle il faut que l'investissement par tête soit égal à la
dépréciation du stock de capital physique par tête. Si on note
$k^\star$ l'état stationnaire du stock de capital physque par tête, on
doit donc avoir :
\[
  s\left. k^{\star} \right.^{\alpha} = (n+\delta) k^{\star}
\]
En excluant le la solution nulle et en supposant que $k^{\star}$ est
strictement positif, on peut diviser les deux membres de l'égalité par
$k^{\star}$ et on obtient :
\[
 \left. k^{\star} \right.^{\alpha} = \frac{n+\delta}{s}
\]
soit de façon équivalente :
\[
k^{\star} = \left(\frac{s}{n+\delta}\right)^{\frac{1}{1-\alpha}}
\]
On note que l'état stationnaire du stock de capital physique par tête
est une fonction monotone croissante du taux d'épargne (car
$\alpha<1$). On en déduit l'état stationnaire de la production par
tête en substituant ce résultât dans la technologie intensive (pour
tout $k$ on a $y=k^{\alpha}$) :
\[
y^{\star} = \left(\frac{s}{n+\delta}\right)^{\frac{\alpha}{1-\alpha}}
\]
Puisque $\alpha>0$, l'état stationnaire de la production par tête est
aussi une fonction monotone croissante du taux d'épargne. On obtient
directement la consommation par tête à l'état stationnaire en
rapellant que la consommation est une part constante de la production,
on a :
\[
c^{\star} = (1-s)\left(\frac{s}{n+\delta}\right)^{\frac{\alpha}{1-\alpha}}
\]
Le lien entre $c^{\star}$ et le taux d'épargne est moins trivial ici.
Une augmentation du taux d'épargne induit une augmentation de la
production à l'état stationnaire \emph{et} une diminution de la part
consommée de la production. L'impact total sur le niveau de la
consommation par tête à l'état stationnaire dépend de l'ampleur de ces
deux effets. \question{6} L'état stationnaire est le niveau de long
terme de l'économie. Dit autrement, pour toute condition initiale
$k(0)>0$ le stock de capital physique par tête converge vers l'état
stationnaire $k^{\star}$,
$\lim_{t\rightarrow\infty}k(t)=k^{\star}$. On peut étayer ce résultat
graphiquement à l'aide d'un diagramme des phases. En reprenant le
graphique du taux de croissance (question 4). L'état stationnaire
$k^{\star}$ est positionné à l'intersection de la courbe du taux de
croissance (c'est-à-dire l'investissement net par unité de capital) et
l'axe des abscisses. On remarque que le taux de croissance est positif
si et seulement si $k<k^{\star}$. Autrement dit, le stock s'accroît
lorsqu'il est inférieur à $k^{\star}$ et décroit lorsqu'il est
supérieur à $k^{\star}$. De plus la variation du stock de capital
physique par tête est d'autant plus faible que l'économie est proche
de l'état stationnaire. Ainsi à long terme l'économie rejoint l'état
stationnaire. Il est possible de retrouver cet argument graphique dans
l'expression du taux de croissance :
\[
  \begin{split}
    g_{k}(t) &= s k(t)^{\alpha-1} - (n+\delta)\\
    &= s \left(k(t)^{\alpha-1} - \frac{n+\delta}{s}\right)\\
    &= s \left(k(t)^{\alpha-1} - \left. k^{\star}\right.^{\alpha-1}\right)\\
    &= s \left. k^{\star}\right.^{\alpha-1} \left( \left(\frac{k(t)}{k^{\star}}\right)^{\alpha-1} - 1\right)\\
    &= (n+\delta)\left( \left(\frac{k(t)}{k^{\star}}\right)^{\alpha-1} - 1\right)
  \end{split}
\]

On voit bien ici, comme sur le graphique, que (\textit{i}) le taux de croissance est positif si
et seulement si l'économie est sous l'état stationnaire, (\textit{ii})
le taux de croissance est d'autant plus faible (en valeur absolue) que
l'économie est proche de l'état stationnaire (c'est-à-dire
$k(t)/k^{\star}$ proche de 1).


\exercice{2} Dans cet exercice on considère un modèle de Solow
augmenté d'une seconde variable d'état: le stock de capital humain.
Soit la fonction de production :
\[
Y(t) = K(t)^{\alpha}K(t)^{\lambda}L(t)^{1-\alpha-\lambda}
\]
Les variables $Y$,  $K$, et $L$ ont  les interprétations usuelles.
$H$ est le niveau de capital humain.  Les paramètres $\alpha$ et
$\lambda$ sont  positifs et  vérifient $\alpha+\lambda<1$.  Le  stock de
capital physique se  déprécie au taux $\delta\in]0,1[$.   La loi d'évolution
du stock de capital physique est donnée par :
\[
\dot{K}(t) = s_{K} Y(t) - \delta K(t)
\]
où $s_{K}\in]0,1[$ est le taux d'épargne en capital physique. La loi
d'évolution du stock de capital humain est donnée par :
\[
\dot{H}(t) = s_{H} Y(t) - \delta H(t)
\]
où $s_{H}\in]0,1[$ est le taux d'épargne en capital
humain. \question{1} On montre facilement que la dynamique jointe des variables par
tête est caractérisée par :
\[
  \begin{cases}
    \dot{k}(t) &= s_{K}k(t)^{\alpha}h^{\lambda} - (n+\delta)k(t)\\
    \dot{h}(t) &= s_{H}k(t)^{\alpha}h^{\lambda} - (n+\delta)h(t)\\
  \end{cases}
\]
Pour la dynamique du stock de capital physique par tête, on a :
\[
  \begin{split}
    \dot{k}(t) &= \frac{\dot K(t)L(t) - K(t)\dot L(t)}{L(t)^2}\\
    &= \frac{\dot K(t)}{L(t)} - nk(t)\\
    &= \frac{s_KY(t)}{L(t)}-(n+\delta)k(t)\\
    &= s_Kk(t)^{\alpha}h^{\lambda}-(n+\delta)k(t)
  \end{split}
\]
La derrnière ligne vient de l'hypothèse de rendement d'échelle
constant (homogénéité de degré un de la fonction de production) qui
nous permet d'écrire la production par tête comme une fonction du
stock de capital physique par tête et su stock de capital humain par
tête :
\[
  \begin{split}
    y(t) &= \frac{K(t)^{\alpha}H(t)^{\lambda}L(t)^{1-\alpha-\lambda}}{L(t)}\\
    &= \left(\frac{K}{L(t)}\right)^{\alpha}\left(\frac{H(t)}{L(t)}\right)^{\lambda}\left(\frac{L(t)}{L(t)}\right)^{1-\alpha-\lambda}\\
    &= k(t)^{\alpha}h^{\lambda}
  \end{split}
\]
Nous avons suivi exactement la même démarche que dans l'exercice 1, la
seule chose qui change c'est l'expression de la fonction de
production. La variation du stock de capital physique par tête est
positive si et seulement si l'investissement en capital physique est
supérieur à la dépréciation du stock de capital physique par tête. On
retrouve la seconde équation en suivant les mêmes étapes. On note que
la variation du capital physique (humain) par tête dépend non
seulement du niveau de stock de capital physique (humain) par tête,
comme dans l'exercice 1, mais aussi du niveau du stock de capital
humain (physique). Les deux dynamiques sont jointes, car la production
est déterminée par les quantités de capital physique et humain, on ne
peut étudier la dynamique du capital physique indépendamment de la
dynamique du capital humain. \question{2} L'état stationnaire $(k^{\star}, h^{\star})$ doit vérifier :
\[
  \begin{cases}
    s_{K}\left. k^{\star}\right.^{\alpha}\left. h^{\star}\right.^{\lambda} &= (n+\delta)k^{\star}\\
    s_{H}\left. k^{\star}\right.^{\alpha}\left. h^{\star}\right.^{\lambda} &= (n+\delta)h^{\star}\\
  \end{cases}
\]
À l'état stationnaire l'épargne par tête (en capital physique ou
humain) doit être égale à la dépréciation (du capital physique ou
humain). \question{3} En faisant le rapport des deux équations
précédantes, on trouve directement :
\[
\frac{k^{\star}}{h^{\star}} = \frac{s_{K}}{s_{H}}
\]
Ce résultat intermédiaire est intuitif. Si une économie épargne
relativement plus en capital physique (par example) elle sera
caractérisée par un état stationnaire plus élevé du capital physique
(relativement au capital humain).

\question{4} En exprimant $k^{\star}$ en fonction de $h^{\star}$, on
utilise la dernière équation, et en substituant dans le seconde
équation du système définissant l'état stationnaire, on obtient :
\[
s_H \left[\frac{s_K}{s_H}h^{\star}\right]^{\alpha}\left. h^{\star}\right. ^{\lambda} = (n+\delta)h^{\star}
\]
soit de façon équivalente :
\[
s_H^{1-\alpha}s_K^{\alpha}\left.h^{\star}\right.^{\alpha+\lambda}
\]
et donc :
\[
h^{\star} = \left(\frac{s_K^{\alpha}s_H^{1-\alpha}}{n+\delta}\right)^{\frac{1}{1-\alpha-\delta}}
\]
De la même façon on trouve
\[
k^{\star} = \left(\frac{s_K^{1-\lambda}s_H^{1-\lambda}}{n+\delta}\right)^{\frac{1}{1-\alpha-\delta}}
\]
En substituant dans la fonction de production par tête :
\[
y^{\star} = \left(\frac{s_K^{1-\lambda}s_H^{1-\lambda}}{n+\delta}\right)^{\frac{\alpha}{1-\alpha-\delta}}\left(\frac{s_K^{\alpha}s_H^{1-\alpha}}{n+\delta}\right)^{\frac{\lambda}{1-\alpha-\delta}}
\]
Soit en réarrangeant :
\[
y^{\star} = \left(\frac{s_{K}}{n+\delta}\right)^{\frac{\alpha}{1-\alpha-\lambda}}\left(\frac{s_{H}}{n+\delta}\right)^{\frac{\lambda}{1-\alpha-\lambda}}
\]
Finalement, que la consommation est une part constante de la
production, plus précisemment $c(t) = (1-s_K-s_H)y(t)$ pour tout $t$,
on obtient l'état stationnaire de la consommation par tête :
\[
c^{\star} = (1-s_{K}-s_{H})\left(\frac{s_{K}}{n+\delta}\right)^{\frac{\alpha}{1-\alpha-\lambda}}\left(\frac{s_{H}}{n+\delta}\right)^{\frac{\lambda}{1-\alpha-\lambda}}
\]
Le paramètre $\nicefrac{\alpha}{1-\alpha-\lambda}$ s'interprête comme
l'élasticité de la production par tête à l'état stationnaire par
rapport au taux d'épargne en capital physique. On remarque que cette
élasticité est plus élevée que dans le modèle de Solow
standard. L'introduction d'un second facteur accumulable magnifie la
sensibilité de la production au taux d'épargne en capital
physique. Cette augmentation de la sensibilité de la production au
taux d'épargne en capital physique s'explique par le fait qu'une
augmentation du taux d'épargne en capital physique affecte aussi
l'accumulation du capital humain, qui à son tour (effet de second
ordre) affecte la production. \question{5} Une augmentation du taux
d'épargne (en capital physique ou humain) augmente le revenu par tête
à l'état stationnaire, mais simultanément décroît la part consommée de
la production. L'effet sur le niveau de la consommation à l'état
stationnaire est donc non trivial. Il faut peser les conséquences
relatives du changement de répartition (entre épargne et consommation)
et l'accroissement de la production (à partager). Considérons les cas
polaires. Choisir une épargne est nulle ($s_K=s_H=0$), c'est décider
de consommer la totalité de la production mais aussi décider de ne
rien épargner. Ainsi à l'état stationnaire les facteurs accumulables
et donc la production sont nuls. Avec un taux d'épargne nul la
consommation par tête à l'état stationnaire est nulle. Choisir
d'épargner toute la production ($s_K+s_H=1$) devrait permettre de
maximiser la production à l'état stationnaire (cela doit dépendre de
la répartition entre épargne en capital physique et épargne en capital
humain). Mais à nouveau dans ce cas aussi, quelque soit le niveau de
la production, le niveau de la consommation par tête sera nul à l'état
stationnaire (puisque le ménage décide d'épargner la totalité de la
production il ne reste rien à manger). Dans les situations
intermédiaires ($0<s_K+s_H<1$, $s_K>0$ et $s_H>0$), on obtient un
niveau de consommation par tête strictement positif à l'état
stationnaire. Par exemple, si on part d'une situation où l'épargne est
nulle et décide d'augmenter d'un montant arbitrairement petit les deux
taux d'épargnes, alors $1-s_K-s_H$ va baisser et la production par
tête $y^{\star}$ va augmenter. Mais l'ampleur de l'augmentation de la
production sera bien plus importante que celle de la baisse de la part
de la production consommée\footnote{En zéro la pente de
  $y^{\star}(s_K)$ est beaucoup plus importante que la pente de
  $1-s_K-s_H$}. À la rêgle d'or, nous devons avoir :
\[
  \begin{cases}
    \frac{\mathrm d}{\mathrm ds_K}c^{\star}(s_K, s_H) &= 0\\
    \frac{\mathrm d}{\mathrm ds_H}c^{\star}(s_K, s_H) &= 0\\
  \end{cases}
\]
c'est-à-dire :
\[
  \begin{cases}
    (1-s_K-s_H)\frac{\mathrm d}{\mathrm ds_K}y^{\star}(s_K, s_H) &= y^{\star}\\
    (1-s_K-s_H)\frac{\mathrm d}{\mathrm ds_H}y^{\star}(s_K, s_H) &= y^{\star}\\
  \end{cases}
\]
On montre facilement que :
\[
\frac{\mathrm d}{\mathrm ds_K}y^{\star}(s_K, s_H) = \frac{1}{s_K}\frac{\alpha}{1-\alpha-\lambda}y^{\star}
\]
et
\[
\frac{\mathrm d}{\mathrm ds_H}y^{\star}(s_K, s_H) = \frac{1}{s_H}\frac{\lambda}{1-\alpha-\lambda}y^{\star}
\]
En substituant dans le système des conditions nécessaires :
\[
  \begin{cases}
    \frac{1-s_K-s_H}{1-\alpha-\lambda} &= \frac{s_K}{\alpha}\\
    \frac{1-s_K-s_H}{1-\alpha-\lambda} &= \frac{s_H}{\lambda}\\
  \end{cases}
\]
En considérant le rapport des deux équations, on doit donc avoir :
\[
\frac{s_K}{s_H} = \frac{\alpha}{\lambda}
\]
En exprimant $s_H$ en fonction de $s_H$ et en substituant dans la première équation du système, on trouve facilement :
\[
s_K^{\mathrm{or}} = \alpha
\]
et donc
\[
s_H^{\mathrm{or}} = \lambda
\]
les taux d'épargnes de la rêgle d'or. Finalement le niveau de consommation de la rêgle d'or est :
\[
c_{\mathrm{or}}^{\star} = (1-\alpha-\lambda)\left(\frac{\alpha}{n+\delta}\right)^{\frac{\alpha}{1-\alpha-\lambda}}\left(\frac{\lambda}{n+\delta}\right)^{\frac{\lambda}{1-\alpha-\lambda}}
\]
\question{6} Supposons que le taux d'épargne en capital humain soit
fixé, égal à $\bar{s}_{H}<s_{H}^{\mathrm{or}}$. Clairement cette
situation est sous optimale. Le seul instrument disponible pour
maximiser le niveau de la consommation par tête à l'état stationnaire
est alors le taux d'épargne en capital physique. À l'optimum contraint, on doit avoir :
\[
\frac{1-s_K-\bar{s}_H}{1-\alpha-\lambda} = \frac{s_K}{\alpha}
\]
En résolvant cette équation pour le taux d'épargne en capital physique, on trouve :
\[
s_K^{\star} = \alpha \left[\frac{1}{1-\lambda}(1-\bar{s}_H)\right]
\]
le taux d'épargne en capital physique qui maximise le niveau de la
consommation à l'état stationnaire sachant que le taux d'épargne en
capital humain est fixé à $\bar{s}_H$. Notons que l'on retrouve le
taux d'épargne de la rêgle d'or si et seulement si
$\bar{s}_H=\lambda$. \question{7} Quand $\bar{s}_H<s_H^{\mathrm{or}}$ on a :
\[
\frac{1-\bar{s}_H}{1-\lambda}>1
\]
et donc :
\[
s_K^{\star}>s_K^{\mathrm{or}}
\]
Autrement dit, on essaye ici de compenser la sous accumulation en
capital humain en accumulant plus de capital physique. Cependant, quelques calculs montrent que :
\[
s_K^{\star}+\bar{s}_H = s_K^{\mathrm{or}}+s_H^{\mathrm{or}} + \frac{1-\alpha-\lambda}{1-\lambda}\left(\bar{s}_H-s_H^{\mathrm{or}}\right) 
\]
Ainsi, lorsque $\bar{s}_H<s_H^{\mathrm{or}}$, même si l'épargne en
capital physique augmente pour compenser la sous accumulation en
capital humain, le taux d'épargne total demeure inférieur à ce qu'il
devrait être pour maximiser la consommation à l'état stationnaire. En
l'absence d'un instrument, on obtient un niveau plus faible de la
consommation à l'état stationnaire (on obtient toujours un <<~score~>>
moins élevé avec une optimisation contrainte qu'avec une optimisation
non contrainte).



\end{document}

%%% Local Variables:
%%% mode: latex
%%% TeX-master: t
%%% End:
