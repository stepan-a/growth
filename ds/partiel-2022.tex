\documentclass[11pt,a4paper,notitlepage, twocolumn]{article}
\usepackage{amsmath}
\usepackage{amssymb}
\usepackage{amsbsy}
\usepackage{float}
\usepackage[french]{babel}
\usepackage{graphicx}
\usepackage{enumerate}

\usepackage{palatino}

 \usepackage[active]{srcltx}
\usepackage{scrtime}

\newcounter{qnumber}
\setcounter{qnumber}{0}


\newcommand{\question}{\textbf{(\addtocounter{qnumber}{1}\theqnumber)}\,}
\setlength{\parindent}{0cm}


\begin{document}

\title{\textsc{Croissance\\ \small{Examen}}}
\date{}%{Le \today\ à \thistime}

\maketitle

\thispagestyle{empty}

\begin{quote}
  Les réponses non commentées ou insuffisamment détaillées ne seront pas considérées.
\end{quote}

\bigskip

Soient $K(t)$ le stock de capital physique d'une économie à l'instant
$t$, $L(t)$ la population qui croît au taux constant $n>0$,
$\alpha\in]0,1[$ un paramètre technologique, $s\in]0,1[$ le taux
d'épargne et $\delta\in[0,1]$ le taux de dépréciation du capital
physique. La dynamique du stock de capital physique est décrite par
l'équation suivante~:
\[
\dot K(t) = sY(t)-\delta K(t)
\]
et la technologie de production est~:
\[
Y(t) = K(t)^\alpha \bigl(A(t)L(t)\bigr)^{1-\alpha}
\]
où $A(t)$, l'efficacité du travail, croît au taux constant $x>0$. \question
Donner une interprétation au paramètre $\alpha$. \question Caractériser la
dynamique du stock de capital physique par tête efficace. \question Représenter
graphiquement le taux de croissance du capital physique par tête efficace (en
prenant soin de justifier la construction du graphique). \question Calculer
l'état stationnaire non trivial du modèle (pour le capital par tête efficace, la
production par tête efficace et la consommation par tête efficace). \question
Expliquer pourquoi l'état stationnaire est aussi le niveau de long terme de
l'économie. Quelle est l'hypothèse au c\oe ur de cette propriété~? \question
Montrer qu'une augmentation permanente du taux d'épargne induit une augmentation
du niveau de long terme de la production par tête efficace. \question Quelle est
la valeur du taux d'épargne qui permet de maximiser le niveau de la production
par tête efficace à long terme~? \question Pourquoi est-il moins simple de
déterminer l'effet de l'augmentation permanente sur le niveau de long terme de
la consommation par tête~? \question Montrer qu'il existe un taux d'épargne
optimal strictement positif et strictement inférieur à un, que nous noterons
$s_{\text{or}}$, qui maximise le niveau de la consommation par tête à long
terme. \question Supposons que le taux d'épargne de l'économie effectif soit
différent de $s_{\text{or}}$ et que l'économie soit initialement (à l'instant 0)
à l'état stationnaire. Représenter graphiquement la dynamique de la consommation
par tête efficace, si le comportement des ménages change de façon à adopter le
comportement d'épargne optimal (il faut représenter la transition de la
consommation par tête efficace vers le nouvel état stationnaire). \question
Est-il toujours évident que les ménages soivent adopter le comportement
d'épargne optimal~? Décrire les arbitrages des ménages. \question Comment la
vitesse de convergence vers l'état stationnaire peut-elle affecter les éventuels
arbitrages des ménages~?

\end{document}

%%% Local Variables:
%%% mode: latex
%%% TeX-master: t
%%% End:
